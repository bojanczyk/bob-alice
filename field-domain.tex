% LTeX: language=en
\section{Field outputs}
\label{sec:field-domain}
In this section, we discuss functions where the output domain is a field, equipped with addition and multiplication. We prove that protocols have  exactly the same expressive power as weighted automata.
We begin by recalling the notion of weighted automata.

\paragraph*{Weighted automata.} \AP A \kl{weighted automaton} is a device that
is used to compute a function from strings to a field (more generally, a
semiring, but we consider the case of fields here). This model was originally
introduced by \schutz~\cite{schutzenberger1961definition}. A weighted automaton is defined like a nondeterministic automaton, except that the transitions have weights in the field, and instead of choosing subsets of initial and final states, we also have assignments of weights. A weighted automaton makes sense not only for fields, but also for semirings, so we define it that way. 
\begin{definition}
    [Weighted automaton] 
    \label{def:weighted-automaton-nondeterministic}
    \AP
    A \intro[wa-nfa]{weighted automaton} over a semiring $\domain$ consists of a finite input alphabet $\Sigma$,  a finite set of states $Q$, and functions: 
    \begin{align}
        \label{eq:weight-functions}
    \myunderbrace{I : Q \to \domain}{initial}
    \quad
    \myunderbrace{F : Q \to \domain}{final}
    \quad
    \myunderbrace{\Delta : Q \times \Sigma \times Q \to \domain}{transitions}.
    \end{align}
\end{definition}
A run of this automaton is defined in the usual way: it is a sequence of
transitions, one for each input letter, such that consecutive transitions
agree on the connecting states. The weight of a run is the product of: (1) the
initial weight of its source state; (2) the weights used by its transitions;
and (3) the final weight of its target state. If the semiring is
non-commutative, the order of multiplication is  important, and transitions
are multiplied in the order corresponding to the input string. For an input
string, the output of the automaton is the sum of weights of all runs over
this automaton (sum is always commutative). 

% The model from
% Definition~\ref{def:weighted-automaton-nondeterministic}  defines the same
% functions as the model from Definition~\ref{def:weighted-automaton}, with the
% appropriate semiring modifications that were described earlier in this
% section, see~\cite[Lemma 8.3]{bojanczyk_automata_2025}. For the purposes of
% this section, we use the nondeterministic  model described from
% Definition~\ref{def:weighted-automaton-nondeterministic}.


% Essentially, this is
% a deterministic automaton where the state space is a vector space of finite
% dimension, and each input letter induces a linear map. Here is the formal
% definition. 

% \ref{def:weighted-automaton-nondeterministic}

% \begin{definition}[Weighted automaton]
%     \label{def:weighted-automaton}
%     \AP
%     A \intro{weighted automaton} over a field $\domain$ is given by: 
%     \begin{enumerate}
%         \item a finite input alphabet $\Sigma$;
%         \item a \intro(wa){dimension} $d \in \set{0,1,\ldots}$;
%         \item an initial state $q_0 \in \domain^d$;
%         \item \label{it:weighted-definition-transitions} 
%           for each letter $a \in \Sigma$, a 
%           corresponding linear map of type $\domain^d \to \domain^d$;
%         \item \label{it:weighted-definition-final} a \intro(wa){final map},
%           which is a linear map of type $\domain^d \to \domain$. 
%     \end{enumerate}
% \end{definition}

% A \kl{weighted automaton} computes a function of type $\Sigma^* \to \domain$,
% which is defined in the same way as for a deterministic finite automaton: we
% begin in the initial state, then we apply the linear maps corresponding to the
% input letters, and finally we apply the \kl(wa){final map}. There is an
% alternative but equivalent way of describing weighted automata, which uses a
% nondeterministic automaton with weights on transitions. This viewpoint will be
% used later in this paper, see
% Definition~\ref{def:weighted-automaton-nondeterministic}.

The main result of this section is that our protocol is equivalent to weighted automata, as stated in the following theorem from the introduction, which we now recall:
\fielddomain*

% \begin{theorem}\label{thm:field-domain}
%     Assume that the domain is a field. Then a function 
%     \begin{align*}
%     f : \Sigma^* \to \domain
%     \end{align*}  is computed by a protocol if and only if it is  computed by a weighted automaton.
% \end{theorem}

As we will see in Example~\ref{ex:non-commutative-semirings} below, the theorem is false for general semirings that are not fields. It is possible that it is true for commutative semirings, but this is only a conjecture at this point. The problem with such generalisations is that our proof uses the  Fliess Theorem, which is only known  for fields.  Before proving the theorem in
Section~\ref{sec:proof-of-thm-field-domain}, we return to the issue of
division, which was already discussed in Example~\ref{ex:division}.

\begin{myexample}[Division, continued]\label{ex:division-continued}
    Because it is undefined for zero, division is not a total operation, and
    therefore technically speaking it does not fall into our framework. We
    could, however try to incorporate it, by making the two parties responsible
    for avoiding division by zero. Under this framework, we could use a
    \kl{protocol} to compute the function $1/|w|$ (a better choice would be
    $1/(|w|+1)$, since it would avoid problems with the empty string). As we have
    discussed in Example~\ref{ex:division}, such a function cannot be computed
    by a \kl{protocol} that uses only addition and multiplication. We do not know
    what functions can be computed if division is also allowed.
\end{myexample}



\begin{myexample}[Semiring outputs]
    \label{ex:non-commutative-semirings} 
    In this example we show that for semirings which are not fields, the \kl{protocol} need not be equivalent to \kl{weighted automata}. The implication 
    \begin{align*}
    \text{protocol} \quad \impliedby \quad \text{weighted automaton}
    \end{align*}
    in \cref{thm:field-domain}, as we will see in a moment, holds for any
    semiring, and therefore the problematic implication is the other one. Here
    is an example where it fails. Let $\domain$ be  the free (non-commutative)
    idempotent semiring generated by two letters $a$ and $b$. Elements of this
    semiring are finite sets of words in $\set{a,b}^*$, such as 
    \begin{align*}
    \set{3ab, 5ba, 7aab}
    \end{align*}
    The addition operation is multiset union, 
    and the multiplication operation is concatenation of words, 
    extended to sets in the natural way, as illustrated on this example:
    \begin{align*}
    \set{a,b}\cdot \set{a,b} = \set{aa, ab, ba, bb}.
    \end{align*}
    \kl{Weighted automata} over this semiring are the same as the 
    rational relations~\cite[Chapter IX]{Eilenberg74}. 
    On the other hand, a protocol can define string-to-$\domain$ functions
    that are not rational. This is witnessed already by functions that produce 
    singleton sets (call these singleton functions), which can be seen as 
    functions of type $\Sigma^* \to \set{a,b}^*$. For example, consider 
    the singleton version of the  reverse function, i.e.
    \begin{align*}
    w \mapsto \set{\text{reverse of $w$}} \in \domain.
    \end{align*}
    This function can be computed by a \kl{protocol}, using the same idea as in 
    Example~\ref{ex:reverse-duplicate}. This function, however, is not
    a rational relation, and therefore it is not computed by a weighted automaton over $\domain$. 

    We know more in the case of singleton functions. In this case, all messages sent during the protocol must be singletons (this is because once a non-singleton is produced, it can never be turned into a singleton). Therefore, the operation $+$ can never be used in a non-trivial way, and thus the protocol can only use muliplication. This means that it coincides with the protocols with outputs that are strings with concatenation, as discussed in Section~\ref{sec:string-outputs}. According to Conjecture~\ref{conj:protocol-regular-string-to-string}, the singleton functions are therefore exactly the \kl{regular functions}.
\end{myexample}

\begin{myexample}[Equality tests]
\label{ex:equality-tests}
In this example, we discuss an extension of the \kl{protocol} which allows for
equality tests, similarly to the algebraic group
model~\cite{fuchsbauer2018algebraic}. Clearly, equality tests cannot be
completely unrestricted. Otherwise, in the presence of a countable output
domain (which is the case for all protocols studied in this paper), the
receiver could compare the message with all possible values one by one, until
the correct one would be identified. This would  invalidate the \kl{black box
discipline}. A reasonable restriction is to allow a constant number of equality
tests for each message; this constant can also be brought down to one, by
possibly sending more copies of the same message. The resulting protocol would
be able of complementing a weighted automaton $\Aa$, in the following sense:
\begin{align*}
w \in \Sigma^* 
\quad \mapsto \quad 
\begin{cases}
    1 & \text{if $\Aa(w) =0$}\\
    0 & \text{otherwise}.
\end{cases}
\end{align*}
This form of complementation is undesirable from the point of view of decidability. For example, language equivalence is undecidable for \kl{weighted automata} that are complemented in this way~\cite[Theorem 4.9]{bojanczyk_automata_2025}. 
Since we strive for protocols that describe ``regular'' functions, and such functions should be decidable, we avoid equality tests.
\end{myexample}

\begin{myexample}[Wrong output domains]\label{ex:wrong-output-domains}
 This discussion of equality tests from Example~\ref{ex:equality-tests} also
 explains why we should not expect results about regularity that work for any
 \kl{output domain}. For example, if we would extend the field domain with a unary
 complementation operation 
 \begin{align*}
 x 
 \quad \mapsto \quad 
 \begin{cases}
    1 & \text{if $x =0$}\\
    0 & \text{otherwise},
\end{cases}
 \end{align*}
 then our protocols could recover the undecidable model discussed in the
 previous paragraph.  Of course, one can come up with even more obviously wrong
 \kl{output domains}, such as a domain that consists of Turing machines with certain
 evaluation operations. We do not know where the dividing line is between
 ``right'' and ``wrong'' \kl{output domains}.
\end{myexample}


\subsection{Proof of \cref{thm:field-domain}}
\label{sec:proof-of-thm-field-domain}
\AP
We now return to the proof of \cref{thm:field-domain}. The right-to-left
implication says that every \kl{weighted automaton} can be simulated by a \kl{protocol}.
This is proved essentially in the same way as in the \kl[Boolean protocol]{Boolean case}, and the proof is valid for all semirings and not just fields. 
Suppose that
the function is computed by a \kl{weighted automaton}, which has state space $Q$. To every input string $w \in \Sigma^*$, we can associate a $Q \times Q$ matrix over the semiring. This matrix is defined by 
\begin{align*}
M[p,q] = \sum_\rho \text{product of weights of transitions used by $\rho$},
\end{align*}
where $\rho$ ranges over runs of the weighte automaton that start in state $p$, read the input string $w$, and end in state $q$. This map is a homomorphism from strings to matrices, i.e.~the matrix corresponding to a string $w_1 w_2$ is obtained by multiplying the matrices corresponding to $w_1$ and $w_2$. In the \kl{protocol}, Alice sends
the matrix  which corresponds to her local string, and Bob sends the  matrix
which corresponds to  his local string. These matrices are multiplied using the
semiring operations, and then multiplied with vectors that represent the initial and final weights of states. This
protocol has one round and is \kl{signal-free}, i.e.~no information is conveyed
using signals.

The rest of this proof is devoted to the left-to-right implication,
i.e.~showing that every function computed by a \kl{protocol} is computed by a
\kl{weighted automaton}. As in the Boolean case, we will do a sequence of
reductions, such that the \kl{protocol} becomes more and more restrictive. In
particular, we will show that the \kl{protocol} can be reduced to a version that has
one-round and is \kl{signal-free}.

\subsubsection{Reduction to a scalar product protocol}
\label{sec:reduction-to-scalar-product-protocols}
\AP
In the first step, we show that each \kl{protocol} can be constrained to have a
special form, which has one round and is \kl{signal-free}. This protocol uses only
the scalar product,  as explained in the following definition. 

\begin{definition}[Scalar product protocol] \label{def:scalar-product-protocol}
    Assume that the \kl{output domain} is a field.
    A \intro{scalar product protocol} is defined as follows.
    First, each of the two parties uses their local string to 
    produce a vector of field elements, 
    of some fixed dimension $d$, as expressed by two functions: 
    \begin{align*}
    \sigma_A, \sigma_B : \Sigma^* \to \domain^d.
    \end{align*}
    Next, the output is defined to be the scalar product of the two vectors. 
\end{definition}

This \kl[scalar product protocol]{protocol} has the same power as \kl{general
protocols}. Furthermore, the proof works not only for fields, but also for the more general case of commutative semirings (these are semirings where both addition and muliplication are commutative, in contrast to general semirings, where only addition is assumed to be commutative). The assumption that the output domain is a field, and not just a commutative semiring, will be used at a later stage, when we apply the Fliess Theorem.

\begin{lemma}\label{lem:scalar-product-reduction}
  Assume that the \kl{output domain} is a commutative semiring. 
  If a function is computed by a \kl{protocol}, 
  then it is computed by a \kl{scalar product protocol}.
\end{lemma}
\begin{proof}
    The proof is a sequence of reductions, 
    where more and more conditions are imposed on the protocol.  
    
    \paragraph*{Step 1. One-round protocol.} 
    The first step is to reduce the protocol to a one-round protocol. 
    This is done using \cref{lemma:one-round-reduction-general}.



 \paragraph*{Step 2. Signal-free protocol.}  \AP 
 We say that a protocol is
 \intro{signal-free} if both of the sets $Q_A$ and $Q_B$ have one element each.
 In other words, the signals do not convey any information, and the only
 messaging activity consists of sending elements of the \kl{output domain}. In a
 \kl{signal-free protocol}, the concept of rounds is irrelevant, since the behaviour
 of one party is not influenced by the communication from the other party. The following claim shows that every protocol can be made signal-free, even if the semiring is not necessarily commutative. 

% \omc{Usually we have signal-free protocols when corresponding Machine model expressive power does not increase when we add two-wayness. Here we have one-way weighted automata, in sequential model also we have one-wayness (it can considered as signal-free because consider algebra to be semiring with concatenation and union and one element for zero, then it is signal-free), in commutative semiring the conjecture is having one-way weighted automat, is it worth mentioning that? Answer my comment wehenever you see it pelase.}
% MB: This is not necessarily true, I think. In a non-commutative semiring, two-way ≠ one-way, and yet we can still have signal-free protocols, by using zero, I think.


 \begin{claim}
    \label{claim:trivial-messages}
    Assume that the \kl{output domain} is a semiring, not necessarily commutative. 
    Then every one-round protocol is equivalent to a \kl{signal-free protocol}.
 \end{claim}
 \begin{proof} 
    Consider a one-round protocol. Without loss of generality, we assume that
    both signal spaces $Q_A$ and $Q_B$ are the same space $Q$. (We can always
    use the union of two signal spaces for both parties.) Assume that each of
    the parties sends $d$ semiring elements in the protocol. In other words, the
    protocol works as follows:
    \begin{enumerate}
        \item Based on her local string, Alice chooses a message $(q_A,\bar x) \in Q \times \domain^d$;
        \item Based on his local string, Bob chooses a message $(q_B,\bar y) \in Q \times \domain^d$;
        \item Based on the signals $q_A$ and $q_B$, a term operation  with $2d$ variables is chosen, call it $t_{q_A,q_B}$, and the output is obtained by applying this term operation to $(\bar x, \bar y).$
    \end{enumerate}
    To prove the claim, we need to show that the protocol can be adapted so
    that always the same term operation is chosen, i.e.~there is no dependence
    of this term operation on the signals $q_A$ and $q_B$. This way the signals
    can be eliminated. To do this, we increase the \kl{protocol dimension} from $d$ to $d +
    |Q|$. 

    This means that for each possible signal $q \in Q$, each party sends a
    semiring element corresponding to this signal. This element will be either $0$ or $1$, which are elements that need to exist in every semiring (see~\cite[p.7]{handbook2009} for a formal definition of semirings). The idea is that instead of
    sending a signal $q \in Q$, each party will set the corresponding semiring
    element to $1$, and the remaining semiring elements to $0$. The corresponding
    \kl{term operation} is then 
    \begin{align*}
    \sum_{\substack{q_A \in Q \\ q_B \in Q}} \myoverbrace{x_{q_A} \cdot y_{q_B}}{variables corresponding \\ to the messages $q_A$ and $q_B$, } \cdot t_{q_A,q_B}(\bar x, \bar y).
    \end{align*}
    When evaluating this \kl{term operation},
    the summands that do not correspond to the intended message $(q_A,q_B)$
    will be eliminated, since they will contain a variable that is set to $0$. 
    Only the summand corresponding to the intended message will be used,
    and thus the correct output will be produced. 
 \end{proof}



 \paragraph*{Step 3. Scalar product.}
 In the previous step, we have reduced the \kl{protocol} to a special case, where
 Alice and Bob send vectors, call them $\bar x, \bar y \in \domain^d$, and then
 some fixed \kl{term operation} $t$ with $2d$ variables is applied to them. To
 complete the proof of the lemma, we show that the term operation can be turned
 into a scalar product. This term operation is a sum of monomials, with each
 monomial being a product of some variables. This part of the proof will work for commutative semirings (the previous step worked for all semirings). 
 
 Consider the monomials in the term
 operation $t$. Since muliplication is assumed to be  commutative, for each monomial, its contribution to the output  is obtained
 by multiplying two numbers: (a) the product of the  variables in the term
 operation that are contributed by Alice; and   (b) the product of the
 variables in the term operation that are contributed by Bob. We can redesign
 the protocol so that for each monomial, Alice sends the contribution (a), and
 Bob sends the contribution (b). In the new protocol, the dimension is the
 number of monomials from the original protocol, and the term operation is a
 scalar product. 
\end{proof}

\subsubsection{From a scalar product protocol to a weighted automaton}
\label{sec:from-scalar-product-protocol-to-weighted-automaton}
\AP
In this section, we complete the proof of \cref{thm:field-domain}, by showing
that \kl{scalar product protocols} can be simulated by \kl{weighted automata}.
Similarly to the \kl[boolean protocol]{Boolean case}, the proof uses a
Myhill-Nerode characterization. In the case of \kl{weighted automata}, this
characterization is called  the Fliess Theorem, which characterizes
functions computed by weighted automata in terms of a certain infinite matrix. It is here where we use the assumption that the output domain is a field, and not just a commutative semiring.


\begin{definition}[Hankel Matrix]\label{def:hankel-matrix}
  \AP
    Let $\domain$ be a field. The \intro{Hankel matrix} of a function 
    \begin{align*}
    f : \Sigma^* \to \domain
    \end{align*}  
    is the matrix where rows are words in $\Sigma^*$, columns are words in $\Sigma^*$, and the entry corresponding to a row $u$ and a column $v$ is $f(uv)$.
\end{definition}

Another perspective on the \kl{Hankel matrix} is that it describes the
\intro{derivatives} of the function $f$. Each row in the \kl{Hankel matrix} can be
seen as a function of type $\Sigma^* \to \domain$, which inputs columns
(i.e.~strings) and outputs the corresponding entries in the Hankel matrix. If
the row corresponds to a word $w$, then this function is
\begin{align*}
v \mapsto f(wv),
\end{align*}
which is called the \intro{left derivative} of $f$ with respect to $w$.
Similarly, the columns of the Hankel matrix describe \intro{right derivatives} of $f$.

The Fliess Theorem~\cite[Theorem 2.1.1]{fliess1974} states that a function 
\begin{align*}
f : \Sigma^* \to \domain
\end{align*}
is computed by a \kl{weighted automaton} if and only if  its \kl{Hankel matrix}
has finite rank, i.e.~its rows (i.e.~the \kl{left derivatives}) are spanned by
a finite subset. (This is equivalent to saying that the columns, or \kl{right
derivatives}, have a finite spanning subset.) Therefore, to complete the proof
of \cref{thm:field-domain}, it is enough to show the following lemma.

\begin{lemma}\label{lem:hankel-finite-rank}
    If a function is computed by a \kl{scalar product protocol},
    then its \kl{Hankel matrix} has finite rank.
\end{lemma}
\begin{proof}
  Essentially by definition, the \kl{Hankel matrix} of a function computed by a
  \kl{scalar product protocol} with dimension $d$ can be obtained as a
  sub-matrix of the following matrix: rows and columns are vectors in
  $\domain^d$, and the entries are obtained by taking scalar products. This
  matrix is easily seen to have finite rank, namely $d$, since the scalar
  product  becomes a linear operation once one of the two arguments is fixed.
\end{proof}
