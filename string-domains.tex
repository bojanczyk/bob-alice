\section{String outputs}
\label{sec:string-outputs}

In this section, we consider the case where the output domain is strings over some finite alphabet. We use the name \emph{string-to-string function} for any function of type $\Sigma^* \to \Gamma^*$, where both alphabets $\Sigma$ and $\Gamma$ are finite. For such functions, the protocols are assumed to use the output domain $\Gamma^*$, equipped with concatenation. 
 In the case of string-to-string functions, we conjecture that protocols define exactly the so-called regular functions, which will be formally defined in Section~\ref{sec:regular-string-to-string-functions} below.
 


\begin{conjecture}\label{conj:protocol-regular-string-to-string}
    A string-to-string function is computed by a protocol if and only if it is regular. 
\end{conjecture}

In Section~\ref{sec:regular-string-to-string-functions} we  define the regular functions from the above conjecture, and we prove the $\impliedby$ implication: every regular function is computed by a protocol. The content of the conjecture is therefore the $\implies$ implication, namely that protocols can only compute regular functions and nothing more.  In Section~\ref{sec:continuity}, we present some evidence for the conjecture, by showing that string-to-string functions computed by protocols share many good properties of the regular functions, such as linear output size and computability. In Section~\ref{sec:unary-output-alphabet}, we present further evidence for the conjecture, namely we  prove it in the special case where the output alphabet has only one letter (the remaining case is two output letters, since more letters do not change the situation). Finally, in Section~\ref{sec:beyond-fields}, we discuss variants of the conjecture that are related to weighted automata which are not over a field, but over an arbitrary semiring. 


\subsection{Regular string-to-string functions}
\label{sec:regular-string-to-string-functions}

In this section, we define the class of regular string-to-string functions, and we prove one implication in the conjecture. 
 Historically, this class of functions was first defined in terms of 
 deterministic two-way automata with output\cite[Note 4]{shepherdson1959reduction}. Let us present this definition.

 \begin{definition}[Two-way automaton]
    A deterministic two-way automaton with output is given by the following ingredients:
    \begin{enumerate}
        \item a finite input alphabet $\Sigma$;
        \item a finite output alphabet $\Gamma$;
        \item a finite set of states $Q$, with an initial state $q_0 \in Q$;
        \item a transition function  
        \begin{align*}
        \delta : 
        \myunderbrace{Q}{old \\ state} \times 
        \myunderbrace{(\Sigma + \{\vdash, \dashv\})}{input letter\\ under  the head} \to  \set{\text{halt}} + (
        \myunderbrace{Q}{new \\ state}
         \times 
         \myunderbrace{\{-1,0,1\}}{head \\ movement} \times 
         \myunderbrace{\Gamma^*}{added \\ output}) .
        \end{align*}
    \end{enumerate}
 \end{definition}

    The automaton works as follows. The input string $w$ is placed on a tape, with the left end marked by $\vdash$ and the right end marked by $\dashv$. The automaton starts in state $q_0$, with its head on the left end of the tape, which contains the marker $\vdash$. In each step, the automaton looks at its current state and the letter under its head, and based on this information, it uses the transition function to decide if it halts, or it continues its computation. In case it continues, it chooses a  new state, the direction in which it moves its head, and a string over the output alphabet $\Gamma$, which is appended to the output tape. We assume that the automaton is always halting, which means that for every input string, the computation eventually halts. In particular, the computation must be well-defined, which means that the head never falls off the input by moving outside the endmarkers.   The semantics of such an automaton is of type $\Sigma^* \to \Gamma^*$. (For automata which are not necessarily halting, the function would be partial, since it would be undefined for inputs where the automaton does not halt.)

    \begin{myexample}[Reverse]
        For each input alphabet $\Sigma$, the reverse function of type $\Sigma^* \to \Sigma^*$ is computed by a  two-way automaton, which first moves its head to the end of the string, and then starts copying it to the output while moving in the left direction.
    \end{myexample}

    The class of functions computed by two-way automata has a remarkable number of equivalent descriptions,  originating in different fields, including:  monadic second-order transductions~\cite[Section 4]{engelfrietMSODefinableString2001}, streaming string transducers~\cite[Section 3]{alurExpressivenessStreamingString2010},  certain kinds of regular expressions~\cite[Section 2]{alur2014regular}, a calculus of functions based on  combinators~\cite[Theorem 6.1]{bojanczykRegularFirstOrderList2018}, a characterisation based on natural transformations~\cite[Theorem 3.2]{bojanczykTitoRegular23}. For this reason, some authors (starting with Engelfriet and Hoogeboom), use the name \emph{regular} for this class of function, with the intended meaning being that these functions play the same role for string-to-string functions, as that which is played by regular languages for string-to-Boolean functions. We adopt this terminology here, as stated in the following definition.


    \begin{definition}[Regular string-to-string function]
        \label{def:regular-string-to-string}
        A string-to-string function is called \emph{regular} if it is computed by a deterministic two-way automaton with output.
    \end{definition}
    
    
    One good  property of the regular string-to-string functions is that they  are closed under composition~\cite[Theorem 2]{chytilSerialComposition2Way1977}. In particular, our conjecture would imply that the same is true for functions computed by protocols. Without proving the conjecture, we do not see any direct way of proving composition for protocols.
     
    
    Another good property of the  regular string-to-string functions is that equivalence is decidable, i.e.~given two functions $f$ and $g$, one can decide if for every input string, the two output strings are equal~\cite[Theorem 1]{gurariEquivalenceProblemDeterministic1982}. This property does not seem to have any direct bearing on protocols, since there is no obvious way of presenting a non-uniform protocol as an input for a decision procedure.



    The following lemma shows one of the implications in the conjecture.

\begin{lemma}\label{lem:from-regular-to-protocol}
    If a string-to-string function is regular, then it is computed by a protocol.
\end{lemma}
\begin{proof}
    The two parties can simulate a two-way automaton with output. The execution of the protocol describes the crossing sequence of the automaton, i.e.~how it crosses the boundary between the two local strings of Alice and Bob. Here is a picture: 
    \mypic{1} 
    More formally, the crossing sequence is defined as follows, given  a split of the input string into two parts $w_1 w_2$. We run the automaton until the first configuration which is in the word $w_2$. Then we run it until the first configuration which is in the word $w_1$. We continue this way, with odd-numbered steps describing runs inside $w_1$ that end in  configurations from $w_2$, and even-numbered steps describing runs inside $w_2$ that end in a configuration from $w_1$. The last step is exceptional, since it ends with an accepting configuration. 
    The number of steps in a crossing sequence is bounded  the number of states, since otherwise the automaton would enter an infinite loop. This bound is the number of rounds in the protocol. In each round, the state corresponding to this round is sent as a signal, and the output value in the message is the  part of the output string that is produced in this  step. At the end of the protocol, the pieces of the output string are concatenated. 
\end{proof}

In view of the above lemma, the content of the conjecture is the opposite implication, namely that every protocol computes are regular function. In the rest of this section, we give some evidence for the opposite implication. In Section~\ref{sec:continuity}, we show that functions computed by protocols share some good properties of the regular functions, which is evidence that they might be the same functions. Then,  in Section~\ref{sec:unary-output-alphabet}, we prove the conjecture in the special case of a unary output alphabet. 



\subsection{Evidence for the conjecture}
\label{sec:continuity}
In this subsection, we show that the string-to-string functions computed by protocols share some  good properties of regular functions, such as linear size outputs and computability. Even computability is not a priori obvious, due to the non-uniformity of the protocols. These results can be seen as evidence of the open implication 
\begin{align*}
\text{protocol} \implies \text{regular}
\end{align*}
in the conjecture. To prove these results, we will leverage the results on weighted automata from Section~\ref{sec:field-domain}.
The point of departure is the following lemma, which connects weighted automata and string-to-string functions that can be computed in our protocol.

    \begin{lemma}
        \label{lem:postcomposition-weighted-automaton}
        Let $\domain$ be a field, and consider two functions
        \[
        \begin{tikzcd}
        \Sigma^* 
        \ar[r,"f"]
        &
        \Gamma^*
        \ar[r,"g"]
        & 
        \domain,
        \end{tikzcd}
        \]
        such that $f$ is computed by a protocol (with string outputs). If $g$  is computed by a weighted automaton, then the same is true for the composition  $f;g$.
    \end{lemma}
    \begin{proof}
        We will show that the composition $f;g$ is computed by a protocol (with field outputs). Thanks to \cref{thm:field-domain}, this will imply that $f;g$ is computed by a weighted automaton.
        For each string in $\Gamma^*$, the weighted automaton for $g$ has an associated matrix over the field $\domain$. We modify the protocol for $f$, so that it uses matrices instead of strings. During the execution, instead of sending strings in $\Gamma^*$, the parties send  the corresponding matrices. At the end, instead of concatenating the output strings, the matrices are multiplied, yielding a matrix for the entire output string. Finally, this matrix is applied to the initial state, and then the output function is applied to the resulting vector. All of this is done using addition and multiplication, which is legitimate in a protocol with the output domain $\domain$.
    \end{proof}


The above lemma establishes a property of $f$, namely that weighted automata (over a field) are closed under precomposition with $f$. We think that this is an important property, and therefore we give it a name.


\begin{definition}[Field continuity]
    \label{def:weighted-continuity}
    A string-to-string function $f : \Sigma^* \to \Gamma^*$ is called \emph{field continuous} if functions computed by weighted automata over a field are closed under precomposition with $f$.
\end{definition}

In the above definition, we only consider weighted automata over a field. The more general setting of semirings is discussed in Section~\ref{sec:beyond-fields}.
The name ``continuous'' is inspired by a similar terminology that is used in automata theory for functions that preserve regularity under inverse images, see~\cite[Theorem 4.1]{PinSilva05} or~\cite[Footnote 2]{continuity20}.  For the latter notion, we use the name \emph{Boolean continuity}.

\begin{definition}[Boolean continuity]
    A string-to-string function $f : \Sigma^* \to \Gamma^*$ is called \emph{Boolean continuous} if preimages of regular languages are regular.
\end{definition}

As we have shown in \cref{lem:postcomposition-weighted-automaton}, all string-to-string functions computed by protocols are field continuous. In particular, since every regular string-to-string function is computed by a protocol, it follows that every regular string-to-string function is field continuous\footnote{To the best of our knowledge, this is a new result. It can also be proved directly, without passing through protocols, and we present such a direct proof in Section~\ref{sec:beyond-fields}, for the more general setting of commutative semirings. In the more general case we need a direct proof, since \cref{thm:field-domain} is not known to be true for this case.}
We conjecture that the converse is also true.

\begin{conjecture}\label{conj:regular-continuous}
    A string-to-string function is field continuous if and only if it is regular.
\end{conjecture}

Later on, in Section~\ref{sec:beyond-fields}, we will dicuss variants of the conjecture, and in particular we show that the conjecture becomes false if the left side is relaxed from field continuous to Boolean continuous.

Conjecture~\ref{conj:regular-continuous} can be seen as  a machine independent characterisation of the regular string-to-string  functions. This would be a very valuable contribution. Almost all known characterisations of the regular string-to-string  functions have somewhat lengthy definitions, based on specific computational models, and it is something of a  miracle that all of these models are equivalent. A possible exception is the characterisation in~\cite{bojanczykTitoRegular23}, which does not use a machine model; however that characterisation uses the abstract language of category theory, and is less elementary than the one in Conjecture~\ref{conj:regular-continuous}.

As in Conjecture~\ref{conj:protocol-regular-string-to-string}, the content of Conjecture~\ref{conj:regular-continuous} is the left-to-right implication
\begin{align*}
\text{field continuous} \implies \text{regular}.
\end{align*}
Conjecture~\ref{conj:regular-continuous} is stronger than Conjecture~\ref{conj:protocol-regular-string-to-string}, as explained in the following diagram, which shows the known relations between three kinds of string-to-string functions:
\[
\begin{tikzcd}
\text{regular}
\ar[d,Rightarrow,shift right=2, "\text{\cref{lem:from-regular-to-protocol}}"']
\\
\text{computed by protocols}
\ar[d,Rightarrow, shift right=2, "\text{\cref{lem:postcomposition-weighted-automaton}}"']
\ar[u,Rightarrow, shift right=2,"\text{Conjecture~\ref{conj:protocol-regular-string-to-string}}"']
\\ 
\text{field continuous} 
\ar[uu,bend right=89, Rightarrow, shift right=2,"\text{Conjecture~\ref{conj:regular-continuous}}"']
\end{tikzcd}
\]


The following theorem gives some evidence for the stronger conjecture,  and therefore also the weaker one, by showing that the field continuous functions share some well-known properties of the regular string-to-string functions. 

\begin{theorem}\label{thm:evidence-for-the-conjecture}
    If a function $f : \Sigma^* \to \Gamma^*$ is  field continuous, then:
    \begin{enumerate}
        \item \label{it:linear-size-outputs} the outputs have at most linear size;
        \item \label{it:linear-time-computable} the outputs can be   computed in linear time;
        \item \label{it:regular-preimages} it is Boolean continuous, i.e.~preimages of regular languages are regular.
    \end{enumerate}
\end{theorem}

% Before proving the theorem, let us comment on the properties that are listed in it.  By Lemma~\ref{lem:from-regular-to-protocol}, every regular string-to-string  function is computed by a protcol, and therefore every regular string-to-string function has the properties that are listed in the theorem. The fact the regular string-to-string functions have the  first three properties is a folklore result and can be seen directly from the definition of regular string-to-string functions, without protocols. (For the third property, it is useful to know that, as language acceptors, two-way automata recognise exactly the regular languages~\cite[Theorem 2]{shepherdson1959reduction}).
% The fact that regular functions have the  property, about postcomposition with weighted automata over a field, is not known in the literature, to the best of our knowledge. A direct proof is possible, 

\begin{proof}
    For properties~\ref{it:linear-size-outputs} and~\ref{it:linear-time-computable}, we embed strings into numbers. 
    An output string over alphabet $\Gamma$ can be seen as a number in base $|\Gamma|$. To avoid the ambiguity that could result from leading zeros, we first prepend the string with the digit 1. Let 
    \begin{align*}
    g : \Gamma^* \to \Nat \subseteq \Rat
    \end{align*} 
    be the corresponding encoding. This encoding can be computed by a weighted automaton over the field $\Rat$, see~\cite[Lemma 8.10]{bojanczyk_automata_2025}. By  the assumption on field continuity, the composition $f;g$ can be computed by a weighted automaton. This is a weighted automaton that works in the field of rationals $\Rat$, but  only produces natural numbers on its output. By~\cite[p. 110]{BerstelReutenauer08}  the automaton can be chosen so that it only uses  integers $\Int$, possibly including negative integers. Summing up, we have a weighted automaton over $\Int$ that outputs the representation, in base $|\Gamma|$, of the output string produced by $g$. We claim that for such an automaton, the output number
    \begin{enumerate}
        \item has a linear number of digits;
        \item can be computed in linear time.
    \end{enumerate}
    These two claims yield the corresponding items in the statement of the theorem. The first claim, about a linear number of digits, is true because it is true for every weighted automaton over $\Int$. This is because applying a fixed linear map can only add a constant number of digits. The second claim is also easy to see, since the weighted automaton can be evaluated in linear time (we assume that we work in a model where addition and subtraction of integers has unit cost). 
    We are left with property~\ref{it:regular-preimages}, about Boolean continuity.  This will follow from the special case of field continuity, where the field is  the two-element field.  This is because of  the following  folklore correspondence between regular languages and weighted automata over the two-element field. 
        
        \begin{claim}\label{claim:regular-weighted-automata}
            A language $L \subseteq \Gamma^*$ is regular iff its characteristic function $\Gamma^* \to \set{0,1}$ is computed by a weighted automaton over the two-element field 
        \end{claim}
        \begin{proof}
            For the lef-to-right implication, we observe that a weighted automaton over a finite field can be simulated by a deterministic finite automaton. For the other direction, we observe that a weighted automaton can count the parity of  the number of runs in a finite automaton, and if the automaton is deterministic then the number of runs is either zero or one, and thus the parity gives the right answer.
        \end{proof}

        In terms of the correspondence from the above claim, preimages of regular languages become precompositions of weighted automata over the two-element field. In particular, regularity is preserved. 
\end{proof}

One could think that already the three properties in the above theorem are not only necessary for regularity, but also sufficient. This is not the case, as shown by the following example.

\begin{myexample}[Factorials]
    \label{ex:not-regular-but-continuous-over-finite-fields}
    Consider a string-to-string function 
    \begin{align*}
    g : \Sigma^* \to \set{a}^*
    \end{align*}
    where both the input and output alphabets are unary. A sufficient condition for Boolean continuity of such functions is given in  \cite[Example 2.12]{bojanczykTitoRegular23}, using \emph{factorials}, i.e.~numbers in the set $\setbuild{n!}{$n \in \Nat$}$. This sufficient condition is that: (a) every output string arises from finitely many inputs; and (b) every output string has length that is a factorial. 
    It is not hard to come up with a non-regular function that has  properties (a) and (b), thus ensuring Boolean continuity, and which has furthermore linear size outputs and is computable in linear time. For example, the function could map an input string $w$ to the longest string of factorial length that is shorter than $w$. 
\end{myexample}

% LTeX: language=en
\subsection{Unary output alphabet}
\label{sec:unary-output-alphabet}
\AP
In this section, we provide further evidence for
Conjectures~\ref{conj:protocol-regular-string-to-string}
and~\ref{conj:regular-continuous}, by showing that they are true for output
alphabets with only one letter. From a technical point of view, this is the
most involved result in the paper, since our proof uses a refined analysis of
the expressive power of weighted automata that output natural numbers of linear
size.

\begin{theorem}\label{thm:unary-string-to-string}
    The following conditions are equivalent for  a string-to-string function where  the output alphabet  has only one letter:
    \begin{enumerate}
        \item computed by a protocol;
        \item \label{it:unary-weighted-continuous} field continuous;
        \item \label{it:unary-regular} regular.
    \end{enumerate}
\end{theorem}

In light of Lemmas \ref{lem:from-regular-to-protocol} and
\ref{lem:postcomposition-weighted-automaton}, the only missing implication is
\ref{it:unary-weighted-continuous}~$\Rightarrow$~\ref{it:unary-regular},
that is the content of \cref{lem:nat-protocols-regular} below.


\begin{lemma}
  \label{lem:nat-protocols-regular}
  Let $f : \Sigma^* \to \{a\}^*$ be a function computed by a protocol
  manipulating natural numbers. Then $f$ is a regular function.
\end{lemma}

Let us first remark that under the assumption that the output is unary,
the class of "rational" and "linear regular functions" coincide.

\begin{lemma}
  \label{lem:unary-linear-regular-rational}
  Let $f : \Sigma^* \to \{a\}^*$ be a function. The following conditions are
  equivalent:
  \begin{enumerate}
    \item \label{it:unary-rational}
      $f$ is a rational function;
    \item \label{it:unary-linear-regular}
        $f$ is a linear regular function.
  \end{enumerate}
\end{lemma}
\begin{proof}
  The implication \ref{it:unary-rational}~$\Rightarrow$~\ref{it:unary-linear-regular}
  is immediate  and holds even without the assumption on the 
  unary output from \cref{fig:transducer-classes}.

  For the converse implication, assume that $f$ is rational (?).
  \omc{For converse implication, the assumption should be $f$ is linear regular, right? Typo?}
  Then it is
  computed by a two-way transducer with outputs. However, this transduction can
  be decomposed into a two step process: first, a rational function computes
  the crossing sequences of the transducer on every letter of the input, and
  then, a two-way transducer that simply follows these crossing sequences to
  produce the output in the correct order. Since the order of the output does
  not matter in the unary case, the second step can be performed in a
  block-wise fashion, producing all outputs corresponding to a letter
  immediately after reading it. This second step is therefore also a rational
  function, and since those are closed under composition, we conclude that $f$ is
  a rational function.
\end{proof}

Let us consider a "protocol" computing a function $f : \Sigma^* \to \{a\}^*$.
Recall that one can assume that the protocol is one round
(\cref{lemma:one-round-reduction-general}). Let us write $\sigma_A : \Sigma^*
\to Q_A \times \domain^k$ for the strategy of Alice, $\sigma_B : \Sigma^*
\times Q_A \to Q_B \times \domain^k$ for the strategy of Bob, and $\theta : Q_A
\times Q_B \to (\domain^{2k} \to \domain)$ for the combining function, that can
only use concatenation of registers. We can place a quasi-order on $Q_A \times
\domain^k$, by saying that $(q, \vec{r}) \leq (q', \vec{r}')$ if $q = q'$ and
for each $i \in \{1, \ldots, k\}$ we have $|r_i|\leq |r_i'|$. This quasi-order
on $Q_A \times \domain^k$ can be lifted to a quasi-order on strings in
$\Sigma^*$, by saying that $u \leq v$ if $\sigma_A(u) \leq \sigma_A(v)$. A key
property of this ordering is the following:

\begin{lemma}
  \label{lemma:order-monotonicity-unary-output}
  Let $u, v \in \Sigma^*$ be such that $u \leq v$. Then for every $w \in
  \Sigma^*$, we have $|f(uw)| \leq |f(vw)|$. Furthermore, the function 
  $\Delta_{u,v} : w \mapsto |f(vw)| - |f(uw)|$ is a bounded function, and for 
  every $c \in \Nat$, the set $\{ w \in \Sigma^* : \Delta_{u,v}(w) = c\}$ is
  a regular language.
\end{lemma}
\begin{proof}
  Assume that $u \leq v$, and fix some $w \in \Sigma^*$.
  Let $q \in Q_A$ be the state that Alice sends to Bob after reading $u$.
  It is the same state after reading $v$. From the point of view of 
  Bob, the two strings behave identically. Now, since the combining function
  can only use concatenation of registers, and since the registers produced by
  Bob on the split $(u, w)$ are the same as those produced on the split
  $(v, w)$, the result follows by monotonicity of concatenation with respect to
  length. Note that the difference in lengths is bounded by the difference 
  between the lengths of the registers after Alice's turn, which does not 
  depend on $w$, proving the boundedness.

  Finally, the set $\{ w \in \Sigma^* : \Delta_{u,v}(w) = c\}$ is regular because it
  can be computed by a Boolean Alice-Bob protocol: Alice simulates the original protocol
  when prepending $u$ and $v$ to its input, sends the difference of the lengths 
  of its registers (that is now a finite message, since the difference is bounded) to Bob,
  who can then check whether the total difference is $c$.
  Because of \cref{thm:boolean-domain}, we conclude that the language is in fact regular.
\end{proof}

The second key property of the ordering is the following: it is in fact a
""well-quasi-order"", meaning that for any infinite sequence $(q_i,
\vec{r}_i)_{i \in \Nat}$, there are two indices $i < j$ such that $(q_i,
\vec{r}_i) \leq (q_j, \vec{r}_j)$. 

\begin{lemma}
  \label{lemma:wqo-unary-output}
  The quasi-order on $\Sigma^*$ defined above is a well-quasi-order.
\end{lemma}
\begin{proof}
  Because the order only compares words with respect to their image by
  $\sigma_A$, it is enough to show that the quasi-order on $Q_A \times
  \domain^k$ is a well-quasi-order.
  Since $Q_A$ is finite, it is enough to show that for each $q \in Q_A$,
  the quasi-order on $\{q\} \times \domain^k$ is a well-quasi-order.
  This follows from Dickson's lemma, since $\domain = \{a\}^*$ is isomorphic
  to $\Nat$ with addition, and since $\Nat^k$ with the product ordering is a
  well-quasi-order by Dickson's lemma.
\end{proof}

From the above two lemmas, we can conclude the proof of
\cref{lem:nat-protocols-regular}. 

\begin{proof}
  We claim that there exists a finite set $L$ of words $u_1, \ldots, u_n \in
  \Sigma^*$, closed under taking prefixes, and such that for every $u \in L$
  and every letter $a \in \Sigma$, either $ua \in L$, or there is some $u_i \in
  L$ such that $u_i \leq ua$. 
  Such a finite set $L$ exists by \cref{lemma:wqo-unary-output}, since one can
  extract a finite set of minimal elements for the quasi-order, and close it
  under prefixes. 

  We can now produce a left-to-right automaton that computes $f$ using regular
  lookaheads as follows. The states of the automaton are the words in $L$.
  The final output of a state $u_i \in L$ is $f(u_i)$.
  The transition from a state $u_i$ on a letter $a$ is defined as follows:
  \begin{itemize}
    \item if $u_i a \in L$, then the transition goes to state $u_i a$;
    \item otherwise, there is some $u_j \in L$ such that $u_j \leq u_i a$,
      and the transition goes to state $u_j$ (this is a choice made uniformly).
  \end{itemize}
  The lookahead for the transition from $u_i$ on $a$ is the regular language
  that tells the precise difference in size between $f(u_i a w)$ and $f(u_j w)$. By
  \cref{lemma:order-monotonicity-unary-output}, this delay is bounded so there 
  are finitely many possibilities, and the precise difference can be checked by a
  regular property on $w$. The automaton outputs this difference on the
  transition.

  An immediate induction proves that this automaton with regular lookaheads
  computes the same function as $f$, and it is a standard result that 
  regular functions are closed under regular lookaheads,
  concluding the proof.
\end{proof}


Let us now use \cref{lem:nat-protocols-regular} to prove a result on
string-to-number functions. Remark that a string-to-string function with a
unary output alphabet can be seen as a string-to-number function, by mapping
the string $a^n$ to the number $n$. Concatenation becomes addition in this
setting. The following theorem states that considering string-to-number
protocols (keeping only addition as operation on numbers) does not change the
class of functions if the outputs are non-negative.

\begin{theorem}
  \label{thm:string-to-number-protocols-nat}
  Let $f : \Sigma^* \to \Nat$ be a function computed by a protocol
  manipulating integers. Then, $f$ is also computed by a protocol
  manipulating natural numbers.
\end{theorem}
\begin{proof}
  Let us assume without loss of generality that $f$ is computed by a one-round
  protocol. Furthermore, up to normalizing the protocol, we can assume that only one
  register of Alice is used in the final computation, and that all registers of
  Alice can be used in some computation. This can be ensured by a powerset
  construction on Alice's side.

  Let us prove the following intermediate claim: the registers of Alice are
  uniformly bounded below. Assuming the contrary, there is a sequence of inputs
  $(u_i)_{i \in \Nat}$ such that the value of some register $r$ of Alice after
  reading $u_i$ tends to $-\infty$ as $i$ tends to infinity. Since there are
  finitely many states of Alice, we can assume without loss of generality that
  the state of Alice after reading $u_i$ is some fixed state $q$ for all $i$.
  Now, there is a fixed word $v$ such that when Bob receives state $q$ from
  Alice and reads $v$, the output of the protocol is the sum of register $r$ of
  Alice and some value computed by Bob. Because $r$ tends to $-\infty$ on the
  sequence $(u_i)_{i \in \Nat}$, the output of the protocol on inputs $u_i v$
  is negative for sufficiently large $i$, which is absurd.


  Hence, one can define a new protocol computing $f$ as follows: Alice
  computes as before, but truncates all registers to zero if they are
  negative. As state, she sends to Bob her state together with the content of
  the registers that were truncated, which is a finite information since we
  have proved that the registers are uniformly bounded below. Bob can then
  simulate the original protocol, if he wishes to use a truncated register,
  he directly substracts the value from his own output. The result of Bob
  remains non-negative, since the original protocol computed a non-negative
  output.
\end{proof}

This theorem allows us to derive a neat characterisation of functions 
computed by weighted automata over $\Nat$ that have linear growth.

\begin{corollary}
  \label{cor:weighted-automata-nat-regular}
  Let $f : \Sigma^* \to \Nat$ be a function computed by a weighted automaton
  over the semiring $\Rat$, such that $f$ has linear growth. Then, $f$ is a regular function.
\end{corollary}
\begin{proof}
  It is well-known that weighted automata over $\Rat$ computing integer values
  can be simulated by weighted automata over the semiring $\Int$: this is
  sometimes referred to as $\Rat$ being a "fatou extension" of $\Int$
  \cite[p. 110]{BerstelReutenauer08}.
  Therefore, we can assume without loss of generality that $f$ is computed by
  a weighted automaton over $\Int$. Then, one can leverage a 
  result from \cite{Zpolyreg23} stating that weighted automata over $\Int$
  with linear growth can be computed as follows: first, a regular function $g : \Sigma^* \to
  \{a,b\}^*$ is computed, then each letter $a$ is mapped to $+1$ and each 
  letter $b$ is mapped to $-1$, and finally all values are summed up. 
  For a precise statement, see \cite[Proposition II.13, Theorem III.3]{Zpolyreg23}.

  Now, it is clear from the above description that $f$ can be computed by a protocol
  manipulating integer values, and combining them only using addition. By
  \cref{thm:string-to-number-protocols-nat}, $f$ can also be computed by a
  protocol manipulating natural numbers. Finally, by \cref{thm:unary-string-to-string},
  $f$ is a regular function.
\end{proof}

Let us conclude this section with an example showing that
\cref{cor:weighted-automata-nat-regular} cannot be extended to weighted
automata over $\Int$ with arbitrary growth.

\begin{myexample}[Quadratic counterexample]\label{ex:quadratic-counterexample}
     We show  a function which: (a) is a linear combination of \mso  counting functions of arity two, with negative coefficients; (b) has only non-negative outputs; and (c) cannot be presented as linear combination with positive coefficients. The idea, which is based on~\cite[Example 2.1]{BerstelReutenauer08}, is to trivially ensure non-negativity by squaring. Take the function
\begin{align*}
w \in \set{a,b}^* 
\quad \mapsto \quad 
(\text{(number of $a$'s in $w$)} - \text{(number of $b$'s in $w$)})^2.
\end{align*}
This function clearly satisfies (a) and (b). As explained in \cite[p.3]{Zpolyreg23}, it also satisfies (c), since the inverse image of $0$ is not a regular language, as would be the case if only positive coefficients were used.
\end{myexample}


\subsection{Semirings that are not fields}
\label{sec:beyond-fields}
In our discussion so far, a prominent role was played by weighted automata over a field. However, weighted automata make also sense in a more general setting, where a semiring is used instead of a field. We discuss this more general setting in the present section.

\paragraph*{Weighted automata over a semiring.}
We begin by recalling the definition  of weighted automaton  in the case of a semiring which is not necessarily a field. One approach is to use Definition~\ref{def:weighted-automaton}, suitably generalised. Since we are working in a semiring that might no longer be a field, we have to be careful when speaking of linear maps in items~\ref{it:weighted-definition-transitions} and~\ref{it:weighted-definition-final} of Definition~\ref{def:weighted-automaton}. The appropriate notion in this case is matrices: for each input letter, the corresponding state transformation is: 
\begin{align*}
q \in \domain^d \quad \mapsto \quad qA \in \domain^d,
\end{align*}
where $q$ is viewed as a row vector and $A$ is a $d \times d$ matrix in the semiring. In the definition, we have to be careful about the order of multiplication, since the semiring might have non-commutative multiplication. In order to be equivalent to a model that will be described below,  we need the new values from the matrix to come after the old values from $q$, hence $qA$ instead of $Aq$.  Similarly, the final map is of the form $q \mapsto q b$, where $b$ is a column vector of dimension $d$.

However, for the purposes of this section, it  will be more convenient to work with an alternative presentation of weighted automata, which is based on the intuition of nondeterministic automata with weights on states and transitions, as defined below. 
\begin{definition}
    [Weighted automaton, nondeterministic presentation] 
    \label{def:weighted-automaton-nondeterministic}
    A \emph{weighted automaton} consists of a finite input alphabet $\Sigma$,  a finite set of states $Q$, and functions: 
    \begin{align*}
    \myunderbrace{I : Q \to \domain}{initial}
    \quad
    \myunderbrace{F : Q \to \domain}{final}
    \quad
    \myunderbrace{\Delta : Q \times \Gamma \times Q \to \domain}{transitions}.
    \end{align*}
\end{definition}
 A run of this automaton is defined in the usual way: it is a sequence of transitions, one for each input letter, such that consecutive transitions agree on the connecting states. The weight of a run is the product of: (1) the initial weight of its source state; (2) the weights used by its transitions; and (3) the final weight of its target state. If the semiring is non-commutative, the order of multiplication is  important, and transitions are multiplied in the order corresponding to the input string. For an input string, the output of the automaton is the sum of weights of all runs over this automaton (sum is always commutative).  The model from Definition~\ref{def:weighted-automaton-nondeterministic}  defines the same functions as the model from Definition~\ref{def:weighted-automaton}, with the appropriate semiring modifications that were described earlier in this section, see~\cite[Lemma 8.3]{bojanczyk_automata_2025}. For the purposes of this seciton, we use the nondeterministic  model described from Definition~\ref{def:weighted-automaton-nondeterministic}.



  Recall Conjecture~\ref{conj:regular-continuous}, which says that a string-to-string function is regular if and only if it is field continuous. One could consider variants of this conjecture for semirings.  For a class $\algclass$ of semirings, let us define \emph{continuity over $\algclass$} to be the same as in Definition~\ref{def:weighted-continuity}, except that instead of fields, we have semirings from the class $\algclass$. This section is devoted to studying   variants of Conjecture~\ref{conj:regular-continuous} for  important classes of semirings, such as finite fields, all fields,  and all semirings. The relationship between the continuity notions is explained in the following diagram. (The diagram also includes rational functions, which are a propert subclass of the regular functions that will be discussed in \cref{sec:rational-functions}.) The diagram also highlights in red the three classes of functions that we conjecture to be the same in Conjecture~\ref{conj:regular-continuous}.
implication immediate, it is one-way by Example~\ref{ex:not-regular-but-continuous-over-finite-fields}
\[
\begin{tikzcd}[row sep=small]
 \text{\blue{continuous over all semirings}}
\arrow[d,blue, equal,"\text{\cref{thm:rational-functions}}"]
\\
 \text{\blue{rational string-to-string functions}}
\ar[d,Rightarrow,"\text{ implication is one-way,  see e.g.~\cite[p. 218]{engelfrietMSODefinableString2001}}"]
\\
 \text{\red{regular string-to-string functions}}
% \ar[d,Rightarrow,"\text{ \cref{thm:regular-continuous-commutative-semirings}}"]
\ar[d,red,Rightarrow,"\text{ \cref{lem:from-regular-to-protocol}}"]
\\
% \text{continuous over commutative semirings}
% \downsymbol{\subseteq}{immediate} 
% 
 \text{\red{computed by a protocol}}
\ar[d,red,Rightarrow,"\text{\cref{lem:postcomposition-weighted-automaton}}"]
% \downsymbol{\subseteq}{\cref{thm:regular-continuous-commutative-semirings}}
\\
 \text{\red{continuous over fields}}
\ar[uu, red, bend left=80,Rightarrow,"\text{ Conjecture~\ref{conj:regular-continuous}}"]
\ar[d,Rightarrow,"\text{ implication is one-way, see Example~\ref{ex:not-regular-but-continuous-over-finite-fields}}"]
\\
 \text{continuous over finite fields}
\arrow[d,equal,"\text{\cref{claim:regular-weighted-automata}}"]
\\
 \text{Boolean continuous}
\end{tikzcd}
\]

There are three parts in the diagram, indicated using colours. The upper part, in blue, corresponds to the strongest continuity condition, namely over all semirings. As we explain in Section~\ref{sec:rational-functions}, this condition characterises rational functions, a well-known transducer class. The middle part, in red, corresponds to the continuity condition over fields, which is conjectured to characterise regular functions. Finally, the lower part, in black, corresponds to the weakest continuity condition, namely over finite fields. As explained in the proof of \cref{thm:evidence-for-the-conjecture}, this condition is the same as Boolean continuity, and it is strictly more general than regularity. There is no machine model for the black part, because there are uncountably many such functions, as explained in \cite{bojanczykTitoRegular23}.

Of course, the diagram does not exhaust all possible notions of continuity. For example, we could consider continuity over a single semiring, such as the field $\Rat$ of rationals, or the field $\Rat(x)$. For all we know, even these notions might also characterise the regular functions. Another possibility is to consider continuity over all commutative semirings, which is discussed in Section~\ref{sec:commutative-semirings}, and again conjectured to characterise the regular functions.

\subsubsection{Rational functions}
\label{sec:rational-functions}
In this section we describe the rational string-to-string functions, and prove that they are exactly the functions which are continuous over all semirings.  We only give a brief description of the class, for a more precise definition the reader is referred to~\cite[Section 14.2]{bojanczyk_automata_2025}. The underlying model is (one-way) nondeterministic automata with output. This is like an \textsc{nfa}, except that there are extra labels for output strings, as  explained in the following picture: 
\mypic{5}
The output of a run is obtained by concatenating the labels, similarly to how it is done in a weighted automaton from Definition~\ref{def:weighted-automaton-nondeterministic}. (This similarity will be revisited in a moment.) The semantics of the automaton is a binary relation that maps an input string to all possible outputs of accepting runs. Any string-to-string relation that can be obtained this way is called a \emph{rational string-to-string relation}, see~\cite[Chapter IX]{Eilenberg74}. If the relation is a function, i.e.~for each input string there is exactly one output string, then it is called a \emph{rational string-to-string function}. (We add the ``string-to-string'' qualifier to avoid confusion with the field  of rational functions, which consists of fractions of polynomials.)  

Below we show that: (1) the rational string-to-string functions are a strict subset of the regular ones; and (2) they are exactly the functions that are continuous over all semirings. Both results are straightforward and could be called folklore.

\begin{myexample}
    [Rational is weaker than regular]
    For the separation between rational and regular, a short argument is given in \cite[p. 218]{engelfrietMSODefinableString2001}. The duplicating function $w \mapsto ww$ is regular. It cannot be rational however, since its image is not a regular language, while rational relations have regular images~\cite[Theorem IX.3.1]{Eilenberg74}.
\end{myexample}





\begin{theorem}\label{thm:rational-functions}
    A string-to-string function is rational if and only if it is continuous over all semirings.
\end{theorem}
\begin{proof}
    The left-to-right implication is proved using a simple product construction, which is particularly easy if we use the nondeterministic presentation of weighted automata from Definition~\ref{def:weighted-automaton-nondeterministic}. The right-to-left implication is proved by viewing a rational string-to-string functions as a special case of weighted automata, for suitable semiring. The semiring is 
    \begin{align*}
    \domain = \pfin(\Gamma^*),
    \end{align*}
    i.e.~finite sets of output strings, with the addition being union and multiplication being concatenation. Essentially by comparing the definitions, one sees that a string-to-string relation is regular if and only if it is computed by a weighted automaton over this semiring. This observation gives us the right-to-left implication in the theorem. Indeed, suppose that a string-to-string function $f : \Sigma^* \to \Gamma^*$ is continuous over all semirings. Consider the weighted automaton 
    \begin{align*}
    \iota : \Gamma^* \to \pfin(\Gamma^*),
    \end{align*}
    which maps a string to itself (as a singleton set). By continuity, we know that $f; \iota$ is computed by a weighted automaton over $\domain$. This means that the composition $f;\iota$ is a rational string-to-string relation; in particular the same is true for $f$. Since the composition is also functional, it follows that $f$ is a rational function. 
\end{proof}


\subsubsection{Commutative semirings}
We finish this section by discussing one more kind of continuity, namely continuity over all commutative semirings. We conjecture that this continuity is also the same as regularity, but we are able to prove only one implication.
\label{sec:commutative-semirings}
\begin{theorem}\label{thm:regular-continuous-commutative-semirings}
    If a string-to-string function is  regular, then it is weighted continuous over commutative semirings.
\end{theorem}
\begin{proof}
    Observe that we have already proved a weaker version of this theorem, namely for field continuity, in \cref{lem:postcomposition-weighted-automaton}. The proof for fields passed through protocols. Unfortunately, we cannot reuse that proof, since we do not know if protocols over a semiring output domain are equivalent to weighted automata. This is because we do not have a strong enough version of the Fliess Theorem for semirings, although some work in that direction has been done~\cite[Corollary 2.15]{daviaud25}. For this reason, we give a direct proof, which does not use protocols. 

        Let $\domain$ be a commutative semiring, and  consider  a composition 
    \[
    \begin{tikzcd}
    \Sigma^* 
    \ar[r,"f"]
    & 
    \Gamma^*
    \ar[r,"g"]
    &
    \domain
    \end{tikzcd}
    \]
    where the first function $f$ is computed by a two-way automaton, and the second function $g$ is computed by a weighted automaton over $\domain$. We need to show that the composition $f;g$ can be computed by a weighted automaton over $\domain$. To prove the lemma, we use a straightforward product construction. To describe the construction, we use \emph{weighted graphs}. Such a graph is a directed graph, where every directed edge has a weight, and every vertex has two weights: an initial and final weight. For an input string $w \in \Sigma^*$, consider the following weighted graph, which call the \emph{run graph}:
    \begin{itemize}
        \item \textbf{Vertices.} Vertices are triples of the form $(p,x,q)$ where $(p,x)$ is a configuration of the two-way automaton for $f$ and  $q$ is a state of the weighted automaton for $g$. Recall that a configuration of the two-way automaton consists of a state $p$, and a position  $x$  in the string obtained from $w$ by adding endmarkers $\vdash$ and $\dashv$ to both ends. 
        \item \textbf{Edges.} In the graph, there is an edge 
        \begin{align*}
        (p,x,q) \xrightarrow{a} (p',x',q')
        \end{align*}
        if the two-way automaton has a single transition which goes from configuration $(p,x)$ to configuration $(p',x')$.
        \item \textbf{Weights of edges.} Consider an edge as in the previous item. The weight of this edge is defined as follows. Let $u$ be the output string, possibly empty, which is produced by the two-way automaton in the transition that goes from $(p,x)$ to $(p',x')$. The weight of the edge is defined to be the  sum of weights of all runs in the weighted automaton that go from $q$ to $q'$ and have input string $u$. In the special case when $u$ is empty, this will mean that $a$ is the $1$ of the semiring, i.e.~the neutral element of multiplication, because there is a unique run over the input string, and this run has weight $1$. 
        \item \textbf{Initial weights of vertices.} The initial weight of a vertex $(p,x,q)$ is zero if $(p,x)$ is not the initial configuration of the two-way automaton, and otherwise it is the initial weight of  the state $q$.
        \item \textbf{Final weights of vertices.} The final weight of a vertex $(p,x,q)$ is zero if $(p,x)$ is not the final configuration of the two-way automaton, and otherwise it is the final weight of the state $q$.
    \end{itemize}
The weight of a path in this graph is defined to be the product of: (1) the initial weight of the first vertex; (2) the weights of all edges on the path; and (3) the final wight of the last vertex. It is  easy to see that for an input string $w \in \Sigma^*$, the output of the composition $f;g$ is the same as the sum of weights of all paths in the corresponding run graph.  (This sum finite and therefore well-defined, since the run graph has finitely many paths by virtue of being  acyclic. This is because the two-way automaton is not allowed to loop.) To complete the proof of the theorem, it is enough to show the following claim.
\begin{claim}
    There is a weighted automaton which inputs a string $w \in \Sigma^*$, and outputs the sum of weights of all paths in the corresponding run graph.
\end{claim}
\begin{proof}
    This claim is the place where we use commutativity of the semiring. Consider a run graph, as in the following picture (to avoid clutter, we only show the vertices, and not the weights):
    \mypic{3}
    Consider a path in the run graph. Since the run graph is necessarily acyclic, such a path is uniquely described by the set of edges that it uses, as in the   following picture:  
    \mypic{4}
    A run graph together with a distinguished path can be viewed as a string, which we call its \emph{string representation}. Each letter in the string representation describes a single column of the picture above. (The string representation includes the weights of the vertices and edges, which are omitted in the picture.)     The function which inputs a string from $\Sigma^*$ and returns the set of all string representations of paths in the corresponding run graph is easily seen to be a rational string-to-string relation. Let this relation be 
    \begin{align*}
    R \subseteq \Sigma^* \times \Delta^*,
    \end{align*}
    where $\Delta$ is the alphabet used for string representation of paths inside run graphs. The function in the present claim is the same as 
    \begin{align*}
    w \in \Sigma^* 
    \quad \mapsto \quad 
    \sum_{\substack{v \in \Delta^* \\ (w,v) \in R}} \text{weight of path described by $v$}.
    \end{align*} 
    Weighted automata are closed under sums as above~\cite[Lemma 8.12]{bojanczyk_automata_2025}, and therefore to complete the proof of the claim it is enough to show a weighted automaton for the function
    \begin{align*}
    v \in \Delta^* 
    \quad \mapsto \quad 
    \text{weight of path described by $v$}.
    \end{align*}
    This weighted automaton is trivial: it simply mutiplies the weights of all highlighted edges, together with the initial weight of the first vertex and the final weight of the last vertex. Here, commutativity of the semiring is crucial, since the multiplication will be done in a left-to-right fashion, which will typically be inconsistent with the order of vertices on the path. 
\end{proof}

\end{proof}

We finish this section by dicussing the relationship between two implications, both of which we conjecture to be true.
\begin{align}
\text{weighted continuous over fields}
& \iff
\text{regular}
\label{eq:weighted-continuous-fields-again}\\
    \text{weighted continuous over commutative semirings}
& \iff
\text{regular}
\label{eq:weighted-continuous-commutative-semirings-again}
\end{align}
Since the implications $\impliedby$ are known to be true by \cref{thm:regular-continuous-commutative-semirings}, the more difficult equivalence is the first one, which has a weaker condition on the left side. 
For all we know, only the harder  first equivalence has any bearing on protocols, since we have established continuity for protocols only in the field case, see \cref{lem:postcomposition-weighted-automaton}. Even though it might not be connected to protocols, the  easier second equivalence would still be interesting on its own, as a characterisation of the regular string-to-string functions.