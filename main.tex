\documentclass[11pt]{article}
\usepackage[letterpaper, margin=1.5in]{geometry}

\usepackage{macros}
\usepackage{stackengine}
\usepackage{centernot}

% macros specific to this paper
\newcommand{\domain}{\mathbb D}
\newcommand{\domainc}{\mathbb C}
\newcommand{\domaine}{\mathbb E}
\newcommand{\pol}{\text{Pol}}
\newcommand{\counter}[1]{\#{#1}}
\newcommand{\lincomb}{\operatorname{Lin}}

% various kinds of function spaces
\newcommand{\lineqfun}[2]{ #1 \underset{\text{lineq}}{\longrightarrow} #2}
\newcommand{\linfsfun}[2]{ #1 \underset{\text{linfs}}{\longrightarrow} #2}
\newcommand{\eqfun}[2]{ #1 \underset{\text{eq}}{\longrightarrow} #2}
\newcommand{\fsfun}[2]{ #1 \underset{\text{fs}}{\longrightarrow} #2}
\newcommand{\leftderivative}[2]{#1(#2\underline{\hspace{2mm}})}
\newcommand{\rightderivative}[2]{#1(\underline{\hspace{2mm}}#2)}
\newcommand{\twoderivative}[3]{#1 #2 #3}
\newcommand{\termop}{ \underset{\text{term}}{\longrightarrow}}
\begin{document}

\title{Two-party computation for functions with string inputs}
\author{Miko{\l}aj Boja\'nczyk, Aliaume Lopez, Rafa{\l} Stefa\'nski, Omid Yaghoubi }

\maketitle 
\begin{abstract}
 Inspired by communication complexity, we introduce a model of computation that defines functions of type $\Sigma^* \to \domain$, where $\Sigma$ is a finite alphabet and $\domain$ is some domain. Domains of interest include: the Booleans, strings, or a field. In the model, the input string $w$ is split into two parts $w=w_1 w_2$, which are sent to two parties, Alice and Bob. The two parties cooperate, exchanging a constant number of messages to compute the output of the function. They must produce the correct output regardless of the split $w = w_1 w_2$. We prove that for some domains, the model coincides with known finite state models: in the case of Boolean outputs it defines exactly the regular languages, and in the case of fields, it defines exactly the functions computable by weighted automata. This is despite the fact that the model is non-uniform and has no computational assumptions. 
\end{abstract}



% LTeX: language=en
\section{Introduction}
\label{sec:introduction}

\AP
This paper is motivated by a desire to understand the notion of  regularity in formal language theory. We take the functional perspective, in which we consider functions 
\begin{align*}
f : \Sigma^* \to \domain
\end{align*}
that input strings, and output values from some domain $\domain$. If the output
domain is the Booleans, then such functions are languages, and there is no
question about which languages should be considered \intro[regular language]{regular}. There are tens --
if not hundreds -- of equivalent definitions, including regular expressions,
finite automata in numerous forms, monoids, monadic second-order logic, and
variants of $\lambda$-calculus. But what about other outputs? Let us review
three examples where the nature of regularity is a topic of genuine
discussion.

\begin{enumerate}
    \item \textbf{String outputs.}
Consider string-to-string functions 
\begin{align*}
f : \Sigma^* \to \Gamma^*.
\end{align*}
Similarly to languages, the literature on automata theory offers countless  models. This time, however, not all of them are equivalent, but there is at least some semblance of order. Let us mention three classes of functions  of particular interest:  the \emph{rational}, \emph{regular}, and \emph{polyregular} functions. These classes are described in~\cref{fig:transducer-classes} in Section~\ref{sec:string-outputs} together with the appropriate references, with each one having at least five different characterisations, using models of varied origins, including logic, algebra and programming language theory. Which of these classes, if any, should be considered ``the'' regular string-to-string functions? We could simply go with the middle one, because  the word ``regular'' is traditionally used for it, but a more principled approach would be preferable.

\item \textbf{Number outputs.}
Consider string-to-number functions, say functions 
\begin{align*}
f : \Sigma^* \to \Rat
\end{align*}
that output rational numbers (more generally, the outputs could be in some field). Here, the literature offers two natural candidates, namely \emph{weighted automata}~\cite{schutzenberger1961definition}, and \emph{polynomial automata}~\cite[Section IV]{DBLP:conf/lics/BenediktDSW17}. In both cases, there is an automaton that reads the input string in one pass, and stores in its state a vector $\Rat^d$ of some fixed dimension. In weighted automata, this vector is updated using linear maps, while polynomial automata can use polynomial maps. These models are not equivalent -- polynomial automata are strictly more powerful -- but both have a good mathematical theory, one based on linear algebra, and the other based on algebraic geometry. Again, we might be tempted to ask: which  is the right one?

\item \textbf{Infinite input alphabets.}
Our final example is of a different kind than the previous ones. We return to functions with Boolean outputs
\begin{align*}
f : \Sigma^* \to \set{\text{true, false}},
\end{align*}
i.e.~languages, but this time the input alphabet is no longer required to be finite. The infinity needs to be somehow tamed, and the standard approach to do this is to use an infinite alphabet where letters can only be compared for equality~\cite{kaminskiFiniteMemoryAutomata1994}. This allows for languages such as ``all letters are different'' or ``the first letter is equal to the last one'', but not for languages such as ``the letters are in increasing order'', since the letters are not equipped with a linear order, or any other kind of structure.
The literature is rife with automata models for such languages, with seventeen examples listed in \cref{fig:automata-infinite-alphabets}, all describing pairwise non-equivalent models. 
 Again, we might be tempted to ask: which  is  the right one?
\end{enumerate}

This type of question can be asked in other settings, with   outputs such as  trees,  graphs or elements of some abstract semiring. One could also vary the inputs, and consider regularity for, say, graph-to-graph functions, but we refrain from considering general outputs in this paper, and we stay with string inputs. This paper attempts to provide a unified theory of regularity for such functions.   We are guided by  the following principle, which we believe to be essential for regularity:
\begin{description}
    \item[Constant information flow.]   If the input is split into parts, then only a constant amount of information  flows between them, as far as the output of the function is concerned.
\end{description}
This principle is only a vague guideline, since it does not identify what
``information flow'' means, or how to quantify its amount. For Boolean outputs,
i.e.~languages, constant information flow has a standard interpretation, which is the
Myhill-Nerode Theorem, and it is known to
correspond exactly to the regular languages. However, in the case of more
complicated outputs, things are less clear.  For example, if the outputs of the
function are rational numbers, then it should be legitimate for the information
flow to contain some rational numbers. How should this be formalised?

We propose a formalisation that is based on communication complexity, in which
two parties, called Alice and Bob, cooperate to compute the output of the
function. Note that a formal study of the communication complexity for
functions of the form $f : \Sigma^* \times \Sigma^* \to \domain$ is well
established since Yao's seminal paper \cite{YAO79}, and the more comprehensive
treatment appearing in the standard text by Kushilevitz \cite{KUSH97}.
\mb{I think that these references should be made precise. What is it that is there and is not there?}
We adapt
this model to our purposes as follows.

\AP
There are \textbf{no uniformity assumptions} and the two parties
have unrestricted computational power; the goal is to measure information 
% \omc{information flow instead of information ?}
% it is a bit of a wordplay on the name of the journal Information and Computation :)
and
not computation.  The input string is split into two parts $w = w_1 w_2$, and
Alice has access to $w_1$ while Bob has access to $w_2$. They exchange messages
in order to compute the output of the  function. There must be a
\textbf{constant number of messages}, where the constant  depends only on the
protocol and not on the input string. Also, the  output of the computation must
be \textbf{split invariant}, which means that the output depends only on the
input string $w$, and not on the split $w = w_1 w_2$. In the communication,
there are two kinds of messages:  either bits in $\set{\text{true, false}}$  or
elements of the output domain.
% \omc{This may be my subjective view (as I mentioned before), but I don’t think it’s a good idea to describe the messages as bits in our paper. I understand that you did this because it’s the convention in the communication complexity community. However, I would prefer saying that either the message comes from a finite set, or … The reason is that, for detail-oriented readers within the communication complexity community, in this setting the parties should either alternate turns when sending messages to each other, or there should be a "Next" function specifying the next communicator based on communication history.
% Using a finite set instead of bits allows parties to send information that might require several rounds rather than just one. I suppose you wrote it this way to appeal to readers already familiar with communication complexity and to make the paper more attractive to them. If that’s the reason, then it makes sense to keep it as is. Otherwise, I think it would be better to take a consistent approach and use a finite set for the visible part of the messages.}. 
% Thanks Omid! I understand your point. However, I did indeed want to start with the more standard terminology in the introduction; and later start using the terminology for our paper.
For example, if the outputs are rational
numbers, then the messages can contain rational numbers. However, there is a
restriction on the access to messages from the output domain, which is called
the \textbf{\intro{black box restriction}}. This restriction (which will be formally
defined later in the paper) is intended to prevent tricks such as Alice sending
her input string to Bob encoded as a rational number. Intuitively speaking,
the  black box restriction says that the parties cannot read the messages from
the output domain, and instead they can only operate on them using predefined
operations. For example, in the case of numbers, the operations are addition
and multiplication.

The model, which we call \emph{protocols}, can be applied to any output domain,
and we study several examples in this paper. As we discover, despite its
non-uniformity, the model  can only define well-behaved functions. In
particular, in all cases that we have studied  these functions are: (1)  always
computable, even in linear time; and (2) the output size is always at most
linear (with the size of a rational number measured by the number of bits
needed to represent it).  This seems to indicate that the split invariance,
together with a  constant number of messages under the black box restriction,
has unexpected computational consequences. In particular, in all cases that we
have studied, protocols coincide -- or are conjectured by us to coincide --
with existing automata models. Since automata are conceptually very different 
than protocols, we believe that those equivalences, summarized in the table below,
justify the protocol-based approach to regularity.

\begin{center}
    \begin{tabular}{ll}
    \textbf{Output} & \textbf{Automata Model} \\
    \hline
    \kl[Boolean domain]{Booleans} & Finite automata (\cref{thm:boolean-domain}) \\
    \kl[String domain]{Field}   & Weighted automata (\cref{thm:field-domain}) \\
    \kl[String domain]{Strings} & Two-way automata with output (\cref{conj:protocol-regular-string-to-string}) \\
    Boolean, but infinite input alphabet & Unambiguous automata (\cref{conj:protocols-unambiguous}) 
\end{tabular}
\end{center}

In the remainder of this introduction, we give a more detailed review of the four rows
in the table, including substantial evidence for the conjectured equivalences.
% One piece of evidence that is worth mentioning here is that
% we were able to prove the equivalence for string outputs in the special case
% of \emph{unary} output alphabets (\cref{thm:unary-string-to-string}).

\subsection{Protocols for Boolean outputs}
\label{sec:intro-boolean}

We begin by studying the protocol model for  functions
\begin{align*}
f : \Sigma^* \to \set{\text{true, false}},
\end{align*}
i.e.~languages. In this case the two parties only exchange bits, 
% \omc{I still think it’s better to say that the messages come from a finite set instead of bits.} 
and they need to decide if the input string is in the language or not.
Here is an example.

\begin{myexample}[Parity]
    \label{ex:three-letters}
Suppose that the language is ``the string has even length''. In a protocol for this language, Alice sends the parity for her part of the input (one bit), and Bob uses this bit to return the final answer.  
\end{myexample}

As mentioned before, the parties have unbounded computational power, which means that  their messages
can contain answers to potentially undecidable questions about their parts of the input.
Despite this, the protocols compute exactly the regular languages.
This was first shown in~\cite[Theorem 5]{hauser1989}, but we restate it here for completeness, and provide a self-contained proof later in the paper.

\begin{restatable}{theorem}{booleandomain}
 \label{thm:boolean-domain} A language $L \subseteq \Sigma^*$ is computed by a \kl[boolean protocol]{protocol} if and only if it is 
 \kl[regular language]{regular}.
\end{restatable}
One implication is immediate: every regular language can be computed by a protocol.  Alice can send to Bob the state of a finite automaton that recognises the language. The other implication is proved in two steps, see Section~\ref{sec:boolean-domain} for more details. In the first step, the protocol is reduced to a one-round non-interactive version, where each party independently sends a message with a constant number of bits, and the decision is then made based on these two messages. In the second step,  the Myhill-Nerode Theorem is used to show that the language must necessarily be regular. 

Theorem~\ref{thm:boolean-domain} is  simple  technically, and its main value for us lies in its role as an inspiration for other results, which use  other output domains  such as  fields or strings. 


% It is worth underlining that, unlike in communication complexity, we are not interested in the exact number of exchanged bits. For example, in the round reduction that is described in the previous paragraph, the number of exchanged bits increases exponentially. One could investigate the constant in more detail, e.g.~how it depends on some parameters of the language, such as the number of states in an automaton that recognises it. We do not pursue this direction in this paper.

Before moving to the other output domains, let us comment on other related work that connects communication complexity with automata in the case of Boolean outputs.  Much of this work is related to \emph{state complexity}, where one studies  the number of states needed for a given language in a given automaton model, and how this number is affected by operations on languages or changes in the model. See  Wikipedia~\cite{stateComplexityWiki} for a comprehensive summary with numerous references, or the recent paper~\cite{goosKiefer2022} which shows how to transfer lower bounds from communication complexity to state complexity of unambiguous automata. 
% \omc{Remove the Wikipedia source and keep only the Goos and Kiefer (2022) reference. I think the Wikipedia source is redundant.}
% MB: I think that the Wikipedia source contains a lot useful references, more so than the Good and Kiefer paper. I would keep it.
 The work on state complexity is mainly about the exact number of states, which is of secondary concern to us. For our purposes, a protocol that exchanges $k$ bits is no different from a protocol that exchanges $2^k$ bits. We only care that this bound should be finite and independent of the input.



\subsection{Field outputs}
\label{sec:intro-field}



After the Booleans, we turn to functions with outputs in a field. This is the first original contribution of this paper. For the sake of concreteness, let us  consider functions with outputs in the  field of rationals
\begin{align*}
f : \Sigma^* \to (\mathbb Q, +, \times).
\end{align*}
We adapt the  protocol model to  compute such functions. Similarly to the Boolean case, the parties exchange messages. However, this time there are two kinds of messages: bits (as in the case of  Boolean outputs), and  elements of the field.  The elements of the field can be added and multiplied.  Division is not allowed, and its role is discussed in Example~\ref{ex:division}. Using bit messages, we can still recognise all regular languages (more formally their characteristic functions).
However, messages from the field allow computing new functions.

\begin{myexample}[Length and exponential length]\label{ex:length}
    Consider the following two functions
\begin{align*}
\myunderbrace{w \mapsto |w|}{length} \qquad \text{and} \qquad \myunderbrace{w \mapsto 2^{|w|}}{exponential length}.
\end{align*}
To compute the length of the input string, Alice  sends the length of her part, and Bob adds this to his length, thus yielding the desired output. For the exponential length, we use a similar protocol, except that multiplication is used instead of addition. 
\end{myexample}


In the presence of an infinite message space, one needs to be careful about the design of the protocol. For example, Alice could try to send her part of the input string in a single message by encoding it as a rational. This would trivialise the model, enabling every function to be computed. To prevent such tricks, we use the \emph{black box restriction} which was discussed before\footnote{The black box postulate is related to  polymorphic parametricity from the theory of programming languages~\cite[Section 7]{reynolds1983types}, or to the recent algebraic group model in cryptography~\cite[Section 1.2]{fuchsbauer2018algebraic}. An important difference with the algebraic group model is that our model does not allow for equality tests, see Example~\ref{ex:equality-tests} for a discussion.}: the messages which use the output domain (in this case, rational numbers) cannot be read directly, but can only be acted on by the operations in the output domain (in this case addition and multiplication).
 If the output domain is finite, e.g.~it is a finite field, then the black box restriction is irrelevant (which is why it was not mentioned when talking about Boolean outputs). This is because one can use bits to sent elements of a finite output domain, and the bits are a preferrable communication channel, since they can be read directly and are not subject to the black box restriction. 

\paragraph*{Definition of the protocol model.} Since there might be some ambiguity as to what exactly is allowed in the protocol, we give a more formal  definition. There is a finite input alphabet $\Sigma$, and a constant number of rounds $k \in \set{1,2,\ldots}$.   Alice and Bob send messages in alternation, with Alice sending the first message\footnote{One could consider other patterns of communication. In fact, in Section~\ref{sec:beyond-boolean-outputs} we use a more symmetric variant where both parties move in parallel in each round. These variants  do not change the expressive power of the model, only the  number of rounds.}. The messages are from 
\begin{align*}
\myunderbrace{\set{\text{true, false}} + \Rat}{disjoint union of bits and rational numbers}.
\end{align*}
When choosing their message in the $i$-th round, the corresponding  party (Alice in odd rounds, Bob in even rounds) has access to their part of the input string, and the  bits from previous messages. The numbers  are rationals, and cannot influence the decision, as per the black box restriction.  The information available  in the $i$-th round is given by  the set 
\begin{align}\label{eq:strategy-input}
\myunderbrace{\Sigma^*}{part of the  input \\ string that is \\ known to the party}
\qquad \times \qquad 
\myunderbrace{\set{\text{true, false, unknown}}^{i-1}}{messages received in  previous rounds, \\ with  numbers from $\Rat$ replaced by ``unknown''}.
\end{align}
Based on this information, the corresponding party chooses a new message to  send, which is either a bit, or a number. The number can be produced in two ways: either  a  fresh number  is produced based on the available information, or otherwise two previously received numbers are combined using addition or multiplication.  Therefore, the possibilities for the message sent in the $i$-th round are described by the set 
\begin{align}
    \label{eq:strategy-output}
\myunderbrace{\set{\text{true, false}}}{bits}
\ + \ 
\myunderbrace{\Rat}{fresh \\ number}
\  + \ 
\myunderbrace{\setbuild{(op,x,y)}{$op \in \set{+,\times}$ and $x,y \in \set{1,\ldots,i-1}$}}{addition or multiplication of previously received numbers}.
\end{align}
If addition or multiplication is used, then the party sending the message is responsible for the operation to be well-defined, i.e.~the messages sent in rounds $x$ and $y$ must have been numbers.
Summing up, the strategy in round $i$ is a function which inputs an element of the set from \eqref{eq:strategy-input}, and outputs an element of the set from \eqref{eq:strategy-output}. This function need not be computable. The protocol is then described by $k$ such strategies, one for each round $i \in \set{1,\ldots,k}$. We assume that the last message, sent in the $k$-th round, is a number and not a bit -- this number is defined to be the output of the protocol. 
Finally, the protocol must be split invariant, i.e.~for every input string $w$, the same output must be produced regardless of the factorization $w = w_1 w_2$ into strings owned by Bob and Alice. This completes the definintion of the protocol model, in the case of field outputs.


\paragraph*{Equivalence with weighted automata.} Our main result for field outputs is the following theorem, which says that protocols are equivalent to weighted automata. The precise definition of weighted automata will be given later in Section~\ref{sec:field-domain}.



\begin{restatable}{theorem}{fielddomain}
    \label{thm:field-domain}
     Assume that the domain $\domain$ is a field. Then a function 
    \begin{align*}
    f : \Sigma^* \to \domain
    \end{align*}  is computed by a protocol if and only if it is  computed by a weighted automaton.
\end{restatable}



  This result might even seem surprising. Our protocol is designed to use polynomial operations on the output domain, and therefore one could expect the relevant automaton model to be also based on polynomials, such as the  polynomial automata of~\cite{DBLP:conf/lics/BenediktDSW17}. As it turns out, the split invariance in the protocols  enforces linearity, and thus it excludes the general polynomials operations that are used in polynomial automata. The linearity phenomenon is true for outputs in a field -- because weighted automata are based on linear maps -- and it will also be true for other output domains, such as strings, see \cref{thm:evidence-for-the-conjecture}. We do not have a fully general understanding of this phenomenon.

  The proof of \cref{thm:field-domain} is given in Section~\ref{sec:proof-of-thm-field-domain}, but here we present a rough outline.  The proof is similar  to the one used for the case of Boolean outputs, and has  two steps:

\begin{enumerate}
    \item We first show in Section~\ref{sec:reduction-to-scalar-product-protocols} that any protocol with outputs in a field can be reduced to a special  form, which we call a \emph{scalar product protocol}. In this protocol, Alice and Bob apply in parallel two functions 
\begin{align*}
\sigma_A, \sigma_B : \Sigma^* \to \Rat^d
\end{align*}
to their parts of the input string, where $d$ is some fixed finite dimension. Then, the output is obtained by combining these two  $d$-dimensional vectors using scalar product. A scalar product protocol can be simulated by the general version of the protocol, but it is subject to certain restrictions: (a) there is no interaction; and (b) bits are not used, only  field elements. Since, as we prove, every product can be reduced to this scalar form, it follows that interaction and bit messages are not needed in the protocol. In fact, the interaction can be removed for all output domains, but the removal of bits is specific to fields. 
\item After reducing to the scalar product protocols, the next step (see Section~\ref{sec:from-scalar-product-protocol-to-weighted-automaton}) is to apply a version of the Myhill-Nerode Theorem for weighted automata. This is called the  Fliess Theorem~\cite{fliess1974}, and it says that recognisability by a weighted automaton is equivalent to having finite rank for a certain matrix, which is called the Hankel matrix. Roughly speaking, the rows in the Hankel matrix, in the context of our protocols, describe strategies of Alice, and the columns describe strategies of Bob. Therefore, it is not hard to show that in a scalar product protocol that uses vectors of dimension $d$,  the Hankel matrix has rank at most $d$. This, together with the Fliess Theorem, shows that protocols are equivalent to weighted automata, completing the proof of \cref{thm:field-domain}.
\end{enumerate}

 




\begin{myexample}[Division]\label{ex:division} What if we added division to the operations?  Consider the function  $w \mapsto 1/(|w|+1)$. This function can easily be computed using a protocol with division. We now argue that it cannot be computed using addition and multiplication only, thus proving that division gives extra power. Since the function depends only on the length of the input, it can be seen as having type $\Nat \to \Rat$. In such a type, weighted automata are the same as linear recurrence sequences. The inverse function $1/(n+1)$ is not a  linear recurrence sequence, which can be shown using the exponential polynomial form~\cite[Theorem 2.1]{BerstelReutenauer08}. Summing up, the choice of operations is important; we use a field, but we only allow the ring operations. We do not know what happens if division is allowed.
\end{myexample}

\paragraph*{Related work.} \cref{thm:field-domain} can be seen as a machine independent characterisation of  weighted automata. This would not be the first such characterisation, e.g.~the Fliess Theorem itself can be seen as a machine independent characterisation. Other research related to the Fliess Theorem is the categorical approach to  minimisation of weighted automata from~\cite{colcombetPetrisan2017}. We think that the value of our approach is that it places weighted automata in a broader context, which is defined purely in terms of communication, and in a way that is applicable to  other output domains, such as strings that will be considered next. As far as we know, the only  work  which takes such a broad view is the cost register automata of~\cite[Section C]{alurDantoniDeshmukhYuan2013}, which are an automaton model that describes functions with outputs in an arbitrary output domain, similarly to our setting. However, unlike our model, cost register automata are defined in terms of a  finite state machine model, and as such they lack the abstract machine independent flavour of our approach.

\subsection{String outputs}
\label{sec:intro-strings}

Our third group of results concerns string-to-string functions
\begin{align*}
f : \Sigma^* \to \Gamma^*.
\end{align*}
We  use the same kind of protocol as in the previous section, except that instead of numbers, the black box messages  contain strings from $\Gamma^*$, and instead of addition and multiplication, we have string concatenation.  Here are some examples.

\begin{myexample}[Reversal and duplication]\label{ex:reverse-duplicate}
    Using the protocol, we can compute string reversal: Alice sends to Bob the reverse of her part of the input, and  Bob concatenates this with  his part of the reverse. Another string-to-string function that can be easily computed by a protocol is string duplication $w \mapsto ww$. 
\end{myexample}

The string-to-string case is of particular interest because, as we have mentioned earlier in the introduction, there is no consensus as to  which string-to-string functions should be considered ``regular''. There are numerous automata models to choose from, some of which are summarized in \cref{fig:transducer-classes}, which contains twenty models, grouped by equivalence into three classes. We can exclude the weakest class (the rational functions), since it is too weak: it  cannot compute the reverse or duplicate functions from Example~\ref{ex:reverse-duplicate}. We can also exclude the strongest class (the polyregular functions), since it is too strong: polyregular functions can have superlinear output size, and as we will see in a moment, protocols can only have linear output size. By a process of elimination, we are left with the final  class from \cref{fig:transducer-classes}, which is traditionally called the ``regular functions''. One of the models which defines this class is  deterministic two-way automata with output~\cite{shepherdson1959reduction}. We conjecture  that this class     is the correct answer (thus validating the traditional name): 

\begin{conjecture}\label{conj:protocol-regular-string-to-string}
    A string-to-string function is computed by a protocol iff it is computed by a deterministic two-way automaton with output.
\end{conjecture}

In Section~\ref{sec:string-outputs} we discuss this conjecture in detail, and provide evidence in its favour, including a proof of the  implication 
\begin{align*}
\text{protocol} \quad \impliedby \quad \text{two-way automaton},
\end{align*} 
This implication is rather easy, since a two-way automaton can be neatly simulated by the repeated interactions of the protocol. The content of the conjecture, and the subject of most of our technical results, is the $\implies$ implication. Most of Section~\ref{sec:string-outputs} is devoted to evidence for this implication. 
Our first argument  is the following result, which shows that functions computed  by protocols have many properties which are known to hold for deterministic two-way automata with output.

\begin{theorem}\label{thm:evidence-for-the-conjecture}
    If a string-to-string function  is  computed by a \kl{protocol}, then:
    \begin{enumerate}
        \item \label{it:linear-size-outputs} outputs have at most linear size;
        \item \label{it:linear-time-computable} outputs can be   computed in linear time;
        \item \label{it:regular-preimages} preimages of regular languages are regular.
    \end{enumerate}
\end{theorem}

One can invent functions which satisfy the three conditions in the above
theorem, but which are not computed by deterministic two-way automata with output, see
Example~\ref{ex:not-regular-but-continuous-over-finite-fields}. However, all
known examples of such functions  are artificial, and none can be computed by
protocols (or any known transducer models).  The proof of
\cref{thm:evidence-for-the-conjecture}, which is given in
Section~\ref{sec:string-outputs}, uses a linear representation of strings as
matrices, and then applies \cref{thm:field-domain} about protocols with field
outputs. In fact, this technique suggests an alternative approach to regularity,
which connects string-to-string functions with the better understood case of
string-to-field functions. This approach is discussed in
Section~\ref{sec:continuity}.

Our second argument in favour of the \cref{conj:regular-continuous} is that we
can prove it in the special case when the output alphabet is unary, i.e.~it has
only one letter. This is the content of
Section~\ref{sec:unary-output-alphabet}. When the output alphabet is unary, the
output strings are commutative, i.e.~the order of letters is irrelevant; this commutativity is essential to our proof of the conjecture in the unary output case. 



\paragraph*{Related work.} One of the consequences of the Myhill-Nerode theorem
for string-to-Boolean functions, or the Fliess Theorem for string-to-number
functions, is the existence of  canonical devices. There have been several
attempts to generalise this to string-to-string functions, with a special
emphasis the canonical devices. Before recalling this work, we observe that our
approach seems to go in a  different direction. Although we think of
Conjecture~\ref{conj:protocol-regular-string-to-string} as being a machine
independent characterisation, it does not necessarily  yield canonical devices.
In particular, our proof of the conjecture for unary alphabets does not yield a
canonical device.

Here is a summary of results on canonical devices for string-to-string
transducers: they have been proposed for subsequential functions~\cite[Théorème
1.1]{choffrut1977}, rational functions in~\cite[Theorem
1]{reutenauerSchutzenberger1991},  and for rational functions on infinite
words~\cite[Section 4]{canonicalRational2018}. A certain drawback of this line
of work is that: (a) the canonical devices are relative to a given automaton
model, which does not help in choosing one model over another; and (b) the
``canonical'' devices are not truly unique, since they depend on extra
parameters, such as the output delay for subsequntial functions or the
lookahead for rational ones. Let us now move to the larger class of regular
functions, which is the subject of our conjecture. Here,  canonical model are
unknown, and the only known way to recover them uses non-standard semantics,
called origin semantics~\cite[Theorem
1]{bojanczykTransducersOriginInformation2014}. Another result of this kind,
which is a machine independent characterisation of the regular functions that
does not yield canonical devices, see \cite[Theorem
3.2]{bojanczykTitoRegular23}, is also implicitly based on origin semantics.
Finally, for the polyregular functions, the situation is of course even harder,
and the only known results concern a unary output alphabet~\cite[Section
IV]{Zpolyreg23}.


\subsection{Infinite input alphabets}
\label{sec:intro-infinite}
Our final group of results is about languages over infinite input alphabets. This is a departure from the previous setting, where  the input alphabet was always finite. Following the standard approach in automata theory,  we assume that letters can only be compared for equality. This is formalised by using invariance under  permutations of the alphabet, i.e.
\begin{align*}
w \in L 
\quad \iff \quad
\pi(w) \in L \qquad \text{for every permutation $\pi$ of the alphabet}.
\end{align*}
An example of such a language is ``all letters are different''.
 As mentioned earlier in the introduction, there is a rich literature on automata for such languages, see the surveys~\cite{neven2003power,segoufin2006automata,bojanczykOrbitFiniteSetsTheir2017} or the lecture notes~\cite{bojanczyk_slightly}.  The  relevant automata models typically use registers to store some letters from the input, so that they can be compared to later letters. Essentially any automaton model for finite alphabets can be lifted this way to infinite alphabets~\cite[Figure 1]{neven2003power}, and there is even a systematic way to do this, which is based on the theory of orbit-finite sets~\cite[Chapter 2]{bojanczyk_slightly}. Unfortunately, after this lifting, previously equivalent models become non-equivalent.  This sad situation is illustrated in~\cref{fig:automata-infinite-alphabets}, which describes seventeen non-equivalent automata models for infinite alphabets; all of these models collapse to the regular languages when considered for finite alphabets. Equally sadly, there are almost no results in the literature on infinite alphabets that prove non-trivial equivalences of models. The only known cases of this kind are about the  weakest of the available models, namely orbit-finite monoids~\cite{bojanczykNominalMonoids2013}, which are known to be equivalent in expressive power to a certain variant of \mso~\cite[Theorems 4.2 and 5.1]{DBLP:journals/corr/ColcombetLP14}, and also to single-use register automata~\cite[Theorem 6]{bojanczykstefanski2020}.
This situation desperately calls for some unifying principles. 

Since protocols have successfully identified important models in the previous
cases, we try to see  what happens in the case of an infinite input alphabet.
When extending protocols to infinite input alphabets, we adapt them as follows:
(1) messages can contain input letters; (2)  messages can only be compared for
equality. Condition (2) is formalised by saying that the execution of the
protocol is invariant under permutations of the input alphabet, similarly to
the languages that we consider. The protocols  work their magic once again, and
they point to (as we conjecture)  one of the models in the literature. Before
revealing this model, let us give two examples.

\begin{myexample}[Deterministic too weak]\label{ex:last-letter-does-not-appear-before} Consider the language ``the first letter does not appear elsewhere in the string''. This language can be computed by a protocol, in which Alice checks the condition on her side, and then sends the first letter to Bob, so that he can check the condition on his side. More generally, a protocol can simulate any  deterministic register automaton~\cite[Definition 3]{kaminskiFiniteMemoryAutomata1994}, using the same idea as for  finite alphabets: Alice  sends her state  to Bob, together with the values of the registers.  However,  protocols can do more. In particular, they are under reversing the input string, and therefore they can compute all reverses of languages recognised by deterministic register automata. Since deterministic register automata are not closed under reversal, it follows  that protocols are strictly more powerful than deterministic register automata. 
\end{myexample}

In the previous example, we  excluded deterministic register automata as being too weak. We now exclude nondeterministic register automata as being too strong. 

\begin{myexample}[Nondeterministic too strong ]
    \label{ex:some-twice}
    Consider the language ``some letter appears twice''. This language can be recognised by a nondeterministic register automaton. However, as we will see later in the paper, it cannot be computed by a protocol. The  intuitive reason  is that if Alice does not see a repetition in her part of the input string, then she should send all of her letters to Bob, since any of them could be repeated in Bob's part. However, Alice cannot do this, since there is a constant number of messages. A formal proof will be given later in the paper.
\end{myexample}

Which automaton model, if any, corresponds to protocols? As explained in the previous two examples, deterministic register automata are too weak, while nondeterministic register automata are too strong.
We conjecture, see Conjecture~\ref{conj:protocols-unambiguous}, that the winner is a seemingly  unexpected candidate, namely unambiguous register automata~\cite[Section 5]{colcombet2015unambiguity}. This  is the special case of  nondeterministic register automata, in which  for every  input string there is  at most  one accepting run. 
Section~\ref{sec:infinite-alphabets} is devoted to our conjecture. We begin by proving one implication, namely 
\begin{align*}
\text{protocol} \quad \impliedby \quad \text{unambiguous automaton}.
\end{align*}
Contrary to  previous variants of this implication, the proof is non-trivial. The issue is with the nondeterminism of the automaton. One  interesting phenomenon is that, in the case of infinite input alphabets, the interactive multi-round nature of the protocols becomes essential, and protocols cannot be reduced to one round, as was the case for finite input alphabets. In our proof of the implication $\impliedby$, we design a protocol where  the two parties  progressively eliminate the uncertainty about letters used in the run of the automaton, until the unique accepting run is identified.
 The proof uses a variant of the sunflower lemma. 

Therefore,  the content of the conjecture is -- as in previous cases --  the other implication, namely that protocols can be simulated by unambiguous register automata. We provide evidence for this implication, using the recently developed theory of orbit-finite vector spaces~\cite{orbitFiniteVectorTheoretics}. We show that every function computed by a protocol can be computed by a weighted automaton with registers. This is almost like an unmabiguous automaton, except that some runs might have negative weights, and the weights always cancel out to give a final result that is either $0$ or $1$. In particular, the functions computed by protocols are computable, which is not a priori clear from the model. Along the way, we again need to develop some new theory, in particular an orbit-finite generalisation of the Fliess Theorem. 



\section{Boolean outputs}
\label{sec:boolean-domain}
In this section, we formally describe our model of computation for the simplest output domain, namely the Booleans, and we prove that it defines exactly the regular languages. 

Our formal definition, see Definition~\ref{def:two-party-protocol-boolean} below has minor differences with respect to the informal description from the introduction. For example, the messages are not necessarily bits, but they can be elements of some finite sets, which are fixed in advance before the input is known. Also, the messages sent by Alice and Bob can come from different sets, and the two parties send their messages in parallel in each round. 
 These generalisations do not change the expressive power of the protocol (they might influence the number of rounds), but they will be useful in later sections, when we consider restrictions and generalisations. 

\begin{definition}[Boolean protocol]
    \label{def:two-party-protocol-boolean}
  A Boolean protocol 
   is given by the following ingredients: 
  \begin{enumerate}
    \item a finite input alphabet $\Sigma$;
    \item a number of rounds $k \in \set{1,2,\ldots}$;
    \item message spaces for Alice and Bob, which are finite sets $Q_A$ and $Q_B$;
    \item for each round $i \in \set{1,\ldots,k}$, two strategies
    \begin{align*}
    \myoverbrace{\sigma_A^i : \myunderbrace{\Sigma^*}{Alice's \\ local\\ string} \times \myunderbrace{Q_B^{i-1}}{message \\ history}  \to \myunderbrace{Q_A}{new\\ message}}{stategy for Alice in the $i$-th round}
    \qquad \text{and} \qquad 
        \myoverbrace{\sigma_B^i : \myunderbrace{\Sigma^*}{Bob's \\ local\\ string} \times \myunderbrace{Q_A^{i-1}}{message \\ history}  \to \myunderbrace{Q_B}{new\\ message}}{stategy for Bob in the $i$-th round};
    \end{align*}
    \item an output function of type $(Q_A \times Q_B)^{k} \to \set{\text{yes, no}}$.
  \end{enumerate}
\end{definition}

Given a pair of input strings $(w_1,w_2)$, which are called the \emph{local strings} of Alice and Bob, respectively, the  protocol is run as follows. There are $k$ rounds. In each round, both parties send messages, and therefore after $i$ rounds are played, the communication history contains $i$ messages sent by Alice and $i$ messages sent by Bob.  In round  $i \in \set{1,\ldots,k}$,  each of the two parties  looks at their local string and the $i-1$ messages sent by the other party in the previous rounds (only the messages sent by the other party are needed, since the party knows their own messages). Based on this information, each party uses their strategy to produce a new message. At the end of the protocol, the output function is used to determine the value of the function, based on all messages in  the communication history. We say that the protocol computes a language $L \subseteq \Sigma^*$ if for every two strings $w_1,w_2$, the output of the protocol tells us if the concatenation $w_1w_2$ belongs to the language. This corresponds to the split invariance condition that was discussed in the introduction.  Also, the reader will recognise the restriction on the total number of bits (this is bounded by the number of rounds times the logarithm of the size of the message spaces), and the non-uniformity (there is no restriction on the strategies of Alice and Bob). By non-uniformity, the first message sent by Alice could contain an answer to some undecidable problem. However, as we will see, the split invariance restriction will make it impossible to use this information, since the protocol can only compute regular languages, as stated in the \cref{thm:boolean-domain} from the introduction, which we now recall.


\booleandomain*




\begin{proof}
  We begin with the easier right-to-left implication, which says that the protocol can compute every regular language. Consider a regular language, which is recognised by a deterministic finite automaton with state space $Q$. To compute this language, we can use a  one-round protocol (as we will see in a moment, this is not a coincidence, since all protocols can be reduced to one round). Alice sends the state in $Q$ of the automaton after reading her local string, and Bob sends the dependency $Q \to \set{\text{yes, no}}$ which says how Alice's state determines acceptance. Once these two pieces of information are known, we can apply the function from Bob's message to the state in Alice's message to determine the output.

  The rest of this proof is devoted to the left-to-right implication, i.e.~to showing that every language computed by the protocol is regular. This is the heart of the proof, and the place where we need to tame the non-uniformity of the protocol. 
  
  Observe that in the right-to-left implication, we have proved that regular languages need only one round. Therefore, if we want to  prove that the protocol can only compute regular languages, then a by-product will be that the protocol collapses to the one-round case.  This is, in fact, the first step of our proof. 
  \begin{lemma}\label{lem:one-round-reduction-boolean}
    For every protocol, there is an equivalent one-round protocol. 
  \end{lemma}
  \begin{proof}
    The general idea is that instead of engaging in interactive communication, each party sends the dependency of their message upon the unknown messages of the other party. Suppose that we are  Alice. The sequence of messages that we send will depend on our local string, and  the messages sent by Bob. Once the local string is fixed, this dependency is captured by a function of type 
    \begin{align*}
    (Q_B)^k \to (Q_A)^k.
    \end{align*}
    Instead of waiting for Bob's messages, Alice sends this function. (This function is not a general function, since it must satisfy the following \emph{causality} constraint: the $i$-th coordinate of the output depends only on the first $i-1$ input coordinates.) At the same time, Bob sends an analogous function of type 
    \begin{align*}
    (Q_A)^k \to (Q_B)^k,
    \end{align*}
    which describes the dependency of his messages. Due to the causality constraints, the two functions combine to create a unique output in $(Q_A \times Q_B)^k$, which can be used to determine the output of the protocol.
  \end{proof}

  Observe that the proof of the above lemma can, in general, incur an exponential cost in the size of the message spaces. This is of little concern to us, since we only care about the protocol having a constant number of rounds.

  To complete the proof of the theorem, we use the  Myhill-Nerode Theorem.  Indeed, consider the strategy of  Alice: 
  \begin{align*}
  \sigma_A : \Sigma^* \to Q_A.
  \end{align*}
  For all we know, this function could be non-computable. However, it classifies all local strings into finitely many categories, with one category for each message in $Q_A$. Furthermore, if two strings $w_1$ and $w'_1$ are in the same category, then they are equivalent in the following sense: 
  \begin{align}\label{eq:myhill-nerode-equivalence}
  w_1 w_2 \in L \Leftrightarrow w'_1 w_2 \in L 
  \qquad \text{for every $w_2 \in \Sigma^*$.}
  \end{align}
  The equivalence described above is the same equivalence as in the Myhill-Nerode Theorem. In particular, equivalence classes of this equivalence are the same as states of the minimal deterministic automaton. Since the number of possible messages in $Q_A$ is finite, it follows that the minimal automaton is finite, and therefore the language is regular.  (Observe that we have shown that the language has at most $Q_A$ equivalence classes of Myhill-Nerode equivalence, which gives an upper bound on the size of the minimal automaton recognizing the language.)
\end{proof}


\section{Beyond Boolean outputs}
\label{sec:beyond-boolean-outputs}

In the previous section, we considered Boolean outputs. In this section, we generalise protocols to account for an arbitrary output domain. The inputs remain unchanged -- they will always be strings in this paper.  For the output domain, we use a very general notion, namely a set with some operations. 
\begin{definition}[Output domain]
    An \emph{output domain} consists of: 
    \begin{enumerate}
        \item an underlying set $\domain$; together with
        \item a family of \emph{operations}, each one having  type $\domain^n \to \domain$ for some $n \in \set{0,1,\ldots}$.
    \end{enumerate}
\end{definition}
An output domain is the same thing as a (non-indexed) algebra, in the sense of universal algebra~\cite[p.5]{hobby1988structure}. 
The output domain will typically  be infinite. In principle, the family of operations might be infinite as well \omc{This is another subjective view: the possibility of having infinitely many operations for the algebra does not seem to be useful anywhere in the paper and makes things more complicated than necessary.}, although any protocol will only use finitely many operations.
By abuse of notation, we use the same symbol $\domain$ to denote the output domain and its underlying set, with the operations being implicit. Here are the output domains that will be studied in this paper: 
\begin{itemize}
    \item \intro{Boolean domain} The set is $\set{\text{true, false}}$, and there are no basic operations. (We could add basic operations, such as the Boolean operations $\lor,\land$ and $\neg$, but this will not affect the expressive power of our protocol, so we choose to have no operations.)
    \item \intro{Field domain} The set is a field, such as the rationals or reals, and there are two operations for addition and multiplication. This is not one domain, but a family of domains, with one for each field.
    \item \intro{Monoid domain} The set is a commutative monoid $(M,+,0)$, and there is one basic operation, which is the binary operation $+$. Again, this is not one domain, but a family of domains, with one for each commutative monoid.
    \omc{The text states, “Here are the output domains that will be studied in this paper:” followed by “commutative monoid.” I think it would be better to explain the relationship between this and the unary output alphabet; otherwise, it is unclear how the commutative monoid part is connected to the rest of the paper.}
    \item \intro{String domain} This is a special case of the monoid domain, where the set is $\Gamma^*$ for some finite alphabet $\Gamma$, and the operation is string concatenation.
\end{itemize}

In the protocol, the output value will be constructed in a constant number of
steps, by using operations from the output domain. We will not distinguish
between the operations that are in the output domain, and other operations
that are derived by composing them, as described in the following definition.

\begin{definition}[Term operation]\label{def:term-operations}
    Consider an output domain $\domain$. An operation 
    \begin{align*}
    t : \domain^n \to \domain
    \end{align*}
    is called a \emph{term operation} if it can be obtained by applying the operations in the domain to variables $x_1,\ldots,x_n$. Each variable can be used multiple times, or not at all.  We write 
    \begin{align*}
    \domain^n \termop \domain
    \end{align*}
    for the set of term operations with $n$ arguments.
\end{definition}

\begin{myexample}
    If the operations are $+$ and $\times$, then after applying the usual distributivity laws,   term operations are the same as polynomials with natural coefficients, such as 
\begin{align*}
x_1x_2^2 + x_1x_2x_3^3  + \myunderbrace{2x_1}{same as $x_1 + x_1$}.
\end{align*}
If the output domain is the Boolean domain, which has no operations, then the only kind of term operation is a single variable, e.g.~$x_2$.  In the case of a string domain, a term operation is some concatenation of the variables, such as $x_1 x_3 x_1 x_2$.
\end{myexample}



We now generalise the protocol to cover functions 
\begin{align*}
f : \Sigma^* \to \domain,
\end{align*}
where $\domain$ is some possibly infinite output domain. As in the Boolean
version of the protocol, the output value is constructed by Alice and Bob, as a
result of an exchange of a constant number of messages. Each message has two
parts, which are called the \emph{signal part} and the \emph{output part}. The
signal part consists of a finite amount of information, and is used to exchange
information between the two parties as in the Boolean protocol. The output part
is a list of elements from the  output domain, and  is meant to be part of the
output value. As in the Boolean case, the above definition slightly deviates
from the informal description in the introduction. In particular, a message can
contain $d$ elements of the output domain. This does not affect the expressive
power of the protocol, but it might affect the number of rounds, and the above
definition will be more convenient later one, where we consider one-round
protocols. Another important difference is that the operations from the output
domain are only applied once at the end of the protocol. This, again does
affect the expressive power, since the two parties can wait with their
operations until all signals have been exchanged.  

\begin{definition}\label{def:two-party-protocol-general} A \intro{two-party protocol}
   is given by the following ingredients: 
  \begin{enumerate}
    \item an \kl{output domain} $\domain$;
    \item a finite input alphabet $\Sigma$;
    \item a \intro{number of rounds} $k \in \set{1,2,\ldots}$;
    \item \intro{signal spaces} for Alice and Bob, which are finite sets $Q_A$ and $Q_B$;
    \item a \intro(protocol){dimension} $d \in \set{0,1,\ldots}$;
    \item for each round $i \in \set{1,\ldots,k}$, a \intro{strategy}
    \begin{align*}
    \myoverbrace{\sigma_A^i : \myunderbrace{\Sigma^*}{Alice's \\ local\\ string} \times \myunderbrace{Q_B^{i-1}}{history of \\ signals \\ from Bob}  \to \myunderbrace{Q_A \times \domain^d}{new\\ message}}{stategy for Alice in the $i$-th round}
    \qquad 
        \myoverbrace{\sigma_B^i : \myunderbrace{\Sigma^*}{Bob's \\ local\\ string} \times \myunderbrace{Q_A^{i-1}}{history of \\ signals \\ from Alice}  \to \myunderbrace{Q_B \times \domain^d}{new\\ message}}{stategy for Bob in the $i$-th round}
    \end{align*}
    \item an \intro[output function@protocol]{output function} of type \begin{align*}
    (Q_A \times Q_B)^{k} \to (\domain^{2dk} \termop \domain).
    \end{align*}
  \end{enumerate}
\end{definition}

\alc{Maybe we should use uniform strategies that take the history $Q_B^{*}$ and
$Q_A^{*}$ instead of fixed number of rounds? This is just to simplify
notations.}



The protocol is executed on a pair of strings $(w_1,w_2)$. In each round, each of the two parties  sends a message (which consists of a signal and some elements of the output domain) that is based on their local string and the history of signals coming from the other party. After all $k$ rounds have been executed, the joint signal history of both players is used, by the output function, to determine a term operation. This operation is then applied to the output history, yielding the final result of the protocol. As in the Boolean case, we are interested in protocols that are split invariant, which means that for every string $w \in \Sigma^*$, the same output is produced for every possible decomposition $w = w_1 w_2$. Such protocols compute a function of type $\Sigma^* \to \domain$. 

\begin{myexample}[Boolean domain]
    Consider the Boolean output domain. In this case, the distinction between output values and signals is irrelevant, since the output values can be sent as signals. Therefore, the general protocol coincides with the Boolean protocol from the previous section, and  can only compute regular languages.  The same remarks apply for a general but finite output domain -- a function $f : \Sigma^* \to \domain$ can be computed by a protocol if and only if for every $d \in \domain$, the inverse image $f^{-1}(d)$ is a regular language.
\end{myexample}

Other examples of output domains are fields and strings. These were discussed  briefly in the introduction in Sections~\ref{sec:intro-field} and~\ref{sec:intro-strings},  and will be discussed at length in Sections~\ref{sec:field-domain} and~\ref{sec:string-outputs}.  

% Nevertheless, we give here one example which concerns strings, and uses both signals and output values in the messages.

% Here are two brief examples.
% \begin{myexample}[Field domain] Assume that the output domain is a field, say the field of rationals, with the operations being addition and multiplication. 
%     Consider the function 
%     \begin{align*}
%     f : \set{0,1}^* \to \mathbb Q,
%     \end{align*}
%     which maps a binary string to the natural number that it represents. For example, the input $01000$ will be mapped to $8$. Here is a protocol. Based on his input string $w_2$, Bob computes two numbers, namely: 
%     \begin{align*}
%     \myunderbrace{y_1 = f(w_2)}{number represented\\  by $w_2$} 
%     \qquad \text{and} \qquad 
%     \myunderbrace{y_2 = 2^{|w_2|}}{the power of two \\ corresponding to the\\ length of $w_2$}.
%     \end{align*}
%     Alice does the same, yielding two numbers $x_1$ and $x_2$, of which only $x_1$ will be used. The final output is then 
%     \begin{align*}
%     x_1\cdot y_2 + x_2.
%     \end{align*}
%     In this protocol, there are no signals. As we will see in the next section, when the output domain is a field, then signals are not needed (unlike the case of strings or Booleans, where signals are needed).
% \end{myexample}


% \begin{myexample}[Conditional reverse]
%     In Example~\ref{ex:reverse-duplicate}, we explained a protocol for reversing the input string. Here is a variant of this protocol, which uses signals. Consider the 
%     \begin{align*}
%     f : \Sigma^* \to \Sigma^*
%     \end{align*}
%     which reverses the string if the first letter is $a$, and otherwise it leaves the string unchanged. To compute this function, Alice needs to send to Bob -- apart from her part of the output string -- a signal to Bob which tells him if the first letter is $a$.
% \end{myexample}



\subsection{One-round protocols}

We have little say to say about  protocols in the case of a general output
domain. The  only result that we have at this level of generality is a
reduction to one-round protocols, similarly to the Boolean case.

\begin{lemma}
\label{lemma:one-round-reduction-general}
  Every protocol is equivalent to a one-round protocol.
\end{lemma}
\begin{proof}
  Same proof as in \cref{lem:one-round-reduction-boolean}. 
\end{proof}

 Later in this paper, we will study protocols for infinite input alphabets. Such protocols will not be a special case of the protocols developped so far, since the signal spaces will be infinite, albeit in a limited way (called orbit-finite).  As we will see, for such signal spaces the reduction to one round will no longer be valid, and the interactive nature of the protocols will be essential.

\section{Field outputs}
\label{sec:field-domain}
In this section, we discuss functions where the output domain is a field, equipped with addition and multiplication. We prove that protocols have  exactly the same expressive power as weighted automata.
We begin by recalling the notion of weighted automata.

\paragraph*{Weighted automata.}  A weighted automaton is a device that is used to compute a function from strings to a field (more generally, a semiring, but we consider the case of fields here). This model was originally introduced by \schutz~\cite{schutzenberger1961definition}. Essentially, this is a deterministic automaton where the state space is a vector space of finite dimension, and each input letter induces a linear map. Here is the formal definition. 

\begin{definition}[Weighted automaton]
    \label{def:weighted-automaton}
    A weighted automaton over a field $\domain$ is given by: 
    \begin{enumerate}
        \item a finite input alphabet $\Sigma$;
        \item a dimension $d \in \set{0,1,\ldots}$;
        \item an initial state $q_0 \in \domain^d$;
        \item \label{it:weighted-definition-transitions} for each letter $a \in \Sigma$, a corresponding linear map of type $\domain^d \to \domain^d$;
        \item \label{it:weighted-definition-final} a \emph{final map}, which is a linear map of type $\domain^d \to \domain$. 
    \end{enumerate}
\end{definition}

A weighted automaton computes a function of type $\Sigma^* \to \domain$, which is defined in the same way as for a deterministic finite automaton: we begin in the initial state, then we apply the linear maps corresponding to the input letters, and finally we apply the final map. There is an alternative but equivalent way of describing weighted automata, which uses a nondeterministic automaton with weights on transitions. This viewpoint will be used later in this paper, see Definition~\ref{def:weighted-automaton-nondeterministic}.

The main result of this section is that our protocol is equivalent to weighted automata, as stated in the following theorem from the introduction, which we now recall:
\fielddomain*

% \begin{theorem}\label{thm:field-domain}
%     Assume that the domain is a field. Then a function 
%     \begin{align*}
%     f : \Sigma^* \to \domain
%     \end{align*}  is computed by a protocol if and only if it is  computed by a weighted automaton.
% \end{theorem}

Weighted automata can be defined not just for fields, but also for rings and even semirings. We do not know how to prove the theorem for such generalisations, since the Fliess Theorem, which is used in the proof is only known for fields. Rings and semirings will be discussed in more detail in Section~\ref{sec:commutative-semirings}.
 Before proving the theorem in Section~\ref{sec:proof-of-thm-field-domain}, we return to the issue of division, which was already discussed in Example~\ref{ex:division}.

\begin{myexample}[Division, continued]\label{ex:division-continued}
    Because it is undefined for zero, division is not a total operation, and therefore technically speaking it does not fall into our framework. We could, however try to incorporate it, by making the two parties responsible for avoiding division by zero. Under this framework, we could use a protocol to compute the function $1/|w|$ (a better choice would be $1/|w|+1$, since it would avoid problems with the empty string). As we have discussed in Example~\ref{ex:division}, such a function cannot be computed by a protocol that uses only addition and multiplication. We do not know what functions can be computed if division is also allowed.
\end{myexample}



\begin{myexample}[Semiring outputs]
    \label{ex:non-commutative-semirings} In this example we show that for semirings which are not fields, the protocol need not be equivalent to weighted automata.  The implication 
    \begin{align*}
    \text{protocol} \quad \impliedby \quad \text{weighted automaton}
    \end{align*}
    in \cref{thm:field-domain}, as we will see in a moment, holds for any semiring, and therefore the problematic implication is the other one. Here is an example where it fails.
    Let $\domain$ be  the free (non-commutative) idempotent semiring generated by two letters $a$ and $b$. Elements of this semiring are finite sets of words in $\set{a,b}^*$, such as 
    \begin{align*}
    \set{3ab, 5ba, 7aab}
    \end{align*}
    The addition operation is multiset union, and the multiplication operation is concatenation of words, extended to sets in the natural way, as illustrated on this example:
    \begin{align*}
    \set{a,b}\cdot \set{a,b} = \set{aa, ab, ba, bb}.
    \end{align*}
    Weighted automata over this semiring are the same as the rational relations~\cite[Chapter IX]{Eilenberg74}. On the other hand, a protocol can define string-to-$\domain$ functions that are not rational. This is witnessed already by functions that produce singleton sets (call these singleton functions), which can be seen as functions of type $\Sigma^* \to \set{a,b}^*$. For example, consider the singleton version of the  reverse function, i.e.
    \begin{align*}
    w \mapsto \set{\text{reverse of $w$}} \in \domain.
    \end{align*}
    This function can be computed by a protocol, using the same idea as in Example~\ref{ex:reverse-duplicate}. This function, however, is not a rational relation, and therefore it is not computed by a weighted automaton over $\domain$. 

    So what exactly can be computed by protocols with outputs in the semiring $\domain$? We do not know the answer to this question, although it is plausible that there is some suitable nondeterministic transducer model. We have a more precise idea in the case of singleton functions. In this case, all messages sent during the protocol must be singletons (this is because once a non-singleton is produced, it can never be turned into a singleton). Therefore, the operation $+$ can never be used in a non-trivial way, and thus the protocol can only use muliplication. This means that it coincides with the protocols with outputs that are strings with concatenation, as discussed in Section~\ref{sec:string-outputs}. According to Conjecture~\ref{conj:protocol-regular-string-to-string}, the singleton functions are therefore exactly the regular functions.
\end{myexample}

\begin{myexample}[Equality tests]
\label{ex:equality-tests} In this example, we discuss an extension of the protocol which allows for equality tests, similarly to the algebraic group model~\cite{fuchsbauer2018algebraic}. Clearly, equality tests cannot be completely unrestricted. Otherwise, in the presence of a countable output domain (which is the case for all protocols studied in this paper), the receiver could compare the message with all possible values one by one, until the correct one would be identified. This would  invalidate the black box discipline. A reasonable restriction is to allow a constant number of equality tests for each message; this constant can also be brought  down to one, by possibly sending more copies of the same message. The resulting protocol would be able of complementing a weighted automaton $\Aa$, in the following sense:
\begin{align*}
w \in \Sigma^* 
\quad \mapsto \quad 
\begin{cases}
    1 & \text{if $\Aa(w) =0$}\\
    0 & \text{otherwise}.
\end{cases}
\end{align*}
This form of complementation is undesirable from the point of view of decidability. For example, language equivalence is undecidable for weighted automata that are complemented in this way~\cite[Theorem 4.9]{bojanczyk_automata_2025}. Since we strive for protocols that describe ``regular'' functions, and such functions should be decidable, we avoid equality tests.
\end{myexample}

\begin{myexample}[Wrong output domains]\label{ex:wrong-output-domains}
 This discussion of equality tests from Example~\ref{ex:equality-tests} also explains why we should not expect results about regularity that work for any output domain. For example, if we would extend the field domain with a unary complementation operation 
 \begin{align*}
 x 
 \quad \mapsto \quad 
 \begin{cases}
    1 & \text{if $x =0$}\\
    0 & \text{otherwise},
\end{cases}
 \end{align*}
 then our protocols could recover the undecidable model discussed in the previous paragraph.  Of course, one can come up with even more obviously wrong output domains, such as a domain that consits of Turing machines with certain evaluation operations. We do not know where the dividing line is between ``right'' and ``wrong'' output domains.
\end{myexample}


\subsection{Proof of \cref{thm:field-domain}}
\label{sec:proof-of-thm-field-domain}
We now return to the proof of \cref{thm:field-domain}. The right-to-left implication says that every weighted automaton can be simulated by a protocol. This is proved essentially in the same way as in the Boolean case. Suppose that the function is computed by a weighted automaton, which uses  dimension $d$. Every input string $\Sigma^*$ induces a linear map of type $\domain^d \to \domain^d$, which is obtained by composing the linear maps for the individual letters in the string.  Such a linear map can be represented as a matrix, and therefore it can be output using $d^2$ messages.  In the protocol, Alice sends the matrix  which corresponds to her local string, and Bob sends the  matrix which corresponds to  his local string. These matrices are multiplied using the field operations, and then multiplied with the initial and final vectors. This protocol has one round and is signal-free, i.e.~no information is conveyed using signals.

The rest of this proof is devoted to the left-to-right implication, i.e.~showing that every function computed by a protocol is computed by a weighted automaton. As in the Boolean case, we will do a sequence of reductions, such that the protocol becomes more and more restrictive. In particular, we will show that the protocol can be reduced to a version that has one-round and  is signal-free.

\subsubsection{Reduction to a scalar product protocol}
\label{sec:reduction-to-scalar-product-protocols}

In the first step, we show that each protocol can be constrained to have a special form, which has one round and is signal-free. This protocol uses only the scalar product,  as explained in the following definition. 
\begin{definition}[Scalar product protocol] \label{def:scalar-product-protocol}
    Assume that the output domain is a field.
    A scalar product protocol is defined as follows. First, each of the two parties uses their local string to  produce a vector of field elements, of some fixed dimension $d$, as expressed by two functions: 
    \begin{align*}
    \sigma_A, \sigma_B : \Sigma^* \to \domain^d.
    \end{align*}
    Next, the output is defined to be the scalar product of the two vectors. 
\end{definition}

This protocol has the same power as general protocols. 

\begin{lemma}\label{lem:scalar-product-reduction}
    Assume that the output domain is a field. 
    If a function is computed by a protocol, then it is computed by a scalar product protocol.
\end{lemma}
\begin{proof}
    The proof is a sequence of reductions, where more and more conditions are imposed on the protocol.  
    
    \paragraph*{Step 1. One-round protocol.} The first step is to reduce the protocol to a one-round protocol. This is done using \cref{lemma:one-round-reduction-general}.



 \paragraph*{Step 2. Signal-free protocol.}  We say that a protocol is \emph{signal-free} if both of the sets $Q_A$ and $Q_B$ have one element each. In other words, the signals do not convey any information, and the only messaging activity consists of sending elements of the output domain. In a signal-free protocol, the concept of rounds is irrelevant, since the behaviour of one party is not influenced by the communication from the other party.

 \begin{claim}
    \label{claim:trivial-messages}
    Assume that the output domain is a field. Then every one-round protocol is equivalent to a signal-free  protocol.
 \end{claim}
 \begin{proof} 
    Consider a one-round protocol. Without loss of generality, we assume that both signal spaces $Q_A$ and $Q_B$ are the same space $Q$. (We can always use the union of two signal spaces for both parties.) Assume that each of the parties sends $d$ field elements in the protocol. In other words, the protocol works as follows:
    \begin{enumerate}
        \item Based on her local string, Alice chooses a message $(q_A,\bar x) \in Q \times \domain^d$;
        \item Based on his local string, Bob chooses a message $(q_B,\bar y) \in Q \times \domain^d$;
        \item Based on the signals $q_A$ and $q_B$, a term operation  with $2d$ variables is chosen, call it $t_{q_A,q_B}$, and the output is obtained by applying this term operation to $(\bar x, \bar y).$
    \end{enumerate}
    To prove the claim, we need to show that the protocol can be adapted so that always the same term operation is chosen, i.e.~there is no dependence of this term operation on the signals $q_A$ and $q_B$. This way the signals can be eliminated. To do this, we increase the dimension from $d$ to $d + |Q|$. 
    This means that for each possible signal $q \in Q$, each party sends a field element corresponding to this signal. The idea is that instead of sending a signal $q \in Q$, each party will set the corresponding field element to $1$, and the remaining field elements to $0$. The corresponding term operation is then 
    \begin{align*}
    \sum_{\substack{q_A \in Q \\ q_B \in Q}} \myoverbrace{x_{q_A} \cdot y_{q_B}}{variables corresponding \\ to the messages $q_A$ and $q_B$, } \cdot t_{q_A,q_B}(\bar x, \bar y).
    \end{align*}
    When evaluating this term operation, the summands that do not correspond to the intended message $(q_A,q_B)$ will be eliminated, since they will contain a variable that is set to $0$. Only the summand corresponding to the intended message will be used, and thus the correct output will be produced. 
 \end{proof}

 In the above claim, the only property of fields that was used is that there are elements $1$ and $0$ with the usual field properties, i.e.~$1$ is neutral for multiplication, while $0$ is neutral for addition and cancellative for multiplication. Therefore, so far our proof would work in any semiring with such elements.

 \paragraph*{Step 3. Scalar product.} In the previous step, we have reduced the protocol to a special case, where Alice and Bob send vectors, call them $\bar x, \bar y \in \domain^d$, and then some fixed term operation $t$ with $2d$ variables is applied to them. 
  To complete the proof of the lemma, we show that the term operation can be turned into a scalar product. This term operation is a sum of monomials, with each monomial being a product of some variables. Consider the monomials in the term operation $t$. For each monomial, its contribution to the output  is obtained by multiplying two numbers: (a) the product of the  variables in the term operation that are contributed by Alice; and   (b) the product of the variables in the term operation that are contributed by Bob. We can redesign the protocol so that for each monomial, Alice sends the contribution (a), and Bob sends the contribution (b). In the new protocol, the dimension is the number of monomials from the original protocol, and the term operation is a scalar product. 

  This completes the third and final step in the proof. In this step, the only property of fields that was used is that multiplication is commutative, and therefore each monomial can be cleanly separated into two parts, one for Alice and one for Bob. Summing up, the entire lemma would work for commutative semirings, and just for fields. However, the second part of the proof of Theorem~\ref{thm:field-domain}, presented in Section~\ref{sec:from-scalar-product-protocol-to-weighted-automaton}, does use the assumption  that the output domain is a field.
\end{proof}

\subsubsection{From a scalar product protocol to a weighted automaton}
\label{sec:from-scalar-product-protocol-to-weighted-automaton}
In this section, we complete the proof of \cref{thm:field-domain}, by showing that scalar product protocols can be simulated by weighted automata. Similarly to the Boolean case, the proof  uses a Myhill-Nerode characterization. In the case of weighted automata, this characterisation is called  the Fliess Theorem. This theorem  characterises functions computed by weighted automata in terms of a certain infinite matrix.

\begin{definition}[Hankel Matrix]\label{def:hankel-matrix}
    Let $\domain$ be a field. The Hankel matrix of a function 
    \begin{align*}
    f : \Sigma^* \to \domain
    \end{align*}  
    is the matrix where rows are words in $\Sigma^*$, columns are words in $\Sigma^*$, and the entry corresponding to a row $u$ and a column $v$ is $f(uv)$.
\end{definition}

Another perspective on the Hankel matrix is that it describes the \emph{derivatives} of the function $f$. Each row in the Hankel matrix can be seen as a function of type $\Sigma^* \to \domain$, which inputs columns (i.e.~strings) and outputs the corresponding entries in the Hankel matrix. If the row corresponds to a word $w$, then this function is
\begin{align*}
v \mapsto f(wv),
\end{align*}
which is called the \emph{left derivative} of $f$ with respect to $w$. Similarly, the columns of the Hankel matrix describe \emph{right derivatives} of $f$.

The Fliess Theorem~\cite[Theorem 2.1.1]{fliess1974} states that a function 
\begin{align*}
f : \Sigma^* \to \domain
\end{align*}
is computed by a weighted automaton if and only if  its  Hankel matrix  has finite rank, i.e.~its rows (i.e.~the left derivatives) are spanned by a finite subset. (This is equivalent to saying that the columns, or right derivatives, have a finite spanning subset.) Therefore, to complete the proof of \cref{thm:field-domain}, it is enough to show the following lemma.

\begin{lemma}\label{lem:hankel-finite-rank}
    If a function is computed by a scalar product protocol, then its Hankel matrix has finite rank.
\end{lemma}
\begin{proof}
    Essentially by definition, the Hankel matrix of a function computed by a scalar product protocol with dimension $d$ can be obtained as a sub-matrix of the following matrix: rows and columns are vectors in $\domain^d$, and the entries are obtained by taking scalar products. This matrix is easily seen to have finite rank, namely $d$, since the scalar product  becomes a linear operation once one of the two arguments is fixed.
\end{proof}

\section{String outputs}
\label{sec:string-outputs}

In this section, we consider the case where the \kl{output domain} is strings over
some finite alphabet. We use the name \emph{string-to-string function} for any
function of type $\Sigma^* \to \Gamma^*$, where both alphabets $\Sigma$ and
$\Gamma$ are finite. For such functions, the \kl{protocol}s are assumed to use the
\kl{output domain} of strings $\Gamma^*$ equipped with concatenation. In the case of
string-to-string functions, we conjectured, see
Conjecture~\ref{conj:protocol-regular-string-to-string},  that \kl{protocol}s define
exactly the so-called \kl{regular function}s, which will be formally defined in
Section~\ref{sec:regular-string-to-string-functions} below.
 
% this file gives contains only one thing: a figure that illustrates the rational, regular and polyregular functions


% the following two macros are used only for one figure
% a class of functions, decorated by a brace on the left
\newcommand{\functionclass}[2]{
    \text{#1} & 
\left\{{\text{\begin{minipage}{11cm}
    \small 
#2
\end{minipage}}}\right. 
}

% a vertical inclusion between classes 
\newcommand{\vertinclusion}[1]{\\ &
\mathrel{\rotatebox[origin=c]{270}{$\subsetneq$}}
\text{\qquad \small(#1)}\\ \\ }

\begin{figure}
    \centering
\begin{align*}
    \functionclass{rational}{
        unambiguous one-way automata with output~\cite[Chapter IX]{Eilenberg74}  =  Eilenberg bimachines~\cite[XI.7]{Eilenberg74} =  compositions of left-to-right and right-to-left sequential functions~\cite[Proposition 7.3]{elgot_relations_1965}   = \mso relabelings~\cite[p.~11]{bloem_comparison_2000} = a certain Myhill-Nerode style characterisation~\cite[Theorem 1]{reutenauerSchutzenberger1991}
    } \\
\vertinclusion{the reverse function is regular but not rational}
\functionclass{regular}{ deterministic two-way automata with output~\cite[Note 4]{shepherdson1959reduction} = streaming string transducers~\cite[Theorems 1,2,3]{alurExpressivenessStreamingString2010} = certain regular expressions for transducers~\cite[Theorem 15]{alur2014regular}  = \mso transductions = a calculus based on combinators~\cite[Theorem 6.1]{bojanczykRegularFirstOrderList2018} = lambda calculus with a certain fold operator~\cite[Theorem VI.1]{polyregular-fold} =   a characterisation in terms of natural transformations~\cite[Theorem 3.2]{bojanczykTitoRegular23} = macro string transducers with linear sized outputs~\cite[Theorem 7.1]{engelfrietMacroTreeTranslations2003} = polyregular functions with linear sized outputs~\cite[Example 11]{polyregular-survey}= transducers with Church encodings of output strings~\cite[Theorem 1.1]{implicit2}  } \\
\vertinclusion{regular functions have linear output size, unlike polyregular ones}
\functionclass{polyregular}{
deterministic pebble automata with output~\cite[p.~235]{engelfriet2002two} = for-transducers~\cite[Theorem 2.1]{polyregular-survey} = compositions of regular functions and squaring~\cite[Theorem 3.2]{polyregular-survey} = a functional programming language~\cite[Theorem 4.1]{polyregular-survey} = \mso interpretations~\cite[Theorem 7]{msoInterpretations} = lambda calculus with a more powerful fold operator~\cite[Theorem V.3]{polyregular-fold} 
} 
\end{align*}
    \caption{Three important classes of string-to-string functions.}
    \label{fig:transducer-classes}
\end{figure}



The content of this section is an extended discussion of this conjecture.
In Section~\ref{sec:regular-string-to-string-functions} we  formally  define the \kl{regular function}s and  prove the implication 
\begin{align*}
\text{\kl{protocol}} \impliedby \text{\kl(function){regular}}.
\end{align*}
The content of the conjecture is therefore the implication 
\begin{align*}
\text{\kl{protocol}} \implies \text{\kl(function){regular}}.
\end{align*}
In Section~\ref{sec:continuity}, we present some evidence for this implication. We  show that string-to-string functions computed by \kl{protocol}s share many good properties of the \kl{regular function}s, such as linear output size and computability. In Section~\ref{sec:unary-output-alphabet}, we present further evidence for the implication, namely we  prove it in the special case where the output alphabet has only one letter (the remaining case is two output letters, since more letters do not change the situation). Finally, in Section~\ref{sec:beyond-fields}, we discuss variants of the conjecture that are related to \kl{weighted automata} which are not over a field, but over an arbitrary semiring. 


\subsection{Regular string-to-string functions}
\label{sec:regular-string-to-string-functions}

In this section, we define the class of regular string-to-string functions, and we prove one implication in the conjecture. 
 Historically, this class of functions was first defined in terms of 
 deterministic two-way automata with output\cite[Note 4]{shepherdson1959reduction}. Let us present this definition.

 \begin{definition}[Two-way automaton]
    A deterministic two-way automaton with output is given by the following ingredients:
    \begin{enumerate}
        \item a finite input alphabet $\Sigma$;
        \item a finite output alphabet $\Gamma$;
        \item a finite set of states $Q$, with an initial state $q_0 \in Q$;
        \item a transition function  
        \begin{align*}
        \delta : 
        \myunderbrace{Q}{old \\ state} \times 
        \myunderbrace{(\Sigma + \{\vdash, \dashv\})}{input letter\\ under  the head} \to  \set{\text{halt}} + (
        \myunderbrace{Q}{new \\ state}
         \times 
         \myunderbrace{\{-1,0,1\}}{head \\ movement} \times 
         \myunderbrace{\Gamma^*}{added \\ output}) .
        \end{align*}
    \end{enumerate}
 \end{definition}

    The automaton works as follows. The input string $w$ is placed on a tape, with the left end marked by $\vdash$ and the right end marked by $\dashv$. The automaton starts in state $q_0$, with its head on the left end of the tape, which contains the marker $\vdash$. In each step, the automaton looks at its current state and the letter under its head, and based on this information, it uses the transition function to decide if it halts, or it continues its computation. In case it continues, it chooses a  new state, the direction in which it moves its head, and a string over the output alphabet $\Gamma$, which is appended to the output tape. We assume that the automaton is always halting, which means that for every input string, the computation eventually halts. In particular, the computation must be well-defined, which means that the head never falls off the input by moving outside the endmarkers.   The semantics of such an automaton is of type $\Sigma^* \to \Gamma^*$. (For automata which are not necessarily halting, the function would be partial, since it would be undefined for inputs where the automaton does not halt.)

    \begin{myexample}[Reverse]
        For each input alphabet $\Sigma$, the reverse function of type $\Sigma^* \to \Sigma^*$ is computed by a  two-way automaton, which first moves its head to the end of the string, and then starts copying it to the output while moving in the left direction.
    \end{myexample}

    The class of functions computed by two-way automata has a remarkable number of equivalent descriptions,  originating in different fields, including:  monadic second-order transductions~\cite[Section 4]{engelfrietMSODefinableString2001}, streaming string transducers~\cite[Section 3]{alurExpressivenessStreamingString2010},  certain kinds of regular expressions~\cite[Section 2]{alur2014regular}, a calculus of functions based on  combinators~\cite[Theorem 6.1]{bojanczykRegularFirstOrderList2018}, a characterisation based on natural transformations~\cite[Theorem 3.2]{bojanczykTitoRegular23}. For this reason, some authors (starting with Engelfriet and Hoogeboom), use the name \emph{regular} for this class of function, with the intended meaning being that these functions play the same role for string-to-string functions, as that which is played by regular languages for string-to-Boolean functions. We adopt this terminology here, as stated in the following definition.


    \begin{definition}[Regular string-to-string function]
        \label{def:regular-string-to-string}
        A string-to-string function is called \emph{regular} if it is computed by a deterministic two-way automaton with output.
    \end{definition}
    
    
    One good  property of the regular string-to-string functions is that they  are closed under composition~\cite[Theorem 2]{chytilSerialComposition2Way1977}. In particular, our conjecture would imply that the same is true for functions computed by protocols. Without proving the conjecture, we do not see any direct way of proving composition for protocols.
     
    
    Another good property of the  regular string-to-string functions is that equivalence is decidable, i.e.~given two functions $f$ and $g$, one can decide if for every input string, the two output strings are equal~\cite[Theorem 1]{gurariEquivalenceProblemDeterministic1982}. This property does not seem to have any direct bearing on protocols, since there is no obvious way of presenting a non-uniform protocol as an input for a decision procedure.



    The following lemma shows one of the implications in the conjecture.

\begin{lemma}\label{lem:from-regular-to-protocol}
    If a string-to-string function is regular, then it is computed by a protocol.
\end{lemma}
\begin{proof}
    The two parties can simulate a two-way automaton with output. The execution of the protocol describes the crossing sequence of the automaton, i.e.~how it crosses the boundary between the two local strings of Alice and Bob. Here is a picture: 
    \mypic{1} 
    More formally, the crossing sequence is defined as follows, given  a split of the input string into two parts $w_1 w_2$. We run the automaton until the first configuration which is in the word $w_2$. Then we run it until the first configuration which is in the word $w_1$. We continue this way, with odd-numbered steps describing runs inside $w_1$ that end in  configurations from $w_2$, and even-numbered steps describing runs inside $w_2$ that end in a configuration from $w_1$. The last step is exceptional, since it ends with an accepting configuration. 
    The number of steps in a crossing sequence is bounded  the number of states, since otherwise the automaton would enter an infinite loop. This bound is the number of rounds in the protocol. In each round, the state corresponding to this round is sent as a signal, and the output value in the message is the  part of the output string that is produced in this  step. At the end of the protocol, the pieces of the output string are concatenated. 
\end{proof}

In view of the above lemma, the content of the conjecture is the opposite implication, namely that every protocol computes are regular function. In the rest of this section, we give some evidence for the opposite implication. In Section~\ref{sec:continuity}, we show that functions computed by protocols share some good properties of the regular functions, which is evidence that they might be the same functions. Then,  in Section~\ref{sec:unary-output-alphabet}, we prove the conjecture in the special case of a unary output alphabet. 



\subsection{Evidence for the conjecture}
\label{sec:continuity}
In this subsection, we show that the string-to-string functions computed by protocols share some  good properties of regular functions, such as linear size outputs and computability. Even computability is not a priori obvious, due to the non-uniformity of the protocols. These results can be seen as evidence of the open implication 
\begin{align*}
\text{protocol} \implies \text{regular}
\end{align*}
in the conjecture. To prove these results, we will leverage the results on weighted automata from Section~\ref{sec:field-domain}.
The point of departure is the following lemma, which connects weighted automata and string-to-string functions that can be computed in our protocol.

    \begin{lemma}
        \label{lem:postcomposition-weighted-automaton}
        Let $\domain$ be a field, and consider two functions
        \[
        \begin{tikzcd}
        \Sigma^* 
        \ar[r,"f"]
        &
        \Gamma^*
        \ar[r,"g"]
        & 
        \domain,
        \end{tikzcd}
        \]
        such that $f$ is computed by a protocol (with string outputs). If $g$  is computed by a weighted automaton, then the same is true for the composition  $f;g$.
    \end{lemma}
    \begin{proof}
        We will show that the composition $f;g$ is computed by a protocol (with field outputs). Thanks to \cref{thm:field-domain}, this will imply that $f;g$ is computed by a weighted automaton.
        For each string in $\Gamma^*$, the weighted automaton for $g$ has an associated matrix over the field $\domain$. We modify the protocol for $f$, so that it uses matrices instead of strings. During the execution, instead of sending strings in $\Gamma^*$, the parties send  the corresponding matrices. At the end, instead of concatenating the output strings, the matrices are multiplied, yielding a matrix for the entire output string. Finally, this matrix is applied to the initial state, and then the output function is applied to the resulting vector. All of this is done using addition and multiplication, which is legitimate in a protocol with the output domain $\domain$.
    \end{proof}


The above lemma establishes a property of $f$, namely that weighted automata (over a field) are closed under precomposition with $f$. We think that this is an important property, and therefore we give it a name.


\begin{definition}[Field continuity]
    \label{def:weighted-continuity}
    A string-to-string function $f : \Sigma^* \to \Gamma^*$ is called \emph{field continuous} if functions computed by weighted automata over a field are closed under precomposition with $f$.
\end{definition}

In the above definition, we only consider weighted automata over a field. The more general setting of semirings is discussed in Section~\ref{sec:beyond-fields}.
The name ``continuous'' is inspired by a similar terminology that is used in automata theory for functions that preserve regularity under inverse images, see~\cite[Theorem 4.1]{PinSilva05} or~\cite[Footnote 2]{continuity20}.  For the latter notion, we use the name \emph{Boolean continuity}.

\begin{definition}[Boolean continuity]
    A string-to-string function $f : \Sigma^* \to \Gamma^*$ is called \emph{Boolean continuous} if preimages of regular languages are regular.
\end{definition}

As we have shown in \cref{lem:postcomposition-weighted-automaton}, all string-to-string functions computed by protocols are field continuous. In particular, since every regular string-to-string function is computed by a protocol, it follows that every regular string-to-string function is field continuous\footnote{To the best of our knowledge, this is a new result. It can also be proved directly, without passing through protocols, and we present such a direct proof in Section~\ref{sec:beyond-fields}, for the more general setting of commutative semirings. In the more general case we need a direct proof, since \cref{thm:field-domain} is not known to be true for this case.}
We conjecture that the converse is also true.

\begin{conjecture}\label{conj:regular-continuous}
    A string-to-string function is field continuous if and only if it is regular.
\end{conjecture}

Later on, in Section~\ref{sec:beyond-fields}, we will dicuss variants of the conjecture, and in particular we show that the conjecture becomes false if the left side is relaxed from field continuous to Boolean continuous.

Conjecture~\ref{conj:regular-continuous} can be seen as  a machine independent characterisation of the regular string-to-string  functions. This would be a very valuable contribution. Almost all known characterisations of the regular string-to-string  functions have somewhat lengthy definitions, based on specific computational models, and it is something of a  miracle that all of these models are equivalent. A possible exception is the characterisation in~\cite{bojanczykTitoRegular23}, which does not use a machine model; however that characterisation uses the abstract language of category theory, and is less elementary than the one in Conjecture~\ref{conj:regular-continuous}.

As in Conjecture~\ref{conj:protocol-regular-string-to-string}, the content of Conjecture~\ref{conj:regular-continuous} is the left-to-right implication
\begin{align*}
\text{field continuous} \implies \text{regular}.
\end{align*}
Conjecture~\ref{conj:regular-continuous} is stronger than Conjecture~\ref{conj:protocol-regular-string-to-string}, as explained in the following diagram, which shows the known relations between three kinds of string-to-string functions:
\[
\begin{tikzcd}
\text{regular}
\ar[d,Rightarrow,shift right=2, "\text{\cref{lem:from-regular-to-protocol}}"']
\\
\text{computed by protocols}
\ar[d,Rightarrow, shift right=2, "\text{\cref{lem:postcomposition-weighted-automaton}}"']
\ar[u,Rightarrow, shift right=2,"\text{Conjecture~\ref{conj:protocol-regular-string-to-string}}"']
\\ 
\text{field continuous} 
\ar[uu,bend right=89, Rightarrow, shift right=2,"\text{Conjecture~\ref{conj:regular-continuous}}"']
\end{tikzcd}
\]


The following theorem gives some evidence for the stronger conjecture,  and therefore also the weaker one, by showing that the field continuous functions share some well-known properties of the regular string-to-string functions. 

\begin{theorem}\label{thm:evidence-for-the-conjecture}
    If a function $f : \Sigma^* \to \Gamma^*$ is  field continuous, then:
    \begin{enumerate}
        \item \label{it:linear-size-outputs} the outputs have at most linear size;
        \item \label{it:linear-time-computable} the outputs can be   computed in linear time;
        \item \label{it:regular-preimages} it is Boolean continuous, i.e.~preimages of regular languages are regular.
    \end{enumerate}
\end{theorem}

% Before proving the theorem, let us comment on the properties that are listed in it.  By Lemma~\ref{lem:from-regular-to-protocol}, every regular string-to-string  function is computed by a protcol, and therefore every regular string-to-string function has the properties that are listed in the theorem. The fact the regular string-to-string functions have the  first three properties is a folklore result and can be seen directly from the definition of regular string-to-string functions, without protocols. (For the third property, it is useful to know that, as language acceptors, two-way automata recognise exactly the regular languages~\cite[Theorem 2]{shepherdson1959reduction}).
% The fact that regular functions have the  property, about postcomposition with weighted automata over a field, is not known in the literature, to the best of our knowledge. A direct proof is possible, 

\begin{proof}
    For properties~\ref{it:linear-size-outputs} and~\ref{it:linear-time-computable}, we embed strings into numbers. 
    An output string over alphabet $\Gamma$ can be seen as a number in base $|\Gamma|$. To avoid the ambiguity that could result from leading zeros, we first prepend the string with the digit 1. Let 
    \begin{align*}
    g : \Gamma^* \to \Nat \subseteq \Rat
    \end{align*} 
    be the corresponding encoding. This encoding can be computed by a weighted automaton over the field $\Rat$, see~\cite[Lemma 8.10]{bojanczyk_automata_2025}. By  the assumption on field continuity, the composition $f;g$ can be computed by a weighted automaton. This is a weighted automaton that works in the field of rationals $\Rat$, but  only produces natural numbers on its output. By~\cite[p. 110]{BerstelReutenauer08}  the automaton can be chosen so that it only uses  integers $\Int$, possibly including negative integers. Summing up, we have a weighted automaton over $\Int$ that outputs the representation, in base $|\Gamma|$, of the output string produced by $g$. We claim that for such an automaton, the output number
    \begin{enumerate}
        \item has a linear number of digits;
        \item can be computed in linear time.
    \end{enumerate}
    These two claims yield the corresponding items in the statement of the theorem. The first claim, about a linear number of digits, is true because it is true for every weighted automaton over $\Int$. This is because applying a fixed linear map can only add a constant number of digits. The second claim is also easy to see, since the weighted automaton can be evaluated in linear time (we assume that we work in a model where addition and subtraction of integers has unit cost). 
    We are left with property~\ref{it:regular-preimages}, about Boolean continuity.  This will follow from the special case of field continuity, where the field is  the two-element field.  This is because of  the following  folklore correspondence between regular languages and weighted automata over the two-element field. 
        
        \begin{claim}\label{claim:regular-weighted-automata}
            A language $L \subseteq \Gamma^*$ is regular iff its characteristic function $\Gamma^* \to \set{0,1}$ is computed by a weighted automaton over the two-element field 
        \end{claim}
        \begin{proof}
            For the lef-to-right implication, we observe that a weighted automaton over a finite field can be simulated by a deterministic finite automaton. For the other direction, we observe that a weighted automaton can count the parity of  the number of runs in a finite automaton, and if the automaton is deterministic then the number of runs is either zero or one, and thus the parity gives the right answer.
        \end{proof}

        In terms of the correspondence from the above claim, preimages of regular languages become precompositions of weighted automata over the two-element field. In particular, regularity is preserved. 
\end{proof}

One could think that already the three properties in the above theorem are not only necessary for regularity, but also sufficient. This is not the case, as shown by the following example.

\begin{myexample}[Factorials]
    \label{ex:not-regular-but-continuous-over-finite-fields}
    Consider a string-to-string function 
    \begin{align*}
    g : \Sigma^* \to \set{a}^*
    \end{align*}
    where both the input and output alphabets are unary. A sufficient condition for Boolean continuity of such functions is given in  \cite[Example 2.12]{bojanczykTitoRegular23}, using \emph{factorials}, i.e.~numbers in the set $\setbuild{n!}{$n \in \Nat$}$. This sufficient condition is that: (a) every output string arises from finitely many inputs; and (b) every output string has length that is a factorial. 
    It is not hard to come up with a non-regular function that has  properties (a) and (b), thus ensuring Boolean continuity, and which has furthermore linear size outputs and is computable in linear time. For example, the function could map an input string $w$ to the longest string of factorial length that is shorter than $w$. 
\end{myexample}

% LTeX: language=en
\subsection{Unary output alphabet}
\label{sec:unary-output-alphabet}
\AP
In this section, we provide further evidence for
Conjectures~\ref{conj:protocol-regular-string-to-string}
and~\ref{conj:regular-continuous}, by showing that they are true for output
alphabets with only one letter. From a technical point of view, this is the
most involved result in the paper, since our proof uses a refined analysis of
the expressive power of weighted automata that output natural numbers of linear
size.

\begin{theorem}\label{thm:unary-string-to-string}
    The following conditions are equivalent for  a string-to-string function where  the output alphabet  has only one letter:
    \begin{enumerate}
        \item computed by a protocol;
        \item \label{it:unary-weighted-continuous} field continuous;
        \item \label{it:unary-regular} regular.
    \end{enumerate}
\end{theorem}

In light of Lemmas \ref{lem:from-regular-to-protocol} and
\ref{lem:postcomposition-weighted-automaton}, the only missing implication is
\ref{it:unary-weighted-continuous}~$\Rightarrow$~\ref{it:unary-regular},
that is the content of \cref{lem:nat-protocols-regular} below.


\begin{lemma}
  \label{lem:nat-protocols-regular}
  Let $f : \Sigma^* \to \{a\}^*$ be a function computed by a protocol
  manipulating natural numbers. Then $f$ is a regular function.
\end{lemma}

Let us first remark that under the assumption that the output is unary,
the class of "rational" and "linear regular functions" coincide.

\begin{lemma}
  \label{lem:unary-linear-regular-rational}
  Let $f : \Sigma^* \to \{a\}^*$ be a function. The following conditions are
  equivalent:
  \begin{enumerate}
    \item \label{it:unary-rational}
      $f$ is a rational function;
    \item \label{it:unary-linear-regular}
        $f$ is a linear regular function.
  \end{enumerate}
\end{lemma}
\begin{proof}
  The implication \ref{it:unary-rational}~$\Rightarrow$~\ref{it:unary-linear-regular}
  is immediate  and holds even without the assumption on the 
  unary output from \cref{fig:transducer-classes}.

  For the converse implication, assume that $f$ is rational (?).
  \omc{For converse implication, the assumption should be $f$ is linear regular, right? Typo?}
  Then it is
  computed by a two-way transducer with outputs. However, this transduction can
  be decomposed into a two step process: first, a rational function computes
  the crossing sequences of the transducer on every letter of the input, and
  then, a two-way transducer that simply follows these crossing sequences to
  produce the output in the correct order. Since the order of the output does
  not matter in the unary case, the second step can be performed in a
  block-wise fashion, producing all outputs corresponding to a letter
  immediately after reading it. This second step is therefore also a rational
  function, and since those are closed under composition, we conclude that $f$ is
  a rational function.
\end{proof}

Let us consider a "protocol" computing a function $f : \Sigma^* \to \{a\}^*$.
Recall that one can assume that the protocol is one round
(\cref{lemma:one-round-reduction-general}). Let us write $\sigma_A : \Sigma^*
\to Q_A \times \domain^k$ for the strategy of Alice, $\sigma_B : \Sigma^*
\times Q_A \to Q_B \times \domain^k$ for the strategy of Bob, and $\theta : Q_A
\times Q_B \to (\domain^{2k} \to \domain)$ for the combining function, that can
only use concatenation of registers. We can place a quasi-order on $Q_A \times
\domain^k$, by saying that $(q, \vec{r}) \leq (q', \vec{r}')$ if $q = q'$ and
for each $i \in \{1, \ldots, k\}$ we have $|r_i|\leq |r_i'|$. This quasi-order
on $Q_A \times \domain^k$ can be lifted to a quasi-order on strings in
$\Sigma^*$, by saying that $u \leq v$ if $\sigma_A(u) \leq \sigma_A(v)$. A key
property of this ordering is the following:

\begin{lemma}
  \label{lemma:order-monotonicity-unary-output}
  Let $u, v \in \Sigma^*$ be such that $u \leq v$. Then for every $w \in
  \Sigma^*$, we have $|f(uw)| \leq |f(vw)|$. Furthermore, the function 
  $\Delta_{u,v} : w \mapsto |f(vw)| - |f(uw)|$ is a bounded function, and for 
  every $c \in \Nat$, the set $\{ w \in \Sigma^* : \Delta_{u,v}(w) = c\}$ is
  a regular language.
\end{lemma}
\begin{proof}
  Assume that $u \leq v$, and fix some $w \in \Sigma^*$.
  Let $q \in Q_A$ be the state that Alice sends to Bob after reading $u$.
  It is the same state after reading $v$. From the point of view of 
  Bob, the two strings behave identically. Now, since the combining function
  can only use concatenation of registers, and since the registers produced by
  Bob on the split $(u, w)$ are the same as those produced on the split
  $(v, w)$, the result follows by monotonicity of concatenation with respect to
  length. Note that the difference in lengths is bounded by the difference 
  between the lengths of the registers after Alice's turn, which does not 
  depend on $w$, proving the boundedness.

  Finally, the set $\{ w \in \Sigma^* : \Delta_{u,v}(w) = c\}$ is regular because it
  can be computed by a Boolean Alice-Bob protocol: Alice simulates the original protocol
  when prepending $u$ and $v$ to its input, sends the difference of the lengths 
  of its registers (that is now a finite message, since the difference is bounded) to Bob,
  who can then check whether the total difference is $c$.
  Because of \cref{thm:boolean-domain}, we conclude that the language is in fact regular.
\end{proof}

The second key property of the ordering is the following: it is in fact a
""well-quasi-order"", meaning that for any infinite sequence $(q_i,
\vec{r}_i)_{i \in \Nat}$, there are two indices $i < j$ such that $(q_i,
\vec{r}_i) \leq (q_j, \vec{r}_j)$. 

\begin{lemma}
  \label{lemma:wqo-unary-output}
  The quasi-order on $\Sigma^*$ defined above is a well-quasi-order.
\end{lemma}
\begin{proof}
  Because the order only compares words with respect to their image by
  $\sigma_A$, it is enough to show that the quasi-order on $Q_A \times
  \domain^k$ is a well-quasi-order.
  Since $Q_A$ is finite, it is enough to show that for each $q \in Q_A$,
  the quasi-order on $\{q\} \times \domain^k$ is a well-quasi-order.
  This follows from Dickson's lemma, since $\domain = \{a\}^*$ is isomorphic
  to $\Nat$ with addition, and since $\Nat^k$ with the product ordering is a
  well-quasi-order by Dickson's lemma.
\end{proof}

From the above two lemmas, we can conclude the proof of
\cref{lem:nat-protocols-regular}. 

\begin{proof}
  We claim that there exists a finite set $L$ of words $u_1, \ldots, u_n \in
  \Sigma^*$, closed under taking prefixes, and such that for every $u \in L$
  and every letter $a \in \Sigma$, either $ua \in L$, or there is some $u_i \in
  L$ such that $u_i \leq ua$. 
  Such a finite set $L$ exists by \cref{lemma:wqo-unary-output}, since one can
  extract a finite set of minimal elements for the quasi-order, and close it
  under prefixes. 

  We can now produce a left-to-right automaton that computes $f$ using regular
  lookaheads as follows. The states of the automaton are the words in $L$.
  The final output of a state $u_i \in L$ is $f(u_i)$.
  The transition from a state $u_i$ on a letter $a$ is defined as follows:
  \begin{itemize}
    \item if $u_i a \in L$, then the transition goes to state $u_i a$;
    \item otherwise, there is some $u_j \in L$ such that $u_j \leq u_i a$,
      and the transition goes to state $u_j$ (this is a choice made uniformly).
  \end{itemize}
  The lookahead for the transition from $u_i$ on $a$ is the regular language
  that tells the precise difference in size between $f(u_i a w)$ and $f(u_j w)$. By
  \cref{lemma:order-monotonicity-unary-output}, this delay is bounded so there 
  are finitely many possibilities, and the precise difference can be checked by a
  regular property on $w$. The automaton outputs this difference on the
  transition.

  An immediate induction proves that this automaton with regular lookaheads
  computes the same function as $f$, and it is a standard result that 
  regular functions are closed under regular lookaheads,
  concluding the proof.
\end{proof}


Let us now use \cref{lem:nat-protocols-regular} to prove a result on
string-to-number functions. Remark that a string-to-string function with a
unary output alphabet can be seen as a string-to-number function, by mapping
the string $a^n$ to the number $n$. Concatenation becomes addition in this
setting. The following theorem states that considering string-to-number
protocols (keeping only addition as operation on numbers) does not change the
class of functions if the outputs are non-negative.

\begin{theorem}
  \label{thm:string-to-number-protocols-nat}
  Let $f : \Sigma^* \to \Nat$ be a function computed by a protocol
  manipulating integers. Then, $f$ is also computed by a protocol
  manipulating natural numbers.
\end{theorem}
\begin{proof}
  Let us assume without loss of generality that $f$ is computed by a one-round
  protocol. Furthermore, up to normalizing the protocol, we can assume that only one
  register of Alice is used in the final computation, and that all registers of
  Alice can be used in some computation. This can be ensured by a powerset
  construction on Alice's side.

  Let us prove the following intermediate claim: the registers of Alice are
  uniformly bounded below. Assuming the contrary, there is a sequence of inputs
  $(u_i)_{i \in \Nat}$ such that the value of some register $r$ of Alice after
  reading $u_i$ tends to $-\infty$ as $i$ tends to infinity. Since there are
  finitely many states of Alice, we can assume without loss of generality that
  the state of Alice after reading $u_i$ is some fixed state $q$ for all $i$.
  Now, there is a fixed word $v$ such that when Bob receives state $q$ from
  Alice and reads $v$, the output of the protocol is the sum of register $r$ of
  Alice and some value computed by Bob. Because $r$ tends to $-\infty$ on the
  sequence $(u_i)_{i \in \Nat}$, the output of the protocol on inputs $u_i v$
  is negative for sufficiently large $i$, which is absurd.


  Hence, one can define a new protocol computing $f$ as follows: Alice
  computes as before, but truncates all registers to zero if they are
  negative. As state, she sends to Bob her state together with the content of
  the registers that were truncated, which is a finite information since we
  have proved that the registers are uniformly bounded below. Bob can then
  simulate the original protocol, if he wishes to use a truncated register,
  he directly substracts the value from his own output. The result of Bob
  remains non-negative, since the original protocol computed a non-negative
  output.
\end{proof}

This theorem allows us to derive a neat characterisation of functions 
computed by weighted automata over $\Nat$ that have linear growth.

\begin{corollary}
  \label{cor:weighted-automata-nat-regular}
  Let $f : \Sigma^* \to \Nat$ be a function computed by a weighted automaton
  over the semiring $\Rat$, such that $f$ has linear growth. Then, $f$ is a regular function.
\end{corollary}
\begin{proof}
  It is well-known that weighted automata over $\Rat$ computing integer values
  can be simulated by weighted automata over the semiring $\Int$: this is
  sometimes referred to as $\Rat$ being a "fatou extension" of $\Int$
  \cite[p. 110]{BerstelReutenauer08}.
  Therefore, we can assume without loss of generality that $f$ is computed by
  a weighted automaton over $\Int$. Then, one can leverage a 
  result from \cite{Zpolyreg23} stating that weighted automata over $\Int$
  with linear growth can be computed as follows: first, a regular function $g : \Sigma^* \to
  \{a,b\}^*$ is computed, then each letter $a$ is mapped to $+1$ and each 
  letter $b$ is mapped to $-1$, and finally all values are summed up. 
  For a precise statement, see \cite[Proposition II.13, Theorem III.3]{Zpolyreg23}.

  Now, it is clear from the above description that $f$ can be computed by a protocol
  manipulating integer values, and combining them only using addition. By
  \cref{thm:string-to-number-protocols-nat}, $f$ can also be computed by a
  protocol manipulating natural numbers. Finally, by \cref{thm:unary-string-to-string},
  $f$ is a regular function.
\end{proof}

Let us conclude this section with an example showing that
\cref{cor:weighted-automata-nat-regular} cannot be extended to weighted
automata over $\Int$ with arbitrary growth.

\begin{myexample}[Quadratic counterexample]\label{ex:quadratic-counterexample}
     We show  a function which: (a) is a linear combination of \mso  counting functions of arity two, with negative coefficients; (b) has only non-negative outputs; and (c) cannot be presented as linear combination with positive coefficients. The idea, which is based on~\cite[Example 2.1]{BerstelReutenauer08}, is to trivially ensure non-negativity by squaring. Take the function
\begin{align*}
w \in \set{a,b}^* 
\quad \mapsto \quad 
(\text{(number of $a$'s in $w$)} - \text{(number of $b$'s in $w$)})^2.
\end{align*}
This function clearly satisfies (a) and (b). As explained in \cite[p.3]{Zpolyreg23}, it also satisfies (c), since the inverse image of $0$ is not a regular language, as would be the case if only positive coefficients were used.
\end{myexample}

\subsection{Regularity with origins}
\label{sec:origin-semantics}

In this section, we provide further evidence for
Conjectures~\ref{conj:protocol-regular-string-to-string}
and~\ref{conj:regular-continuous}, by showing that they are true under the
so-called ``origin semantics''
\cite{bojanczykTransducersOriginInformation2014}.

\begin{theorem}
  \label{thm:origin-string-to-string}
  A function $f: \Sigma^* \to \Gamma^*$ is computable by a
  two-party protocol with string inputs under the origin semantics if and only if
  it is a regular function with origins.
\end{theorem}




\subsection{Semirings that are not fields}
\label{sec:beyond-fields}
In our discussion so far, a prominent role was played by weighted automata over a field. However, weighted automata make also sense in a more general setting, where a semiring is used instead of a field. We discuss this more general setting in the present section.

\paragraph*{Weighted automata over a semiring.}
We begin by recalling the definition  of weighted automaton  in the case of a semiring which is not necessarily a field. One approach is to use Definition~\ref{def:weighted-automaton}, suitably generalised. Since we are working in a semiring that might no longer be a field, we have to be careful when speaking of linear maps in items~\ref{it:weighted-definition-transitions} and~\ref{it:weighted-definition-final} of Definition~\ref{def:weighted-automaton}. The appropriate notion in this case is matrices: for each input letter, the corresponding state transformation is: 
\begin{align*}
q \in \domain^d \quad \mapsto \quad qA \in \domain^d,
\end{align*}
where $q$ is viewed as a row vector and $A$ is a $d \times d$ matrix in the semiring. In the definition, we have to be careful about the order of multiplication, since the semiring might have non-commutative multiplication. In order to be equivalent to a model that will be described below,  we need the new values from the matrix to come after the old values from $q$, hence $qA$ instead of $Aq$.  Similarly, the final map is of the form $q \mapsto q b$, where $b$ is a column vector of dimension $d$.

However, for the purposes of this section, it  will be more convenient to work with an alternative presentation of weighted automata, which is based on the intuition of nondeterministic automata with weights on states and transitions, as defined below. 
\begin{definition}
    [Weighted automaton, nondeterministic presentation] 
    \label{def:weighted-automaton-nondeterministic}
    A \emph{weighted automaton} consists of a finite input alphabet $\Sigma$,  a finite set of states $Q$, and functions: 
    \begin{align*}
    \myunderbrace{I : Q \to \domain}{initial}
    \quad
    \myunderbrace{F : Q \to \domain}{final}
    \quad
    \myunderbrace{\Delta : Q \times \Gamma \times Q \to \domain}{transitions}.
    \end{align*}
\end{definition}
 A run of this automaton is defined in the usual way: it is a sequence of transitions, one for each input letter, such that consecutive transitions agree on the connecting states. The weight of a run is the product of: (1) the initial weight of its source state; (2) the weights used by its transitions; and (3) the final weight of its target state. If the semiring is non-commutative, the order of multiplication is  important, and transitions are multiplied in the order corresponding to the input string. For an input string, the output of the automaton is the sum of weights of all runs over this automaton (sum is always commutative).  The model from Definition~\ref{def:weighted-automaton-nondeterministic}  defines the same functions as the model from Definition~\ref{def:weighted-automaton}, with the appropriate semiring modifications that were described earlier in this section, see~\cite[Lemma 8.3]{bojanczyk_automata_2025}. For the purposes of this seciton, we use the nondeterministic  model described from Definition~\ref{def:weighted-automaton-nondeterministic}.



  Recall Conjecture~\ref{conj:regular-continuous}, which says that a string-to-string function is regular if and only if it is field continuous. One could consider variants of this conjecture for semirings.  For a class $\algclass$ of semirings, let us define \emph{continuity over $\algclass$} to be the same as in Definition~\ref{def:weighted-continuity}, except that instead of fields, we have semirings from the class $\algclass$. This section is devoted to studying   variants of Conjecture~\ref{conj:regular-continuous} for  important classes of semirings, such as finite fields, all fields,  and all semirings. The relationship between the continuity notions is explained in the following diagram. (The diagram also includes rational functions, which are a propert subclass of the regular functions that will be discussed in \cref{sec:rational-functions}.) The diagram also highlights in red the three classes of functions that we conjecture to be the same in Conjecture~\ref{conj:regular-continuous}.
implication immediate, it is one-way by Example~\ref{ex:not-regular-but-continuous-over-finite-fields}
\[
\begin{tikzcd}[row sep=small]
 \text{\blue{continuous over all semirings}}
\arrow[d,blue, equal,"\text{\cref{thm:rational-functions}}"]
\\
 \text{\blue{rational string-to-string functions}}
\ar[d,Rightarrow,"\text{ implication is one-way,  see e.g.~\cite[p. 218]{engelfrietMSODefinableString2001}}"]
\\
 \text{\red{regular string-to-string functions}}
% \ar[d,Rightarrow,"\text{ \cref{thm:regular-continuous-commutative-semirings}}"]
\ar[d,red,Rightarrow,"\text{ \cref{lem:from-regular-to-protocol}}"]
\\
% \text{continuous over commutative semirings}
% \downsymbol{\subseteq}{immediate} 
% 
 \text{\red{computed by a protocol}}
\ar[d,red,Rightarrow,"\text{\cref{lem:postcomposition-weighted-automaton}}"]
% \downsymbol{\subseteq}{\cref{thm:regular-continuous-commutative-semirings}}
\\
 \text{\red{continuous over fields}}
\ar[uu, red, bend left=80,Rightarrow,"\text{ Conjecture~\ref{conj:regular-continuous}}"]
\ar[d,Rightarrow,"\text{ implication is one-way, see Example~\ref{ex:not-regular-but-continuous-over-finite-fields}}"]
\\
 \text{continuous over finite fields}
\arrow[d,equal,"\text{\cref{claim:regular-weighted-automata}}"]
\\
 \text{Boolean continuous}
\end{tikzcd}
\]

There are three parts in the diagram, indicated using colours. The upper part, in blue, corresponds to the strongest continuity condition, namely over all semirings. As we explain in Section~\ref{sec:rational-functions}, this condition characterises rational functions, a well-known transducer class. The middle part, in red, corresponds to the continuity condition over fields, which is conjectured to characterise regular functions. Finally, the lower part, in black, corresponds to the weakest continuity condition, namely over finite fields. As explained in the proof of \cref{thm:evidence-for-the-conjecture}, this condition is the same as Boolean continuity, and it is strictly more general than regularity. There is no machine model for the black part, because there are uncountably many such functions, as explained in \cite{bojanczykTitoRegular23}.

Of course, the diagram does not exhaust all possible notions of continuity. For example, we could consider continuity over a single semiring, such as the field $\Rat$ of rationals, or the field $\Rat(x)$. For all we know, even these notions might also characterise the regular functions. Another possibility is to consider continuity over all commutative semirings, which is discussed in Section~\ref{sec:commutative-semirings}, and again conjectured to characterise the regular functions.

\subsubsection{Rational functions}
\label{sec:rational-functions}
In this section we describe the rational string-to-string functions, and prove that they are exactly the functions which are continuous over all semirings.  We only give a brief description of the class, for a more precise definition the reader is referred to~\cite[Section 14.2]{bojanczyk_automata_2025}. The underlying model is (one-way) nondeterministic automata with output. This is like an \textsc{nfa}, except that there are extra labels for output strings, as  explained in the following picture: 
\mypic{5}
The output of a run is obtained by concatenating the labels, similarly to how it is done in a weighted automaton from Definition~\ref{def:weighted-automaton-nondeterministic}. (This similarity will be revisited in a moment.) The semantics of the automaton is a binary relation that maps an input string to all possible outputs of accepting runs. Any string-to-string relation that can be obtained this way is called a \emph{rational string-to-string relation}, see~\cite[Chapter IX]{Eilenberg74}. If the relation is a function, i.e.~for each input string there is exactly one output string, then it is called a \emph{rational string-to-string function}. (We add the ``string-to-string'' qualifier to avoid confusion with the field  of rational functions, which consists of fractions of polynomials.)  

Below we show that: (1) the rational string-to-string functions are a strict subset of the regular ones; and (2) they are exactly the functions that are continuous over all semirings. Both results are straightforward and could be called folklore.

\begin{myexample}
    [Rational is weaker than regular]
    For the separation between rational and regular, a short argument is given in \cite[p. 218]{engelfrietMSODefinableString2001}. The duplicating function $w \mapsto ww$ is regular. It cannot be rational however, since its image is not a regular language, while rational relations have regular images~\cite[Theorem IX.3.1]{Eilenberg74}.
\end{myexample}





\begin{theorem}\label{thm:rational-functions}
    A string-to-string function is rational if and only if it is continuous over all semirings.
\end{theorem}
\begin{proof}
    The left-to-right implication is proved using a simple product construction, which is particularly easy if we use the nondeterministic presentation of weighted automata from Definition~\ref{def:weighted-automaton-nondeterministic}. The right-to-left implication is proved by viewing a rational string-to-string functions as a special case of weighted automata, for suitable semiring. The semiring is 
    \begin{align*}
    \domain = \pfin(\Gamma^*),
    \end{align*}
    i.e.~finite sets of output strings, with the addition being union and multiplication being concatenation. Essentially by comparing the definitions, one sees that a string-to-string relation is regular if and only if it is computed by a weighted automaton over this semiring. This observation gives us the right-to-left implication in the theorem. Indeed, suppose that a string-to-string function $f : \Sigma^* \to \Gamma^*$ is continuous over all semirings. Consider the weighted automaton 
    \begin{align*}
    \iota : \Gamma^* \to \pfin(\Gamma^*),
    \end{align*}
    which maps a string to itself (as a singleton set). By continuity, we know that $f; \iota$ is computed by a weighted automaton over $\domain$. This means that the composition $f;\iota$ is a rational string-to-string relation; in particular the same is true for $f$. Since the composition is also functional, it follows that $f$ is a rational function. 
\end{proof}


\subsubsection{Commutative semirings}
We finish this section by discussing one more kind of continuity, namely continuity over all commutative semirings. We conjecture that this continuity is also the same as regularity, but we are able to prove only one implication.
\label{sec:commutative-semirings}
\begin{theorem}\label{thm:regular-continuous-commutative-semirings}
    If a string-to-string function is  regular, then it is weighted continuous over commutative semirings.
\end{theorem}
\begin{proof}
    Observe that we have already proved a weaker version of this theorem, namely for field continuity, in \cref{lem:postcomposition-weighted-automaton}. The proof for fields passed through protocols. Unfortunately, we cannot reuse that proof, since we do not know if protocols over a semiring output domain are equivalent to weighted automata. This is because we do not have a strong enough version of the Fliess Theorem for semirings, although some work in that direction has been done~\cite[Corollary 2.15]{daviaud25}. For this reason, we give a direct proof, which does not use protocols. 

        Let $\domain$ be a commutative semiring, and  consider  a composition 
    \[
    \begin{tikzcd}
    \Sigma^* 
    \ar[r,"f"]
    & 
    \Gamma^*
    \ar[r,"g"]
    &
    \domain
    \end{tikzcd}
    \]
    where the first function $f$ is computed by a two-way automaton, and the second function $g$ is computed by a weighted automaton over $\domain$. We need to show that the composition $f;g$ can be computed by a weighted automaton over $\domain$. To prove the lemma, we use a straightforward product construction. To describe the construction, we use \emph{weighted graphs}. Such a graph is a directed graph, where every directed edge has a weight, and every vertex has two weights: an initial and final weight. For an input string $w \in \Sigma^*$, consider the following weighted graph, which call the \emph{run graph}:
    \begin{itemize}
        \item \textbf{Vertices.} Vertices are triples of the form $(p,x,q)$ where $(p,x)$ is a configuration of the two-way automaton for $f$ and  $q$ is a state of the weighted automaton for $g$. Recall that a configuration of the two-way automaton consists of a state $p$, and a position  $x$  in the string obtained from $w$ by adding endmarkers $\vdash$ and $\dashv$ to both ends. 
        \item \textbf{Edges.} In the graph, there is an edge 
        \begin{align*}
        (p,x,q) \xrightarrow{a} (p',x',q')
        \end{align*}
        if the two-way automaton has a single transition which goes from configuration $(p,x)$ to configuration $(p',x')$.
        \item \textbf{Weights of edges.} Consider an edge as in the previous item. The weight of this edge is defined as follows. Let $u$ be the output string, possibly empty, which is produced by the two-way automaton in the transition that goes from $(p,x)$ to $(p',x')$. The weight of the edge is defined to be the  sum of weights of all runs in the weighted automaton that go from $q$ to $q'$ and have input string $u$. In the special case when $u$ is empty, this will mean that $a$ is the $1$ of the semiring, i.e.~the neutral element of multiplication, because there is a unique run over the input string, and this run has weight $1$. 
        \item \textbf{Initial weights of vertices.} The initial weight of a vertex $(p,x,q)$ is zero if $(p,x)$ is not the initial configuration of the two-way automaton, and otherwise it is the initial weight of  the state $q$.
        \item \textbf{Final weights of vertices.} The final weight of a vertex $(p,x,q)$ is zero if $(p,x)$ is not the final configuration of the two-way automaton, and otherwise it is the final weight of the state $q$.
    \end{itemize}
The weight of a path in this graph is defined to be the product of: (1) the initial weight of the first vertex; (2) the weights of all edges on the path; and (3) the final wight of the last vertex. It is  easy to see that for an input string $w \in \Sigma^*$, the output of the composition $f;g$ is the same as the sum of weights of all paths in the corresponding run graph.  (This sum finite and therefore well-defined, since the run graph has finitely many paths by virtue of being  acyclic. This is because the two-way automaton is not allowed to loop.) To complete the proof of the theorem, it is enough to show the following claim.
\begin{claim}
    There is a weighted automaton which inputs a string $w \in \Sigma^*$, and outputs the sum of weights of all paths in the corresponding run graph.
\end{claim}
\begin{proof}
    This claim is the place where we use commutativity of the semiring. Consider a run graph, as in the following picture (to avoid clutter, we only show the vertices, and not the weights):
    \mypic{3}
    Consider a path in the run graph. Since the run graph is necessarily acyclic, such a path is uniquely described by the set of edges that it uses, as in the   following picture:  
    \mypic{4}
    A run graph together with a distinguished path can be viewed as a string, which we call its \emph{string representation}. Each letter in the string representation describes a single column of the picture above. (The string representation includes the weights of the vertices and edges, which are omitted in the picture.)     The function which inputs a string from $\Sigma^*$ and returns the set of all string representations of paths in the corresponding run graph is easily seen to be a rational string-to-string relation. Let this relation be 
    \begin{align*}
    R \subseteq \Sigma^* \times \Delta^*,
    \end{align*}
    where $\Delta$ is the alphabet used for string representation of paths inside run graphs. The function in the present claim is the same as 
    \begin{align*}
    w \in \Sigma^* 
    \quad \mapsto \quad 
    \sum_{\substack{v \in \Delta^* \\ (w,v) \in R}} \text{weight of path described by $v$}.
    \end{align*} 
    Weighted automata are closed under sums as above~\cite[Lemma 8.12]{bojanczyk_automata_2025}, and therefore to complete the proof of the claim it is enough to show a weighted automaton for the function
    \begin{align*}
    v \in \Delta^* 
    \quad \mapsto \quad 
    \text{weight of path described by $v$}.
    \end{align*}
    This weighted automaton is trivial: it simply mutiplies the weights of all highlighted edges, together with the initial weight of the first vertex and the final weight of the last vertex. Here, commutativity of the semiring is crucial, since the multiplication will be done in a left-to-right fashion, which will typically be inconsistent with the order of vertices on the path. 
\end{proof}

\end{proof}

We finish this section by dicussing the relationship between two implications, both of which we conjecture to be true.
\begin{align}
\text{weighted continuous over fields}
& \iff
\text{regular}
\label{eq:weighted-continuous-fields-again}\\
    \text{weighted continuous over commutative semirings}
& \iff
\text{regular}
\label{eq:weighted-continuous-commutative-semirings-again}
\end{align}
Since the implications $\impliedby$ are known to be true by \cref{thm:regular-continuous-commutative-semirings}, the more difficult equivalence is the first one, which has a weaker condition on the left side. 
For all we know, only the harder  first equivalence has any bearing on protocols, since we have established continuity for protocols only in the field case, see \cref{lem:postcomposition-weighted-automaton}. Even though it might not be connected to protocols, the  easier second equivalence would still be interesting on its own, as a characterisation of the regular string-to-string functions.

\section{Infinite alphabets}
\label{sec:infinite-alphabets}


In this section, we present a variant of our model which deals with an input alphabet. This direction is rooted in the tradition of language theory for infinite alphabets, which dates back to the work of Kaminski and Francez~\cite{kaminskiFiniteMemoryAutomata1994}, and has been developed in many subsequent papers, see e.g.~the survey~\cite{bojanczykOrbitFiniteSetsTheir2017}. The general idea is that we have an infinite alphabet $\atoms$, and the languages that we care about refer only to equality between letters, as in the following examples
\begin{align}
\setbuild{ w \in \atoms^*}{the first letter is equal to the last letter}
\label{eq:first-last}
\\
\setbuild{ w \in \atoms^*}{some letter appears at least twice}
\label{eq:some-twice}
\end{align}
There are numerous models of automata for such languages, which typically involve some kind of finite memory and registers that store letters from $\atoms$. For example, the language in~\eqref{eq:first-last} is recognised by an automaton which loads the first letter into a register, and then toggles acceptance depending on comparison of the register with the current input letter. On the other hand, the language~\eqref{eq:some-twice} is recognised by an automaton which nondeterministically guesses a position, puts its letter into a register, and then waits for this letter to appear again. 

There are many models for infinite alphabets in the literature,  including deterministic and nondeterministic automata with registers~\cite[Section 2]{kaminskiFiniteMemoryAutomata1994}, two-way variants of these~\cite[Definition 2.1]{nevenFiniteStateMachines2004}, unambiguous register automata~\cite[Section 5]{colcombet2015unambiguity}, single-use register automata~\cite[Definition 3]{bojanczykstefanski2020}, alternating automata with registers~\cite[Section 2.5]{DBLP:journals/tocl/DemriL09}, data automata~\cite[Section 4.2]{bojanczykTwovariableLogicData2011}, or various kinds of regular expressions~\cite{regexpKaminskiTan2004,regexpLibkin2015,KleeneNominal2019}. Unfortunately, none of these models (and many others in the literature) are equivalent to each other, including the three kinds of regular expression. This is in contrast to the theory for finite alphabets, where all models are equivalent, and define the regular languages.
%  One attempt to put order into the chaotic zoo of automata models for infinite alphabets is the theory of orbit-finite sets~\cite{bojanczyk_slightly}, which builds on the ideas of nominal sets~\cite{PittsAM:nomsns}. 

In this section, we describe a version of our two-party protocols that uses infinite alphabets. There are two sources of motivation for this study, namely: (a) a search for a canonical model of regular languages for infinite alphabets; and (b) mathematical interest. Regarding the point (a), we hope that the ready adaptability of two-party protocols to various settings will help us  find a canonical model of regular languages for infinite alphabets. This seems to be at least partially successful, since there is evidence -- which we present in this section -- that the protocols are equivalent to of the known automata models, namely unambiguous register automata. If true, this equivalence would be unexpected, since  there does not seem to be any direct connection between the two models. Regarding point (b), one of the exciting features of our protocol model for infinite alphabets is that the interaction between the two parties becomes essential, and the protocol cannot be reduced to the one-round case as in \cref{lemma:one-round-reduction-general}.

\begin{figure}
    \begin{enumerate}
    \item deterministic register automata~\cite[Definition 3]{kaminskiFiniteMemoryAutomata1994}
    \item nondeterministic register automata~\cite[Definition 1]{kaminskiFiniteMemoryAutomata1994}
    \item nondeterministic register automata with guessing~\cite[Definition 2.7]{bojanczyk_slightly}
    \item weighted register automata over the two-element field~\cite[Definition 3.1]{orbitFiniteVectorTheoretics}
    \item two-way deterministic register automata~\cite[Definition 5]{kaminskiFiniteMemoryAutomata1994}
    \item two-way nondeterministic register automata~\cite[Definition 2.1]{nevenFiniteStateMachines2004}
    \item alternating register automata~\cite[p.~16:8]{lazicDemri09}
    \item alternating register automata with one register~\cite[p.~16:19]{lazicDemri09}
    \item unambiguous register automata~\cite[Section 5]{colcombet2015unambiguity}
    \item register automata with pebbles~\cite[Section 2.2]{nevenFiniteStateMachines2004}
    \item \label{it:single-use} single-use register automata~\cite[Definition 2]{bojanczykstefanski2020}
    \item data automata~\cite[Section 4.2]{bojanczykTwovariableLogicData2011}
    \item class automata~\cite[Section III]{bojanczykExtensionDataAutomata2010} 
    \item regular expressions~\cite[Definition 2]{regexpKaminskiTan2004}
    \item three other kinds of regular expressions~\cite[Sections 4, 5, 6]{regexpLibkin2015}
    \item yet another kind of regular expressions~\cite[Section 5]{KleeneNominal2019}
    \item monadic second-order logic with equality~\cite[Section 2.4]{nevenFiniteStateMachines2004}
\end{enumerate}
    \caption{A non-exhaustive list of models of automata for infinite alphabets. All models in the list are pairwise non-equivalent. In contrast, for finite alphabets, all models in this list are equivalent, and define exactly the regular languages. }
    \label{fig:automata-infinite-alphabets}
\end{figure}




% . Regarding the point (b), we note that the theory of automata for infinite alphabets is closely related to the theory of orbit-finite sets~\cite{bojanczyk_slightly}, which builds on the ideas of nominal sets~\cite{PittsAM:nomsns}. This theory has many interesting mathematical aspects, and we hope that our model will be useful in this context.



\subsubsection{Protocols for an infinite alphabet}
\label{sec:protocols-infinite-alphabet}
We now give a more detailed description of our model. The purpose of the protocol is to compute a language $L \subseteq \atoms^*$, where $\atoms$ is an infinite alphabet. The idea is that this language can only refer to equality between letters. Following the literature on nominal sets and orbit-finite sets, this idea is formalised by requiring that the language is invariant under permutations of the alphabet, as described in the following definition.

\begin{definition}[Equivariant language] \label{def:equivariant-language}
    A language $L \subseteq \atoms^*$ is called \emph{equivariant} if 
    \begin{align*}
    w \in L \quad \iff \quad \pi(w) \in L
    \end{align*}
    holds for every permutation $\pi$ of the alphabet $\atoms$.
\end{definition}

Examples of equivariant languages include the languages in~\eqref{eq:first-last} and~\eqref{eq:some-twice}. On the other hand, the language ``the first letter is a vowel'' or ``the letters are strictly increasing'' are not equivariant, since there is no such thing as a ``vowel'', or an ordering of the letters. The principle of equivariance will also be applied to protocols, as described below, by restricting the two parties to use only equivariant strategies.





The idea is to use the same kind of protocol as in Definition \ref{def:two-party-protocol-boolean}, except that apart from bits, the parties can also send letters  from the alphabet $\atoms$. Before giving a formal definition, let us consider some examples. 



Let us give a more formal definition of the protocol. The input alphabet is $\atoms$, while the allowed messages are from the set $\set{0,1} + \atoms$, i.e.~the disjoint union of the Booleans and the input alphabet.
There is a fixed number of rounds $k$. In round $i \in \set{1,\ldots,k}$, each of the two parties chooses a new message according to a strategy, which is a function of type
\begin{align*}
\myunderbrace{\atoms^*}{local \\ string} \times \myunderbrace{(\set{0,1} + \atoms)^{i-1}}{messages received \\ in previous rounds} 
\to
\myunderbrace{\set{0,1} + \atoms}{message sent \\ in this round}
\end{align*}
In the last round, Bob must send a bit, and this bit is the output of the protocol. A key restriction of the protocol is that the strategies of both parties must be equivariant. A strategy $\sigma$  is equivariant if for every permutation $\pi$ of the atoms and every round $i \in \set{1,\ldots,k}$, the following condition holds:  
\begin{align*}
\sigma(w, m_1, \ldots, m_{i-1}) = m_i 
\quad \Rightarrow \quad
\sigma(\pi(w), \pi(m_1), \ldots, \pi(m_{i-1})) = \pi(m_i).
\end{align*}
In the above, when $\pi$ is extended from atoms to messages in the natural way, by leaving bits unchanged. 

\begin{myexample}[Repetitions cannot be detected]
    Let us formalise the claim from Example \ref{ex:some-twice} that the language ``some letter appears at least twice'' cannot be computed by a protocol. Suppose, towards a contradiction, that there is a protocol with $k$ rounds that computes this language. Consider an input string which has $2k+2$ different letters, and such that Alice and Bob get $k+1$ letters each. In the execution of the protocol there are at most $k$ atoms which are sent as messages. In particular, there must be some atom $a$ that appears in Alice's part of the input string, but is not sent as a message, and similarly there must be some atom $b$ that appears in Bob's part of the input string, but is not sent as a message. Consider an atom permutation $\pi$ which swaps $a$ with $b$. If we apply this atom permutation to Alice's part of the string (but not Bob's), then the communication history will remain unchanged. In particular, the output of the protocol will be the same on both inputs. However, after applying this permutation, the input string has a repetition, unlike the original one. 
\end{myexample}


\subsubsection{Orbit-finite sets}
\label{sec:orbit-finite-sets}
In this section, we do a more systematic analysis of automata models for infinite alphabets, and their relationship to protocols. As an organising principle,  we use  the approach of orbit-finite sets~\cite{bojanczykOrbitFiniteSetsTheir2017}. In this approach, we define a new notion of finite sets, namely the orbit-finite set. Once this has been done, any model of computation can be lifted by using orbit-finite sets instead of finite sets. This allows us for a clean comparison of two settings: the classical setting of finite sets, and the lifted setting of orbit-finite sets. 

%  To simplify the exposition, instead of the fully general notion of orbit-finite sets, we use a special case, which is called \emph{polynomial orbit-finite sets}. This special case has a more concrete defnition, and is sufficient for our purposes. 
\begin{definition}[Orbit-finite sets] \label{def:orbit-finite-sets}
    An orbit-finite set\footnote{
This definition is weaker than the usual notion of orbit-finite sets~\cite[Section 5]{bojanczyk_slightly}; in fact it is the special case of the usual notion that is called \emph{polynomial orbit-finite sets} in~\cite[Section 1]{bojanczyk_slightly}.
The usual notion  allows for two extra features: (a) restricting to equivariant subsets (e.g.~one could limit $\atoms^2$ to pairs which are non-repeating); and (b)  symmetries (e.g.~one could identify pairs in $\atoms^2$ if they agree up to swapping of coordinates, thus yielding unordered pairs). In some cases, the extra features are desirable, in particular they establish a connection with set theory~\cite{blassDedekind2016} and  nominal sets~\cite[Section 5]{PittsAM:nomsns}. However, these extra features  do not play any role in the analysis of protocols, and so we use the  simpler polynomial version in Definition~\ref{def:orbit-finite-sets}. All results would continue to be true with the extra features.
} is any set of the form 
    \begin{align*}
    \atoms^{d_1} + \cdots + \atoms^{d_n},
    \end{align*}
    for some natural numbers $d_1,\ldots,d_n \in \set{0,1,\ldots}$. 
\end{definition}

The special case of $\atoms^0$ describes a set with exactly one element, namely the empty tuple. Therefore, orbit-finite sets generalise finite sets, since a  finite set with $n$ elements can be seen as the orbit-finite set which has $n$ disjoint copies of $\atoms^0$. We will only be interested in properties that are equivariant, i.e.~invariant under permutations of the atoms, in the following sense:
\begin{align*}
\myunderbrace{x \in X \iff \pi(x) \in X}{equivariant subset $X$ \\ of an orbit-finite set}
\qquad 
\myunderbrace{f(x) = y \iff f(\pi(x)) = \pi(y)}{equivariant function $f$ \\ between two orbit-finite sets}
\end{align*}
We can now discuss various orbit-finite models of computation, by generalising finite sets to orbit-finite ones, and requiring all subsets and relations to be equivariant. As a first example of this approach, we can revisit the definition of Boolean protocols from Section~\ref{sec:protocols-infinite-alphabet}, and define it in terms of orbit-finiteness.

\begin{definition}[Orbit-finite protocol]
    \label{def:orbit-finite-protocol}
  An orbit-finite Boolean two-party protocol  is defined in the same way as in Definition \ref{def:two-party-protocol-boolean}, except that:
  \begin{enumerate}
    \item the input alphabet $\Sigma$, and the message spaces $Q_A$ and $Q_B$ are orbit-finite; and 
    \item the strategies of both players and the output function are equivariant.
  \end{enumerate}
\end{definition}

Indeed, the description from Section~\ref{sec:protocols-infinite-alphabet} is  the special case of the above definition where the input alphabet is $\atoms$, and the message spaces are both equal to 
\begin{align*}
\myunderbrace{\atoms}{letter} + \myunderbrace{\atoms^0}{bit 0} + \myunderbrace{\atoms^0}{bit 1}.
\end{align*}
On the other hand, the special case is also equivalent to the general case, since an element of a general orbit-finite set can be sent using messages that are either bits or individual letters. Therefore, protocol definition from Definition~\ref{def:orbit-finite-protocol} has the same expressive power as the one that we have presented in Section~\ref{sec:protocols-infinite-alphabet}. From now on, when talking about protocols, we will use the formalisation from Definition~\ref{def:orbit-finite-protocol}.

Orbit-finiteness can also be used to define automata. The following definition has the same expressive power as the standard (nondeterministic and deterministic) register automata for infinite alphabets from~\cite{kaminskiFiniteMemoryAutomata1994}; this equivalence was shown in~\cite[Lemma 6.3]{bojanczykAutomataTheoryNominal2014} and is in fact one of the original motivations for studying orbit-finiteness.

\begin{definition}
    [Orbit-finite automata]
    \label{def:orbit-finite-automata}
    A nondeterministic orbit-finite automaton is defined in the same way as a nondeterministic finite automaton, except that all sets are orbit-finite, and all subsets and functions are equivariant: 
\begin{align*}
    \myoverbrace{
        \myunderbrace{Q}{states} \quad 
        \myunderbrace{\Sigma}{input \\ alphabet}
    }
    {orbit-finite}
    \qquad
    \myoverbrace{
        \myunderbrace{I \subseteq Q}{initial \\ states} \quad 
        \myunderbrace{F \subseteq Q}{final \\ states} \quad 
        \myunderbrace{\Delta \subseteq Q \times \Sigma \times Q}{transitions}
    }{equivariant}.
\end{align*}
A deterministic orbit-finite automaton is the special case which has exactly one initial state, and where the transition relation is a function.
\end{definition}
 


As we mentioned earlier in this section, deterministic and nondeterministic models from the above definition have different expressive power. Therefore, if we want to ask if protocols are equivalent to automata, then we have not one question, but at least two. As we show in the following examples, both questions have negative answers.



\begin{myexample}
    [Deterministic too weak]\label{ex:protocol-not-dofa}
     Every determinstic orbit-finite automaton can be simulated by a protocol, in the same way as for finite alphabets. Alice sends her state to Bob, which will typically require sending some atoms. Bob then uses this state to continue the computation and report the answer. 
    
    However, the inclusion is strict: there protocols are strictly more powerful than deterministic orbit-finite automata. One reason is that protocols are symmetric, i.e.~there is no difference between left-to-right and right-to-left. On the other hand, deterministic orbit-finite automata are not symmetric. For example, the language ``the first letter appears twice'' can be recognised by a deterministic orbit-finite automaton, but its reverse ``the last letter appears at least twice'' cannot. The latter language can be computed by a protocol, and hence witnesses that the two models are different. 
\end{myexample}

\begin{myexample}
    [Nondeterministic too strong] \label{ex:protocol-not-nofa}  
    The language ``some letter appears at least twice'', from Example \ref{ex:some-twice}, can be recognised by a nondeterministic orbit-finite automaton, but cannot be computed by a protocol. We do not know yet if nondeterministic orbit-finite can simulate all protocols, and therefore for all we know the two models could be incmparable. However, as we will explain later in this section, we conjecture that there is indeed an inclusion, since we believe that protocols are equivalent to the special case of nondeterministic orbit-finite automata which are unambiguous.
\end{myexample}

In Definition~\ref{def:orbit-finite-automata}, we have defined one-way orbit-finite automata, which read the input string from left to right.
Another candidate automaton model could be two-way orbit-finite automata. This idea sounds natural, due to the interactive two-way nature of communication in the protocol. However, this model is too strong as well,  already in the deterministic case, as explained in the following example. 

\begin{myexample}[Two-way too strong]\label{ex:protocol-not-2dofa}
    The language ``some letter appears at least twice'' can also be recognised by a deterministic two-way orbit-finite automaton~\cite[Example 18]{bojanczyk_slightly}. Therefore, this automaton model cannot be simulated by protocols. Intuitively speaking, the issue is that in the orbit-finite case, a two-way automaton can make an unbounded number of visits to any given position, and therefore the constant number of messages in a protocol is insufficient to trace the execution of the automaton.  (For example, the automaton that checks if some letter appears at least twice will have a run of quadratic length, which visits the last position a linear number of times.) One idea to resolve this discrepancy would be to consider a variant of the automaton which has bounded crossing, see~\cite[p.~92]{neven2003power}, i.e.~there is some fix bound $k$ on the number of times that the automaton can visit any position. For all we know, this model could be equivalent to protocols.
\end{myexample}

\subsection{Unambiguous orbit-finite automata}
\label{sec:unambiguous-orbit-finite-automata}
As we have shown in Examples~\ref{ex:protocol-not-dofa}, \ref{ex:protocol-not-nofa} and \ref{ex:protocol-not-2dofa}, protocols are not equivalent to one-way deterministic or nondeterministic orbit-finite automata, or their two-way variants automata. So what is the right automaton model?  We conjecture that the answer is  \emph{unambiguous orbit-finite automata}, i.e.~the special case of nondeterministic orbit-finite automata that have zero or one accepting runs for every input string.

\begin{conjecture}
    \label{conj:protocols-unambiguous} A language over an orbit-finite alphabet is computed by an orbit-finite protocol if and only if it is recognised by an unambiguous orbit-finite automaton.
\end{conjecture}

One corollary of this conjecture would be that unambiguous orbit-finite automata are closed under complement, since protocols can be complemented by flipping the output bit. This corollary has been conjectured in~\cite[p.9]{colcombet2012forms}, and is open to the best of our knowledge, despite claims to the contrary~\cite[Footnote 5]{colcombet2015unambiguity}.

In this section, we prove impliccation $\impliedby$ in the conjecture, i.e.~we show that orbit-finite protocols can simulate  unambiguous orbit-finite automata. Contrary to similar results earlier in this paper, the simulation is non-trivial. Also, despite the one-way nature of the automata, the simulation crucially uses  the interactive nature of protocols, i.e.~there is more than one round of communication. In particular, the simulation cannot be done by a one-round protocol, as was the case for finite sets, see \cref{lemma:one-round-reduction-general}.

\begin{theorem}
    \label{thm:unambiguous-to-protocol}
    If a language $L$ over an orbit-finite alphabet is recognised by an unambiguous orbit-finite automaton, then it is computed by an orbit-finite protocol.
\end{theorem}
\begin{proof}
Fix for the rest of this proof an unambiguous orbit-finite automaton, whose state space is the orbit-finite set $Q$.
Suppose that the input string is factorized as $w = w_1 w_2$. The idea is that Alice and Bob will exchange a constant number of messages, which will allow them to determine the  intermediate state, i.e.~the state $q$ which satisfies 
\begin{align*}
\myunderbrace{I \xrightarrow{w_1}q}{there is a run over $w_1$\\ from an initial state to $q$} \qquad \text{and} \qquad
\myunderbrace{q \xrightarrow{w_2} F}{there is a run over $w_2$\\ from $q$ to a final state.}
\end{align*}
By unambiguity, there is at most one intermediate state, and it exists if and only if the string is accepted. 
To determine this state, Alice and Bob will narrow down the set of possible candidates, by storing a list of possible orbits, as described in the following definition. 


\begin{definition}[Orbit] \label{def:orbit}
    For a finite set $S \subseteq \atoms$, the orbit of $q \in Q$ with support $S$ is
    \begin{align*}
    \setbuild{ \pi(q)}{$\pi$ is a permutation of $\atoms$ such that $\pi(a)=a$ for all $a \in S$}.
    \end{align*}
\end{definition}


 
\begin{myexample}\label{ex:tau-disjoint}
    Let $Q = \atoms^5$ and consider the orbit of
    \begin{center}
        (\red{John}, Tom, Mary, Tom, \red{Eve})
    \end{center}
    which has support $\set{\text{\red{John}, \red{Eve}}}$. We put the support in red to underline its role. An element of this orbit is any tuple of the form 
    \begin{center}
        (\red{John}, $a$, $b$, $a$, \red{Eve})
    \end{center}
    where $a$ and $b$ are distinct atoms, which are not \red{John} or \red{Eve}. 
    % Two elements of this orbit are $\tau$-disjoint if the corresponding choices $\set{a_1,b_1}$ and $\set{a_2,b_2}$ are disjoint sets. 
\end{myexample}

 As the support increases, the orbit becomes smaller; in particular the biggest orbits are the ones with empty support, i.e.~the equivariant orbits. It is not hard to see that every orbit-finite set has a finite number of equivariant orbits~\cite[Lemma 1.4]{bojanczyk_slightly}; in fact this is the reason for the name.  
Each orbit in an orbit-finite set is a subset of $\atoms^d$ for some $d$. In such an orbit, we partition the coordinates $\set{1,\ldots,d}$ into two parts: the \emph{fixed coordinates}, which used the atoms from the support, and the \emph{free coordinates}, which do not use these atoms. In Example~\ref{ex:tau-disjoint}, the fixed coordinates are  the first and last ones, while the free coordinates are the middle three. The \emph{dimension} of an orbit is the number of distinct atoms in the free coordinates. In Example~\ref{ex:tau-disjoint}, the dimension is two, corresponding to the atoms $a$ and $b$. 
An important special case is then the dimension is zero; in this case the orbit contains only one state.


In the protocol, the two parties will share list of orbits, such that the intermediate state -- if it exists -- must belong to one of these orbits.  Initially, this list consists of the finitely many equivariant orbits in the state space. They will then exchange a constant number of messages, in order to decrease the dimension of these orbits, until the dimension becomes zero for all  the orbits. At this point, there will be  a finite list of candidates for the intermediate state, which has constant length and is known to both parties. Then, Alice and Bob can exchange a constant number of messages to determine which of these candidates is the actual intermediate state, if any.  
 
To decrease the dimension, we will use the following lemma.


    \begin{lemma}\label{lem:fixed-atoms}
        Let  $\varphi \subseteq Q$ be an orbit.  Consider   input string  $w = w_1 w_2$, and the sets
        \begin{align*}
        X_1 = \setbuild{ q \in \varphi}{$ I \xrightarrow{w_1} q$}
        \qquad
        X_2 = \setbuild{ q \in \varphi}{$ q \xrightarrow{w_2} F$}.
        \end{align*}
        There is a set $S \subseteq \atoms$, whose size is at most the dimension of the orbit $\varphi$, such that either: 
\begin{enumerate}
    \item   every state from $X_1$ uses some atom from $S$ on some free coordinate; or 
    \item   every state from $X_2$ uses some atom from $S$ on some free coordinate.
\end{enumerate}
    \end{lemma}
    \begin{proof}
        In the proof of the lemma, we use an analysis of disjointness, which is inspired by the sunflower lemma.
 We say that two states $p,q$ in this orbit are $\varphi$-disjoint if, after removing the fixed coordinates of $\varphi$, there is no atom that appears in both states. For example, if we take the orbit from Example~\ref{ex:tau-disjoint}, then the two states
\begin{center}
    (\red{John}, Tom, Mary, Tom, \red{Eve}) \qquad
    (\red{John}, Ann, Timmy, Ann, \red{Eve})
\end{center}
are $\varphi$-disjoint, because the sets $\set{\text{Tom, Mary}}$ and $\set{\text{Ann, Timmy}}$ are disjoint. In other words, the atoms from the  red coordinates can repeat (in fact, they must), but the atoms from the  black coordinates must be disjoint in the two states. 

The following straightforward claim shows that the only obstruction for having two disjoint tuples is having a constant number of shared atoms.
        \begin{claim}\label{claim:sunflower}
            Let $\varphi$ be a type of dimension $d$. If a set $X \subseteq \varphi$ does not contain two $\varphi$-disjoint elements, then there is a set $S$ of at most $d$ atoms such that  every element of $X$ uses at least one of these atoms on a free coordinate.
        \end{claim}
        \begin{proof}
            Take some element $x \in X$. Either there is an element  of $X$ that is completely disjoint with $x$, or otherwise some atom from $x$ must appear in every other element of $X$ on a free coordinate.
        \end{proof}

        In the light of the above claim, to prove the lemma it is enough to show that the assumption of the claim is satisfied by at least one of the sets $X_1$ or $X_2$. We prove this alternative by contradiction: if both sets $X_1$ and $X_2$ would violate the assumption of the claim, then the automaton would not be unambiguous. 
        Indeed, suppose that each of the two sets contains two $\varphi$-disjoint elements, say $p_1,p_2 \in X_1$ and $q_1,q_2 \in X_2$. The key observation is that the two pairs $(p_1,p_2)$ and $(q_1,q_2)$ would be  in the same equivariant orbit, i.e.~there would be some atom permutation $\pi$ which sends $p_1$ to $q_1$ and $p_2$ to $q_2$. Therefore, if we would  apply this atom permutation to the first part $w_1$ of the input string, then we would get some other input string $\pi(w_1) w_2$, such that  the same two states $q_1$ and $q_2$ are can be reached from both sides, contradicting  unambiguity.     \end{proof}


        Using the above lemma, we will construct a protocol that simulates the automaton. As explained before, the idea is to narrow down orbit which contains the intermediate state. This idea is formalised in the following lemma. 


\begin{lemma}\label{lem:narrow-down-orbit}
    Let $\varphi \subseteq Q$ be an orbit. Alice and Bob can exchange a constant number of messages -- which depends only on the dimension of the orbit --  to determine if the intermediate state belongs to $X$. 
\end{lemma}

Before proving the lemma, we clarify one issue in its statement:  we assume both Alice and Bob know the orbit $\varphi$, as a result of a previous exchange of messages. Having made this clarification, we can easily use the lemma to complete the proof of the theorem. Indded, at the beginning of the protocol, when no messages have been exchanged yet, the parties have no knowledge. However, we do know that the state space splits into a constant number of equivariant orbits. Therefore, the protocol from the lemma can be run for each of these orbits, which results in a constant number of rounds. It remains to prove the lemma. 


\begin{proof}[Proof of \cref{lem:narrow-down-orbit}]
Induction on the  {dimension} of the orbit.  
    
    The induction basis is when the  dimension is zero. In this case, the orbit has exactly one state, and Alice and Bob can simply check separately if the state is reachable on their side.

    Consider now the induction step. Apply \cref{lem:fixed-atoms}, to the orbit.  In the factorisation $w = w_1 w_2$, at least one of the two alternatives in the conclusion of \cref{lem:fixed-atoms} must hold. Alice can check if the first alternative holds, and Bob can check if the second alternative holds.  At least one of the two parties must report success, which is witnessed by some set $S$ of atoms. Let $T$ be the support of the orbit $\varphi$. The orbit $\varphi$ splits into finitely many orbits $\varphi_1,\ldots,\varphi_n$ with the larger support $S \cup T$, see~\cite[Lemma 10.9]{bojanczyk_slightly}. The number $n$ is bounded by a constant which is bounded by $S$ and the dimension of $\varphi$. We are only interested in the orbits among $\varphi_1,\ldots,\varphi_n$ which use at least one atom from $S$ on a coordinate that was free in $\varphi$. These orbits have lower dimension, and  we can apply the induction assumption to them. This completes the proof of the lemma, and therefore also of \cref{thm:unambiguous-to-protocol}.
\end{proof}
\end{proof}




\subsection{Weighted automata}
\label{sec:weighted-automata-atoms}

In \cref{thm:unambiguous-to-protocol}, we have shown one implication in Conjecture~\ref{conj:protocols-unambiguous}. This section is devoted to presenting some evidence for  the other implication, i.e.
\begin{align}\label{eq:missing-orbit-finite-implication}
\text{protocol} \quad \implies \quad \text{unambiguous automaton}.
\end{align}
 We begin by explaining why the techniques that we used to prove this implication in the finite case do not extend to  the orbit-finite case. 

\paragraph*{What goes wrong in the orbit-finite case?}
In the finite case, the proof had two parts: (a) a reduction to one-round protocols, and (b) the Myhill-Nerode Theorem. Part (b) does not seem to be problematic, as orbit-finite versions of the Myhill-Nerode Theorem are known in many variants, including monoids~\cite[Lemma 3.3]{bojanczykNominalMonoids2013}, automata~\cite[Section 3.2]{bojanczykAutomataTheoryNominal2014}, and -- as we will prove later in this section -- also for weighted automata. The problematic part is (a), in which the number of rounds is reduced to one. The key argument in this reduction  was that the sets of strategies 
  \begin{align*}
    (Q_B)^k \to (Q_A)^k \qquad \text{and} \qquad (Q_A)^k \to (Q_B)^k
    \end{align*}
are finite, and thus each party could simply send their strategy as a message.  This argument fails to carry over from finite sets to orbit-finite sets. The reason is  that orbit-finite sets are not closed under taking function spaces $X \to Y$, see~\cite{functionSpaces2024} for an extended discussion of this phenomenon.  The following example shows that the one-round reduction is indeed impossible in the orbit-finite case.

\begin{myexample}
    [No reduction to one round]\label{ex:no-one-round-reduction} Consider a language that is computed by an orbit-finite protocol with one round. Using the same argument as in \cref{lem:one-round-reduction-boolean}, we can show that the Myhill-Nerode equivalence relation for the language, as defined in~\eqref{eq:myhill-nerode-equivalence}, has an orbit-finite set of equivalence classes. As mentioned above, \cite[Section 3.2]{bojanczykAutomataTheoryNominal2014} can be used to conclude that the language is recognised by a deterministic orbit-finite automaton. As we have seen in Example~\ref{ex:protocol-not-dofa}, such automata are not strong enough to capture all protocols. The reasoning in this example shows that a language is recognised by a one-round orbit-finite protocol if and only if it is recognised by a deterministic orbit-finite automaton in both directions, i.e.~both the language and its reverse are recognised by deterministic orbit-finite automata. 
\end{myexample}

In light of the above example, it is no longer surprising that the proof of \cref{thm:unambiguous-to-protocol} used multi-round protocols. In fact, we believe that the number of needed rounds can be arbitrarily large, as suggested by the following example. 

\begin{myexample}[Back and forth]
    A string over the alphabet $\atoms^2$ can be seen as a directed graph, where each letter represents an edge. For $k \in \set{1,2,\ldots}$, define  $L_k$ to be the set of strings over this alphabet such that: (1) the string is functional, i.e.~for each atom $a$ there is at most letter in the string that begins with $a$; and (2) in the corresponding graph, there is a path with $k$ edges that uses the edges from the first and last letter. This language can be computed by an orbit-finite protocol with $k-1$ rounds, with each round corresponding to a step in the path. It seems unlikely that a smaller number of rounds would suffice, but we do not prove this claim here. 
\end{myexample}



\paragraph*{Weighted orbit-finite automata.} As explained above, the missing implication in Conjecture~\ref{conj:protocols-unambiguous} cannot be proved in the same way as in the finite case. The rest of this section is devoted to presenging some evidence for this missing implication conjecture, using an orbit-finite version of weighted automata. In \cref{thm:orbit-finite-protocol-to-weighted}, we will show  that  if a language is computed by a protocol, then it is recognised by a weighted orbit-finite automaton over the two-element field. For finite alphabets, this would be enough to ensure regularity, see \cref{claim:regular-weighted-automata}. This is no longer true in the orbit-finite case, and therefore \cref{thm:orbit-finite-protocol-to-weighted} can only be considered as evidence for the conjecture. However, at the very least it shows that languages computed by orbit-finite protocols are decidable, which was not a priori clear from the definition.

Let us begin by defining the orbit-finite version of weighted automata. 
\begin{definition}[Weighted orbit-finite automata]
    \label{def:weighted-orbit-finite-automata}
    A weighted orbit-finite automaton over a semiring $\domain$ is defined in the same way as in Definition~\ref{def:weighted-automaton-nondeterministic}, except that:
    \begin{enumerate}
        \item the input alphabet and state space are orbit-finite, instead of finite;
        \item the weight functions are equivariant.
    \end{enumerate}
     We require that for every input string, there are finitely many runs with non-zero weight.
\end{definition}

For the purpose of this section, already the special case of the two-element field $\set{0,1}$ is interesting. In this case, the automaton defines a function $\Sigma^* \to \set{0,1}$, which can be seen as the characteristic function of a language. Therefore, we can compare weighted orbit-finite automata to other models, such as nondeterministic orbit-finite automata. The following example shows that these two  models are incomparable. 

\begin{myexample}
    The language ``some letter appears twice'' is recognised by a nondeterministic orbit-finite automaton, but its characteristic function cannot be  recognised by a weighted orbit-finite automaton over the two-element field. The non-expressivity can be proved using the orbit-finite version of the Fliess Theorem, see \cref{thm:orbit-finite-fliess}. On the other hand, the  language ``an even number of distinct letters'' is not recognised by a nondeterministic orbit-finite automaton, while its characterisatic function can  be computed by a weighted orbit-finite automaton, see~\cite[Example 3.2]{orbitFiniteVectorTheoretics}. On the other hand, for every language recognised by an unambiguous orbit-finite automaton, its characteristic function is computed by a weighted orbit-finite automaton, since for unambiguous automata, counting the runs modulo two gives the same result as checking if a run exists. This discussion is summed up in the following picture:
    \mypic{2}
\end{myexample}

The following theorem is the main result of Section~\ref{sec:weighted-automata-atoms}.
\begin{theorem}\label{thm:orbit-finite-protocol-to-weighted}
    Let $\Sigma$ be an orbit-finite input alphabet, and let $\domain$ be a field.
    If a language $L \subseteq \Sigma^*$ is computed by a protcol, then the corresponding characteristic function of type $\Sigma^* \to \set{0,1} \subseteq \domain$  is computed by a weighted orbit-finite automaton.
\end{theorem}


In the proof of the theorem, we use the recently developped theory of orbit-finite vector spaces. In order to streamline the development, we will work with a special case of these spaces, namely spaces which have an orbit-finite basis.
% \begin{definition}
%     [Orbit-finite dimension] A \emph{vector space of orbit-finite dimension} is any vector space of the form $\lincomb X$, where $X$ is some orbit-finite set. 
% \end{definition}
For an orbit-finite set $Q$, let us write $\lincomb Q$ for the vector space which consists of finite formal linear combinations of elements of $Q$. In other words, an element of this space is a vector of the form 
\begin{align*}
\alpha_1 q_1 + \cdots + \alpha_n q_n,
\end{align*}
where the coefficients $\alpha_i$ are from the field, and the element $q_i$ (which can be seen as basis vectors) are from $Q$. Any space of the form $\lincomb Q$ is called a \emph{vector space of orbit-finite dimension}. Such a space has two kinds of structure: one can take linear combinations, and one can apply atom permutations. We will typically be interested in functions between such spaces that preserve both kinds of structure.

In the proof of \cref{thm:orbit-finite-protocol-to-weighted}, we will use an orbit-finite version of the protocols that were developped in Section~\ref{sec:field-domain}. We could use a general version of the protocol, corresponding to an orbit-finite generalisation of Definition~\ref{def:two-party-protocol-general} with field outputs, and prove that it is equivalent to weighted orbit-finite automata. This is indeed true.  However, in order to streamline the exposition, we treat these protocols as a tool to prove \cref{thm:orbit-finite-protocol-to-weighted}, and not an object of independent interest. Therefore, we only define the simplest kind of protocol that is needed for our purposes, namely an orbit-finite variant of the scalar product protocol from Definition~\ref{def:scalar-product-protocol}. It will be more convenient to use a generalisation of scalar products, namely bilinear maps. Recall that a bilinear map inputs two vectors, and is linear in both arguments. 

% This space is equipped with a scalar product, 
% \begin{align*}
% \langle v, w \rangle = \sum_{q \in Q} (\text{coefficient of $q$ in $v$}) \cdot ( \text{coefficient of $q$ in $w$}).
% \end{align*}
% The sum in the above definition is in fact finite, since only finitely many basis vectors will have nonzero coefficients. Using this scalar product, we can define an orbit-finite version of the scalar product protocols from Definition~\ref{def:scalar-product-protocol}.



\begin{definition}
    [Orbit-finite bilinear protocol] 
    \label{def:orbit-finite-scalar-product-protocol}
    An orbit-finite bilinear protocol is given by:
    \begin{enumerate}
        \item two vector spaces $V_A$ and $V_B$ of orbit-finite dimension  and two strategies, which are  equivariant functions
        \begin{align*}
        \sigma_A : \Sigma^* \to V_A 
        \quad \text{and} \quad
        \sigma_B : \Sigma^* \to V_B
        \end{align*}
        \item an output map, which is an equivariant bilinear map 
        \begin{align*}
        \text{out} : V_A \times V_B \to \domain.
        \end{align*}
    \end{enumerate}
\end{definition}
The output of the protocol is defined in the expected way. Alice and Bob apply their strategies to their local strings, yielding two vectors, and the output of the protocol is obtained using the output map. As usual, we require split invariance, i.e.~the output of the protocol should depend only on the input string $w$ and not on its factorisation $w = w_1 w_2$ into local strings. 

Before continuing, let us comment on the difference between scalar products and bilinear maps. Scalar protducts can be extended to vector spaces of orbit-finite dimension in the obvious way, namely
\begin{align*}
\langle v, w \rangle = \sum_{q} (\text{coefficient of $q$ in $v$}) \cdot ( \text{coefficient of $q$ in $w$}),
\end{align*}
where the sum ranges over basis vectors. This sum is in fact finite, since each of the vectors $v$ and $w$ involves only finitely many basis vectors with non-zero coefficients.
Scalar products are a special case of bilinear maps, and therefore every scalar product protocol is a special case of a bilinear protocol. In the context of finite dimension (and not orbit-finite) dimension, using bilinear maps instead of scalar products does not increase the expressive power of the protocols, since bilinear maps are still a special case of the general version from Definition~\ref{def:two-party-protocol-general}. Also in the orbit-finite setting, bilinear maps can be replaced by scalar products, but we do not know a direct proof of this result, other than showing that both models are equivalent to weighted orbit-finite automata, as we will show later in this section. 

Having defined the bilinear protocols, we resume the proof of \cref{thm:orbit-finite-protocol-to-weighted}. The proof has two steps, as described in the following diagram.
\[
\begin{tikzcd}
\text{orbit-finite protocol}
\ar[d,Rightarrow,"\text{Lemma~\ref{lem:orbit-finite-protocol-to-scalar}}"]
\\
\text{orbit-finite bilinear protocol}
\ar[d,Rightarrow, "\text{Lemma~\ref{lem:scalar-to-weighted}}"]
\\
\text{orbit-finite weighted automaton}
\end{tikzcd}
\]

We begin with the first step, which can be seen as form of reduction to one round, since the bilinear protocols have only one round. Recall that without vector spaces, a reduction to one round was not possible, see Example~\ref{ex:no-one-round-reduction}. This phenomenon is connected to closure under taking function spaces: orbit-finite sets are not closed under taking function spaces, but this closure is recovered once one moves to vector spaces, see~\cite[Section 8.3]{bojanczyk_slightly}. 

\begin{lemma}\label{lem:orbit-finite-protocol-to-scalar}
    If $L \subseteq \Sigma^*$ is computed by an orbit-finite protocol, then the characteristic function is computed by an orbit-finite scalar product protocol, over any field.
\end{lemma}


\begin{proof}
    We first introduce a common generalisation of the two models in the lemma. This genneralisation is called \emph{hybrid protocols}, and it has both multiple rounds (as in the protocols from the assumption of the lemma) and linear combinations (as in the protocols from the  conclusion of the lemma). We then show that the multiple rounds can be eliminated, yielding a protocol as in the conclusion of the lemma. 

    We begin by describing the hybrid protocols. The protocol has $k$ rounds.  In the first $k-1$ rounds, the two parties exchange messages  as in the orbit-finite protocol. Then, in the last round, they create two vectors, which are combined using a bilinear map to get the final number. 
    Here is a formal definition of the hybrid protocol. For each round $i \in \set{1,\ldots,k-1}$, the  parties exchange messages just as in an orbit-finite protocol, using message spaces $Q_A$ and $Q_B$ and strategies 
        \begin{align*}
        \sigma_{A,i} & : \Sigma^* \times (Q_B)^{i-1} \to Q_A\\
        \sigma_{B,i} & : \Sigma^* \times (Q_A)^{i-1} \to Q_B
        \end{align*}
    Then, in  the last $k$-th round, the message histories are used to produce vectors in two vector spaces $V_A$ and $V_B$ of orbit-finite dimension, using strategies 
    \begin{align*}
        \sigma_{A,k} & : \Sigma^* \times (Q_B)^{k-1} \to V_A\\
        \sigma_{B,k} & : \Sigma^* \times (Q_A)^{k-1} \to V_B.
        \end{align*}
    Finally, from the two vectors, the output is computed using a bilinear map
    \begin{align*}
        \text{out} : V_A \times V_B \to \domain.
    \end{align*}
    

    The hybrid protocol is meant to generalise both orbit-finite protocols and orbit-finite bilinear protocols. For the latter, this is clear: we simply use $k=1$ and there is no message exchange. For the former, we proceed as follows. We use  trivial vector spaces, i.e.~both vector spaces are the field. The bilinear map is multiplication. Once the two parties have agreed on a Boolean decision, they can both send $1$ (in the case of a ``yes'' decision) or $0$ (in the case of a ``no'' decision), and the bilinear map will give the correct output. 
    
    In order to complete the proof of the lemma, we will show that the number of rounds can always be reduced to one, thus yielding a bilinear protocol. 

    \begin{claim}\label{claim:reduce-round}
        For every $k > 1$, a hybrid protocol with $k$ rounds can be simulated by a hybrid protocol with $k-1$ rounds.
    \end{claim}
    \begin{proof}
        We will to eliminate round $k-1$, where the last message is sent. Once Alice has recieved the first $k-2$ messages from Bob, here contribution to the rest of the protocol is described by an object of type 
          \begin{align}\label{eq:contribution-last-two-rounds}
            \myunderbrace{Q_A}{message \\ sent in \\ round $k-1$} \quad \times \quad  \myunderbrace{(\fsfun  {Q_B} {V_A})}{message sent in \\ round $k$, as a function \\ of the message sent \\ in  round $k-1$}
        \end{align}
        We want to turn the above type into a vector space. The second coordinate is already a vector space, since functions with outputs in a vector space can be added and scaled pointwise. What is more, the  second coordinate  is  a vector space of orbit-finite dimension, which is a nontrivial result~\cite[Section 8.3]{bojanczyk_slightly}, i.e.~it has an orbit-finite basis  
        \begin{align*}
        F_A \subseteq \fsfun  {Q_B} {V_A}.  
        \end{align*}
        The first coordinate $Q_A$ can be turned into a vector space by  allowing linear combinations, i.e.~$\lincomb Q_A$. We combine these two using tensor product, yielding a vector space of orbit-finite dimension
        \begin{align*}
           W_A =  (\lincomb Q_A) \otimes (\lincomb F_A).
        \end{align*}
        We can do the same thing for Bob, thus giving a vector space 
        \begin{align*}
           W_B =  (\lincomb Q_B) \otimes (\lincomb F_B),
        \end{align*}
        where $F_B$ is an orbit-finite basis of the vector space $\fsfun  {Q_A} {V_B}$. Define a linear map
        \begin{align*}
        \varphi : W_A \otimes W_B \to \domain
        \end{align*}
        as follows. A basis of the input space for $\varphi$ is 
        \begin{align*}
        Q_A \times F_B \times Q_B \times F_A,
        \end{align*}
        and therefore it is enough to define the map $\varphi$ on the basis. (We want $\varphi$ to be equivariant, and therefore the definition on the basis will need to be equivariant as well.) This definition is the only one that types, namely 
        \begin{align*}
        (q_A, f_B, q_B, f_A) \quad 
        \mapsto \quad 
        \text{out}(f_A(q_B), f_B(q_A)).
        \end{align*}
        The map $\varphi$ is a linear map with domain $W_A \otimes W_B$, and therefore it is a bilinear map with domain $W_A \times W_B$. Because the output map is bilinear, one can check that $\varphi$ defined this way is consistent with the original protocol, i.e.~if we take functions 
        \begin{align*}
        f_A : Q_B \to V_A \qquad \text{and} \qquad f_B : Q_A \to V_B,
        \end{align*}
        which are not necessariy basis vectors from $F_A$ and $F_B$, then we have 
        \begin{align*}
        \text{out}(f_A(q_B), f_B(q_A)) = \varphi((q_A, f_B), (q_B, f_A)).
        \end{align*}
        Therefore, we can implement the last two round of the original hybrid protocol using a single round. The message spaces and the strategies for the first $k-2$ rounds are unchanged. In the last round $k-1$, the new strategies
        \begin{align*}
        \sigma'_{A,k-1} & : \Sigma^* \times (Q_B)^{k-2} \to W_A\\
        \sigma'_{B,k-1} & : \Sigma^* \times (Q_A)^{k-2} \to W_B
        \end{align*}
        output the tensor pairs consisting of the contribution that was described in~\eqref{eq:contribution-last-two-rounds}. Finally, the output  map for the new protocol is $\varphi$. 
    \end{proof}

    By repeatedly applying the above claim, we can reduce the number of rounds to one, in which case we get a bilinear protocol, as required in the statement of the lemma. 
\end{proof}

The second step in the proof of \cref{thm:orbit-finite-protocol-to-weighted} is to show that orbit-finite bilinear protocols are equivalent to weighted orbit-finite automata.  This will be proved similarly to the finite case, see \cref{sec:from-scalar-product-protocol-to-weighted-automaton}, by appealing to the Fliess Theorem. However, we first  need to prove the appropriate  orbit-finite generalisation  of this theorem.

\subsubsection{Fliess Theorem}
In this section, we prove an orbit-finite version of the Fliess Theorem, which characterises functions $\Sigma^* \to \domain$ that are computed by weighted orbit-finite automata. As in the original Fliess Theorem, we will be interested in derivatives of the function, which live in the space  
\begin{align*}
\Sigma^* \to \domain.
\end{align*}
This set has three kinds of structure, all of which will are used in the Fliess Theorem. The first kind of structure is that of a vector space, since we can take linear combinations of functions. The second kind of structure is that of left derivatives, i.e.~for each function $f$ and input string $w \in \Sigma^*$, we can consider the left derivative $v \mapsto f(wv)$. The third kind of structure is that of atom permutations: for each function $f$ and atom permutation $\pi$, we can consider the function $\pi(f)$, which is the composition $\pi;f$.

We say that a subset  $U \subseteq \Sigma^* \to \domain$  is \emph{orbit-finitely spanned} if there is some orbit-finite set $Q$, such that  every element of $U$ is a finite linear combination of elements from $Q$. We do not require the linear combination to be unique, i.e.~we do not require $Q$ to be a basis. (Choosing a basis can be problematic in the context of orbit-finite sets, see~\cite[Example 77]{bojanczyk_slightly}.) We are now ready to state the orbit-finite version of the Fliess Theorem.

\begin{theorem}[Orbit-finite Fliess Theorem]\label{thm:orbit-finite-fliess}
    The following conditions are equivalent for every  function
    \begin{align*}
    f : \Sigma^* \to \domain,
    \end{align*}
    where $\Sigma^*$ is an orbit-finite alphabet, and $\domain$ is a field: 
    \begin{enumerate}
        \item \label{it:fliess-weighted} $f$ is computed by a weighted orbit-finite automaton;
        \item \label{it:fliess-derivatives} $f$ is equivariant and its set of derivatives is orbit-finitely spanned.
        %  there is a finite set $\Gamma$ of derivatives of $f$, such that every derivative of $f$ can be expressed as linear combination
        % \begin{align*}
        % \alpha_1 \pi_1(f_1) + \cdots + \alpha_k \pi_k(f_k),
        % \end{align*}
        % where each $\alpha_i$ is in the field, each $\pi_i$ is an atom permutation, and each $f_i$ is in $\Gamma$.
    \end{enumerate}
\end{theorem}

Before proving this theorem, we use it to complete the proof of \cref{thm:orbit-finite-protocol-to-weighted}. In light of  \cref{lem:orbit-finite-protocol-to-scalar}, it is enough to show that if (the characteristic function of) a  language is computed by an orbit-finite bilinear protocol, then it has  an orbit-finite space of left derivatives. This is explained in the same way as in Section~\ref{sec:beyond-fields}: the vector produced by Alice in a bilinear protocol uniquely determines the left derivative of her part of the input. 

\begin{proof}[Proof of orbit-finite Fliess Theorem] In this proof, the orbit-finite generalisation is proved in the same way as the original theorem, without any significant changes.

    We begin with the implication \ref{it:fliess-weighted} $\implies$ \ref{it:fliess-derivatives}. Consider a weighted orbit-finite automaton with state space $Q$. 
    Define the \emph{pre-weight} of a run in the same way as its weight, except that we do not use the final weight. In other words, this is the product of: (1) the initial weight of the first state; and (2) the weights of all transitions. Consider an input string $w$. Define the \emph{configuration} of $w$ to be the linear combination
    \begin{align}
        \label{eq:configuration-wa}
        \sum_\rho \alpha_\rho \cdot q_\rho,
    \end{align}
    where $\rho$ ranges over runs that have input $w$ and non-zero pre-weight, $\alpha_\rho$ is the pre-weight of the run $\rho$, and $q_\rho$ is the last state in this run (this state is a string, since states are strings). By the assumption that each input string has finitely many runs with non-zero weight, the configuration is a finite sum, i.e.~it belongs to the vector space $\lincomb Q$. The left derivative which corresponds to the input string is uniquely determined by this configuration, and the space of configurations is orbit-finitely spanned. Hence, we get~\ref{it:fliess-derivatives}.

    We now prove the other implication, \ref{it:fliess-derivatives} $\implies$ \ref{it:fliess-weighted}. Assume~\ref{it:fliess-weighted}, which means that   there is an orbit-finite set $Q \subseteq \Sigma^*$ such that every derivative of $f$ can be decomposed as a finite linear combination of left derivatives
    \begin{align*}
            \sum_i \alpha_i \leftderivative{f}{w_i},
        \end{align*}
        where each string $w_i$ is in $Q$. The following claim discusses this decomposition for strings that are obtained by taking a string in $Q$ and appending one letter. 
        \begin{claim}
            There is an equivariant function 
        \begin{align*}
       \delta :  Q \times \Sigma \to \lincomb Q
        \end{align*}
        such that the following conditions holds for every $w \in Q$ and $a \in \Sigma$:
        \begin{align*}
        \delta(w,a) = 
        \sum_i \alpha_i w_i 
        \qquad \Rightarrow \qquad 
        \leftderivative f {wa} = \sum_i \alpha_i \leftderivative{f}{w_i}.
        \end{align*}
        \end{claim}
    \begin{proof}
        Condition~\ref{it:fliess-derivatives} in the theorem says that there is such a function, not necessarily equivariant. The content of the claim is to show that this function can be assumed to be equivariant. Indeed, we can start with any function $\delta$, which is not necessarily equivariant, and then improve it as follows:  for every orbit in $Q \times \Sigma$,  pick a representative $(w,a)$, apply the original function to it, and then extend the result to the whole orbit by equivariance. This improvement  procedure is legitimate thanks to the following equivalence: 
        \begin{align*}
                    \leftderivative{f}{wa} = \sum_i \alpha_i \leftderivative{f}{w_i}
                    \quad \iff \quad 
                            \leftderivative{f}{\pi(wa)} = \sum_i \alpha_i \leftderivative{f}{\pi(w_i)},
        \end{align*}
        The equivalence holds  because equivariance of $f$ ensures that  derivatives commute with atom permutations.
    \end{proof}        

    Using the function $\delta$ from the abvove claim, we define a weighted orbit-finite automaton. The state space is the set $Q$. (We assume without loss of generality that $Q$ contains the empty string $\varepsilon$. This is not really necessary for the construction, but it makes more intuitive.) The weights are defined as follows: 
    \begin{itemize}
        \item \textbf{Initial weights.} The initial weight of   $\varepsilon$ is $1$. All other states have initial weight $0$.
        \item \textbf{Transition weights.} The weight of a transition 
        \begin{align*}
        w \xrightarrow{a} v
        \end{align*}
    is the coefficient next $v$ in the linear decomposition $\delta(w,a)$.
        \item \textbf{Final weights.} The final weight of a state $w \in Q$ is $f(w)$.
    \end{itemize}
    In order for this to be a well-defined automaton, there need to be  finitely many runs with non-zero weight for every input string. This is because all linear combinations are finite, which ensures that for each state $w$ there are finitely many outgoing transitions over any input letter $a$ that have nonzero weight.

    Finally, we justify why this automaton computes the function $f$.   A simple inductive proof shows that  
    \begin{align*}
\text{configuration of $w$} = \sum_\rho \alpha_\rho \cdot w_\rho
\quad \implies \quad 
        \leftderivative f w = 
    \sum_\rho \alpha_\rho \cdot \leftderivative f {w_\rho}.
    \end{align*}
    By choice of final weights,  the output of the automaton is equal to $f(w)$. 
\end{proof}


\section{Conclusions}
\label{sec:conclusions}
One could possibly consider other inputs, such as trees. It seems that at least some of our results could generalise to such inputs, as long as there would be a suitable theory of automata for the inputs, with theorems in the style of Myhill-Nerode (which is the case for trees). Nevertheless, we leave the the exploration of non-string inputs to future work.



\bibliographystyle{alpha}
\bibliography{bib}



\end{document}
