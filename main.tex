\documentclass[11pt]{article}
\usepackage[letterpaper, margin=1.5in]{geometry}

\usepackage{macros}
\usepackage{stackengine}
\usepackage{centernot}


% macros specific to this paper
\newcommand{\domain}{\mathbb D}
\newcommand{\domainc}{\mathbb C}
\newcommand{\domaine}{\mathbb E}
\newcommand{\pol}{\text{Pol}}
\newcommand{\counter}[1]{\#{#1}}
\newcommand{\lincomb}{\operatorname{Lin}}

% various kinds of function spaces
\newcommand{\lineqfun}[2]{ #1 \underset{\text{lineq}}{\longrightarrow} #2}
\newcommand{\linfsfun}[2]{ #1 \underset{\text{linfs}}{\longrightarrow} #2}
\newcommand{\eqfun}[2]{ #1 \underset{\text{eq}}{\longrightarrow} #2}
\newcommand{\fsfun}[2]{ #1 \underset{\text{fs}}{\longrightarrow} #2}
\newcommand{\leftderivative}[2]{#1(#2\underline{\hspace{2mm}})}
\newcommand{\rightderivative}[2]{#1(\underline{\hspace{2mm}}#2)}
\newcommand{\twoderivative}[3]{#1 #2 #3}
\newcommand{\termop}{ \underset{\text{term}}{\longrightarrow}}
\begin{document}

\title{Two-party computation for functions with string inputs}
\author{Miko{\l}aj Boja\'nczyk, Aliaume Lopez, Rafa{\l} Stefa\'nski, Omid Yaghoubi }

\maketitle 
\begin{abstract}
 Inspired by communication complexity, we introduce a model of computation that defines functions of type $\Sigma^* \to \domain$, where $\Sigma$ is a finite alphabet and $\domain$ is some domain. Domains of interest include: the Booleans, strings, or a field. In the model, the input string $w$ is divided into two parts $w=w_1 w_2$, which are sent to two parties, Alice and Bob. These two parties cooperate, by exchanging a constant number of messages, to compute the output of the function. The two parties must be able to produce the correct output regardless of the partition $w = w_1 w_2$. We prove that for some domains, the model coincides with known finite state models: in the case of Boolean outputs it defines exactly the regular languages, and in the case of fields, it defines exactly the functions computable by weighted automata. This is despite the fact that the model is non-uniform and has no computational assumptions. 
\end{abstract}



% LTeX: language=en
\section{Introduction}
\label{sec:introduction}

\AP
This paper is motivated by a desire to understand the notion of  regularity in formal language theory. We take the functional perspective, in which we consider functions 
\begin{align*}
f : \Sigma^* \to \domain
\end{align*}
that input strings, and output values from some domain $\domain$. If the output
domain is the Booleans, then such functions are languages, and there is no
question about which languages should be considered \intro[regular language]{regular}. There are tens --
if not hundreds -- of equivalent definitions, including regular expressions,
finite automata in numerous forms, monoids, monadic second-order logic, and
variants of $\lambda$-calculus. But what about other outputs? Let us review
three examples where the nature of regularity is a topic of genuine
discussion.

\begin{enumerate}
    \item \textbf{String outputs.}
Consider string-to-string functions 
\begin{align*}
f : \Sigma^* \to \Gamma^*.
\end{align*}
Similarly to languages, the literature on automata theory offers countless  models. This time, however, not all of them are equivalent, but there is at least some semblance of order. Let us mention three classes of functions  of particular interest:  the \emph{rational}, \emph{regular}, and \emph{polyregular} functions. These classes are described in~\cref{fig:transducer-classes} in Section~\ref{sec:string-outputs} together with the appropriate references, with each one having at least five different characterisations, using models of varied origins, including logic, algebra and programming language theory. Which of these classes, if any, should be considered ``the'' regular string-to-string functions? We could simply go with the middle one, because  the word ``regular'' is traditionally used for it, but a more principled approach would be preferable.

\item \textbf{Number outputs.}
Consider string-to-number functions, say functions 
\begin{align*}
f : \Sigma^* \to \Rat
\end{align*}
that output rational numbers (more generally, the outputs could be in some field). Here, the literature offers two natural candidates, namely \emph{weighted automata}~\cite{schutzenberger1961definition}, and \emph{polynomial automata}~\cite[Section IV]{DBLP:conf/lics/BenediktDSW17}. In both cases, there is an automaton that reads the input string in one pass, and stores in its state a vector $\Rat^d$ of some fixed dimension. In weighted automata, this vector is updated using linear maps, while polynomial automata can use polynomial maps. These models are not equivalent -- polynomial automata are strictly more powerful -- but both have a good mathematical theory, one based on linear algebra, and the other based on algebraic geometry. Again, we might be tempted to ask: which  is the right one?

\item \textbf{Infinite input alphabets.}
Our final example is of a different kind than the previous ones. We return to functions with Boolean outputs
\begin{align*}
f : \Sigma^* \to \set{\text{true, false}},
\end{align*}
i.e.~languages, but this time the input alphabet is no longer required to be finite. The infinity needs to be somehow tamed, and the standard approach to do this is to use an infinite alphabet where letters can only be compared for equality~\cite{kaminskiFiniteMemoryAutomata1994}. This allows for languages such as ``all letters are different'' or ``the first letter is equal to the last one'', but not for languages such as ``the letters are in increasing order'', since the letters are not equipped with a linear order, or any other kind of structure.
The literature is rife with automata models for such languages, with seventeen examples listed in \cref{fig:automata-infinite-alphabets}, all describing pairwise non-equivalent models. 
 Again, we might be tempted to ask: which  is  the right one?
\end{enumerate}

This type of question can be asked in other settings, with   outputs such as  trees,  graphs or elements of some abstract semiring. One could also vary the inputs, and consider regularity for, say, graph-to-graph functions, but we refrain from considering general outputs in this paper, and we stay with string inputs. This paper attempts to provide a unified theory of regularity for such functions.   We are guided by  the following principle, which we believe to be essential for regularity:
\begin{description}
    \item[Constant information flow.]   If the input is split into parts, then only a constant amount of information  flows between them, as far as the output of the function is concerned.
\end{description}
This principle is only a vague guideline, since it does not identify what
``information flow'' means, or how to quantify its amount. For Boolean outputs,
i.e.~languages, constant information flow has a standard interpretation, which is the
Myhill-Nerode Theorem, and it is known to
correspond exactly to the regular languages. However, in the case of more
complicated outputs, things are less clear.  For example, if the outputs of the
function are rational numbers, then it should be legitimate for the information
flow to contain some rational numbers. How should this be formalised?

\AP
We choose to use a formalisation that is based on communication complexity~\cite{YAO79,KUSH97}. In our model, the function is computed by two cooperating parties, called Alice and Bob. 
There are \textbf{no uniformity assumptions} and the two parties
have unrestricted computational power; the goal is to measure information 
% \omc{information flow instead of information ?}
% it is a bit of a wordplay on the name of the journal Information and Computation :)
and
not computation.  The input string is split into two parts $w = w_1 w_2$, and
Alice has access to $w_1$ while Bob has access to $w_2$. The two parties  exchange  a
\textbf{constant number of messages} in order to compute the function, where the constant  depends only on the
protocol and not on the input string. The  output of the computation must
be \textbf{split invariant}, which means that the output depends only on the
input string $w$, and not on the split $w = w_1 w_2$. In the communication,
there are two kinds of messages:  either bits in $\set{\text{true, false}}$  or
elements of the output domain (as far as we can tell, the messages from the domain are the novelty of our approach).
% \omc{This may be my subjective view (as I mentioned before), but I don’t think it’s a good idea to describe the messages as bits in our paper. I understand that you did this because it’s the convention in the communication complexity community. However, I would prefer saying that either the message comes from a finite set, or … The reason is that, for detail-oriented readers within the communication complexity community, in this setting the parties should either alternate turns when sending messages to each other, or there should be a "Next" function specifying the next communicator based on communication history.
% Using a finite set instead of bits allows parties to send information that might require several rounds rather than just one. I suppose you wrote it this way to appeal to readers already familiar with communication complexity and to make the paper more attractive to them. If that’s the reason, then it makes sense to keep it as is. Otherwise, I think it would be better to take a consistent approach and use a finite set for the visible part of the messages.}. 
% Thanks Omid! I understand your point. However, I did indeed want to start with the more standard terminology in the introduction; and later start using the terminology for our paper.
For example, if the outputs are rational
numbers, then the messages can contain rational numbers. However, there is a
restriction on the access to messages from the output domain, which is called
the \emph{\textbf{\intro{black box restriction}}}. This restriction (which will be formally
defined later in the paper) is intended to prevent tricks such as Alice sending
her input string to Bob encoded as a rational number. Intuitively speaking,
the  black box restriction says that the parties cannot read the messages from
the output domain, and instead they can only operate on them using predefined
operations. For example, in the case of numbers, the operations are addition
and multiplication.

The model, which we call \emph{protocols}, can be applied to any output domain,
and we study several examples in this paper. In the case of Boolean outputs, the model and its connection to finite automata have already been studied before~\cite[Theorem 5]{hauser1989}, however the results on other output domains are new up to our best knowledge.   As we discover, despite its
non-uniformity, the model  can only define well-behaved functions. In
particular, in all cases that we have studied  these functions are: (1)  always
computable, even in linear time; and (2) the output size is always at most
linear (with the size of a rational number measured by the number of bits
needed to represent it).  This seems to indicate that the split invariance,
together with a  constant number of messages under the black box restriction,
has unexpected computational consequences. In particular, in all cases that we
have studied, protocols coincide -- or are conjectured by us to coincide --
with existing automata models. Since automata are conceptually very different 
than protocols, we believe that those equivalences, summarized in the table below,
justify the protocol-based approach to regularity.

\begin{center}
    \begin{tabular}{ll}
    \textbf{Output} & \textbf{Automata Model} \\
    \hline
    \kl[Boolean domain]{Booleans} & Finite automata (\cref{thm:boolean-domain}) \\
    \kl[String domain]{Field}   & Weighted automata (\cref{thm:field-domain}) \\
    \kl[String domain]{Strings} & Two-way automata with output (\cref{conj:protocol-regular-string-to-string}) \\
    Boolean, but infinite input alphabet & Unambiguous automata (\cref{conj:protocols-unambiguous}) 
\end{tabular}
\end{center}

In the remainder of this introduction, we give a more detailed review of the four rows
in the table, including substantial evidence for the conjectured equivalences.
% One piece of evidence that is worth mentioning here is that
% we were able to prove the equivalence for string outputs in the special case
% of \emph{unary} output alphabets (\cref{thm:unary-string-to-string}).

\subsection{Protocols for Boolean outputs}
\label{sec:intro-boolean}

We begin by studying the protocol model for  functions
\begin{align*}
f : \Sigma^* \to \set{\text{true, false}},
\end{align*}
i.e.~languages. This case is not new, and it has already been studied in~\cite{hauser1989}. For Boolean outputs, the two parties only exchange bits, 
% \omc{I still think it’s better to say that the messages come from a finite set instead of bits.} 
and they need to decide if the input string is in the language or not.
Here is an example.

\begin{myexample}[Parity]
    \label{ex:three-letters}
Suppose that the language is ``the string has even length''. In a protocol for this language, Alice sends the parity for her part of the input (one bit), and Bob uses this bit to return the final answer.  
\end{myexample}

As mentioned before, the parties have unbounded computational power, which means that  their messages
can contain answers to potentially undecidable questions about their parts of the input.
Despite this, the protocols compute exactly the regular languages.
This was first shown in~\cite[Theorem 5]{hauser1989}, but we restate it here for completeness, and provide a self-contained proof later in the paper.

\begin{restatable}{theorem}{booleandomain}
 \label{thm:boolean-domain} A language $L \subseteq \Sigma^*$ is computed by a \kl[boolean protocol]{protocol} if and only if it is 
 \kl[regular language]{regular}.
\end{restatable}
One implication is immediate: every regular language can be computed by a protocol.  Alice can send to Bob the state of a finite automaton that recognises the language. The other implication is proved in two steps, see Section~\ref{sec:boolean-domain} for more details. In the first step, the protocol is reduced to a one-round non-interactive version, where each party independently sends a message with a constant number of bits, and the decision is then made based on these two messages. In the second step,  the Myhill-Nerode Theorem is used to show that the language must necessarily be regular. 

Theorem~\ref{thm:boolean-domain} is  simple  technically, and its main value for us lies in its role as an inspiration for other results, which use  other output domains  such as  fields or strings. 


% It is worth underlining that, unlike in communication complexity, we are not interested in the exact number of exchanged bits. For example, in the round reduction that is described in the previous paragraph, the number of exchanged bits increases exponentially. One could investigate the constant in more detail, e.g.~how it depends on some parameters of the language, such as the number of states in an automaton that recognises it. We do not pursue this direction in this paper.

Before moving to the other output domains, let us comment on other related work that connects communication complexity with automata in the case of Boolean outputs.  Much of this work is related to \emph{state complexity}, where one studies  the number of states needed for a given language in a given automaton model, and how this number is affected by operations on languages or changes in the model. See  Wikipedia~\cite{stateComplexityWiki} for a comprehensive summary with numerous references, or the recent paper~\cite{goosKiefer2022} which shows how to transfer lower bounds from communication complexity to state complexity of unambiguous automata. 
% \omc{Remove the Wikipedia source and keep only the Goos and Kiefer (2022) reference. I think the Wikipedia source is redundant.}
% MB: I think that the Wikipedia source contains a lot useful references, more so than the Good and Kiefer paper. I would keep it.
 The work on state complexity is mainly about the exact number of states, which is of secondary concern to us. For our purposes, a protocol that exchanges $k$ bits is no different from a protocol that exchanges $2^k$ bits. We only care that this bound should be finite and independent of the input.



\subsection{Field outputs}
\label{sec:intro-field}



After the Booleans, we turn to functions with outputs in a field. This is the first original contribution of this paper. For the sake of concreteness, let us  consider functions with outputs in the  field of rationals
\begin{align*}
f : \Sigma^* \to (\mathbb Q, +, \times).
\end{align*}
We adapt the  protocol model to  compute such functions. Similarly to the Boolean case, the parties exchange messages. However, this time there are two kinds of messages: bits (as in the case of  Boolean outputs), and  elements of the field.  The elements of the field can be added and multiplied.  Division is not allowed, and its role is discussed in Example~\ref{ex:division}. Using bit messages, we can still recognise all regular languages (more formally their characteristic functions).
However, messages from the field allow computing new functions.

\begin{myexample}[Length and exponential length]\label{ex:length}
    Consider the following two functions
\begin{align*}
\myunderbrace{w \mapsto |w|}{length} \qquad \text{and} \qquad \myunderbrace{w \mapsto 2^{|w|}}{exponential length}.
\end{align*}
To compute the length of the input string, Alice  sends the length of her part, and Bob adds this to his length, thus yielding the desired output. For the exponential length, we use a similar protocol, except that multiplication is used instead of addition. 
\end{myexample}


In the presence of an infinite message space, one needs to be careful about the design of the protocol. For example, Alice could try to send her part of the input string in a single message by encoding it as a rational. This would trivialise the model, enabling every function to be computed. To prevent such tricks, we use the \emph{black box restriction} which was discussed before\footnote{The black box postulate is related to  polymorphic parametricity from the theory of programming languages~\cite[Section 7]{reynolds1983types}, or to the recent algebraic group model in cryptography~\cite[Section 1.2]{fuchsbauer2018algebraic}. An important difference with the algebraic group model is that our model does not allow for equality tests, see Example~\ref{ex:equality-tests} for a discussion.}: the messages which use the output domain (in this case, rational numbers) cannot be read directly, but can only be acted on by the operations in the output domain (in this case addition and multiplication).
 If the output domain is finite, e.g.~it is a finite field, then the black box restriction is irrelevant (which is why it was not mentioned when talking about Boolean outputs). This is because one can use bits to sent elements of a finite output domain, and the bits are a preferrable communication channel, since they can be read directly and are not subject to the black box restriction. 

\paragraph*{Definition of the protocol model.} Since there might be some ambiguity as to what exactly is allowed in the protocol, we give a more formal  definition. There is a finite input alphabet $\Sigma$, and a constant number of rounds $k \in \set{1,2,\ldots}$.   Alice and Bob send messages in alternation, with Alice sending the first message\footnote{One could consider other patterns of communication. In fact, in Section~\ref{sec:beyond-boolean-outputs} we use a more symmetric variant where both parties move in parallel in each round. These variants  do not change the expressive power of the model, only the  number of rounds.}. The messages are from 
\begin{align*}
\myunderbrace{\set{\text{true, false}} + \Rat}{disjoint union of bits and rational numbers}.
\end{align*}
When choosing their message in the $i$-th round, the corresponding  party (Alice in odd rounds, Bob in even rounds) has access to their part of the input string, and the  bits from previous messages. The numbers  are rationals, and cannot influence the decision, as per the black box restriction.  The information available  in the $i$-th round is given by  the set 
\begin{align}\label{eq:strategy-input}
\myunderbrace{\Sigma^*}{part of the  input \\ string that is \\ known to the party}
\qquad \times \qquad 
\myunderbrace{\set{\text{true, false, unknown}}^{i-1}}{messages received in  previous rounds, \\ with  numbers from $\Rat$ replaced by ``unknown''}.
\end{align}
Based on this information, the corresponding party chooses a new message to  send, which is either a bit, or a number. The number can be produced in two ways: either  a  fresh number  is produced based on the available information, or otherwise two previously received numbers are combined using addition or multiplication.  Therefore, the possibilities for the message sent in the $i$-th round are described by the set 
\begin{align}
    \label{eq:strategy-output}
\myunderbrace{\set{\text{true, false}}}{bits}
\ + \ 
\myunderbrace{\Rat}{fresh \\ number}
\  + \ 
\myunderbrace{\setbuild{(op,x,y)}{$op \in \set{+,\times}$ and $x,y \in \set{1,\ldots,i-1}$}}{addition or multiplication of previously received numbers}.
\end{align}
If addition or multiplication is used, then the party sending the message is responsible for the operation to be well-defined, i.e.~the messages sent in rounds $x$ and $y$ must have been numbers.
Summing up, the strategy in round $i$ is a function which inputs an element of the set from \eqref{eq:strategy-input}, and outputs an element of the set from \eqref{eq:strategy-output}. This function need not be computable. The protocol is then described by $k$ such strategies, one for each round $i \in \set{1,\ldots,k}$. We assume that the last message, sent in the $k$-th round, is a number and not a bit -- this number is defined to be the output of the protocol. 
Finally, the protocol must be \kl{split invariant}, i.e.~for every input string $w$, the same output must be produced regardless of the factorization $w = w_1 w_2$ into strings owned by Bob and Alice. This completes the definition of the protocol model, in the case of field outputs.


\paragraph*{Equivalence with weighted automata.} Our main result for field outputs is~\cref{thm:field-domain} below, which says that protocols are equivalent to weighted automata. The precise definition of weighted automata will be given later in Section~\ref{sec:field-domain}, but the rough idea is that a weighted automaton maintains a vector  $\domain^d$ of field elements, with each input letter updating the vector via some fixed linear map.  


\begin{restatable}{theorem}{fielddomain}
    \label{thm:field-domain}
     Assume that the domain $\domain$ is a field. Then a function 
    \begin{align*}
    f : \Sigma^* \to \domain
    \end{align*}  is computed by a \kl{protocol} if and only if it is  computed by a \kl{weighted automaton} over the same field.
\end{restatable}



  This result might even seem surprising. Our protocol is designed to use polynomial operations on the output domain, and therefore one could expect the relevant automaton model to be also based on polynomials, such as the  polynomial automata of~\cite{DBLP:conf/lics/BenediktDSW17}, which are the extension of weighted automata that allow polynomial maps instead of linear ones. As it turns out, the split invariance in the protocols  enforces linearity, and thus it excludes the general polynomials operations that are used in polynomial automata. The linearity phenomenon is true for outputs in a field -- because weighted automata are based on linear maps -- and it will also be true for other output domains, such as strings, see \cref{thm:evidence-for-the-conjecture}. We do not have a fully general understanding of this phenomenon.

  The proof of \cref{thm:field-domain} is given in Section~\ref{sec:proof-of-thm-field-domain}, but here we present a rough outline.  The proof is similar to the one used for the case of Boolean outputs, and has  two steps:

\begin{enumerate}
  \item We first show in Section~\ref{sec:reduction-to-scalar-product-protocols} that any protocol with outputs in a field can be reduced to a special  form, which we call a \emph{\kl{scalar product protocol}}. In this protocol, Alice and Bob apply in parallel two functions 
\begin{align*}
\sigma_A, \sigma_B : \Sigma^* \to \Rat^d
\end{align*}
to their parts of the input string, where $d$ is some fixed finite dimension. Then, the output is obtained by combining these two  $d$-dimensional vectors using scalar product. A scalar product protocol can be simulated by the general version of the protocol, but it is subject to certain restrictions: (a) there is no interaction; and (b) bits are not used, only  field elements. Since, as we prove, every product can be reduced to this scalar form, it follows that interaction and bit messages are not needed in the protocol. In fact, the interaction can be removed for all output domains, but the removal of bits is specific to fields. 
\item After reducing to the scalar product protocols, the next step (see Section~\ref{sec:from-scalar-product-protocol-to-weighted-automaton}) is to apply a version of the Myhill-Nerode Theorem for weighted automata. This is called the  Fliess Theorem~\cite{fliess1974}, and it says that recognisability by a weighted automaton is equivalent to having finite rank for a certain matrix, which is called the Hankel matrix. Roughly speaking, the rows in the Hankel matrix, in the context of our protocols, describe strategies of Alice, and the columns describe strategies of Bob. Therefore, it is not hard to show that in a scalar product protocol that uses vectors of dimension $d$,  the Hankel matrix has rank at most $d$. This, together with the Fliess Theorem, shows that protocols are equivalent to weighted automata, completing the proof of \cref{thm:field-domain}.
\end{enumerate}

 




\begin{myexample}[Division]\label{ex:division} What if we added division to the operations?  Consider the function  $w \mapsto 1/(|w|+1)$. This function can easily be computed using a protocol with division. We now argue that it cannot be computed using addition and multiplication only, thus proving that division gives extra power. Since the function depends only on the length of the input, it can be seen as having type $\Nat \to \Rat$. In such a type, weighted automata are the same as linear recurrence sequences. The inverse function $1/(n+1)$ is not a  linear recurrence sequence, which can be shown using the exponential polynomial form~\cite[Theorem 2.1]{BerstelReutenauer08}. Summing up, the choice of operations is important; we use a field, but we only allow the ring operations. We do not know what happens if division is allowed.
\end{myexample}

\paragraph*{Related work.} \cref{thm:field-domain} can be seen as a machine independent characterisation of  weighted automata. This would not be the first such characterisation, e.g.~the Fliess Theorem itself can be seen as a machine independent characterisation. Other research related to the Fliess Theorem is the categorical approach to  minimisation of weighted automata from~\cite{colcombetPetrisan2017}. We think that the value of our approach is that it places weighted automata in a broader context, which is defined purely in terms of communication, and in a way that is applicable to  other output domains, such as strings that will be considered next. As far as we know, the only  work  which takes such a broad view is the cost register automata of~\cite[Section C]{alurDantoniDeshmukhYuan2013}, which are an automaton model that describes functions with outputs in an arbitrary output domain, similarly to our setting. However, unlike our model, cost register automata are defined in terms of a  finite state machine model, and as such they lack the abstract machine independent flavour of our approach.

\subsection{String outputs}
\label{sec:intro-strings}

Our third group of results concerns string-to-string functions
\begin{align*}
f : \Sigma^* \to \Gamma^*.
\end{align*}
We  use the same kind of protocol as in the previous section, except that instead of numbers, the black box messages  contain strings from $\Gamma^*$, and instead of addition and multiplication, we have string concatenation.  Here are some examples.

\begin{myexample}[Reversal and duplication]\label{ex:reverse-duplicate}
    Using the protocol, we can compute string reversal: Alice sends to Bob the reverse of her part of the input, and  Bob concatenates this with  his part of the reverse. Another string-to-string function that can be easily computed by a protocol is string duplication $w \mapsto ww$. 
\end{myexample}

The string-to-string case is of particular interest because, as we have mentioned earlier in the introduction, there is no consensus as to  which string-to-string functions should be considered ``regular''. There are numerous automata models to choose from, some of which are summarized in \cref{fig:transducer-classes}, which contains twenty models, grouped by equivalence into three classes. We can exclude the weakest class (the rational functions), since it is too weak: it  cannot compute the reverse or duplicate functions from Example~\ref{ex:reverse-duplicate}. We can also exclude the strongest class (the polyregular functions), since it is too strong: polyregular functions can have superlinear output size, and as we will see in a moment, protocols can only have linear output size. By a process of elimination, we are left with the final  class from \cref{fig:transducer-classes}, which is traditionally called the ``regular functions''. One of the definitions of the class is in terms of   deterministic two-way automata with output~\cite{shepherdson1959reduction}; this model  is formally defined in Section~\ref{sec:string-outputs}. We conjecture  that this class     is the correct answer (thus validating the traditional name): 

\begin{conjecture}\label{conj:protocol-regular-string-to-string}
    A string-to-string function is computed by a protocol iff it is computed by a deterministic two-way automaton with output.
\end{conjecture}

In Section~\ref{sec:string-outputs} we discuss this conjecture in detail, and provide evidence in its favour, including a proof of the  implication 
\begin{align*}
\text{protocol} \quad \impliedby \quad \text{two-way automaton},
\end{align*} 
This implication is rather easy, since a two-way automaton can be neatly simulated by the repeated interactions of the protocol. The content of the conjecture, and the subject of the more technical results, is the $\implies$ implication. Most of Section~\ref{sec:string-outputs} is devoted to evidence for this implication. 
Our first argument  is the following result, which shows that functions computed  by protocols have many properties which are known to hold for deterministic two-way automata with output.


\begin{restatable}{theorem}{evidencefortheconjecture}
    \label{thm:evidence-for-the-conjecture}
    If a string-to-string function  is  computed by a \kl{protocol}, then:
    \begin{enumerate}
        \item \label{it:linear-size-outputs} outputs have at most linear size;
        \item \label{it:linear-time-computable} outputs can be   computed in linear time (ignoring logarithmic factors);
        \item \label{it:regular-preimages} preimages of regular languages are regular.
    \end{enumerate}
\end{restatable}

One can invent functions which satisfy the three conditions in the above
theorem, but which are not computed by deterministic two-way automata with output, see
Example~\ref{ex:not-regular-but-continuous-over-finite-fields}. However, all
known examples of such functions  are artificial, and none can be computed by
protocols (or any known transducer models).  The proof of
\cref{thm:evidence-for-the-conjecture}, which is given in
Section~\ref{sec:string-outputs}, uses a linear representation of strings as
matrices, and then applies \cref{thm:field-domain} about protocols with field
outputs. In fact, this technique suggests an alternative approach to regularity,
which connects string-to-string functions with the better understood case of
string-to-field functions. This approach is discussed in
Section~\ref{sec:continuity}.

Our second argument in favour of the \cref{conj:regular-continuous} is that we
can prove it in the special case when the output alphabet is unary, i.e.~it has
only one letter. This is the content of
Section~\ref{sec:unary-output-alphabet}. When the output alphabet is unary, the
output strings are commutative, i.e.~the order of letters is irrelevant; this commutativity is essential to our proof of the conjecture in the unary output case.  The proof is based on well-quasi orders, plus some results on weighted automata.



\paragraph*{Related work.} One of the consequences of the Myhill-Nerode theorem
for string-to-Boolean functions, or the Fliess Theorem for string-to-number
functions, is the existence of  canonical devices. There have been several
attempts to generalise this to string-to-string functions, with a special
emphasis the canonical devices. Before recalling this work, we observe that our
approach seems to go in a  different direction. Although we think of
Conjecture~\ref{conj:protocol-regular-string-to-string} as being a machine
independent characterisation, it does not necessarily  yield canonical devices.
In particular, our proof of the conjecture for unary alphabets does not yield a
canonical device.

Here is a summary of results on canonical devices for string-to-string
transducers: they have been proposed for subsequential functions~\cite[Théorème
1.1]{choffrut1977}, rational functions in~\cite[Theorem
1]{reutenauerSchutzenberger1991},  and for rational functions on infinite
words~\cite[Section 4]{canonicalRational2018}. A certain drawback of this line
of work is that: (a) the canonical devices are relative to a given automaton
model, which does not help in choosing one model over another; and (b) the
``canonical'' devices are not truly unique, since they depend on extra
parameters, such as the output delay for subsequential functions or the
lookahead for rational ones. Let us now move to the larger class of regular
functions, which is the subject of our conjecture. Here,  canonical model are
unknown, and the only known way to recover them uses non-standard semantics,
called origin semantics~\cite[Theorem
1]{bojanczykTransducersOriginInformation2014}. Another result of this kind,
which is a machine independent characterisation of the regular functions that
does not yield canonical devices, see \cite[Theorem
3.2]{bojanczykTitoRegular23}, is also implicitly based on origin semantics.
Finally, for the polyregular functions, the situation is of course even harder,
and the only known results concern a unary output alphabet~\cite[Section
IV]{Zpolyreg23}.


\subsection{Infinite input alphabets}
\label{sec:intro-infinite}
Our final group of results is about languages over infinite input alphabets. This is a departure from the previous setting, where  the input alphabet was always finite. Following the standard approach in automata theory, we assume that letters can only be compared for equality. Formally, we only want to consider languages that are invariant under permutations of the
alphabet, i.e.
\begin{align*}
w \in L 
\quad \iff \quad
\pi(w) \in L \qquad \text{for every permutation $\pi$ of the alphabet}.
\end{align*}
An example of such a language is ``all letters are different'', but not ``the letters are in the increasing order''.
 As mentioned earlier in the introduction, there is a rich literature on automata for such languages, see the surveys~\cite{neven2003power,segoufin2006automata,bojanczykOrbitFiniteSetsTheir2017} or the lecture notes~\cite{bojanczyk_slightly}.  The  relevant automata models typically use registers to store some letters from the input, so that they can be compared to later letters. Essentially any automaton model for finite alphabets can be lifted this way to infinite alphabets~\cite[Figure 1]{neven2003power}, and there is even a systematic way to do this, which is based on the theory of orbit-finite sets~\cite[Chapter 2]{bojanczyk_slightly}. Unfortunately, after this lifting, previously equivalent models become non-equivalent.  This sad situation is illustrated in~\cref{fig:automata-infinite-alphabets}, which describes seventeen non-equivalent automata models for infinite alphabets; all of these models collapse to the regular languages when considered for finite alphabets. Equally sadly, there are almost no results in the literature on infinite alphabets that prove non-trivial equivalences of models. The only known cases of this kind are about the  weakest of the available models, namely orbit-finite monoids~\cite{bojanczykNominalMonoids2013}, which are known to be equivalent in expressive power to a certain variant of \mso~\cite[Theorems 4.2 and 5.1]{DBLP:journals/corr/ColcombetLP14}, and also to single-use register automata~\cite[Theorem 6]{bojanczykstefanski2020}.
This situation desperately calls for some unifying principles. 

Since protocols have successfully identified important models in the previous
cases, we try to see  what happens in the case of an infinite input alphabet.
When extending protocols to infinite input alphabets, we adapt them as follows:
(1) messages can contain input letters; (2)  input letters can only be compared for
equality. Condition (2) is formalised by saying that the execution of the
protocol is invariant under permutations of the input alphabet, similarly to
the languages that we consider. The protocols  work their magic once again, and
they point to (as we conjecture)  one of the models in the literature. Before
revealing this model, let us briefly compare protocols to the two most prominent models 
for infinite alphabets.

\begin{myexample}[Deterministic too weak]\label{ex:reg-det-too-weak} We begin by considering a popular deterministic automaton model for infinite alphabets, namely i.e. \emph{deterministic register automata} \cite[Definition~3]{kaminskiFiniteMemoryAutomata1994}. This model is defined formally in Section~\ref{sec:infinite-alphabets}, but roughly speaking it is nondeterministic a finite automaton that is additionally equipped  with a fixed number of registers, which can be used to store input letters and compare them for equality with later input letters.  For example, the automaton can store the first input letter in a register, and then compare it with all later letters, thus recognising the language ``the first letter appears elsewhere in the word''. 

    A protocol can simulate any deterministic register automaton, using the same idea as for finite alphabets,
    i.e. Alice sends the intermediate state, including the register values, to Bob, who 
    continues the run on his side. To see that the inclusion is strict, consider the language
    ``the last letter appears elsewhere in the word'', which is the reverse of the language described in the previous paragraph. It is computed by a protocol, 
    in which Bob checks the condition on his side, and sends the last letter to Alice, so that she can verify
    that it does not appear on her side. On the other hand, it is a well-know fact that this language
    is not recognised by a deterministic register automata, as it proves that deterministic register automata 
    are not closed under reverse \cite[Examples~4~and~8]{kaminskiFiniteMemoryAutomata1994}. 
\end{myexample}

% \begin{myexample}[Deterministic too weak]\label{ex:reg-det-too-weak}

%     Deterministic register automata \cite[Definition~3]{kaminskiFiniteMemoryAutomata1994} is 
%     an automaton model equipped with a finite state, and a fixed number of register used for
%     storing and comparing input letters. An example of a language that can be recognised by this 
%     model is ``the fist letter does not appear elsewhere in the string''. The automaton 
%     can save the first letter in its register compare it with every other input letter, remembering in 
%     its finite state if it has found a reappearance. However, the reverse of this language,
%     i.e. ``the last letter does not appear elsewhere in the string'' is no longer recognized 
%     by deterministic register automata: In order to compare the last letter with all
%     other letters that appear in the word, the automaton would have to store all those letters in its memory, which cannot be done 
%     using a fixed number of registers.

%     A protocol can simulate any deterministic register automaton, using the same idea as for finite alphabets: Alice computes the intermediate state and register values and sends them to Bob, who continues the run of the automaton on his side. In particular, in order to compute the language
%     ``the fist letter does not appear elsewhere in the string'' Alice can check the condition on her side, and then send the first letter to Bob,
%     so that he can check the condition on his side. However, the inclusion is strict.
%     In particular, as a symmetric model they can compute the reverse of the language above, i.e.
%     ``the last letter does not appear elsewhere in the string'' (in the protocol Alice takes the role of Bob and vice versa).
% \end{myexample}
    
% In the previous example, we excluded deterministic register automata as being too weak. We now exclude nondeterministic register automata as being too strong. 

\begin{myexample}[Nondeterministic too strong]
    \label{ex:reg-ndet-too-strong} Consider the nondeterministic variant of the automata from the previous paragraph~\cite[Definition~1]{kaminskiFiniteMemoryAutomata1994}. This model is already too strong for the protocols, as attested by the language
    ``some letter appears twice'', which can be computed by an automaton, see~\cite[Example~1]{kaminskiFiniteMemoryAutomata1994}, but not by  a protocol. The intuitive reason for the negative result is that if neither Alice or Bob sees a repetition in their parts of the input string, then they should exchange all their letters to check for cross-part repetitions, which cannot be done in a constant number of messages -- 
    a detailed proof is given in \cref{sec:infinite-alphabets}. 
\end{myexample}

Which automaton model, if any, corresponds to protocols? As explained in the previous two examples, deterministic register automata are too weak, while nondeterministic register automata are too strong.
We conjecture (in Conjecture~\ref{conj:protocols-unambiguous}), that the winner is a seemingly  unexpected candidate, namely unambiguous register automata~\cite[Section 5]{colcombet2015unambiguity}. This is the special case of  nondeterministic register automata, in which for every input string there is at most one accepting run.
We discuss this conjecture in \cref{sec:infinite-alphabets}, and provide evidence in its favour. We start with an actual proof of one implication, namely:
\begin{align*}
\text{protocol} \quad \impliedby \quad \text{unambiguous automaton}.
\end{align*}
Contrary to  previous variants of this implication, the proof is non-trivial -- the usual construction does not work, because the automaton is nondeterministic. One  interesting phenomenon is that, in the case of infinite input alphabets, the interactive multi-round nature of the protocols becomes essential, and protocols cannot be reduced to one round, as was the case for finite input alphabets. In our proof of the implication $\impliedby$, we design a protocol where  the two parties  progressively eliminate the uncertainty about letters used in the run of the automaton, until the unique accepting run is identified or its existence disproved.
The proof uses a variant of the sunflower lemma.

Therefore,  the content of the conjecture is -- as in previous cases --  the other implication, namely that protocols can be simulated by unambiguous register automata. We provide evidence for this implication, using the recently developed theory of orbit-finite vector spaces~\cite{orbitFiniteVectorTheoretics}. We show that every function computed by a protocol can be computed by a weighted automaton with registers. This is almost like an unmabiguous automaton, except that some runs might have negative weights, and the weights always cancel out to give a final result that is either $0$ or $1$. In particular, the functions computed by protocols are computable, which is not a priori clear from the model. Along the way, we develop some new theory, in particular an orbit-finite generalisation of the Fliess Theorem. 



\section{Boolean outputs}
\label{sec:boolean-domain}
In this section, we describe our model of computation for the simplest output domain, namely the Booleans. In this case, the model computes a function of type $\Sigma^* \to \set{\text{yes, no}}$, which is the same as a language. Before giving a more formal definition of the protocol, we give an informal description.

Suppose that we want to compute a language over some input alphabet $\Sigma$. The computation will be distributed across two parties, called Alice and Bob, who cooperate and have infinite computational power. An evil adversary chooses an input word $w \in \Sigma^*$, and a factorisation $w = w_1 w_2$ of this input into two consecutive parts. The first part is sent to Alice, and the second part is sent to Bob. Next Alice and Bob exchange a constant number of bits, where the constant depends only on the protocol, and not on the input word $w$ or its factorisation $w=w_1 w_2$. After this exchange, the two parties must decide if the input word $w$ belongs to the language. 

\begin{myexample}
Suppose that the language is 
\begin{align*}
\setbuild{ w \in \set{a,b}^*}{the first $k$ letters of $w$ are equal to the last $k$ letters of $w$}.
\end{align*}
To decide this language, the parties can exchange $\Oo(k)$ messages. For example, Alice can send the first $k$ letters of her part to Bob. Alternatively, Bob can send his first $k$ letters, and other variants are possible as well. At any rate, the number of exchanged messages is a function of $k$, which is part of the language and not of the input string. 
\end{myexample}


We now give a formal definition of the protocol. 

\begin{definition}
    \label{def:two-party-protocol-boolean}
  A Boolean two-party protocol 
   is given by the following ingredients: 
  \begin{enumerate}
    \item a finite input alphabet $\Sigma$;
    \item a number of rounds $k \in \set{1,2,\ldots}$;
    \item message spaces for Alice and Bob, which are finite sets $Q_A$ and $Q_B$;
    \item for each round $i \in \set{1,\ldots,k}$, two strategies
    \begin{align*}
    \myoverbrace{\sigma_A^i : \myunderbrace{\Sigma^*}{Alice's \\ local\\ string} \times \myunderbrace{Q_B^{i-1}}{message \\ history}  \to \myunderbrace{Q_A}{new\\ message}}{stategy for Alice in the $i$-th round}
    \qquad \text{and} \qquad 
        \myoverbrace{\sigma_B^i : \myunderbrace{\Sigma^*}{Bob's \\ local\\ string} \times \myunderbrace{Q_A^{i-1}}{message \\ history}  \to \myunderbrace{Q_B}{new\\ message}}{stategy for Bob in the $i$-th round};
    \end{align*}
    \item an output function of type $(Q_A \times Q_B)^{k} \to \set{\text{yes, no}}$.
  \end{enumerate}
\end{definition}

Given a pair of input strings $(w_1,w_2)$, which are called the \emph{local strings} of Alice and Bob, respectively, the  protocol is run as follows. There are $k$ rounds. In each round, both parties send messages, and therefore after $i$ rounds are played, the communication history contains $i$ messages sent by Alice and $i$ messages sent by Bob.  In round  $i \in \set{1,\ldots,k}$,  each of the two parties  looks at their local string and the $i-1$ messages sent by the other party in the previous rounds (only the messages sent by the other party are needed, since the party knows their own messages). Based on this information, each of the two parties uses their strategy to produce a new message. At the end of the protocol, the output function is used to determine the value of the function, based on all messages in  the communication history. We say that the protocol computes a language $L \subseteq \Sigma^*$ if for every two strings $w_1,w_2$, the output of the protocol tells us if the concatenation $w_1w_2$ belongs to the language. In particular, the answer given by the protocol must satisfy a property that we call \emph{split invariance}: the output for a pair $(w_1,w_2)$ depends only on the concatenation $w_1w_2$. 

Let us comment on two design choices in Definition~\ref{def:two-party-protocol-boolean}. The first design choice is that messages sent by the two parties belong to some fixed finite sets $Q_A$ and $Q_B$. Without changing the expressive power, one could simplify the protocol by assuming that both of these sets are the same, and they are $\set{0,1}$, which means that the parties simply exchange individual bits of information. However, the variant with sets $Q_A$ and $Q_B$ will be more amenable to generalisations and modifications, so we use this variant. The second design choice is the in each round, the two parties send their messages simultaneously, without seeing the message sent by the other party in the same round. We do this for the purpose of symmetry, but the protocol could be changed, without affecting its expressive power, so that the parties send messages in alternation.

An important part of the protocol is that there are no uniformity assumptions or computational restrictions. For example, the first message sent by Alice could contain an answer to some undecidable problem. However, as we will see, the split invariance restriction will make it impossible to use this information, since the protocol can only compute regular languages. 

\begin{theorem}
  A language $L \subseteq \Sigma^*$ is computed by a protocol if and only if it is regular.
\end{theorem}
\begin{proof}
  We begin with the easier right-to-left implication, which says that the protocol can compute every regular language. Consider a regular language, which is recognised by a deterministic automaton with state space $Q$. To compute this language, we will use a  one-round protocol. Alice sends the state in $Q$ of the automaton after reading her local string, and Bob sends the dependency $Q \to \set{\text{yes, no}}$ which says how Alice's state determines acceptance. Once these two pieces of information are known, we can apply the function from Bob's message to the state in Alice's message to determine the output.

  The rest of this proof is devoted to the left-to-right implication, i.e.~to showing that every language computed by the protocol is regular. This is the heart of the proof, and the place where we need to tame the hypothetically infinite computational power of the protocol. 
  
  Observe that in the right-to-left implication, we have proved that regular languages need only one round. Therefore, if we want to  prove that the protocol can only compute regular languages, then a by-product will be that the protocol collapses to the one-round case.  This is, in fact, the first step of our proof. 
  \begin{lemma}\label{lem:one-round-reduction-boolean}
    For every protocol, there is a one-round protocol that computes the same function. 
  \end{lemma}
  \begin{proof}
    The general idea is that instead of engaging in interactive communication, each party sends the dependency of their message upon the messages of the other party. Suppose that we are  Alice, and we know our local input string. The sequence of messages that we send will depend on the messages sent by Bob. This dependency is captured by a function of type 
    \begin{align*}
    (Q_B)^k \to (Q_A)^k,
    \end{align*}
    which  itself depends on the local string of Alice.  Instead of waiting for Bob's messages, Alice sends this function. (This function is not a general function, since it must satisfy the following \emph{causality} constraint: the $i$-th coordinate of the output depends only on the first $i-1$ input coordinates.) At the same time, Bob sends an analogous function of type 
    \begin{align*}
    (Q_A)^k \to (Q_B)^k,
    \end{align*}
    which describes the dependency of his messages. Due to the causality constraints, the two functions combine to create a unique output in $(Q_A \times Q_B)^k$, which can be used to determine the output of the protocol.
  \end{proof}

  Observe that the proof of the above lemma can, in general, incur an exponential cost in the size of the message spaces. This is of little concern to us, since we only care about the protocol having a constant number of rounds, without  any regard as to the size of the constant. 

  To complete the proof of the theorem, we use the  Myhill-Nerode Theorem to that one-round protocols can only compute regular languages.  Indeed, consider the strategy of  Alice: 
  \begin{align*}
  \sigma_A : \Sigma^* \to Q_A.
  \end{align*}
  For all we know, this function is non-computable. However, it classifies all local strings into finitely many categories, and furthermore, if two strings $w_1$ and $w'_1$ are classified into the same category, then they are equivalent in the following sense: 
  \begin{align}\label{eq:myhill-nerode-equivalence}
  w_1 w_2 \in L \Leftrightarrow w'_1 w_2 \in L 
  \qquad \text{for every $w_2 \in \Sigma^*$.}
  \end{align}
  The equivalence described above is the same equivalence as in the Myhill-Nerode Theorem. In particular, equivalence classes of this equivalence are the same as states of the minimal deterministic automaton. In particular, this automaton is finite, and therefore the language is regular.  (Observe that we have shown that the language has at most $Q_A$ equivalence classes of Myhill-Nerode equivalence, which gives an upper bound on the size of the minimal automaton recognizing the language.)
\end{proof}


\section{Beyond Boolean outputs}
\label{sec:beyond-boolean-outputs}

In the previous section, we have defined a protocol to compute functions with string inputs and Boolean outputs. In this section, we consider a generalisation, with possibly infinite output domains. The inputs remain unchanged -- they will always be strings in this paper.  For the output domain, we use a very general notion, namely a set with some operations. 
\begin{definition}[Output domain]
    An \emph{output domain} consists of: 
    \begin{enumerate}
        \item an underlying set $\domain$; together with
        \item a family of \emph{operations}, each one having  type $\domain^n \to \domain$ for some $n \in \set{0,1,\ldots}$.
    \end{enumerate}
\end{definition}
An output domain is the same thing as a (non-indexed) algebra, in the sense of universal algebra~\cite[p.5]{hobby1988structure}. 
The output domain might be infinite. In principle, the family of operations might be infinite as well, although any protocol will only use finitely many operations.
By abuse of notation, we use the same symbol $\domain$ to denote the output domain and its underlying set, with the operations being implicit. Here are the output domains that will be studied in this paper: 
\begin{itemize}
    \item \textbf{Boolean domain.} The set is $\set{\text{true, false}}$, and there are no basic operations. (We could add basic operations, such as the Boolean operations $\lor,\land$ and $\neg$, but this will not affect the expressive power of our protocol, so we choose to have no operations.)
    \item \textbf{Field domain.} The set is a field, such as the rationals or reals, and there are two operations for addition and multiplication. This is not one domain, but a family of domains, with one for each field.
    \item \textbf{String domain.} The set is $\Gamma^*$ for some finite alphabet $\Gamma$, and there is one basic operation, which is concatenation. Again, this is not one domain, but a family of domains, with one for each alphabet.
\end{itemize}

In the protocol, the output value will be constructed in a constant number of steps, by using operations from the output domain. Insofar as the protocols are concerned, we will not distinguish between the  operations that are in the output domain, and other operations that are derived by composing them, as described in the following definition.

\begin{definition}[Term operation]\label{def:term-operations}
    Consider an output domain $\domain$. An operation 
    \begin{align*}
    t : \domain^n \to \domain
    \end{align*}
    is called a \emph{term operation} if it can be obtained by applying the operations in the domain to variables $x_1,\ldots,x_n$. Each variable can be used multiple times, or not at all.  We write 
    \begin{align*}
    \domain^n \termop \domain
    \end{align*}
    for the set of term operations with $n$ arguments.
\end{definition}

\begin{myexample}
    If the operations are $+$ and $\times$, then after applying the usual distributivity laws,  a term operation becomes a special case of a polynomial with natural coefficients, such as 
\begin{align*}
x_1x_2^2 + x_1x_2x_3^3  + \myunderbrace{2x_1}{same as $x_1 + x_1$}.
\end{align*}
If the output domain is the Boolean domain, which has no operations, then the only kind of term operation is a single variable, e.g.~$x_2$.  In the case of a string domain, a term operation is some concatenation of the variables, such as $x_1 x_3 x_1 x_2$.
\end{myexample}



In this section, we generalise the protocol to cover functions 
\begin{align*}
f : \Sigma^* \to \domain,
\end{align*}
where $\domain$ is some possibly infinite output domain. As in the Boolean version of the protocol, the output value is constructed by Alice and Bob, as a result of an exchange of a constant number of messages. Each message has two parts, which are called the \emph{signal part} and the \emph{output part}. The signal part consists of a finite amount of information, and is used to exchange information between the two parties as in the Boolean protocol. The output part is an element of the output domain, and  is meant to be part of the output value. At the end of the protocol, the final output is constructed by applying some term operation to the output parts, with the term operation depending only on the signal parts. 

\begin{definition}\label{def:two-party-protocol-general} A two-party protocol   
   is given by the following ingredients: 
  \begin{enumerate}
    \item an output domain $\domain$;
    \item a finite input alphabet $\Sigma$;
    \item a number of rounds $k \in \set{1,2,\ldots}$;
    \item signal spaces for Alice and Bob, which are finite sets $Q_A$ and $Q_B$;
    \item an  dimension $d \in \set{0,1,\ldots}$;
    \item for each round $i \in \set{1,\ldots,k}$, a strategy
    \begin{align*}
    \myoverbrace{\sigma_A^i : \myunderbrace{\Sigma^*}{Alice's \\ local\\ string} \times \myunderbrace{Q_B^{i-1}}{history of \\ signals \\ from Bob}  \to \myunderbrace{Q_A \times \domain^d}{new\\ message}}{stategy for Alice in the $i$-th round}
    \qquad 
        \myoverbrace{\sigma_B^i : \myunderbrace{\Sigma^*}{Bob's \\ local\\ string} \times \myunderbrace{Q_A^{i-1}}{hostory of \\ signals \\ from Alice}  \to \myunderbrace{Q_B \times \domain^d}{new\\ message}}{stategy for Bob in the $i$-th round}
    \end{align*}
    \item an output function of type \begin{align*}
    (Q_A \times Q_B)^{k} \to (\domain^{2dk} \termop \domain).
    \end{align*}
  \end{enumerate}
\end{definition}

As in the Boolean case, the above definition slightly deviates from the informal description in the introduction. In particular, a message can contain $d$ elements of the output domain. This does not affect the expressive power of the protocol, but it might affect the number of rounds, and the above definition will be more convenient later one, where we consider one-round protocols.

The protocol is executed on a pair of strings $(w_1,w_2)$. In each round, each of the two parties  sends a message (which consists of a signal and an output) that is based on their local string and the history of signals coming from the other party. After all $k$ rounds have been executed, the joint signal history of both players is used, by the output function, to determine a term operation. This operation is then applied to the output history, yielding the final result of the protocol. As in the Boolean case, we are interested in protocols that are split invariant, which means that for every string $w \in \Sigma^*$, the same output is produced for every possible decomposition $w = w_1 w_2$. Such protocols compute a function of type $\Sigma^* \to \domain$. 

\begin{myexample}[Boolean domain]
    Consider the Boolean output domain, or more generally a finite output domain. In this case, the distinction between output values and signals is irrelevant, since the output values can be sent as signals. Therefore, the general protocol coincides with the Boolean protocol from the previous section, and  can only compute regular languages. 
\end{myexample}

Other examples of output domains are fields and strings. These were discussed  briefly in the introduction in Sections~\ref{sec:intro-field} and~\ref{sec:intro-strings},  and will be discussed at length in Sections~\ref{sec:field-domain} and~\ref{sec:string-outputs}.  Nevertheless, we give here one example which concerns strings, and uses both signals and output values in the messages.

% Here are two brief examples.
% \begin{myexample}[Field domain] Assume that the output domain is a field, say the field of rationals, with the operations being addition and multiplication. 
%     Consider the function 
%     \begin{align*}
%     f : \set{0,1}^* \to \mathbb Q,
%     \end{align*}
%     which maps a binary string to the natural number that it represents. For example, the input $01000$ will be mapped to $8$. Here is a protocol. Based on his input string $w_2$, Bob computes two numbers, namely: 
%     \begin{align*}
%     \myunderbrace{y_1 = f(w_2)}{number represented\\  by $w_2$} 
%     \qquad \text{and} \qquad 
%     \myunderbrace{y_2 = 2^{|w_2|}}{the power of two \\ corresponding to the\\ length of $w_2$}.
%     \end{align*}
%     Alice does the same, yielding two numbers $x_1$ and $x_2$, of which only $x_1$ will be used. The final output is then 
%     \begin{align*}
%     x_1\cdot y_2 + x_2.
%     \end{align*}
%     In this protocol, there are no signals. As we will see in the next section, when the output domain is a field, then signals are not needed (unlike the case of strings or Booleans, where signals are needed).
% \end{myexample}


\begin{myexample}[Conditional reverse]
    In Example~\ref{ex:reverse-duplicate}, we explained a protocol for reversing the input string. Here is a variant of this protocol, which uses signals. Consider the 
    \begin{align*}
    f : \Sigma^* \to \Sigma^*
    \end{align*}
    which reverses the string if the first letter is $a$, and otherwise it leaves the string unchanged. To compute this function, Alice needs to send to Bob -- apart from her part of the output string -- a signal to Bob which tells him if the first letter is $a$.
\end{myexample}



\subsection{One-round protocols}

We have little say to say about  protocols in the case of a general output domain. The  only result that we have at this level of generality is a reduction to one-round protocols, similarly to the Boolean case.       \begin{lemma}\label{lemma:one-round-reduction-general}
        Every protocol is equivalent to a one-round protocol.
 \end{lemma}
 \begin{proof}
    Same proof as in \cref{lem:one-round-reduction-boolean}. 
 \end{proof}

 Later in this paper, we will study protocols for infinite input alphabets. Such protocols will not be a special case of the protocols developped so far, since the signal spaces will be infinite, albeit in a limited way (called orbit-finite).  As we will see, for such signal spaces the reduction to one round will no longer be valid, and the interactive nature of the protocols will be essential.
\section{Field outputs}
\label{sec:field-domain}
In this section, we discuss functions where the output domain is a field, equipped with addition and multiplication. We prove that protocols have  exactly the same expressive power as weighted automata.
We begin by recalling the notion of weighted automata.

\paragraph*{Weighted automata.}  A weighted automaton is a device that is used to compute a function from strings to a field (more generally, a semiring, but we consider the case of fields here). This model was originally introduced by \schutz~\cite{schutzenberger1961definition}. Essentially, this is a deterministic automaton where the state space is a vector space of finite dimension, and each input letter induces a linear map. Here is the formal definition. 

\begin{definition}[Weighted automaton]
    \label{def:weighted-automaton}
    A weighted automaton over a field $\domain$ is given by: 
    \begin{enumerate}
        \item a finite input alphabet $\Sigma$;
        \item a dimension $d \in \set{0,1,\ldots}$;
        \item an initial state $q_0 \in \domain^d$;
        \item \label{it:weighted-definition-transitions} for each letter $a \in \Sigma$, a corresponding linear map of type $\domain^d \to \domain^d$;
        \item \label{it:weighted-definition-final} a \emph{final map}, which is a linear map of type $\domain^d \to \domain$. 
    \end{enumerate}
\end{definition}

A weighted automaton computes a function of type $\Sigma^* \to \domain$, which is defined in the same way as for a deterministic finite automaton: we begin in the initial state, then we apply the linear maps corresponding to the input letters, and finally we apply the final map. There is an alternative but equivalent way of describing weighted automata, which uses a nondeterministic automaton with weights on transitions. This viewpoint will be used later in this paper, see Definition~\ref{def:weighted-automaton-nondeterministic}.

The main result of this section is that our protocol is equivalent to weighted automata, as stated in the following theorem from the introduction, which we now recall:
\fielddomain*

% \begin{theorem}\label{thm:field-domain}
%     Assume that the domain is a field. Then a function 
%     \begin{align*}
%     f : \Sigma^* \to \domain
%     \end{align*}  is computed by a protocol if and only if it is  computed by a weighted automaton.
% \end{theorem}

Weighted automata can be defined not just for fields, but also for rings and even semirings. We do not know how to prove the theorem for such generalisations, since the Fliess Theorem, which is used in the proof is only known for fields. Rings and semirings will be discussed in more detail in Section~\ref{sec:commutative-semirings}.
 Before proving the theorem in Section~\ref{sec:proof-of-thm-field-domain}, we return to the issue of division, which was already discussed in Example~\ref{ex:division}.

\begin{myexample}[Division, continued]\label{ex:division-continued}
    Because it is undefined for zero, division is not a total operation, and therefore technically speaking it does not fall into our framework. We could, however try to incorporate it, by making the two parties responsible for avoiding division by zero. Under this framework, we could use a protocol to compute the function $1/|w|$ (a better choice would be $1/|w|+1$, since it would avoid problems with the empty string). As we have discussed in Example~\ref{ex:division}, such a function cannot be computed by a protocol that uses only addition and multiplication. We do not know what functions can be computed if division is also allowed.
\end{myexample}



\begin{myexample}[Semiring outputs]
    \label{ex:non-commutative-semirings} In this example we show that for semirings which are not fields, the protocol need not be equivalent to weighted automata.  The implication 
    \begin{align*}
    \text{protocol} \quad \impliedby \quad \text{weighted automaton}
    \end{align*}
    in \cref{thm:field-domain}, as we will see in a moment, holds for any semiring, and therefore the problematic implication is the other one. Here is an example where it fails.
    Let $\domain$ be  the free (non-commutative) idempotent semiring generated by two letters $a$ and $b$. Elements of this semiring are finite sets of words in $\set{a,b}^*$, such as 
    \begin{align*}
    \set{3ab, 5ba, 7aab}
    \end{align*}
    The addition operation is multiset union, and the multiplication operation is concatenation of words, extended to sets in the natural way, as illustrated on this example:
    \begin{align*}
    \set{a,b}\cdot \set{a,b} = \set{aa, ab, ba, bb}.
    \end{align*}
    Weighted automata over this semiring are the same as the rational relations~\cite[Chapter IX]{Eilenberg74}. On the other hand, a protocol can define string-to-$\domain$ functions that are not rational. This is witnessed already by functions that produce singleton sets (call these singleton functions), which can be seen as functions of type $\Sigma^* \to \set{a,b}^*$. For example, consider the singleton version of the  reverse function, i.e.
    \begin{align*}
    w \mapsto \set{\text{reverse of $w$}} \in \domain.
    \end{align*}
    This function can be computed by a protocol, using the same idea as in Example~\ref{ex:reverse-duplicate}. This function, however, is not a rational relation, and therefore it is not computed by a weighted automaton over $\domain$. One could perhaps conjecture that the appropriate model here is two-way weighted automata (for commutative semirings, two-way weighted automata are equivalent to one-way weighted automata, see~\cite[Theorem 1]{anselmo1990two}), but we have not investigated this direction further.
    We know more in the case of singleton functions. In this case, all messages sent during the protocol must be singletons (this is because once a non-singleton is produced, it can never be turned into a singleton). Therefore, the operation $+$ can never be used in a non-trivial way, and thus the protocol can only use muliplication. This means that it coincides with the protocols with outputs that are strings with concatenation, as discussed in Section~\ref{sec:string-outputs}. According to Conjecture~\ref{conj:protocol-regular-string-to-string}, the singleton functions are therefore exactly the regular functions.
\end{myexample}

\begin{myexample}[Equality tests]
\label{ex:equality-tests} In this example, we discuss an extension of the protocol which allows for equality tests, similarly to the algebraic group model~\cite{fuchsbauer2018algebraic}. Clearly, equality tests cannot be completely unrestricted. Otherwise, in the presence of a countable output domain (which is the case for all protocols studied in this paper), the receiver could compare the message with all possible values one by one, until the correct one would be identified. This would  invalidate the black box discipline. A reasonable restriction is to allow a constant number of equality tests for each message; this constant can also be brought  down to one, by possibly sending more copies of the same message. The resulting protocol would be able of complementing a weighted automaton $\Aa$, in the following sense:
\begin{align*}
w \in \Sigma^* 
\quad \mapsto \quad 
\begin{cases}
    1 & \text{if $\Aa(w) =0$}\\
    0 & \text{otherwise}.
\end{cases}
\end{align*}
This form of complementation is undesirable from the point of view of decidability. For example, language equivalence is undecidable for weighted automata that are complemented in this way~\cite[Theorem 4.9]{bojanczyk_automata_2025}. Since we strive for protocols that describe ``regular'' functions, and such functions should be decidable, we avoid equality tests.
\end{myexample}

\begin{myexample}[Wrong output domains]\label{ex:wrong-output-domains}
 This discussion of equality tests from Example~\ref{ex:equality-tests} also explains why we should not expect results about regularity that work for any output domain. For example, if we would extend the field domain with a unary complementation operation 
 \begin{align*}
 x 
 \quad \mapsto \quad 
 \begin{cases}
    1 & \text{if $x =0$}\\
    0 & \text{otherwise},
\end{cases}
 \end{align*}
 then our protocols could recover the undecidable model discussed in the previous paragraph.  Of course, one can come up with even more obviously wrong output domains, such as a domain that consits of Turing machines with certain evaluation operations. We do not know where the dividing line is between ``right'' and ``wrong'' output domains.
\end{myexample}


\subsection{Proof of \cref{thm:field-domain}}
\label{sec:proof-of-thm-field-domain}
We now return to the proof of \cref{thm:field-domain}. The right-to-left implication says that every weighted automaton can be simulated by a protocol. This is proved essentially in the same way as in the Boolean case. Suppose that the function is computed by a weighted automaton, which uses  dimension $d$. Every input string $\Sigma^*$ induces a linear map of type $\domain^d \to \domain^d$, which is obtained by composing the linear maps for the individual letters in the string.  Such a linear map can be represented as a matrix, and therefore it can be output using $d^2$ messages.  In the protocol, Alice sends the matrix  which corresponds to her local string, and Bob sends the  matrix which corresponds to  his local string. These matrices are multiplied using the field operations, and then multiplied with the initial and final vectors. This protocol has one round and is signal-free, i.e.~no information is conveyed using signals.

\omc{Usually we have signal-free protocols when corresponding Machine model expressive power does not increase when we add two-wayness. Here we have one-way weighted automata, in sequential model also we have one-wayness (it can considered as signal-free because consider algebra to be semiring with concatenation and union and one element for zero, then it is signal-free), in commutative semiring the conjecture is having one-way weighted automat, is it worth mentioning that? Answer my comment wehenever you see it pelase.}

The rest of this proof is devoted to the left-to-right implication, i.e.~showing that every function computed by a protocol is computed by a weighted automaton. As in the Boolean case, we will do a sequence of reductions, such that the protocol becomes more and more restrictive. In particular, we will show that the protocol can be reduced to a version that has one-round and  is signal-free.

\subsubsection{Reduction to a scalar product protocol}
\label{sec:reduction-to-scalar-product-protocols}

In the first step, we show that each protocol can be constrained to have a special form, which has one round and is signal-free. This protocol uses only the scalar product,  as explained in the following definition. 
\begin{definition}[Scalar product protocol] \label{def:scalar-product-protocol}
    Assume that the output domain is a field.
    A scalar product protocol is defined as follows. First, each of the two parties uses their local string to  produce a vector of field elements, of some fixed dimension $d$, as expressed by two functions: 
    \begin{align*}
    \sigma_A, \sigma_B : \Sigma^* \to \domain^d.
    \end{align*}
    Next, the output is defined to be the scalar product of the two vectors. 
\end{definition}

This protocol has the same power as general protocols. 

\begin{lemma}\label{lem:scalar-product-reduction}
    Assume that the output domain is a field. 
    If a function is computed by a protocol, then it is computed by a scalar product protocol.
\end{lemma}
\begin{proof}
    The proof is a sequence of reductions, where more and more conditions are imposed on the protocol.  
    
    \paragraph*{Step 1. One-round protocol.} The first step is to reduce the protocol to a one-round protocol. This is done using \cref{lemma:one-round-reduction-general}.



 \paragraph*{Step 2. Signal-free protocol.}  We say that a protocol is \emph{signal-free} if both of the sets $Q_A$ and $Q_B$ have one element each. In other words, the signals do not convey any information, and the only messaging activity consists of sending elements of the output domain. In a signal-free protocol, the concept of rounds is irrelevant, since the behaviour of one party is not influenced by the communication from the other party.

 \begin{claim}
    \label{claim:trivial-messages}
    Assume that the output domain is a field. Then every one-round protocol is equivalent to a signal-free  protocol.
 \end{claim}
 \begin{proof} 
    Consider a one-round protocol. Without loss of generality, we assume that both signal spaces $Q_A$ and $Q_B$ are the same space $Q$. (We can always use the union of two signal spaces for both parties.) Assume that each of the parties sends $d$ field elements in the protocol. In other words, the protocol works as follows:
    \begin{enumerate}
        \item Based on her local string, Alice chooses a message $(q_A,\bar x) \in Q \times \domain^d$;
        \item Based on his local string, Bob chooses a message $(q_B,\bar y) \in Q \times \domain^d$;
        \item Based on the signals $q_A$ and $q_B$, a term operation  with $2d$ variables is chosen, call it $t_{q_A,q_B}$, and the output is obtained by applying this term operation to $(\bar x, \bar y).$
    \end{enumerate}
    To prove the claim, we need to show that the protocol can be adapted so that always the same term operation is chosen, i.e.~there is no dependence of this term operation on the signals $q_A$ and $q_B$. This way the signals can be eliminated. To do this, we increase the dimension from $d$ to $d + |Q|$. 
    This means that for each possible signal $q \in Q$, each party sends a field element corresponding to this signal. The idea is that instead of sending a signal $q \in Q$, each party will set the corresponding field element to $1$, and the remaining field elements to $0$. The corresponding term operation is then 
    \begin{align*}
    \sum_{\substack{q_A \in Q \\ q_B \in Q}} \myoverbrace{x_{q_A} \cdot y_{q_B}}{variables corresponding \\ to the messages $q_A$ and $q_B$, } \cdot t_{q_A,q_B}(\bar x, \bar y).
    \end{align*}
    When evaluating this term operation, the summands that do not correspond to the intended message $(q_A,q_B)$ will be eliminated, since they will contain a variable that is set to $0$. Only the summand corresponding to the intended message will be used, and thus the correct output will be produced. 
 \end{proof}

 In the above claim, the only property of fields that was used is that there are elements $1$ and $0$ with the usual field properties, i.e.~$1$ is neutral for multiplication, while $0$ is neutral for addition and cancellative for multiplication. Therefore, so far our proof would work in any semiring with such elements.

 \paragraph*{Step 3. Scalar product.} In the previous step, we have reduced the protocol to a special case, where Alice and Bob send vectors, call them $\bar x, \bar y \in \domain^d$, and then some fixed term operation $t$ with $2d$ variables is applied to them. 
  To complete the proof of the lemma, we show that the term operation can be turned into a scalar product. This term operation is a sum of monomials, with each monomial being a product of some variables. Consider the monomials in the term operation $t$. For each monomial, its contribution to the output  is obtained by multiplying two numbers: (a) the product of the  variables in the term operation that are contributed by Alice; and   (b) the product of the variables in the term operation that are contributed by Bob. We can redesign the protocol so that for each monomial, Alice sends the contribution (a), and Bob sends the contribution (b). In the new protocol, the dimension is the number of monomials from the original protocol, and the term operation is a scalar product. 

  This completes the third and final step in the proof. In this step, the only property of fields that was used is that multiplication is commutative, and therefore each monomial can be cleanly separated into two parts, one for Alice and one for Bob. Summing up, the entire lemma would work for commutative semirings, and just for fields. However, the second part of the proof of Theorem~\ref{thm:field-domain}, presented in Section~\ref{sec:from-scalar-product-protocol-to-weighted-automaton}, does use the assumption  that the output domain is a field.
\end{proof}

\subsubsection{From a scalar product protocol to a weighted automaton}
\label{sec:from-scalar-product-protocol-to-weighted-automaton}
In this section, we complete the proof of \cref{thm:field-domain}, by showing that scalar product protocols can be simulated by weighted automata. Similarly to the Boolean case, the proof  uses a Myhill-Nerode characterization. In the case of weighted automata, this characterisation is called  the Fliess Theorem. This theorem  characterises functions computed by weighted automata in terms of a certain infinite matrix.

\begin{definition}[Hankel Matrix]\label{def:hankel-matrix}
    Let $\domain$ be a field. The Hankel matrix of a function 
    \begin{align*}
    f : \Sigma^* \to \domain
    \end{align*}  
    is the matrix where rows are words in $\Sigma^*$, columns are words in $\Sigma^*$, and the entry corresponding to a row $u$ and a column $v$ is $f(uv)$.
\end{definition}

Another perspective on the Hankel matrix is that it describes the \emph{derivatives} of the function $f$. Each row in the Hankel matrix can be seen as a function of type $\Sigma^* \to \domain$, which inputs columns (i.e.~strings) and outputs the corresponding entries in the Hankel matrix. If the row corresponds to a word $w$, then this function is
\begin{align*}
v \mapsto f(wv),
\end{align*}
which is called the \emph{left derivative} of $f$ with respect to $w$. Similarly, the columns of the Hankel matrix describe \emph{right derivatives} of $f$.

The Fliess Theorem~\cite[Theorem 2.1.1]{fliess1974} states that a function 
\begin{align*}
f : \Sigma^* \to \domain
\end{align*}
is computed by a weighted automaton if and only if  its  Hankel matrix  has finite rank, i.e.~its rows (i.e.~the left derivatives) are spanned by a finite subset. (This is equivalent to saying that the columns, or right derivatives, have a finite spanning subset.) Therefore, to complete the proof of \cref{thm:field-domain}, it is enough to show the following lemma.

\begin{lemma}\label{lem:hankel-finite-rank}
    If a function is computed by a scalar product protocol, then its Hankel matrix has finite rank.
\end{lemma}
\begin{proof}
    Essentially by definition, the Hankel matrix of a function computed by a scalar product protocol with dimension $d$ can be obtained as a sub-matrix of the following matrix: rows and columns are vectors in $\domain^d$, and the entries are obtained by taking scalar products. This matrix is easily seen to have finite rank, namely $d$, since the scalar product  becomes a linear operation once one of the two arguments is fixed.
\end{proof}

\section{String outputs}
\label{sec:string-outputs}

In this section, we consider the case where the output domain is strings over some finite alphabet. We use the name \emph{string-to-string function} for any function of type $\Sigma^* \to \Gamma^*$, where both alphabets $\Sigma$ and $\Gamma$ are finite. For such functions, the protocols are assumed to use the output domain $\Gamma^*$, equipped with concatenation. 
 In the case of string-to-string functions, we conjecture that protocols define exactly the so-called regular functions, which will be formally defined in Section~\ref{sec:regular-string-to-string-functions} below.
 


\begin{conjecture}\label{conj:protocol-regular-string-to-string}
    A string-to-string function is computed by a protocol if and only if it is regular. 
\end{conjecture}

In Section~\ref{sec:regular-string-to-string-functions} we  define the regular functions from the above conjecture, and we prove the $\impliedby$ implication: every regular function is computed by a protocol. The content of the conjecture is therefore the $\implies$ implication, namely that protocols can only compute regular functions and nothing more.  In Section~\ref{sec:continuity}, we present some evidence for the conjecture, by showing that string-to-string functions computed by protocols share many good properties of the regular functions, such as linear output size and computability. In Section~\ref{sec:unary-output-alphabet}, we present further evidence for the conjecture, namely we  prove it in the special case where the output alphabet has only one letter (the remaining case is two output letters, since more letters do not change the situation). Finally, in Section~\ref{sec:beyond-fields}, we discuss variants of the conjecture that are related to weighted automata which are not over a field, but over an arbitrary semiring. 


\subsection{Regular string-to-string functions}
\label{sec:regular-string-to-string-functions}

In this section, we define the class of regular string-to-string functions, and we prove one implication in the conjecture. 
 Historically, this class of functions was first defined in terms of 
 deterministic two-way automata with output\cite[Note 4]{shepherdson1959reduction}. Let us present this definition.

 \begin{definition}[Two-way automaton]
    A deterministic two-way automaton with output is given by the following ingredients:
    \begin{enumerate}
        \item a finite input alphabet $\Sigma$;
        \item a finite output alphabet $\Gamma$;
        \item a finite set of states $Q$, with an initial state $q_0 \in Q$;
        \item a transition function  
        \begin{align*}
        \delta : 
        \myunderbrace{Q}{old \\ state} \times 
        \myunderbrace{(\Sigma + \{\vdash, \dashv\})}{input letter\\ under  the head} \to  \set{\text{halt}} + (
        \myunderbrace{Q}{new \\ state}
         \times 
         \myunderbrace{\{-1,0,1\}}{head \\ movement} \times 
         \myunderbrace{\Gamma^*}{added \\ output}) .
        \end{align*}
    \end{enumerate}
 \end{definition}

    The automaton works as follows. The input string $w$ is placed on a tape, with the left end marked by $\vdash$ and the right end marked by $\dashv$. The automaton starts in state $q_0$, with its head on the left end of the tape, which contains the marker $\vdash$. In each step, the automaton looks at its current state and the letter under its head, and based on this information, it uses the transition function to decide if it halts, or it continues its computation. In case it continues, it chooses a  new state, the direction in which it moves its head, and a string over the output alphabet $\Gamma$, which is appended to the output tape. We assume that the automaton is always halting, which means that for every input string, the computation eventually halts. In particular, the computation must be well-defined, which means that the head never falls off the input by moving outside the endmarkers.   The semantics of such an automaton is of type $\Sigma^* \to \Gamma^*$. (For automata which are not necessarily halting, the function would be partial, since it would be undefined for inputs where the automaton does not halt.)

    \begin{myexample}[Reverse]
        For each input alphabet $\Sigma$, the reverse function of type $\Sigma^* \to \Sigma^*$ is computed by a  two-way automaton, which first moves its head to the end of the string, and then starts copying it to the output while moving in the left direction.
    \end{myexample}

    The class of functions computed by two-way automata has a remarkable number of equivalent descriptions,  originating in different fields, including:  monadic second-order transductions~\cite[Section 4]{engelfrietMSODefinableString2001}, streaming string transducers~\cite[Section 3]{alurExpressivenessStreamingString2010},  certain kinds of regular expressions~\cite[Section 2]{alur2014regular}, a calculus of functions based on  combinators~\cite[Theorem 6.1]{bojanczykRegularFirstOrderList2018}, a characterisation based on natural transformations~\cite[Theorem 3.2]{bojanczykTitoRegular23}. For this reason, some authors (starting with Engelfriet and Hoogeboom), use the name \emph{regular} for this class of function, with the intended meaning being that these functions play the same role for string-to-string functions, as that which is played by regular languages for string-to-Boolean functions. We adopt this terminology here, as stated in the following definition.


    \begin{definition}[Regular string-to-string function]
        \label{def:regular-string-to-string}
        A string-to-string function is called \emph{regular} if it is computed by a deterministic two-way automaton with output.
    \end{definition}
    
    
    One good  property of the regular string-to-string functions is that they  are closed under composition~\cite[Theorem 2]{chytilSerialComposition2Way1977}. In particular, our conjecture would imply that the same is true for functions computed by protocols. Without proving the conjecture, we do not see any direct way of proving composition for protocols.
     
    
    Another good property of the  regular string-to-string functions is that equivalence is decidable, i.e.~given two functions $f$ and $g$, one can decide if for every input string, the two output strings are equal~\cite[Theorem 1]{gurariEquivalenceProblemDeterministic1982}. This property does not seem to have any direct bearing on protocols, since there is no obvious way of presenting a non-uniform protocol as an input for a decision procedure.



    The following lemma shows one of the implications in the conjecture.

\begin{lemma}\label{lem:from-regular-to-protocol}
    If a string-to-string function is regular, then it is computed by a protocol.
\end{lemma}
\begin{proof}
    The two parties can simulate a two-way automaton with output. The execution of the protocol describes the crossing sequence of the automaton, i.e.~how it crosses the boundary between the two local strings of Alice and Bob. Here is a picture: 
    \mypic{1} 
    More formally, the crossing sequence is defined as follows, given  a split of the input string into two parts $w_1 w_2$. We run the automaton until the first configuration which is in the word $w_2$. Then we run it until the first configuration which is in the word $w_1$. We continue this way, with odd-numbered steps describing runs inside $w_1$ that end in  configurations from $w_2$, and even-numbered steps describing runs inside $w_2$ that end in a configuration from $w_1$. The last step is exceptional, since it ends with an accepting configuration. 
    The number of steps in a crossing sequence is bounded  the number of states, since otherwise the automaton would enter an infinite loop. This bound is the number of rounds in the protocol. In each round, the state corresponding to this round is sent as a signal, and the output value in the message is the  part of the output string that is produced in this  step. At the end of the protocol, the pieces of the output string are concatenated. 
\end{proof}

In view of the above lemma, the content of the conjecture is the opposite implication, namely that every protocol computes are regular function. In the rest of this section, we give some evidence for the opposite implication. In Section~\ref{sec:continuity}, we show that functions computed by protocols share some good properties of the regular functions, which is evidence that they might be the same functions. Then,  in Section~\ref{sec:unary-output-alphabet}, we prove the conjecture in the special case of a unary output alphabet. 



\subsection{Evidence for the conjecture}
\label{sec:continuity}
The rest of Section~\ref{sec:string-outputs} is devoted to  evidence for Conjecture~\ref{conj:protocol-regular-string-to-string}.
In this subsection, we show that the string-to-string functions computed by protocols share some of the good properties of regular functions, such as linear size outputs and computability. (A priori, it is not clear why the functions should be computable, since the protocols are non-uniform.) The point of departure is the following lemma, which connects weighted automata and string-to-string functions that can be computed in our protocol.

    \begin{lemma}
        \label{lem:postcomposition-weighted-automaton}
        Let $\domain$ be a field, and consider two functions
        \[
        \begin{tikzcd}
        \Sigma^* 
        \ar[r,"f"]
        &
        \Gamma^*
        \ar[r,"g"]
        & 
        \domain,
        \end{tikzcd}
        \]
        such that $f$ is computed by a protocol (with string outputs). If $g$  is computed by a weighted automaton, then the same is true for  $f;g$.
    \end{lemma}
    \begin{proof}
        We will show that the composition $f;g$ is. computed by a protocol (with field outputs). Thanks to \cref{thm:field-domain}, this will imply that $f;g$ is computed by a weighted automaton.

        For each string in $\Gamma^*$, the weighted automaton for $g$ has an associated matrix over the field $\domain$. We modify the protocol for $f$, so that it uses matrices instead of strings. During the execution, instead of sending strings in $\Gamma^*$, the parties send  the corresponding matrices. At the end, instead of concatenating the output strings, the matrices are multiplied, yielding a matrix for the entire output string. Finally, this matrix is applied to the initial state, and then the output function is applied to the resulting vector. All of this is done using addition and multiplication, which is legitimate in a protocol with the output domain $\domain$.
    \end{proof}


The above lemma establishes a property of $f$, namely that weighted automata (over a field) are closed under precomposition with $f$. We think that this is an important property, and therefore we give it a name.


\begin{definition}[Field continuity]
    \label{def:weighted-continuity}
    A string-to-string function $f : \Sigma^* \to \Gamma^*$ is called \emph{field continuous} if functions computed by weighted automata over a field are closed under precomposition with $f$.
\end{definition}

In the above definition, we only consider weighted automata over a field. The more general setting of semirings is dicussed in Section~\ref{sec:beyond-fields}.
The name ``continuous'' is inspired by a similar terminology that is used in automata theory for functions that preserve regularity under inverse images, see~\cite[Theorem 4.1]{PinSilva05} or~\cite[Footnote 2]{continuity20}.  For the latter notion, we use the name \emph{Boolean continuity}.

As we have shown in \cref{lem:postcomposition-weighted-automaton}, all string-to-string functions computed by protocols are field continuous. In particular, since every regular string-to-string function is computed by a protocol, it follows that every regular string-to-string function is field continuous\footnote{To the best of our knowledge, this is a new result. It can also be proved directly, without passing through protocols, and we present such a direct proof in Section~\ref{sec:beyond-fields}, for the more general setting of commutative semirings. In the more general case we need a direct proof, since \cref{thm:field-domain} is not known to be true for this case.}
We conjecture that the converse is also true.

\begin{conjecture}\label{conj:regular-continuous}
    A string-to-string function is field continuous if and only if it is regular.
\end{conjecture}

In Section~\ref{sec:beyond-fields}, we dicuss variants of the conjecture, and in particular we show that the conjecture becomes false if the left side is relaxed from field continuous to Boolean continuous.
The above conjecture can be seen as  a machine independent characterisation of the regular string-to-string  functions. This would be a very valuable contribution. Almost all known characterisations of the regular string-to-string  functions have somewhat lengthy definitions, based on specific computational models, and it is something of a  miracle that all of these models are equivalent. A possible exception is the characterisation in~\cite{bojanczykTitoRegular23}, which does not use a machine model; however that characterisation uses the abstract language of category theory, and is less elementary than the one in Conjecture~\ref{conj:regular-continuous}.

As in Conjecture~\ref{conj:protocol-regular-string-to-string}, the content of the conjecture is the left-to-right implication. 
Conjecture~\ref{conj:regular-continuous} is stronger than Conjecture~\ref{conj:protocol-regular-string-to-string}, as explained in the following diagram, which shows the known relations between three kinds of string-to-string functions:
\[
\begin{tikzcd}
\text{regular}
\ar[d,Rightarrow,shift right=2, "\text{\cref{lem:from-regular-to-protocol}}"']
\\
\text{computed by protocols}
\ar[d,Rightarrow, shift right=2, "\text{\cref{lem:postcomposition-weighted-automaton}}"']
\ar[u,Rightarrow, shift right=2,"\text{Conjecture~\ref{conj:protocol-regular-string-to-string}}"']
\\ 
\text{field continuous} 
\ar[uu,bend right=89, Rightarrow, shift right=2,"\text{Conjecture~\ref{conj:regular-continuous}}"']
\end{tikzcd}
\]


The following theorem gives some evidence for the stronger conjecture, by showing that the field coniniuous functions share some well-known properties of the regular string-to-string functions. 

\begin{theorem}\label{thm:evidence-for-the-conjecture}
    If a function $f : \Sigma^* \to \Gamma^*$ is  field continuous, then:
    \begin{enumerate}
        \item \label{it:linear-size-outputs} the outputs have at most linear size;
        \item \label{it:linear-time-computable} the outputs can be   computed in linear time;
        \item \label{it:regular-preimages} it is Boolean continuous, i.e.~preimages of regular languages are regular.
    \end{enumerate}
\end{theorem}

% Before proving the theorem, let us comment on the properties that are listed in it.  By Lemma~\ref{lem:from-regular-to-protocol}, every regular string-to-string  function is computed by a protcol, and therefore every regular string-to-string function has the properties that are listed in the theorem. The fact the regular string-to-string functions have the  first three properties is a folklore result and can be seen directly from the definition of regular string-to-string functions, without protocols. (For the third property, it is useful to know that, as language acceptors, two-way automata recognise exactly the regular languages~\cite[Theorem 2]{shepherdson1959reduction}).
% The fact that regular functions have the  property, about postcomposition with weighted automata over a field, is not known in the literature, to the best of our knowledge. A direct proof is possible, 

\begin{proof}
    For properties~\ref{it:linear-size-outputs} and~\ref{it:linear-time-computable}, we embed strings into numbers. 
    An output string over alphabet $\Gamma$ can be seen as a number in base $|\Gamma|$. To avoid with ambiguity that could result from leading zeros, we first prepend the string with the digit 1. Let 
    \begin{align*}
    g : \Gamma^* \to \Nat \subseteq \Rat
    \end{align*} 
    be the corresponding encoding. This encoding can be computed by a weighted automaton over the field $\Rat$, see~\cite[Lemma 8.10]{bojanczyk_automata_2025}. By  the assumption on field continuity, the composition $f;g$ can be computed by a weighted automaton. This is a weighted automaton that works in the field of rationals $\Rat$, but  only produces natural numbers on its output. By~\cite[p. 110]{BerstelReutenauer08}  the automaton can be chosen so that it only uses  integers $\Int$, possibly including negative integers. Summing up, we have a weighted automaton over $\Int$ that outputs the representation, in base $|\Gamma|$, of the output string produced by $g$. We claim that for such an automaton, the output number
    \begin{enumerate}
        \item has a linear number of digits;
        \item can be computed in linear time.
    \end{enumerate}
    These two claims yield the corresponding items in the statement of the theorem. The first claim, about a linear number of digits, is true because it is true for every weighted automaton over $\Int$. This is because applying a fixed linear map can only add a constant number of digits. THe second claim is also easy to see, since the weighted automaton can be evaluated in linear time (we assume that we work in a model where addition and subtraction of integers has unit cost). 

    We are left with property~\ref{it:regular-preimages}, about Boolean continuity.  This will follow from the special case of field continuity, where the field is  the two-element field.  This is because of  the following  folklore correspondence between regular languages and weighted automata over the two-element field. 
        
        \begin{claim}\label{claim:regular-weighted-automata}
            A language $L \subseteq \Gamma^*$ is regular iff its characteristic function $\Gamma^* \to \set{0,1}$ is computed by a weighted automaton over the two-element field 
        \end{claim}
        \begin{proof}
            For the lef-to-right implication, we observe that a weighted automaton over a finite field can be simulated by a deterministic finite automaton. For the other direction, we observe that a weighted automaton can count the parity of  the number of runs in a finite automaton, and if the automaton is deterministic then the number of runs is either zero or one, and thus the parity gives the right answer.
        \end{proof}

        In terms of the correspondence from the above claim, preimages of regular languages become precompositions of weighted automata over the two-element field. In particular, regularity is preserved. 
\end{proof}

One could think that already the two properties in the above theorem are not only necessary for field continuous functions, but also sufficient. This is not the case, as shown by the following example.

\begin{myexample}[Factorials]
    \label{ex:not-regular-but-continuous-over-finite-fields}
    Consider a string-to-string function 
    \begin{align*}
    g : \Sigma^* \to \set{a}^*
    \end{align*}
    where both the input and output alphabets are unary. A sufficient condition for regularity of such functions is given in  \cite[Example 2.12]{bojanczykTitoRegular23}, using \emph{factorials}, i.e.~numbers in the set $\setbuild{n!}{$n \in \Nat$}$. This sufficient condition is that: (a) every output string arises from finitely many inputs; and (b) every output string has length that is a factorial. 
    It is not hard to come up with a function that has  properties (a) and (b), thus ensuring Boolean continuity, and which has furthermore linear size outputs and is computable in linear time. For example, the function could map an input string $w$ to the longest string of factorial length that is shorter than $w$. 
\end{myexample}

\subsection{Unary output alphabet}
\label{sec:unary-output-alphabet}
In this section, we prove the special case of Conjectures~\ref{conj:protocol-regular-string-to-string} and~\ref{conj:regular-continuous}, for output alphabets with only one letter.
\begin{theorem}\label{thm:unary-string-to-string}
    The following conditions are equivalent for  a string-to-string function $f$ where  the output alphabet  has only one letter:
    \begin{enumerate}
        \item $f$ is computed by a protocol;
        \item \label{it:unary-weighted-continuous} $f$ is weighted continuous;
        \item \label{it:unary-regular} $f$ is regular.
    \end{enumerate}
\end{theorem}

The rest of this section is devoted to proving the above theorem. As discussed above, the only missing implication is \ref{it:unary-weighted-continuous}~$\Rightarrow$~\ref{it:unary-regular}. In the case of a unary output alphabet, we will be able to prove this implication, since outputs can be seen as representing natural numbers.  Under this representation, concatenation of strings is the same as addition of natural numbers. Therefore, we can think of the output domain in the protocol as being $(\Nat,+)$.



\subsubsection{Linear output size}
\label{sec:linear-output-size}

Another corollary  of \cref{lem:nat-in-rat} is that the output of a protocol $f : \Sigma^* \to \Nat$ can be at most exponential, since this is true for weighted automata over the field of rationals.
The next step in our construction is a stronger bound on the output size, namely linear. This result is true also for output alphabets that have more than one letter. 




\begin{lemma}\label{lem:linear-output-size}
    If a function $f : \Sigma^* \to \Gamma^*$ is computed by a protocol, then it has linear output size, i.e.~the length of the output string is at most linear in the length  of the input string.
\end{lemma}
\begin{proof}
    In the proof, we again use weighted automata over a field. We begin by observing that string-to-string functions computed by protocols can be postcomposed with string-to-field functions computed by weighted automata, and there result will be a string-to-field function computed by a weighted automaton. 



    Observe also that, since all regular string-to-string functions are computed by protocols (Lemma~\ref{lem:from-regular-to-protocol}), the above claim implies that weighted automata are closed under precomposition with regular string-to-string functions. As far as we know, this is a new result, and we are not of any direct proof of this result which would be substantially simpler than the one given here (although the result can be proved directly, without passing through protocols). 
    We believe that the  claim describes an essential property of the protocols, and we will  discuss this in more detail in Section~\ref{sec:continuity}.    For the moment, we use the claim to prove linear growth. 

    Consider the ring  $\Rat[x]$ of polynomials in one variable, and its field extension which is the field of rational functions $\Rat(x)$. An element of this field extension is a fraction of two polynomials, modulo equivalence of fractions.  We use this field represent the length of an output string, via the map 
    \begin{align}\label{eq:length-as-polynomial-degree}
    w \in \Gamma^* 
    \quad \mapsto \quad 
    x^{|w|} \in \Rat(x).
    \end{align}
    This map is easily seen to be computed by a weighted automaton (the state stores the output of the function, and in each step it is multiplied by the scalar $x$, which is a linear map). By \cref{claim:postcomposition-weighted-automaton}, the composition of the protocol $f$ with the map from~\eqref{eq:length-as-polynomial-degree} is computed by a weighted automaton. This map is 
    \begin{align*}
    w \in \Sigma^* \quad \mapsto \quad x^{|f(w)|} \in \Rat(x).
    \end{align*}
    The size of the outuput string for $f$ is the same as the degree of the output of the output of the above function. The above function happens to produce only polynomials (and not fractions of polynomials, which is the general case of elements in $\Rat(x)$), but the notion of degree can easily be extended to fractions: this is the degree of the enumerator minus the degree of the denominator. The following claim shows that the degree can be at most linear in the input size, thus proving the lemma. 
    \begin{claim}
        Consider a function $h : \Sigma^* \to \Rat(x)$ which is computed by a weighted automaton. Then the degree of the output is at most linear in the length of the input.
    \end{claim}
    \begin{proof}
        When reading one input letter, the degrees of the field elements stored in the state can only increase by a constant amount.
    \end{proof}
\end{proof}




\subsubsection{Regularity}
\label{sec:regularity-of-string-to-number}
We now proceed to the final stage of the proof of \cref{thm:unary-string-to-string}. So far, we have established that if a function $f : \Sigma^* \to \Nat$ is computed by a protocol, then: (1) it is computed by a weighted automaton over the rationals (\cref{lem:nat-in-rat}); and (2) it has linear output size (\cref{lem:linear-output-size}). In this section, we show that these two properties ensure regularity, which completes the proof of \cref{thm:unary-string-to-string}.
\begin{lemma}
    Consider a function $f : \Sigma^* \to \Nat$ which:
    \begin{enumerate}
        \item is computed by a weighted automaton over the field $\Rat$;
        \item produces  output number that are at most linear in the length of the input string.
    \end{enumerate}
    Then the corresponding string-to-string function $w \mapsto a^{f(w)}$ is regular. 
\end{lemma}

\begin{proof} Following~\cite{Zpolyreg23}, we  use an approach that is based counting positions selected by formulas of  monadic second-order logic \mso. We assume that the reader is familiar with this logic, see~\cite[Section 2.1]{bojanczyk_recobook} for an introduction. 
A \emph{query of arity $d \in \set{0,1,\ldots}$} 
over an iput alphabet $\Sigma$ is defined to be a function which inputs a string  $w \in \Sigma^*$ and output a subset of $d$-tuples of positions in $w$. We will be interested mainly in the cases of arity zero or one. 
% We use the name \emph{unary query} for a query of arity one, while queries of arity zero are the same as languages, but we call them \emph{Boolean queries} when viewed as queries. 
Also, we will  only be interested in queries that can be defined in \mso. This means that for each query there must be an \mso formula $\varphi(x_1,\ldots,x_d)$, which tells us when a tuple of positions is selected by the query. Such queries will play the role of regular languages, and will be called \emph{\mso definable}. For example, a unary query $\varphi(x)$ could say that the position has at least one $a$ before it, and at least one $a$ after it, which can clearl be expressed in \mso.  

In the context of this paper, queries will be used to define string-to-number functions, by counting the number of selected tuples, as described in the  following definition, which is based on~\cite{Zpolyreg23}.


\begin{definition}
    [Basic counting functions]   For an \mso definable query $\varphi(x_1,\ldots,x_k)$ over input alphabet $\Sigma$, the corresponding \emph{counting function}, denoted by
    \begin{align*}
    \counter\varphi : \Sigma^* \to \Nat,
    \end{align*}
    is the function that outputs the number of selected tuples in the input string. Any function of this type is called a \emph{basic counting function of arity $d$}.
\end{definition}

The following claim shows that the regular functions, as in the conclusion of the present lemma, are exactly those that can be obtained using positive linear combinations of basic counting functions of arity zero or one.
    \begin{claim}\label{claim:mso-counting-regular}
        Consider a function $f : \Sigma^* \to \Nat$. Then the corresponding string-to-string function with a unary output alphabet is regular if and only if $f$  can be decomposed as  a linear combination 
        \begin{align*}
        f = \alpha_1 \counter{\varphi_1} + \cdots +  \alpha_k \counter{\varphi_k},
        \end{align*}        
        where the coefficents $\alpha_i$ are in $\Nat$, and the queries $\varphi_i$ have arity zero or one.
    \end{claim}
    \begin{proof}
        To prove this claim, we use the characterisation of regular functions in terms of \mso transductions from~\cite[Section 4]{engelfrietMSODefinableString2001}. If the output is unary, then the order on output positions is irrelevant, and only the number of output positions is relevant. In an \mso transduction, the output positions are defined using \mso formulas with at most one free variable, thus yielding the result. 
    \end{proof}

In light of the above claim, to prove the lemma it will be enough to show that if a function is computed by a weighted automaton, and its  outputs are natural numbers of linear size, then it is a sum of basic counting functions of arity zero or one. The first step in our construction will be to prove that there is such a decomposition, but possibly using negative coefficients.

\begin{claim}\label{claim:mso-counting-regular-with-negative}
    Consider a function $f : \Sigma^* \to \Nat$ which satisfies the assumption of the lemma, i.e.~it is computed by a weighted automaton over the rationals, and it has linear output size. Then $f$ admits a decomposition 
    \begin{align*}
    f = \alpha_1 \counter{\varphi_1} + \cdots +  \alpha_k \counter{\varphi_k},
    \end{align*}        
        where the coefficents $\alpha_i$ are in $\Int$, and the queries $\varphi_i$ have arity zero or one.
\end{claim}
\begin{proof}
    In the proof, we use weighted automata over the integer ring $\Int$. This is the special case of weighted automata over the rational field $\Rat$, in which all coefficients are integers. If a function is computed by a weighted automaton over $\Rat$, and all of its outputs are integers,  then it is also computed by a weighted automaton over $\Int$,  see~\cite[p. 110]{BerstelReutenauer08}. Therefore, we can assume that $f$ is computed by a weighted automaton over $\Int$.
    As remarked in~\cite[Remark II.21]{Zpolyreg23}, if a function is computed by a weighted automaton over the ring $\Int$, and it has polynomial output size, then it can be obtained as a linear combination of basic counting functions, possibly using negative coefficients. Furthermore, the cited remark can be strengthened, with the same proof, to a linear version: if the output is linear, then the basic counting functions can be chosen to have arity zero or one, as required by the claim.
\end{proof}

To finish the proof of the lemma, we will show that negative coefficients are not necessary, assuming that the outputs are non-negative. Since the  proof of this last part is long, and the corresponding result  might be of independent interest, we state it as a separate theorem in its own separate section, see \cref{thm:int-to-nat} in Section~\ref{sec:non-negative}. This theorem completes the proof of the lemma, and thus also of \cref{thm:unary-string-to-string}.
\end{proof}



\subsubsection{Proof of \cref{thm:int-to-nat}}
\label{sec:non-negative}
In this section, we prove \cref{thm:int-to-nat},  which says that negative coefficients can be eliminated from linear combinations of query counting functions, as long as the outputs are non-negative. This result is non-trivial, and it crucially depends on linearity (i.e.~the query counting functions have arity at most one), as explained in the following example.

\begin{myexample}[Quadratic counterexample]\label{ex:quadratic-counterexample}
     We show  a function which: (a) is a linear combination of \mso  counting functions of arity two, with negative coefficients; (b) has only non-negative outputs; and (c) cannot be presented as linear combination with positive coefficients. The idea, which is based on~\cite[Example 2.1]{BerstelReutenauer08}, is to trivially ensure non-negativity by squaring. Take the function
\begin{align*}
w \in \set{a,b}^* 
\quad \mapsto \quad 
(\text{(number of $a$'s in $w$)} - \text{(number of $b$'s in $w$)})^2.
\end{align*}
This function clearly satisfies (a) and (b). As explained in \cite[p.3]{Zpolyreg23}, it also satisfies (c), since the inverse image of $0$ is not a regular language, as would be the case if only positive coefficients were used.
\end{myexample}

The rest of Section~\ref{sec:non-negative} is devoted to proving \cref{thm:int-to-nat}. We use the following terminology\footnote{The terminology is based on~\cite{Zpolyreg23}, where the polynomial (instead of linear) generalisations of these functions are called $\Int$-polyregular and $\Nat$-polyregular, respectively. As remarked in Example~\ref{ex:quadratic-counterexample}, the polynomial version of~\eqref{eq:nat-to-int} does not hold.
}. A function with a decomposition as in the assumption of the theorem is called \emph{$\Int$-regular}. In other words, this is a linear combination of \mso counting functions that have arities zero or one, with the coefficient in the linear combination being from $\Int$. If the coefficients are positive, then the funciton is called \emph{$\Nat$-regular}. In this terminology, the theorem says 
\begin{align}\label{eq:nat-to-int}
\text{$\Int$-regular} \ \land \ \text{non-negative outputs} \implies \text{$\Nat$-regular}.
\end{align}
For technical reasons, it will be easier to work with the following variant of \eqref{eq:nat-to-int}, where both the assumption and conclusion are weakened as follows: 
\begin{align}\label{eq:nat-to-int-weaker}
\text{$\Int$-regular} \ \land \ 
\myunderbrace{\text{
\begin{tabular}{c}
        outputs are $\ge c$ \\ 
        for some $c \in \Int$
    \end{tabular}
}}{call this \emph{lower bounded outputs}}
 \implies 
 \myunderbrace{\text{
\begin{tabular}{c}
         $\Int$-regular, but  negative \\ coefficients are allowed 
        only\\ for queries of arity zero
    \end{tabular}
}}{call this \emph{almost $\Nat$-regular}}
\end{align}
The implication~\eqref{eq:nat-to-int-weaker}, which uses the weaker properties, will turn out to be more amenable to our proof techniques, and therefore this will be the implication that we care about.  Before proving~\eqref{eq:nat-to-int-weaker}, we show how to deduce~\eqref{eq:nat-to-int} from it. The main observation is in the following lemma, which bridges the gap between almost $\Nat$-regular and $\Nat$-regular functions.



        \begin{lemma}\label{lem:remove-almost}
            If a  $\Int$-regular function  is almost $\Nat$-regular, and has only non-negative outputs, then  is $\Nat$-regular.
    \end{lemma}

    Before proving the lemma, we use it to show how the implication~\eqref{eq:nat-to-int-weaker} gives us the implication~\eqref{eq:nat-to-int}. This is explained in the following diagram:
    \[
    \begin{tikzcd}
    \text{$\Int$-regular and  non-negative outputs}
    \ar[d,Rightarrow, "\text{special case}"]
    \ar[ddd,Rightarrow,  bend right=62]
    \\
 \text{$\Int$-regular and lower bounded outputs} 
\ar[d,Rightarrow, "\text{\eqref{eq:nat-to-int-weaker}}"] \\
 \text{almost $\Nat$-regular}
\ar[d,Rightarrow]
\\
\text{almost $\Nat$-regular and non-negative outputs}
\ar[d,Rightarrow, "\text{\cref{lem:remove-almost}}"] 
\\
 \text{$\Nat$-regular}
    \end{tikzcd}
    \]

    Therefore, to finish the proof of the theorem, it remains to prove~\eqref{eq:nat-to-int-weaker} and the lemma. We begin with the lemma.

    \begin{proof}[Proof of \cref{lem:remove-almost}]
        It will be easier to prove a generalisation of the lemma,  where the input domain of the function, instead of being  all strings over the input alphabet, is restricted to  some regular language. The notions of $\Int$-regularity and $\Nat$-regularity are adapted in the natural way for such functions: for a regular language $L \subseteq \Sigma^*$, a function of type  $ L \to \Int$ is called $\Int$-regular if it coincides with some $\Int$-regular function of type $ \Sigma^* \to \Int$ over the domain $L$. In the same way, we extend  $\Nat$-regularity to functions whose domains are regular languages. This more general setup allows us to do case analysis and split the domain into several regular parts, as shown in the  following claim.


    \begin{claim}\label{claim:aggregate-two-domains} If a  function  is  $\Nat$-regular when its domain is restricted to $L_1$,  and also when restricted to $L_2$, then it is  $\Nat$-regular when its domain is restricted to $L_1 \cup L_2$. Likewise for almost $\Nat$-regular.
    \end{claim}
    \begin{proof}
        The queries for $L_2$ can be modified so that they return ``false'' (in the case of arity zero) or ``no positions'' (in the case of arity one) on inputs from $L_1$. After this modification, we can simply add the two function to each other.
    \end{proof}

     We now return to the proof of the lemma.  
         Consider a $\Int$-regular function $g : L \to \Int$, which is almost $\Nat$-regular and has non-negative outputs. Consider  a decomposition into a linear combination of query counting functions 
        \begin{align*}
        g = \sum_{\varphi \in \Phi} \alpha_\varphi \cdot \counter{\varphi}.
        \end{align*}
        By assumption on being almost $\Nat$-regular, the negative coefficients are only allowed next to queries of arity zero. 
        By possibly repeating queries, we can assume that each coefficient is either $1$ or $-1$. Our goal is to improve the decomposition so that it only uses positive coefficients. We do this by induction on the number of queries with a negative coefficient. In the induction base, all queries have positive coefficients, and there is nothing to do.

        Consider now the induction step, where at least one coefficient is negative. We will split the domain into finitely many regular parts, so that certain information is uniform across each part. 
        For each input string, we are interested in the following information: (a) for each query $\varphi \in \Phi$ of arity zero, is it true in the string; and (b) for each query of $\varphi \in \Phi$ of arity one, is it true for at least one position in the string. This information can assume finitely many values, namely $2^{|\Phi|}$, and for each such value the corresponding set of strings is a regular language. Therefore, we can split the domain $L$ into finitely many regular parts, such that for each part this information is fixed. It is then enough to prove the lemma for each of the parts, since the results from the conclusion of the lemma can then be aggregated using  \cref{claim:aggregate-two-domains}. Summing up, we  can assume without loss of generality that the information (a) and (b) is the same for all strings in the entire domain. 
        
        If some query with a negative coefficient is false in the domain, then it can be removed, and we can call on the inductive assumption. Otherwise, there is some query $\varphi \in \Phi$ which has a negative coefficient, and which is true in the entire domain. By the assumption that the function only has outputs in $\Nat$, this negative contriubtion must be canclelled out by some positive contribution. This means  there must be some $j$ such that the query $\psi \in \Phi$ has a positive coefficient and either: (i) has arity zero and is true in the entire domain; or (ii) has arity one and is true for at least one position in every string from the domain. In case (i), we can remove both queries $\varphi$ and $\psi$, and the output of the function will be unchanged, while the induction parameter is decreased. In case (ii), we can modify the query $\psi$ so that it does not select some chosen element, say the leftmost element that $\psi$ originally chose. This way the contribution of $\psi$ is decremented by one, and we can then remove the query $\varphi$ to balance out the output, thus decreasing the induction parameter.
    \end{proof}

It remains to prove~\eqref{eq:nat-to-int-weaker}, which says that if a $\Int$-regular function has lower bounded outputs, then it is almost $\Nat$-regular. The advantage of almost $\Nat$-regular functions is that they are more robust, and in particular they are amenable to our proof strategy, which is to use a tool from algebraic language theory, called the Factorisation Forest Theorem~\cite[Theorem 6.1]{simonFactorizationForestsFinite1990}. We now describe this tool.

\paragraph*{The Factorisation Forest Theorem.} In the proof below, it will sometimes be more convenient to define queries using monoids instead of \mso formulas. We assume that the reader is familiar with the definition of a monoid and  a monoid homomorphism, see~\cite[Section 1]{bojanczyk_recobook}. Here is how a monoid homomorphism $h : \Sigma^* \to M$ can be used to recognise a query. We describe this notion only for arities zero and one, which are the only ones that we need. For arity zero, a query  is recognised
 by $h$  if for every input string $w$, the output (which is true or false) depends only on the monoid element $h(w)$ produced by the  monoid homomorphism. For arity one, we say that  $\varphi(x)$ is recognised by $h$ 
    if for every input string $w$,  whether a position $x$ in the  string $w$ is selected by $\varphi(x)$ depends only on the following three pieces of information: 
    \begin{enumerate}[(a)]
        \item the value of $h$ on the part of the string before $x$;
        \item the label of $x$;
        \item the value of $h$ on the part of the string after $x$.
    \end{enumerate}
    A standard compositionality argument for \mso, see \cite[Section 2.1]{bojanczyk_recobook}, shows that every \mso query of arity zero or one is recognised by some homomorphism into a finite monoid,  and the converse is also true, i.e.~if a query is recognised by a homomorphism into some finite monoid, then it is can be expressed in \mso. (The result is also true for higher arities, but we do not use it here.)
    % We extend the notion of recognisability to $\Int$-regular functions: such a function is said to be recognised by a monoid homomorphism if all the  queries in the corresponding linear combination are recognised by this monoid homomorphism. Again, every $\Int$-regular function is recognised by some homomorphism into a finite monoid.

    Thanks to using monoids, we can appeal to the  Factorisation Forest Theorem, which we now describe. The general idea behind the theorem is to recursively split the string into simpler factors, with a special role being played by factors whose corresponding monoid element is idempotent, i.e.~it satisfies $aa=a$.  
    
    \begin{definition}
        [Height]     Fix a monoid homomorphism $h : \Sigma^* \to M$. For an input string $w$,  define its \emph{height (with respect to $h$)} to be the smallest number $k \in \set{0,1,\ldots}$ that can be obtained as follows:
\begin{enumerate}
    \item the empty string and single letters have  height $0$;
    \item if $w_1,w_2$ have height  $\leq k$, then their concatenation $w_1w_2$ has height  $\leq k+1$;
    \item if $w_1,\ldots,w_n$ have height $ \leq k$, and all these strings are mapped by $h$ to the same monoid element which is furthermore idempotent, 
 then the concatenation $w_1 \cdots w_n$ has height  $\leq k$.
\end{enumerate}
    \end{definition}

In other words, we can concatenate words without increasing the height, but only if the concatenated words are mapped by the homomorphism to the same idempotent. 

\begin{theorem}
    [{Factorisation Forest Theorem \cite[Theorem 6.1]{simonFactorizationForestsFinite1990}}]  For every monoid homomorphism $h : \Sigma^* \to M$ into a finite monoid, there is some $k \in \set{0,1,\ldots}$ such that every string has height at most $k$.
\end{theorem}

Using the Factorisation Forest Theorem, we prove implication~\eqref{eq:nat-to-int-weaker}, and therefore also \cref{thm:int-to-nat}. The proof is  by induction on the height of input strings, as stated in the following lemma.      

\begin{lemma}\label{lem:induction-on-height}
    Let $f : \Sigma^* \to \Int$ be a lower bounded counting function which is recognised by a  monoid homomorphism $h : \Sigma^* \to M$, which means that $f$ can be decomposed into a linear combination, with coefficients in $\Int$, of query counting functions that are recognised by $h$ and have arity zero or one. For every monoid element $a \in M$ and every $k \in \set{0,1,\ldots}$, the function $f$ is almost $\Nat$-regular when its domain is restricted to the set
    \begin{align*}
    L(k,a) = \setbuild{ w \in \Sigma^*}{$w$ has height at most $k$ and $h(w) = a$}.
    \end{align*}
\end{lemma}

Before proving the lemma, we use it to conclude the proof of implication~\eqref{eq:nat-to-int-weaker}. Suppose that $f$ is $\Int$-regular, and has lower bounded outputs. Choose some monoid homomorphism $h$ that recognises it. By the above lemma, $f$ is almost $\Nat$-regular when its domain is restricted to input strings that have some specific value in the monoid, and some specific height. By the Factorisation Forest Theorem, there is a finite bound on the heights, and there are also finitely many values in the monoid. Therefore, the set of all input strings is covered by a finite union of languages of the form $L(k,a)$. By \cref{lem:induction-on-height}, $f$ is almost $\Nat$-regular on each of these languages, and we can apply  \cref{claim:aggregate-two-domains} to aggregate these languages into their union, thus proving that  $f$ is almost $\Nat$-regular on all input strings. This completes the proof of the implication in~\eqref{eq:nat-to-int-weaker}.

\begin{proof}[Proof of \cref{lem:induction-on-height}]
In the proof, it will be convenient to avoid queries of arity zero. Removing all queries of arity zero affects neither the assumption of the lemma (having lower bounded outputs)  nor the conclusion (being almost $\Nat$-regular). Therefore, we will only consider linear combinations of query counting functions of arity one; such linear combinations will be called \emph{constant-free}.
An advantage of such functions is that we can use the following notion of derivative. 



\begin{definition}[Derivative]
    Consider a query counting function $\#\varphi$ of arity one. For two input strings $w_1, w_2$, we define the \emph{derivative} of this function, denoted by $\twoderivative {w_1} {(\#\varphi)}{w_2}$, to be the function which maps an input string $w$ to the number of positions which are selected by $\varphi$ in the string $w_1 w w_2$, and which are in the part $w$.
\end{definition}

% Let $f : \Sigma^* \to \Int$ be a function as in the assumption of the theorem. We say that a $f$ is \emph{constant-free} if it does not use any basic functions that have arity 0, i.e.~all the basic functions have arity 1. We can assume without loss of generality that $f$ is constant-free, since basic functions of arity 0 can be changed into arity 1, by selecting the first position of the input string. (This works for nonempty input strings, but we can handle the empty string separately, since it is only one input.)



One of the two strings $w_1$ or $w_2$ might be empty, in which case we get derivatives like $w_1 (\#\varphi)$ or $(\# \varphi)w_2$. 
There is a caveat to the above definition, which is worth discussing. 
This notion is similar to the derivatives of that were discussed in the proof of the Fliess Theorem, however it is not exactly the same, since the contribution of the words $w_1$ and $w_2$ is subtracted from the result. Therefore, the equality
\begin{align*}
\#\varphi(w_1 w w_2) = 
(\twoderivative {w_1}{(\#\varphi)}{w_1}) (w)
\end{align*}
fails for the derivatives from the above Definition, while it  would be true in the case of the derivatives from the Fliess Theorem. 



The following claim collects the basic properties of derivatives. The proof is a straightforward consequence of the definitions, and is left to the reader.  
\begin{claim}\label{claim:derivatives}
    Let $\counter \varphi$ be a query counting function of arity one. Then
    \begin{enumerate}
        \item \label{it:derivative-add} for every input string we have
         \begin{align*}
\counter \varphi(w_1 w_2) = 
(\twoderivative{}{(\counter \varphi)}{w_1}) (w_2) 
+ 
(\twoderivative{}{(\counter \varphi)}{w_2}) (w_1)
\end{align*}
        % \item  \label{it:derivative-lower-bounded} if $\counter \varphi$ has lower bounded outputs, then so are its derivatives;
        \item \label{it:derivative-homo} each derivative of $\counter \varphi$ is a query counting function of arity one;
        \item \label{it:derivative-homo2} if $\counter \varphi$ is recognised by a monoid homomorphism $h$, then so are  its derivatives;
                \item \label{it:derivative-depends} the derivative $\twoderivative{w_1} {\counter \varphi} {w_2}$ depends only on $h(a_1)$ and $h(a_2)$;
                \item \label{it:derivative-finite} there are finitely many derivatives;
        \item  \label{it:derivative-associative} taking derivatives is associative in the following sense
        \begin{align*}
        v_1 (w_1 (\counter\varphi) w_2) v_2 = (v_1 w_1) (\counter\varphi) (w_2 v_2).
        \end{align*}
    \end{enumerate}
\end{claim}


We extend the notion of derivatives to linear combinations of query counting functions,  by distributing out the linear combinations:
\begin{align*}
w_1(\sum_i \alpha_i \cdot  \counter{\varphi_i}) w_2 = \sum_i \alpha_i \cdot (w_1(\counter{\varphi_i}) w_2).
\end{align*} 
In order to be meaningful,  the linear combination must be constant-free, i.e.~it must use only queries of arity one, since only such queries have  defined derivatives. 
The results of \cref{claim:derivatives} extend to such  linear combinations in a straightforward manner. Thanks to item~\ref{it:derivative-depends} in the claim, we can use monoid elements instead of strings in the derivatives, i.e.~for a constant-free $\Int$-regular function $f$ that is recognised by a monoid homomorphism $h$, and monoid elements $a_1, a_2$, we write $a_1 f a_2$ for the derivative which arises from some (equivalently, every) strings $w_1$ and $w_2$ that are mapped to $a_1$ and $a_2$. 


The following claim shows that the property of having lower bounded outputs is preserved by taking derivatives.

\begin{claim}\label{claim:derivative-lower-bounded}
    Let $f$ be a $\Int$-regular function which is constant-free. If $f$ has lower bounded outputs, then the same is true for all of its derivatives.
\end{claim}
\begin{proof}
    The words $w_1$ and $w_2$ in the derivative can change the output of the function only by a constant amount.
\end{proof}




We are now ready to prove the lemma. The proof is by induction on $k$, i.e.~on the height of input strings. In the induction base of $k=0$, there is not much  to do, since each set $L(k,a)$ is finite, and every function is almost $\Nat$-regular on a finite domain. 

It remains to prove the induction step. 
Suppose that we have proved the lemma for  $k-1$, and we want to prove it for $k$.  We want to show that for every monoid element $a$, the function $f$ is almost $\Nat$-regular when its domain is restricted $L(a,k)$. We consider two cases, depending on whether $a$ is an idempotent or not.


\paragraph*{Non-idempotent.} Suppose first that $a$ is not an idempotent. In this case, by definition of  height we have 
\begin{align*}
L(a,k) = \bigcup_{a_1,a_2} L(a_1,k-1) \cdot L(a_2,k-1),
\end{align*}
where the sum ranges over pairs of monoid elements such that $a_1 a_2 =a$.
Thanks to \cref{claim:aggregate-two-domains}, it is enough to prove that $f$ is almost $\Nat$-regular when restricted to each  summand. Fix one summand, which corresponds to some monoid elements $a_1,a_2$. Consider an input string $w$ of type $a$ (we use the name type for the output of the homomorphism) which admits  a factorisation $w = w_1 w_2$ into strings of types $a_1$ and $a_2$, respectively. This factorisation is not necessarily unique. However, there is a unary query $\varphi(x)$ which chooses one such  decomposition, if it exists, and then describes it by selecting the first positions from the second part $w_1$. One way of doing it is to consider a \emph{leftmost factorisation}, where the part $w_1$ has  minimal length. 
 By item~\ref{it:derivative-add} of \cref{claim:aggregate-two-domains}, we know that 
\begin{align*}
f(w_1 w_2) = \myunderbrace{(f a_2)}{$f_1$}(w_1) + \myunderbrace{(a_1  f)}{$f_2$}(w_2).
\end{align*}
The functions $f_1$ and $f_2$ are derivatives of $f$. They have lower bounded ouptuts thanks to \cref{claim:derivative-lower-bounded}, and  therefore we can use the induction assumption to conclude that $f_1$ is almost $\Nat$-regular on $L(a_1,k-1)$ and $f_2$ is almost $\Nat$-regular on $L(a_2,k-1)$. Since the leftmost factorisation $w = w_1 w_2$ can be defined in \mso, we can modify the functions $f_1$ and $f_2$ so that they are applied only to the parts $w_1$ and $w_2$ in the leftmost factorisation. The result is a sum of two almost $\Nat$-regular functions, and is therefore also almost $\Nat$-regular.

\paragraph*{Idempotent.} We are left with the case when $a$ is idempotent. In this case, except for the summands discussed in the non-idemoptent case,  we also have to deal with a summand of the form 
\begin{align*}
(L(a,k-1))^*.
\end{align*}
The summands from the non-idempotent case are treated in the same way as previously, so we focus on this new summand.
For an input string $w$ in the new summand, consisder a factorisation 
\begin{align*}
w = w_1 \cdots w_n,
\end{align*}
where all factors belong to $L(a,k-1)$. As previously, this factorisation is not necessarily unique. However, also as previously,  we can make it unique, by requiring that $w_1$ has minimal length, then $w_2$ has minimal length, and so on. Call this the \emph{leftmost factorisation}. The leftmost factorisation is again definable  in \mso, i.e.~there is a query $\varphi(x)$ that selects the first positions of the factors $w_1,\ldots,w_n$.  We have 
\begin{align*}
f(w) = (fa)(w_1) + \myunderbrace{(afa)(w_2) +  \cdots + (afa)(w_{n-1})}{in this part, thanks to idempotence, \\ the same derivative $afa$ is always applied} + (a f)(w_n).
\end{align*}
We will be interested in the partition of $w$ into three parts: the left part $w_1$, the middle part $w_2,\ldots,w_{n-1}$, and the right part $w_n$. The partition into these three parts, in the leftmost factorisation,  can be defined in \mso.  We can use the same reasoning for the left and right part is in the non-idempotent case. We are left with the middle part. 

Using the induction assumption, we know that the contribution of each string $w_i$ in the middle part is almost $\Nat$-regular. However, we cannot simply add these contributions to each other, since they might have negative components of arity zero, and these components might accumulate in an unbounded way. Fortunately, this accumulation cannot arise, because the strings in the middle part will have necessarily non-negative contributions, as shown in the following claim. 



\begin{claim}
    If $a\in M$ is an idempotent, then the derivative $afa$ has non-negative outputs when applied to strings with type $a$. 
\end{claim}
\begin{proof}
    Since $f$ has lower bounded outputs, the same is true for $afa$, thanks to \cref{claim:derivative-lower-bounded}.
    Suppose, toward a contradiction, that there is some string $w$ of type $a$ which has negative output under the derivative $afa$. By concatenating many copies of this string with itself, we get an output 
    \begin{align*}
    (afa)(w^n) = (afa)(w) + (afa)(w) + \cdots + (afa)(w) = n \cdot (afa)(w),
    \end{align*}
    thus contradicting the assumption that $afa$ has lower bounded outputs.
\end{proof}

By the above claim, the derivative $afa$ is not only lower bounded, but it has non-negative outputs. Therefore, we can apply \cref{lem:remove-almost} to conclude that $afa$ is $\Nat$-regular, i.e.~it has a decomposition where only positive coefficients are used. Without loss of generality, we can assume that all coefficients are 1, i.e.~the decomposition is 
\begin{align*}
afa = \sum_{\varphi \in \Phi} \counter \varphi.
\end{align*}
Let us investigate the contribution of each query counting function $\counter \varphi$, when computing the output of the middle part, i.e. the value 
\begin{align*}
    (\counter \varphi)(w_2) +  \cdots + (\counter \varphi)(w_{n-1}).
\end{align*}

Suppose first that $\varphi_i$ has arity zero. Since it is recognised by the homomorphism $h$, and all strings $w_2,\ldots,w_{n-1}$ have the same type, this contribution is eiter $0$ or $n-2$, depending on whether $\varphi_i$ is true in strings of type $a$. This value can be computed by a query counting function, since the decomposition of the middle part into strings $w_2,\ldots,w_{n-1}$ can be defined in \mso. 

Suppose now that $\varphi_i$ has arity one. If we take a position $x$ inside some string $w_i$ from the middle part, then this position will satisfy $\varphi_i$ inside $w_i$ if and only if the same position $x$, but inside the entire string $w$, satisfies the same query $\varphi_i$. Therefore, the contribution of $\varphi_i$ can be obtained by counting the number of positions in $w$ that satisfy $\varphi_i$ and belong to the middle part. This is a query counting function. 

This concludes the idempotent case in the induction step, and thus also the proof of \cref{lem:induction-on-height}, \cref{thm:int-to-nat} and \cref{thm:unary-string-to-string}. 
\end{proof}





\subsection{Semirings that are not fields}
\label{sec:beyond-fields}
In our discussion so far, a prominent role was played by weighted automata over a field. However, weighted automata make also sense in a more general setting, where a semiring is used instead of a field. We discuss this more general setting in the present section.

When talking about weighted automata over a semiring, it  will be more convenient to work with an alternative presentation of weighted automata.  Instead of using a deterministic automaton that operates on tuples in $\domain^k$, we use a nondeterministic automaton with weights on states and transitions, as defined below. 
\begin{definition}
    [Weighted automaton, nondeterministic presentation] 
    \label{def:weighted-automaton-nondeterministic}
    A \emph{weighted automaton} consists of a finite input alphabet $\Sigma$,  a finite set of states $Q$, and functions: 
    \begin{align*}
    \myunderbrace{I : Q \to \domain}{initial}
    \quad
    \myunderbrace{F : Q \to \domain}{final}
    \quad
    \myunderbrace{\Delta : Q \times \Gamma \times Q \to \domain}{transitions}.
    \end{align*}
\end{definition}
 A run of this automaton is defined in the usual way: it is a sequence of transitions, one for each input letter, such that consecutive transitions agree on the connecting states. The weight of a run is the product of: (1) the initial weight of its source state; (2) the weights used by its transitions; and (3) the final weight of its target state. If the semiring is commutative, the order of transitions is unimportant when taking the product. For an input string, the output of the automaton is the sum of weights of all runs over this automaton.  For fields, the model from Definition~\ref{def:weighted-automaton-nondeterministic}  defines the same functions as the model from Definition~\ref{def:weighted-automaton}, see~\cite[Lemma 8.3]{bojanczyk_automata_2025}. The equivalence also extends to semirings, for a suitable adaptation of Definition~\ref{def:weighted-automaton}. For the purposes of this seciton, we use the nondeterministic  model described from Definition~\ref{def:weighted-automaton-nondeterministic}.



  Recall Conjecture~\ref{conj:regular-continuous}, which says that a string-to-string function is regular if and only if it is weighted continuous, with respect to weighted automata over a field. One could consider variants of this conjecture for semirings.  For a class $\algclass$ of semirings, let us define \emph{weighted continuity over $\algclass$} to be the same as in Definition~\ref{def:weighted-continuity}, except that instead of fields, we have semirings from the class $\algclass$. This section is devoted to studying   variants of Conjecture~\ref{conj:regular-continuous} for  important classes of semirings, namely:
\begin{align}
\text{weighted continuous over finite fields}
& \iff
\text{regular}
\label{eq:weighted-continuous-finite-fields}\\
\text{weighted continuous over fields}
& \iff
\text{regular}
\label{eq:weighted-continuous-fields}\\
    \text{weighted continuous over commutative semirings}
& \iff
\text{regular}
\label{eq:weighted-continuous-commutative-semirings}\\
\text{weighted continuous over all semirings}
& \iff
\text{regular}
\label{eq:weighted-continuous-all-semirings}
\end{align}

As the class of semirings grows, the left condition becomes stronger, and thus the implication $\implies$ becomes easier to prove, while the converse implication $\impliedby$ becomes harder to prove. The following diagram explains what we know and conjecture about the implications (known facts are in black, and conjectures are in red).

\begin{eqnarray*}
\text{weighted continuous over finite fields}
& \raisebox{-3pt}{\stackon[1pt]{\(\centernot\implies\)}{\(\impliedby\)}}
& \text{regular}\\ [1.5ex]
\text{weighted continuous over fields}
& \raisebox{-3pt}{\stackon[3pt]{\(\red\implies\)}{\(\impliedby\)}}
&  \text{regular}\\ [1.5ex]
    \text{weighted continuous over commutative semirings}
& \raisebox{-3pt}{\stackon[3pt]{\(\red\implies\)}{\(\impliedby\)}}
& \text{regular}\\ [1.5ex]
\text{weighted continuous over all semirings}
& \raisebox{-3pt}{\stackon[1pt]{\(\red\implies\)}{\({\centernot \impliedby}\)}}
& \text{regular}
\end{eqnarray*}

To justify the description above, we need to show that: (a) there is a function that is continuous over finite fields, but which is not regular; (b) there is a function that is regular, but not continuous over all semirings; and (c) if a function is regular, then it is continuous over commutative semirings. These are justified in Example~\ref{ex:not-regular-but-continuous-over-finite-fields}, Example~\ref{ex:all-semirings} and Theorem~\ref{thm:regular-continuous-commutative-semirings} below.


\begin{myexample}
    \label{ex:not-regular-but-continuous-over-finite-fields}
    We give a counterexample that witnesses
    \begin{align*}
    \text{weighted continuous over finite fields}
    & \centernot \implies \text{regular}.
    \end{align*}
    A weighted automaton over a finite field is the same thing as a regular language. More precisely if $\domain$ is a finite field, then a  function 
    \begin{align*}
    g : \Gamma^* \to \domain
    \end{align*}
    is computed by a weighted automaton over $\field$ if and only if for every field element $x \in \field$, the set of strings that are mapped to $x$ is a regular language. This is because a finite automaton can simulate a weighted automaton over a finite field, and conversely a deterministic finite automaton can be seen as a special case of a weighted automaton over any field with at least two elements. Therefore, being weighted continuous over finite fields is the same thing as preserving regularity under inverse images. This is an established notion of continuity for functions, as we have explained before.  As shown in \cite{bojanczykTitoRegular23}, there are functions which are not regular but still preserve regularity under inverse images, in fact there are uncountably many such functions.
\end{myexample}


\begin{myexample} 
    \label{ex:all-semirings}
    We give a counterexample that witnesses
    \begin{align*}
    \text{weighted continuous over all semirings}
    & {\centernot \impliedby} \text{regular}.
    \end{align*}
    Fix a two-letter alphabet $\set{a,b}$, and  consider the semiring 
    \begin{align*}
    \domain = \pfin(\set{a,b}^*),
    \end{align*}
    in which elements are finite sets of words, addition is union, and multiplication is concatenation. Over this semiring, a weighted automaton 
    \begin{align*}
    f : \Sigma^* \to \domain
    \end{align*}
    gives a function from input strings in $\Sigma^*$ to finite sets of output strings in $\set{a,b}^*$. This function can be viewed as a binary relation between input strings in $\Sigma^*$ and output strings in $\set{a,b}^*$. The binary relations that arise this way are exactly the \emph{rational relations}~\cite[Chapter IX]{Eilenberg74}, i.e.~relations that can be  described by a nondeterministic automaton with output strings on transitions. If regular functions would be countinuous over the semiring $\domain$, then rational relations would be closed under precomposition with regular functions. This is not the case. For example, the identity function $\set{a,b}^* \to \set{a,b}^*$ is a rational relation, but if we precompose it with the reverse function, which is regular, we get the reverse function again, which is not a rational relation. The non-rationality of reverse can formally be proved by appealing to the decidable characterisation of rationality in \cite[Theorem 1]{filiot2013two}.
\end{myexample}

\begin{theorem}\label{thm:regular-continuous-commutative-semirings}
    If a string-to-string function is  regular, then it is weighted continuous over commutative semirings.
\end{theorem}
\begin{proof}

    Having described the model, let us prove the lemma.
        Let $\domain$ be a commutative semiring, and  consider  a composition 
    \[
    \begin{tikzcd}
    \Sigma^* 
    \ar[r,"f"]
    & 
    \Gamma^*
    \ar[r,"g"]
    &
    \domain
    \end{tikzcd}
    \]
    where the first function $f$ is computed by a two-way automaton, and the second function $g$ is computed by a weighted automaton over $\domain$. We need to show that the composition $f;g$ can be computed by a weighted automaton over $\domain$. To prove the lemma, we use a straightforward product construction. To describe the construction, we use \emph{weighted graphs}. Such a graph is a directed graph, where every directed edge has a weight, and every vertex has two weights: an initial and final weight. For an input string $w \in \Sigma^*$, consider the following weighted graph, which call the \emph{run graph}:
    \begin{itemize}
        \item \textbf{Vertices.} Vertices are triples of the form $(p,x,q)$ where $(p,x)$ is a configuration of the two-way automaton for $f$ and  $q$ is a state of the weighted automaton for $g$. Recall that a configuration of the two-way automaton consists of a state $p$, and a position  $x$  in the string obtained from $w$ by adding endmarkers $\vdash$ and $\dashv$ to both ends. 
        \item \textbf{Edges.} In the graph, there is an edge 
        \begin{align*}
        (p,x,q) \xrightarrow{a} (p',x',q')
        \end{align*}
        if the two-way automaton has a single transition which goes from configuration $(p,x)$ to configuration $(p',x')$.
        \item \textbf{Weights of edges.} Consider an edge as in the previous item. The weight of this edge is defined as follows. Let $u$ be the output string, possibly empty, which is produced by the two-way automaton in the transition that goes from $(p,x)$ to $(p',x')$. The weight of the edge is defined to be the  sum of weights of all runs in the weighted automaton that go from $q$ to $q'$ and have input string $u$. In the special case when $u$ is empty, this will mean that $a$ is the $1$ of the semiring, i.e.~the neutral element of multiplication, because there is a unique run over the input string, and this run has weight $1$. 
        \item \textbf{Initial weights of vertices.} The initial weight of a vertex $(p,x,q)$ is zero if $(p,x)$ is not the initial configuration of the two-way automaton, and otherwise it is the initial weight of  the state $q$.
        \item \textbf{Final weights of vertices.} The final weight of a vertex $(p,x,q)$ is zero if $(p,x)$ is not the final configuration of the two-way automaton, and otherwise it is the final weight of the state $q$.
    \end{itemize}
The weight of a path in this graph is defined to be the product of: (1) the initial weight of the first vertex; (2) the weights of all edges on the path; and (3) the final wight of the last vertex. It is  easy to see that for an input string $w \in \Sigma^*$, the output of the composition $f;g$ is the same as the sum of weights of all paths in the corresponding run graph.  (This sum finite and therefore well-defined, since the run graph has finitely many paths by virtue of being  acyclic. This is because the two-way automaton is not allowed to loop.) To complete the proof of the theorem, it is enough to show the following claim.
\begin{claim}
    There is a weighted automaton which inputs a string $w \in \Sigma^*$, and outputs the sum of weights of all paths in the corresponding run graph.
\end{claim}
\begin{proof}
    This claim is the place where we use commutativity of the semiring. Consider a run graph, as in the following picture (to avoid clutter, we only show the vertices, and not the weights):
    \mypic{3}
    Consider a path in the run graph. Since the run graph is necessarily acyclic, such a path is uniquely described by the set of edges that it uses, as in the   following picture:  
    \mypic{4}
    A run graph together with a distinguished path can be viewed as a string, which we call its \emph{string representation}. Each letter in the string representation describes a single column of the picture above. (The string representation includes the weights of the vertices and edges, which are omitted in the picture.)     The function which inputs a string from $\Sigma^*$ and returns the set of all string representations of paths in the corresponding run graph is easily seen to be a rational relation, in the sense that was  discussed in Example~\ref{ex:all-semirings}. Let this relation be 
    \begin{align*}
    R \subseteq \Sigma^* \times \Delta^*,
    \end{align*}
    where $\Delta$ is the alphabet used for string representation of paths inside run graphs. The function in the present claim is the same as 
    \begin{align*}
    w \in \Sigma^* 
    \quad \mapsto \quad 
    \sum_{\substack{v \in \Delta^* \\ (w,v) \in R}} \text{weight of path described by $v$}.
    \end{align*} 
    Weighted automata are closed under sums as above~\cite[Lemma 8.12]{bojanczyk_automata_2025}, and therefore to complete the proof of the claim it is enough to show a weighted automaton for the function
    \begin{align*}
    v \in \Delta^* 
    \quad \mapsto \quad 
    \text{weight of path described by $v$}.
    \end{align*}
    This weighted automaton is trivial: it simply mutiplies the weights of all highlighted edges, together with the initial weight of the first vertex and the final weight of the last vertex. Here, commutativity of the semiring is crucial, since the multiplication will be done in a left-to-right fashion, which will typically be inconsistent with the order of vertices on the path. 
\end{proof}

\end{proof}

We finish this section by dicussing the relationship between two implications, both of which we conjecture to be true.
\begin{align}
\text{weighted continuous over fields}
& \iff
\text{regular}
\label{eq:weighted-continuous-fields-again}\\
    \text{weighted continuous over commutative semirings}
& \iff
\text{regular}
\label{eq:weighted-continuous-commutative-semirings-again}
\end{align}
Since the implications $\impliedby$ are known to be true by \cref{thm:regular-continuous-commutative-semirings}, the more difficult equivalence is the first one, which has a weaker condition on the left side. 
For all we know, only the harder  first equivalence has any bearing on protocols, since we have established continuity for protocols only in the field case, see \cref{lem:postcomposition-weighted-automaton}. Even though it might not be connected to protocols, the  easier second equivalence would still be interesting on its own, as a characterisation of the regular string-to-string functions.
% LTeX: language=en
\section{Infinite alphabets}
\label{sec:infinite-alphabets}
\AP
In this section, we present a variant of our model which deals with an input
alphabet. This direction is rooted in the tradition of language theory for
infinite alphabets, which dates back to the work of Kaminski and
Francez~\cite{kaminskiFiniteMemoryAutomata1994}, and has been developed in many
subsequent papers, see e.g.~the
survey~\cite{bojanczykOrbitFiniteSetsTheir2017}. The general idea is that we
have an infinite alphabet $\atoms$ (whose elements are called \intro{atoms}), and the
languages  refer only to equality between letters, as in the following examples
\begin{align}
\setbuild{ w \in \atoms^*}{the first letter is equal to the last letter}
\label{eq:first-last}
\\
\setbuild{ w \in \atoms^*}{some letter appears at least twice}
\label{eq:some-twice}
\end{align}

There are numerous models of automata for such languages, which typically
involve some kind of finite memory, as well as registers that store letters
from $\atoms$. For example, the language~\eqref{eq:first-last} is recognised by
an automaton which loads the first letter into a register, and then toggles
acceptance depending on comparison of the register with the current input
letter. The language~\eqref{eq:some-twice} is recognised by an automaton which
nondeterministically guesses a position, puts its letter into a register, and
then waits for this letter to appear again. 

Numerous models for infinite alphabets have been proposed in the literature,
see Figure~\ref{fig:automata-infinite-alphabets} which contains a sample of
seventeen models. Interestingly, all of those models are pairwise
non-equivalent. This sharply contrasts with the finite-alphabet case, where
virtually all models coincide and capture the regular languages. 

In this section, we describe an infinite-alphabet version of our two-party
protocols. The motivation for this study is twofold: (a) a search for a
canonical model of regular languages for infinite alphabets; and (b)
mathematical interest. Regarding the point (a), we hope that the adaptability
of two-party protocols to various settings will help us  find a canonical model
of regular languages for infinite alphabets. This seems to be at least
partially successful, since there is evidence --- which we present in this
section --- that the protocols are equivalent to one of the existing automaton
models\footnote{
  Thus avoiding proliferation of standards, humanistically depicted in a famous XKCD comic.
}, namely unambiguous register automata, see item~\ref{it:orbit-finite-unamb} in
Figure~\ref{fig:automata-infinite-alphabets}. If true, this
equivalence would be unexpected, since there does not seem to be any syntactic
connection between unambiguous register automata and protocols. Regarding point
(b), one of the exciting features of our protocol model for infinite alphabets
is that the interaction between the two parties becomes essential, and the
protocol cannot be reduced to the one-round case as in
\cref{lemma:one-round-reduction-general}.

\begin{figure}
    \begin{enumerate}
    \item deterministic register automata~\cite[Definition 3]{kaminskiFiniteMemoryAutomata1994}
    \item nondeterministic register automata~\cite[Definition 1]{kaminskiFiniteMemoryAutomata1994}
    \item nondeterministic register automata with guessing~\cite[Definition 2.7]{bojanczyk_slightly}
    \item weighted register automata over the two-element field~\cite[Definition 3.1]{orbitFiniteVectorTheoretics}
    \item two-way deterministic register automata~\cite[Definition 5]{kaminskiFiniteMemoryAutomata1994}
    \item two-way nondeterministic register automata~\cite[Definition 2.1]{nevenFiniteStateMachines2004}
    \item alternating register automata~\cite[p.~16:8]{lazicDemri09}
    \item alternating register automata with one register~\cite[p.~16:19]{lazicDemri09}
    \item \label{it:orbit-finite-unamb} unambiguous register automata~\cite[Section 5]{colcombet2015unambiguity}
    \item register automata with pebbles~\cite[Section 2.2]{nevenFiniteStateMachines2004}
    \item \label{it:single-use} single-use register automata~\cite[Definition 2]{bojanczykstefanski2020}
    \item data automata~\cite[Section 4.2]{bojanczykTwovariableLogicData2011}
    \item class automata~\cite[Section III]{bojanczykExtensionDataAutomata2010} 
    \item regular expressions~\cite[Definition 2]{regexpKaminskiTan2004}
    \item three other kinds of regular expressions~\cite[Sections 4, 5, 6]{regexpLibkin2015}
    \item yet another kind of regular expressions~\cite[Section 5]{KleeneNominal2019}
    \item monadic second-order logic with equality~\cite[Section 2.4]{nevenFiniteStateMachines2004}
    % \item session automata
\end{enumerate}
    \caption{A non-exhaustive list of models of automata for infinite alphabets. All models in the list are pairwise non-equivalent. In contrast, for finite alphabets, all models in this list are equivalent, and define exactly the regular languages. }
    \label{fig:automata-infinite-alphabets}
\end{figure}




% . Regarding the point (b), we note that the theory of automata for infinite alphabets is closely related to the theory of orbit-finite sets~\cite{bojanczyk_slightly}, which builds on the ideas of nominal sets~\cite{PittsAM:nomsns}. This theory has many interesting mathematical aspects, and we hope that our model will be useful in this context.



\subsection{Protocols for an infinite alphabet}
\label{sec:protocols-infinite-alphabet}
\AP
We now give a more detailed description of our model. 
As explained in the introduction, we only care about languages that are closed 
under \intro{permutations of the alphabet}, according to the following definition. 

\begin{definition}[Equivariant language] \label{def:equivariant-language}
  \AP
  A language $L \subseteq \atoms^*$ is called \intro[equivariant language]{equivariant} if 
  \begin{align*}
  w \in L \quad \iff \quad \pi(w) \in L
  \end{align*}
  holds for every permutation $\pi$ of the alphabet $\atoms$.
\end{definition}

Examples of \kl{equivariant languages} include the languages
in~\eqref{eq:first-last} and~\eqref{eq:some-twice}. On the other hand, the
language ``the first letter is a vowel'' or ``the letters are strictly
increasing'' are not \kl{equivariant}, since there is no such thing as a vowel,
or an ordering of the letters. The principle of \kl[equivariant
language]{equivariance} will also be applied to protocols, as described below.


\AP To define \intro{(simple) equivariant protocols} for infinite alphabets, we use the
same kind of \kl{protocols} as in Definition \ref{def:two-party-protocol-boolean},
except that apart from bits, the parties can also send letters from the
alphabet $\atoms$. It follows that the allowed set of messages is
$\set{\text{true, false}} + \atoms$, i.e.~the disjoint union of the Booleans
and the input alphabet. Similarly, to
Definition~\ref{def:two-party-protocol-boolean}, the number of rounds is a
fixed number $k$ and in the $i$-th round each party chooses a new message
according a strategy which is a function of the following type: 

\begin{align*}
\myunderbrace{\atoms^*}{local \\ string} \times \myunderbrace{(\set{\text{true, false}} + \atoms)^{i-1}}{messages received \\ in previous rounds} 
\to
\myunderbrace{\set{\text{true, false}} + \atoms}{message sent \\ in this round}
\end{align*}
In the last round, Bob must send a bit, and this bit is the output of the \kl[simple equivariant protocol]{protocol}.
An important restriction in the protocol is that the strategies of both parties must be \emph{equivariant} in the following sense: 
a strategy $\sigma$ is equivariant if for every permutation $\pi$ of $\atoms$ and for every $i$, it satisfies the following condition:
\begin{align*}
\sigma(w, m_1, \ldots, m_{i-1}) = m_i 
\quad \Rightarrow \quad
\sigma(\pi(w), \pi(m_1), \ldots, \pi(m_{i-1})) = \pi(m_i).
\end{align*}
In the above, $\pi$ is applied to messages in the natural way -- 
it modifies the \kl{atoms} and leaves the Booleans unchanged.

\begin{myexample}[Repetitions cannot be detected]
\label{ex:protocol-not-repetitions}
    Having formally defined the protocols for infinite alphabets,
    we can now revisit Example~\ref{ex:reg-ndet-too-strong} from the introduction
    and prove that the language ``some letter appears at least twice'' cannot be computed by a
    \kl[simple equivariant protocol]{protocol}. 
    Suppose, towards a contradiction, that there is a protocol with
    $k$ rounds that computes this language. Consider an input string with
    $2k+2$ pairwise different letters, split so that Alice and Bob get $k+1$
    letters each. In the execution of the protocol there are at most $k$ \kl{atoms}
    which are sent as messages. In particular, there must be some atom $a$ that
    appears in Alice's part of the input string, but is not sent as a message,
    and similarly there must be some atom $b$ that appears in Bob's part of the
    input string, but is not sent as a message. Consider an \kl{atom permutation}
    $\pi$ which swaps $a$ with $b$. If we apply this atom permutation to
    Alice's part of the string (but not Bob's), then the communication history
    will remain unchanged. In particular, the output of the protocol will be
    the same on both inputs. However, after applying this permutation, the
    input string has a repetition, unlike the original one. 
\end{myexample}


\subsubsection{Orbit-finite sets}
\label{sec:orbit-finite-sets}
\AP
In this section, we do a more systematic analysis of automata models for
infinite alphabets, and their relationship to our protocols. As an organising
principle, we use the approach of orbit-finite sets,
see~\cite{bojanczykOrbitFiniteSetsTheir2017} for a longer survey, which is a
generalisation of finite sets suitable for infinite alphabets sets. Using this
notion, we can lift any model of computation that uses finite sets, to a model
that uses orbit-finite sets, which allows us for a clean comparison of the two
setting. For the purposes of this paper, we use a simplified definition of
orbit-finite sets, which is sufficient for our purposes\footnote{
Definition~\ref{def:orbit-finite-sets} is weaker than the usual notion of orbit-finite sets~\cite[Section 5]{bojanczyk_slightly}; in fact it is the special case of the usual notion that is called \emph{polynomial orbit-finite sets} in~\cite[Section 1]{bojanczyk_slightly}.
The stronger  notion that is usually used  allows for two extra features: (a) restricting to equivariant subsets (e.g.~one could limit $\atoms^2$ to pairs which are non-repeating); and (b)  symmetries (e.g.~one could identify pairs in $\atoms^2$ if they agree up to swapping of coordinates, thus yielding unordered pairs). In some cases, the extra features are desirable, in particular they establish a connection with set theory~\cite{blassDedekind2016} and  nominal sets~\cite[Section 5]{PittsAM:nomsns}. However, those features do not play any role in the analysis of protocols and automata and, to avoid technicalities, we use the simpler polynomial version from Definition~\ref{def:orbit-finite-sets}. This simplification is purely technical --- all results continue to hold for the usual notion of orbit-finite sets.
}.
%   instead of the fully general notion of orbit-finite sets, we use a special case, which is called \emph{polynomial orbit-finite sets}. This special case has a more concrete defnition, and is sufficient for our purposes. 
\begin{definition}[Orbit-finite sets] \label{def:orbit-finite-sets}
  \AP
  An \intro{orbit-finite set} is any set of the form 
    \begin{align*}
    \atoms^{d_1} + \cdots + \atoms^{d_n},
    \end{align*}
    for some natural numbers $d_1,\ldots,d_n \in \set{0,1,\ldots}$. 
\end{definition}

\AP
When a summand $\atoms^{d_i}$ uses $d_i =0$, then it  describes a set with
exactly one element, namely the empty tuple. Therefore, \kl{orbit-finite sets}
generalise finite sets, since a  finite set with $n$ elements can be seen as
the orbit-finite set which has $n$ disjoint copies of $\atoms^0$. We will only
be interested in subsets of orbit-finite sets and functions between them that
are \intro{equivariant}, i.e.~invariant under permutations of the \kl{atoms}, in
the following sense:
\begin{align*}
\myunderbrace{x \in X \iff \pi(x) \in X}{equivariant subset $X$ \\ of an orbit-finite set}
\qquad 
\myunderbrace{f(x) = y \iff f(\pi(x)) = \pi(y)}{equivariant function $f$ \\ between two orbit-finite sets}
\end{align*}
We can now discuss various orbit-finite models of computation, by generalising
finite sets to orbit-finite ones, and requiring all subsets and relations to be
\kl{equivariant}. As a first example of this approach, we can revisit the definition
of \kl{simple equivariant protocols} from Section~\ref{sec:protocols-infinite-alphabet}, and
define it in terms of \kl{orbit-finite sets}:

\begin{definition}[Orbit-finite protocol]
    \label{def:orbit-finite-protocol}
    An \intro{orbit-finite Boolean two-party protocol}
    is defined in the same way as in Definition \ref{def:two-party-protocol-boolean}, except that:
  \begin{enumerate}
    \item the input alphabet $\Sigma$, and the message spaces $Q_A$ and $Q_B$ are \kl{orbit-finite}; and 
    \item the \kl{strategies} of both players and the output function are \kl{equivariant}.
  \end{enumerate}
\end{definition}

Indeed, the \kl{simple equivariant protocols} from
Section~\ref{sec:protocols-infinite-alphabet} are the special case of the above
definition where the input alphabet is $\atoms$, and the message spaces are
both equal to 
\begin{align*}
\myunderbrace{\atoms}{letter} + \myunderbrace{\atoms^0}{bit 0} + \myunderbrace{\atoms^0}{bit 1}.
\end{align*}
On the other hand, the \kl[simple equivariant protocol]{special case} is also
equivalent to the \kl[orbit-finite protocols]{general case}, since an element of a general \kl{orbit-finite set}
can be transmitted using a constant number of bits and \kl{atoms} (by first sending
the index of the summand, and then sending the tuple of atoms). It follows that
the protocols from Definition~\ref{def:orbit-finite-protocol} and those from
Section~\ref{sec:protocols-infinite-alphabet} have the same expressive power.
From now on we will use the formalisation from
Definition~\ref{def:orbit-finite-protocol}.

Orbit-finiteness can also be used to define automata. The following definition
has the same expressive power as the standard (nondeterministic and
deterministic) register automata for infinite alphabets
from~\cite{kaminskiFiniteMemoryAutomata1994}; this equivalence was shown
in~\cite[Lemma 6.3]{bojanczykAutomataTheoryNominal2014} and is one of the
original motivations for studying orbit-finiteness.

\begin{definition}
    [Orbit-finite automata]
    \label{def:orbit-finite-automata}
    \AP
    A \intro{nondeterministic orbit-finite automaton} is defined in the same way as a 
    nondeterministic finite automaton, except that all sets are orbit-finite, 
    and all subsets and functions are \kl{equivariant}:
\begin{align*}
    \myoverbrace{
        \myunderbrace{Q}{states} \quad 
        \myunderbrace{\Sigma}{input \\ alphabet}
    }
    {orbit-finite}
    \qquad
    \myoverbrace{
        \myunderbrace{I \subseteq Q}{initial \\ states} \quad 
        \myunderbrace{F \subseteq Q}{final \\ states} \quad 
        \myunderbrace{\Delta \subseteq Q \times \Sigma \times Q}{transitions}
    }{equivariant}.
\end{align*}
A \intro{deterministic orbit-finite automaton}
is the special case which has exactly one initial state, and where the transition relation is a function.
\end{definition}

As stated in Figure~\ref{fig:automata-infinite-alphabets}, deterministic and
nondeterministic orbit-finite automata have different expressive power.
Moreover, as stated in Examples~\ref{ex:reg-det-too-weak}~and~\ref{ex:reg-ndet-too-strong}
none of these models is equivalent to \kl{orbit-finite protocols}:
\kl[of dfa]{deterministic automata} are too weak, and 
\kl[of nfa]{nondeterministic automata} are too
strong. Example~\ref{ex:reg-ndet-too-strong} was already revisited in Example~\ref{ex:protocol-not-repetitions}, and we now revisit Example~\ref{ex:reg-det-too-weak}:  

\begin{myexample} [Deterministic too weak]\label{ex:protocol-not-dofa}
  Every \kl{deterministic orbit-finite automaton} can be simulated by a 
  \kl[of protocol]{protocol}, in the same way as for finite alphabets.
  Alice simulates the run of the automaton on her part of the input, 
  and sends the intermediate orbit-finite state (as an orbit-finite message)
  to Bob who continues the simulation.  
  
  The inclusion is strict: The language ``the last letter appears at least
  twice'' can be computed by a \kl[of protocol]{protocol} (as explained in
  Example~\ref{ex:reg-det-too-weak}), but cannot be recognised by a 
  \kl{deterministic orbit-finite automaton}. Indeed, in order to compare the
  last letter with all previous letters, the automaton would need to store all
  those letters in its orbit-finite state, which is impossible since the number of letters 
  is unbounded. (See \cite[Example~4]{kaminskiFiniteMemoryAutomata1994} for a formal proof).
\end{myexample}

In Definition~\ref{def:orbit-finite-automata}, we have defined \kl{one-way
orbit-finite automata}, which read the input string from left to right. A
natural extension are the two-way automata, which can move their reading head
in both directions according to their transition function. This extension is
particularly natural in the context of protocols, with their two-way
interaction between the communicating parties. However, in the orbit-finite
setting, two-way automata are very strong:

\begin{myexample}[Two-way too strong]\label{ex:protocol-not-2dofa}
    The language ``some letter appears at least twice'' can also be recognised
    by a deterministic two-way orbit-finite automaton~\cite[Example
    18]{bojanczyk_slightly}. Therefore, this automaton model cannot be
    simulated by protocols.

    The reason why two-way orbit-finite automata are so strong is that they can
    make an unbounded number of visits to any given position -- for example the
    automaton for the language ``some letter appears at least twice'' will
    visit the last position a linear number of times (the length of its run is
    quadratic). One idea to tame this power is to consider the bounded-crossing
    variant of two-way automaton, which has a fixed bound $k$ on the number of
    times that the automaton can visit a position \cite[p.~92]{neven2003power}.
    We believe that this model can actually be equivalent to the protocols.
    However, in our conjecture, we will focus on the better studied model
    presented in the following section.
\end{myexample}

\subsection{Unambiguous orbit-finite automata}
\label{sec:unambiguous-orbit-finite-automata}
\AP
As we have shown in Examples~\ref{ex:protocol-not-dofa},
\ref{ex:protocol-not-nofa} and \ref{ex:protocol-not-2dofa},
\kl[of protocol]{protocols} are not equivalent to one-way deterministic or
nondeterministic orbit-finite automata, or their two-way variants automata. So
what is the right automaton model?  We conjecture that the answer is
\intro{unambiguous orbit-finite automata}, i.e.~the special case of
\kl{nondeterministic orbit-finite automata} that have zero or one accepting runs for
every input string.

\begin{conjecture}
    \label{conj:protocols-unambiguous}
    A language over an \kl{orbit-finite} alphabet is computed by 
    an \kl{orbit-finite protocol} if and only if it is recognised by an \kl{unambiguous orbit-finite automaton}.
\end{conjecture}

One corollary of this conjecture would be that \kl{unambiguous orbit-finite
automata} are closed under complement, since \kl[of protocol]{protocols} can
be complemented by flipping the output bit. This corollary has been conjectured
in~\cite[p.9]{colcombet2012forms}, and to the best of our knowledge remains
open, despite apparent claims to the contrary in~\cite[Footnote
5]{colcombet2015unambiguity}.

In this section, we prove implication $\impliedby$ in the conjecture, i.e.~we
show that \kl{orbit-finite protocols} can simulate \kl{unambiguous orbit-finite
automata}. Unlike similar results earlier in this paper, this simulation is
non-trivial. Also, despite the one-way nature of the automata, the simulation
crucially depends on the interactive nature of protocols, i.e.~it requires more
than one round of communication. In particular multi-round protocols cannot be
reduced to one round, as was the case for finite alphabets
(i.e.~\cref{lemma:one-round-reduction-general} is no longer true for the
orbit-finite case).

\begin{theorem}
    \label{thm:unambiguous-to-protocol}
    If a language $L$ over an \kl{orbit-finite} alphabet is recognised by 
    \kl{an unambiguous orbit-finite automaton}, then 
    it is also computed by an \kl{orbit-finite protocol}.
\end{theorem}
\begin{proof}
  For the rest of this proof fix an \kl{unambiguous orbit-finite automaton}, whose state space is the \kl{orbit-finite} set $Q$.
Suppose that the input string is factorized as $w = w_1 w_2$. The general idea of the protocol is that Alice and Bob will jointly
compute the intermediate state (if it exists), i.e.~the state $q$ which satisfies:
\begin{align*}
\myunderbrace{I \xrightarrow{w_1}q}{there is a run over $w_1$\\ from an initial state to $q$} \qquad \text{and} \qquad
\myunderbrace{q \xrightarrow{w_2} F}{there is a run over $w_2$\\ from $q$ to a final state.}
\end{align*}
By \kl{unambiguity}, there is at most one intermediate state, and it exists if and only if the string is accepted.

Observe that Alice can compute the set of states that are reachable from an
initial state by reading her string $w_1$, and Bob can compute the set of
states from which a final state is reachable by reading his string $w_2$. So,
the challenge is to compute their intersection, which is either a singleton
with the intermediate state, or the empty set. Before we explain how to do
this, let us first explain why this is non-trivial, i.e. why Alice cannot send
her set of states to Bob, or vice versa. The problem is that the set of all
reachable subsets of $Q$ is not \kl{orbit-finite}, and therefore it cannot be sent
in a constant number of messages. This issue is illustrated in the following
example.

\begin{myexample}
    \mb{rozwinac to tutaj}
    For example, if we consider the automaton for the language ``the last letter appears at least twice'' that non-deterministically guesses the penultimate appearance of the last letter, 
then the set of all reachable states after reading $w_1$ contains all the atoms that appear in $w_1$, which is unbounded (as it can grow with the length of $w_1$).
 It follows
that the set of all reachable subsets of $Q$ is orbit-infinite\footnote{
This observation can be summarized by saying that orbit-finiteness
is not preserved by the powerset construction. Is one of the main obstacles when working with orbit-finite sets. For example, it is the reason why nondeterministic orbit-finite automata are more powerful than deterministic ones.}
\end{myexample}

To work around the issue explained above, Alice and Bob will engage in interactive communication, which will narrow down the set of possible candidates. At each stage, the set of possible candidates will be represented using orbits, as defined below.

\begin{definition}[Orbit] \label{def:orbit}
  \AP
  For a finite set $S \subseteq \atoms$, the \intro{$S$-orbit} of $q \in Q$ is the following set:
    \begin{align*}
    \setbuild{ \pi(q)}{$\pi$ is a permutation of $\atoms$ such that $\pi(a)=a$ for all $a \in S$}.
    \end{align*}
  The set $S$ is called the \intro{support} of the orbit.
\end{definition}

\begin{myexample}\label{ex:tau-disjoint}
    Let $Q = \atoms^5$, and consider the $\set{\text{{John}, {Eve}}}$-orbit of the following tuple
    \begin{center}
        (\red{John}, Tom, Mary, Tom, \red{Eve})
    \end{center}
    An element of this \kl{orbit} is any tuple of the form 
    \begin{center}
        (\red{John}, $a$, $b$, $a$, \red{Eve})
    \end{center}
    where $a$ and $b$ are distinct \kl{atoms}, which are not \red{John} or \red{Eve}. 
    % Two elements of this orbit are $\tau$-disjoint if the corresponding choices $\set{a_1,b_1}$ and $\set{a_2,b_2}$ are disjoint sets. 
\end{myexample}

\AP As the \kl{support} increases, the \kl{orbit} becomes smaller; in
particular the biggest orbits are the ones with empty support, i.e. the
$\emptyset$-orbits, which we call the \intro{equivariant orbits}. It is not
hard to see that every \kl{orbit-finite} set has a finite number of
\kl{equivariant orbits}~\cite[Lemma 1.4]{bojanczyk_slightly}; in fact this is
the reason for the name. Each \kl{orbit} in an \kl{orbit-finite} set is a
subset of $\atoms^d$ for some $d$. In such an orbit, we partition the
coordinates $\set{1,\ldots,d}$ into two parts: the \intro{fixed coordinates},
which use the \kl{atoms} from the \kl{support}, and the \intro{free
coordinates}, which do not use these atoms. In Example~\ref{ex:tau-disjoint},
the \kl{fixed coordinates} are  the first and last ones, while the \kl{free
coordinates} are the middle three. The \intro[orbit dimension]{dimension} of an
orbit is the number of distinct atoms in the \kl{free coordinates}. In
Example~\ref{ex:tau-disjoint}, the \kl[orbit dimension]{dimension} is two,
corresponding to the atoms $a$ and $b$. An important special case is when the
\kl[orbit dimension]{dimension} is zero; in this case the orbit contains only one state.

In the \kl[of protocol]{protocol}, Alice and Bob will jointly maintain a set
$S \subseteq \atoms$ and list of \kl{$S$-orbits} which may contain the
intermediate state (starting with $S = \emptyset$ the list of all
$\emptyset$-orbits). They will ensure that if the intermediate state exists,
i.e.~if the input string is accepted,  then the intermediate state is contained
in one of the \kl{orbits} on the list. The goal is to decrease the \kl[orbit
dimension]{dimension} of the orbits on the list until they become
\kl{zero-dimensional}, by gradually computing the set $S$ of atoms that appear in
the intermediate state. Once the orbits become zero-dimensional, they will
contain only a finite (and bounded) number of the candidates. At this point,
Alice can compute which of these candidates are reachable from an initial state
over $w_1$ and send this (bounded) information to Bob, who can then check if
one of these candidates can reach a final state over $w_2$. 

To decrease the \kl[orbit dimension]{dimension} and increase $S$, we will use the following lemma.
\begin{lemma}\label{lem:fixed-atoms}
  Let $S \subset \atoms$ be a finite set, and let $\varphi \subseteq Q$ be an infinite \kl{$S$-orbit}
  of \kl[orbit dimension]{dimension} $k$. 
  Consider an input string $w = w_1 w_2$, and the sets:
  \begin{align*}
  \myunderbrace{X_1 = \setbuild{ q \in \varphi}{$ I \xrightarrow{w_1} q$}}{states reachable on Alice's side}
  \qquad
  \myunderbrace{X_2 = \setbuild{ q \in \varphi}{$ q \xrightarrow{w_2} F$}}{states reachable on Bob's side}
  \end{align*}
  There is a set $T \subseteq \atoms$ of size at most $k$, such that either: 
  \begin{enumerate}
    \item   every state from $X_1$ contains an \kl{atom} of $T$ on some \kl{free coordinate}; or 
    \item   every state from $X_2$ contains an \kl{atom} of $T$ on some \kl{free coordinate}.
  \end{enumerate}
\end{lemma}
    \begin{proof}
      \AP
 In the proof of the lemma, we use an analysis of disjointness, which is
 inspired by the sunflower lemma. We say that two states $p,q$ in an \kl{$S$-orbit}
 are \intro{$S$-disjoint} if 
 \begin{align*}
    \forall a \in \atoms 
    \qquad 
 \text{$a$ appears in both $p$ and $q$} \quad \Rightarrow \quad a \in S.
 \end{align*}
 For example, if we take the $\{\textrm{John}, \textrm{Eve}\}$-orbit from Example~\ref{ex:tau-disjoint}, then the two states
\begin{center}
    (\red{John}, Tom, Mary, Tom, \red{Eve}) \qquad
    (\red{John}, Ann, Timmy, Ann, \red{Eve})
\end{center}
are $\{\textrm{John}, \textrm{Eve}\}$-disjoint, because the sets
$\set{\text{Tom, Mary}}$ and $\set{\text{Ann, Timmy}}$ are disjoint. In other
words, the atoms from $S$ can repeat (in fact, they must), but all other atoms
must be disjoint in the two states.

The following claim characterises subsets of orbits that do not contain any pair of disjoint elements:
\begin{claim}\label{claim:sunflower}
  Let $Q$ be an \kl{$S$-orbit} type of \kl[orbit dimension]{dimension} $d$, and let $X$ be a subset of $Q$.
  If $X$ does not contain two \kl{$S$-disjoint} elements, then there is a
  set $T$ of at most $d$ atoms such that every element of $X$ uses at last one atom from $T$ on a \kl{free coordinate}.
\end{claim}
\begin{proof}
        Take some element $x \in X$. Either there is an element of $X$ that is completely disjoint with $x$, or otherwise some atom from $x$ must appear in every other element of $X$ on a free coordinate.
\end{proof}

The claim leaves us with showing that at least one of $X_1$ or $X_2$ does not
contain an \kl{$S$-disjoint} pair of elements. Suppose, towards a contradiction that
both $X_1$ and $X_2$ contain \kl{$S$-disjoint} pairs of elements, say $p_1, p_2 \in
X_1$ and $q_1, q_2 \in X_2$. It follows that the two pairs $(p_1,p_2)$ and
$(q_1,q_2)$ are in the same \kl{equivariant orbit} (of $Q \times Q$), i.e. there is
some atom permutation $\pi$ which sends $p_1$ to $q_1$ and $p_2$ to $q_2$.
Applying $\pi$ to Alices's part of the input string, we obtain a new input
string $\pi(w_1) w_2$, in which both $q_1$ and $q_2$ are valid intermediate
states. It follows that there are at least two accepting runs (one that passes
through $q_1$ and one that passes through $q_2$), contradicting the \kl{unambiguity}
assumption. 
\end{proof}

Using the above lemma, we will construct a \kl[of protocol]{protocol} that
simulates the automaton. As explained before, the idea is to narrow down orbit
which contains the intermediate state. This idea is formalised in the following
lemma. 


\begin{lemma}\label{lem:narrow-down-orbit}
  Let $S$ be a finite subset of \kl{atoms}, and let $X \subseteq Q$ be an \kl{orbit}. 
  Alice and Bob can exchange a constant number of messages
  -- which depends only on the \kl[orbit dimension]{dimension} of $X$ -- 
  to determine if the intermediate state belongs to $X$. 
\end{lemma}

Before proving the lemma, let us explain how to use it to complete the proof of
\cref{thm:unambiguous-to-protocol}. We know that the set of all states $Q$
splits into constant number of \kl{equivariant orbits}, so the two parties can run
the protocol the lemma for each of these orbits with $S=\emptyset$. Each run of
the protocol uses a constant number of rounds, so the total number of rounds is
also constant. It remains to prove the lemma.

\begin{proof}[Proof of \cref{lem:narrow-down-orbit}]
  The proof proceeds by induction on the \kl{dimension of the orbit} $X$.
    
  The induction basis is when the dimension is zero. In this case, the orbit
  has exactly one state, and Alice and Bob can simply check  if the state is
  reachable on their side and exchange this bit of information.

  Consider now the induction step. Apply \cref{lem:fixed-atoms}, to the
  orbit. In the factorisation $w = w_1 w_2$, at least one of the two
  alternatives in the conclusion of \cref{lem:fixed-atoms} must hold. Alice
  can check if the first alternative holds, and Bob can check if the second
  alternative holds.  At least one of the two parties must report success,
  which is witnessed by some finite set $T$ of atoms. The successful party sends the
  set $T$ to the other party. This is possible since the size of $T$ is
  bounded by the \kl[orbit dimension]{dimension} of $X$. 
  The orbit $X$ splits into finitely many
  orbits $X_1,\ldots,X_n$ with the larger \kl[of support]{support} $S \cup T$, see~\cite[Lemma
  10.9]{bojanczyk_slightly}. The number $n$ depends only on the \kl[orbit dimension]{dimension} 
  of $X$ (as the size of $T$ is bounded by the dimension of $X$).

  We know that the intermediate state contains at least one atom from $T$, so
  we are only interested in the orbits among $X_1,\ldots,X_n$ which use at
  least one atom from $T$ on a coordinate that was \kl[free coordinate]{free} in $X$. 
  These orbits
  have lower \kl[of dimension]{dimension}, so the parties can sequentially apply the induction
  assumption to check if the intermediate state belongs to any of these
  orbits. This completes the proof of the lemma, and therefore also of
  \cref{thm:unambiguous-to-protocol}.
\end{proof}
\end{proof}


\subsection{Weighted automata}
\label{sec:weighted-automata-atoms}

\AP
In \cref{thm:unambiguous-to-protocol}, we have proved one implication in Conjecture~\ref{conj:protocols-unambiguous}.
This section is devoted to presenting some evidence for the other implication, i.e.
\begin{align}\label{eq:missing-orbit-finite-implication}
\text{protocol} \quad \implies \quad \text{unambiguous automaton}.
\end{align}
We begin by explaining why the orbit-finite case cannot be handled using the
techniques that were used to prove this implication in the finite case.

\paragraph*{What goes wrong in the orbit-finite case?}
In the finite case, the proof had two parts: (a) a reduction to \kl{one-round protocols}, and (b) the Myhill-Nerode Theorem.
Part (b) does not seem to be problematic, as orbit-finite versions of the Myhill-Nerode Theorem are known in many variants, 
including monoids~\cite[Lemma 3.3]{bojanczykNominalMonoids2013}, automata~\cite[Section 3.2]{bojanczykAutomataTheoryNominal2014}, 
and -- as we will prove later in this section -- also for weighted automata. 
The problematic part is (a), in which the number of rounds is reduced to one.
The key argument in this reduction  was that the sets of strategies 
  \begin{align*}
    (Q_B)^k \to (Q_A)^k \qquad \text{and} \qquad (Q_A)^k \to (Q_B)^k
    \end{align*}
are finite, and thus each party could simply send their strategy as a message.
This argument fails to carry over from finite sets to orbit-finite sets. 
The reason is that \kl{orbit-finite} sets are not closed under taking function spaces $X \to Y$, see~\cite{functionSpaces2024} 
for an extended discussion of this phenomenon.
The following example shows that the one-round reduction 
is indeed impossible in the orbit-finite case.

\begin{myexample}
    [No reduction to one round]\label{ex:no-one-round-reduction}
    Consider a language $L$ that is computed by an \kl{orbit-finite protocol} with
    one round. Using the same argument as in
    \cref{lem:one-round-reduction-boolean}, we can show that the Myhill-Nerode
    equivalence relation for the language, as defined
    in~\eqref{eq:myhill-nerode-equivalence}, has an \kl{orbit-finite} set of
    equivalence classes. As mentioned above, it follows from \cite[Section
    3.2]{bojanczykAutomataTheoryNominal2014} that $L$ is also recognised by a
    \kl{deterministic orbit-finite automaton}. As we have seen in
    Example~\ref{ex:protocol-not-dofa}, such automata are not strong enough to
    capture all \kl[of protocol]{protocols}.
    %In fact, the reasoning in this example shows that a language is recognised by a one-round orbit-finite protocol if and only if it is recognised by a deterministic orbit-finite automaton in both directions, i.e.~both the language and its reverse are recognised by deterministic orbit-finite automata. 
\end{myexample}

In light of the above example, it is no longer surprising that the proof of
\cref{thm:unambiguous-to-protocol} used multi-round protocols. In fact, we
believe that the number of needed rounds can be arbitrarily large, as suggested
by the following example. 

\begin{myexample}[Back and forth]
    A string over the alphabet $\atoms^2$ can be seen as a directed graph, where each letter represents an edge. For $k \in \set{1,2,\ldots}$, define  $L_k$ to be the set of strings over this alphabet such that: (1) the string is functional, i.e.~for each atom $a$ there is at most letter in the string that begins with $a$; and (2) in the corresponding graph, there is a path with $k$ edges that uses the edges from the first and last letter. This language can be computed by an orbit-finite protocol with $k-1$ rounds, with each round corresponding to a step in the path. It seems unlikely that a smaller number of rounds would suffice, but we do not prove this claim here. 
\end{myexample}



\paragraph*{Weighted orbit-finite automata.} In the reminder of this section,
we present some evidence for the missing implication in
Conjecture~\ref{conj:protocols-unambiguous}, using the orbit-finite version of
weighted automata. Namely, we will prove
\cref{thm:orbit-finite-protocol-to-weighted}, which states that every language
computed by an \kl{orbit-finite protocol} is also recognised by a \kl{weighted
orbit-finite automaton} over the two-element field. For finite alphabets, this
would be enough to ensure \kl[regular language]{regularity}, see
\cref{claim:regular-weighted-automata}. This claim is no longer true in the
orbit-finite case, see Example~\ref{ex:weighted-vs-nondet-orbit-finite}, and
therefore \cref{thm:orbit-finite-protocol-to-weighted} can only be considered
as evidence for the conjecture. However, at the very least it shows that
languages computed by orbit-finite protocols are decidable, which was not a
priori clear from the definition.

Let us begin by defining the orbit-finite version of weighted automata. 
\begin{definition}[Weighted orbit-finite automata]
    \label{def:weighted-orbit-finite-automata}
    \AP
    A \intro{weighted orbit-finite automaton} over a semiring $\domain$ is 
    defined in the same way as in Definition~\ref{def:weighted-automaton-nondeterministic}, except that:
    \begin{enumerate}
      \item the input alphabet and state space are \kl{orbit-finite}, instead of finite;
      \item the functions in~\eqref{eq:weight-functions} are \kl{equivariant}.
    \end{enumerate}
     We require that for every input string, there are only finitely many runs with non-zero weight.
\end{definition}

For the purpose of this section, already the special case of the two-element
field $\set{0,1}$ is interesting. In this case, the automaton defines a
function $\Sigma^* \to \set{0,1}$, which can be seen as the characteristic
function of a language. Therefore, we can compare \kl{weighted orbit-finite
automata} to other models, such as \kl{nondeterministic orbit-finite automata}. The
following example shows that these two  models are incomparable. 

\begin{myexample}\label{ex:weighted-vs-nondet-orbit-finite}
  The language ``some letter appears twice'' is recognised by a
  \kl{nondeterministic orbit-finite automaton}, but its characteristic function
  cannot be  recognised by a \kl{weighted orbit-finite automaton} over the
  two-element field. The non-expressivity can be proved using the orbit-finite
  version of the Fliess Theorem, see \cref{thm:orbit-finite-fliess}. On the
  other hand, the  language ``an even number of distinct letters'' is not
  recognised by a nondeterministic orbit-finite automaton, while its
  characteristic function can be computed by a \kl{weighted orbit-finite
  automaton}, see~\cite[Example 3.2]{orbitFiniteVectorTheoretics}. Therefore,
  the two models -- nondeterministic and weighted in the two-element field --
  are incomparable. Inside the intersection of these two classes we will find
  the \kl[of unambiguous automata]{unambiguous automata}, since for unambiguous automata, counting the runs
  modulo two gives the same result as checking if a run exists. This discussion
  is summed up in the following picture:
    \mypic{2}
\end{myexample}

The following theorem is the main result of Section~\ref{sec:weighted-automata-atoms}.
\begin{theorem}\label{thm:orbit-finite-protocol-to-weighted}
  Let $\Sigma$ be an \kl{orbit-finite} input alphabet, and let $\domain$ be a field.
  If a language $L \subseteq \Sigma^*$ is computed by a \kl[of protocol]{protocol}, 
  then the corresponding characteristic function of type $\Sigma^* \to \set{0,1} \subseteq \domain$
  is computed by a \kl{weighted orbit-finite automaton}.
\end{theorem}

In the proof of the theorem, we use the recently developed theory of
orbit-finite vector spaces \cite{orbitFiniteVectorTheoretics}. To streamline
the presentation, we will use a special case of these spaces, namely those that
have an orbit-finite basis. We begin by summarizing the necessary background.

For an orbit-finite set $Q$, let us write $\lincomb Q$ for the vector space
which consists of finite formal linear combinations of elements of $Q$. In
other words, an element of this space is a vector of the form 
\begin{align*}
\alpha_1 q_1 + \cdots + \alpha_n q_n,
\end{align*}
where the coefficients $\alpha_i$ are from the field, and the element $q_i$ (which can be seen as basis vectors) are from $Q$. We use such spaces as the orbit-finite generalisation of finite dimension\footnote{Similarly to the case of orbit-finite sets, we use a simplified definition for this paper, as compared to the literature. The usual notion of vector spaces for orbit-finite sets, see \cite[Definition 8.1]{bojanczyk_slightly} allows for more features; these features are irrelevant for our application and hence not discussed here.}.

\begin{definition}
    [Vector space of orbit-finite dimension]\label{def:orbit-finite-vector-space}
    \AP
    A \intro{vector space of orbit-finite dimension} 
    is a vector space of the form $\lincomb Q$ for some \kl{orbit-finite} set $Q$.
\end{definition}
A space as in the above definition is equipped with two structures: as a vector
space it is closed under linear combinations, and as a set with \kl{atoms} it
is closed under applying \kl{atom permutations}. We will typically be
interested in functions between such spaces that preserve both of those
structures.

In the proof of \cref{thm:orbit-finite-protocol-to-weighted}, we will use an
orbit-finite version of \kl{protocols} with field outputs. Recall that there
are two variants: the general version from Section~\ref{sec:intro-field} and
the simpler \kl{scalar-product protocols} from Section~\ref{sec:field-domain}.
We could begin with the general version and show that it is equivalent to the
scalar-product one -- this is indeed the case. However, to keep the exposition
concise, we treat \kl{orbit-finite protocols} with field outputs merely as a tool
for proving \cref{thm:orbit-finite-protocol-to-weighted}, rather than as an
object of independent interest. Therefore, we focus on the simplest form of
protocol required for the proof, namely an orbit-finite version of the scalar
product protocols from Section~\ref{sec:field-domain}. In the orbit-finite
case, instead of scalar products we will use the slightly more general notion
of bilinear maps, i.e.~maps that have two arguments and are linear in each
argument separately\footnote{It is possible to define orbit-finite scalar
  product protocols, but they are harder to work with. In particular, from our
  results it will follow that orbit-finite scalar product protocols (suitably
  defined) and orbit-finite bilinear protocols are equivalent, but we are not
  aware of a direct proof of this fact.}. 

% This space is equipped with a scalar product, 
% \begin{align*}
% \langle v, w \rangle = \sum_{q \in Q} (\text{coefficient of $q$ in $v$}) \cdot ( \text{coefficient of $q$ in $w$}).
% \end{align*}
% The sum in the above definition is in fact finite, since only finitely many basis vectors will have nonzero coefficients. Using this scalar product, we can define an orbit-finite version of the scalar product protocols from Definition~\ref{def:scalar-product-protocol}.

\begin{definition}
    [Orbit-finite bilinear protocol] 
    \label{def:orbit-finite-scalar-product-protocol}
    \AP
    An \intro{orbit-finite bilinear protocol} consists of:
    \begin{enumerate}
      \item two \kl{vector spaces of orbit-finite dimension}
        $V_A$ and $V_B$ and two strategies, which are equivariant functions
        \begin{align*}
        \sigma_A : \Sigma^* \to V_A 
        \quad \text{and} \quad
        \sigma_B : \Sigma^* \to V_B
        \end{align*}
        \item an output map, which is an equivariant bilinear map 
        \begin{align*}
        \text{out} : V_A \times V_B \to \domain.
        \end{align*}
    \end{enumerate}
\end{definition}
The output of the \kl[of bilinear protocol]{protocol} is defined in the natural way: 
Alice and Bob apply their strategies to their local strings, yielding two vectors, 
and the output of the protocol is obtained using the output map. 
As usual, we require split invariance, i.e.~the output of the protocol should
depend only on the input string $w$ and not on its factorisation $w = w_1 w_2$.

% Before continuing with the proof, let us briefly use explain why we chose to work with bilinear maps instead of scalar products.
% First, let us observe that scalar products can be defined in the orbit-finite in a natural way:
% \begin{align*}
% \langle v, w \rangle = \sum_{q} (\text{coefficient of $q$ in $v$}) \cdot ( \text{coefficient of $q$ in $w$}),
% \end{align*}
% where the sum ranges over basis vectors. Observe that this sum is in fact finite, since each of the vectors $v$ and $w$ involves only finitely many basis vectors with non-zero coefficients.
% Scalar products are a special case of bilinear maps, and therefore every scalar product protocol is a special case of a bilinear protocol. In the finite case it is easy to show
% directly that bilinear protocols are equivalent to scalar product protocols. In the orbit-finite case the picture is a bit more complex, the two models are still equivalent,
% but we are no aware of a direct proof -- the equivalence follows from the fact that both models are equivalent to weighted orbit-finite automata as we will show later in this section.

Let us go back to the proof of \cref{thm:orbit-finite-protocol-to-weighted}. The proof has two steps, as described in the following diagram.
\[
\begin{tikzcd}
\text{orbit-finite protocol}
\ar[d,Rightarrow,"\text{Lemma~\ref{lem:orbit-finite-protocol-to-scalar}}"]
\\
\text{orbit-finite bilinear protocol}
\ar[d,Rightarrow, "\text{\cref{claim:bilinear-prot-to-of-automaton}}"]
\\
\text{orbit-finite weighted automaton}
\end{tikzcd}
\]

We begin with the first step, which can be seen as form of reduction to one round. Recall that without vector spaces, a reduction to one round was not possible, see Example~\ref{ex:no-one-round-reduction}. This phenomenon is connected to closure under taking function spaces: orbit-finite sets are not closed under taking function spaces, but this closure is recovered once one moves to vector spaces, see~\cite[Section 8.3]{bojanczyk_slightly}. 

\alc{orbit finite bilinear protocol should probably be used here instead of 
orbit finite scalar product protocol}
\begin{lemma}\label{lem:orbit-finite-protocol-to-scalar}
  If $L \subseteq \Sigma^*$ is computed by an \kl{orbit-finite protocol}, then its characteristic function is computed by an \kl{orbit-finite bilinear product protocol}, over any field.
\end{lemma}


\begin{proof}
  \AP
    We first introduce a common generalisation of the two models called \intro{hybrid protocols}. 
    In such a protocol, there are multiple rounds of messages, 
    followed by a bilinear operation. We will show that the multiple rounds can be eliminated,
    yielding a plain \kl{orbit-finite bilinear protocol}.
    
    Here is the formal definition of the \kl{hybrid protocol}:
    For each round $i \in \set{1, \ldots, k-1}$, the parties exchange messages just as in an 
    \kl[of protocol]{orbit-finite protocol} from Definition~\ref{def:orbit-finite-protocol}, 
    using message spaces $Q_A$ and $Q_B$ and strategies of the following types: 
        \begin{align*}
        \sigma_{A,i} & : \Sigma^* \times (Q_B)^{i-1} \to Q_A\\
        \sigma_{B,i} & : \Sigma^* \times (Q_A)^{i-1} \to Q_B
        \end{align*}
    Then, in  the last $k$-th round, the message histories are used to produce vectors 
    in two \kl[ofd vector spaces]{vector spaces} $V_A$ and $V_B$ of orbit-finite dimension, 
    see Definition~\ref{def:orbit-finite-vector-space}, using strategies 
    of types
    \begin{align*}
        \sigma_{A,k} & : \Sigma^* \times (Q_B)^{k-1} \to V_A\\
        \sigma_{B,k} & : \Sigma^* \times (Q_A)^{k-1} \to V_B.
        \end{align*}
    Finally, from the two vectors, the output is computed using a bilinear map
    \begin{align*}
        \text{out} : V_A \times V_B \to \domain.
    \end{align*}
    

    The hybrid protocol generalises both \kl{orbit-finite protocols} and 
    \kl{orbit-finite bilinear protocols}. 
    For the latter, this is clear: we simply use $k=1$ and there is no message exchange. 
    For the former, we proceed as follows.
    We use  trivial vector spaces, i.e.~both $V_A$ and $V_B$ are the field. 
    The bilinear map is multiplication. 
    Once the two parties have agreed on a Boolean decision, 
    they can both send $1$ (in the case of a ``yes'' decision) or $0$ (in the case of a ``no'' decision), 
    and the bilinear map will give the correct output. 
    
    In order to complete the proof of the lemma, we will show that the number
    of rounds can always be reduced to one, thus yielding a \kl[of bilinear protocol]{bilinear protocol}.

    \begin{claim}\label{claim:reduce-round}
      For every $k > 1$, a \kl{hybrid protocol} with $k$ rounds can be simulated by a 
      \kl{hybrid protocol} with $k-1$ rounds.
    \end{claim}
    \begin{proof}
        We will eliminate round $k-1$, where the last message is sent. 
        Once Alice has received the first $k-2$ messages from Bob, 
        her contribution to the rest of the protocol is described by an object of type 
          \begin{align}\label{eq:contribution-last-two-rounds}
            \myunderbrace{Q_A}{message \\ sent in \\ round $k-1$} \quad \times \quad  \myunderbrace{(\fsfun  {Q_B} {V_A})}{message sent in \\ round $k$, as a function \\ of the message sent \\ in  round $k-1$}
        \end{align}
        We want to turn the above type into a vector space. 
        The second coordinate is already a vector space, since functions with outputs in a vector space can be added and scaled pointwise. What is more, the  second coordinate  is  a \kl{vector space of orbit-finite dimension}, which is a nontrivial result~\cite[Theorem 8.16]{bojanczyk_slightly}, i.e.~it has an orbit-finite basis  
        \begin{align*}
        F_A \subseteq \fsfun  {Q_B} {V_A}.  
        \end{align*}
        \rs{expand what is the cited theorem and how it is applied}
        The first coordinate $Q_A$ in~\eqref{eq:contribution-last-two-rounds} can be turned into a vector space by  allowing linear combinations, i.e.~$\lincomb Q_A$. We combine these two using tensor product, yielding a vector space of orbit-finite dimension
        \begin{align*}
           W_A =  (\lincomb Q_A) \otimes (\lincomb F_A).
        \end{align*}
        We can do the same thing for Bob, obtaining a vector space
        \begin{align*}
           W_B =  (\lincomb Q_B) \otimes (\lincomb F_B),
        \end{align*}
        where $F_B$ is an orbit-finite basis of the vector space $\fsfun  {Q_A} {V_B}$.
        In order to define an equivariant bilinear map of type
        \begin{align*}
        \varphi : W_A \times W_B \to \domain
        \end{align*}
        it is enough to define an equivariant linear map of type 
        \begin{align*}
        \varphi : W_A \otimes W_B \to \domain
        \end{align*}
        for which it is enough to define an equivariant function on its basis
        \begin{align*}
        Q_A \times F_B \times Q_B \times F_A \to \domain
        \end{align*}
        This definition is the only one that types, namely 
        \begin{align*}
        (q_A, f_B, q_B, f_A) \quad 
        \mapsto \quad 
        \text{out}(f_A(q_B), f_B(q_A)).
        \end{align*}
        Because the output map is bilinear, one can check that $\varphi$ defined this way is consistent with the original protocol, i.e.~if we take functions 
        \begin{align*}
        f_A : Q_B \to V_A \qquad \text{and} \qquad f_B : Q_A \to V_B,
        \end{align*}
        which are not necessarily basis vectors from $F_A$ and $F_B$, then we have 
        \begin{align*}
        \text{out}(f_A(q_B), f_B(q_A)) = \varphi((q_A, f_B), (q_B, f_A)).
        \end{align*}
        Therefore, we can implement the last two round of the original hybrid protocol using a single round. The message spaces and the strategies for the first $k-2$ rounds are unchanged. In the last round $k-1$, the new strategies
        \begin{align*}
        \sigma'_{A,k-1} & : \Sigma^* \times (Q_B)^{k-2} \to W_A\\
        \sigma'_{B,k-1} & : \Sigma^* \times (Q_A)^{k-2} \to W_B
        \end{align*}
        output the tensor pairs consisting of the contribution that was described in~\eqref{eq:contribution-last-two-rounds}. Finally, the output  map for the new protocol is $\varphi$. 
    \end{proof}

    By repeatedly applying the above claim, we can reduce the number of rounds to one, in which case we get a bilinear protocol, as required in the statement of the lemma. 
\end{proof}


\subsubsection{Orbit-Finite Fliess Theorem}
In this section, we prove an orbit-finite version of the Fliess Theorem, which characterises functions $\Sigma^* \to \domain$ that are computed by \kl{weighted orbit-finite automata}. 
This result will be used to complete the proof of \cref{thm:orbit-finite-protocol-to-weighted}.

As in the original Fliess Theorem, we will be interested in \kl{derivatives} of the function,
which live in the space  
\begin{align*}
\Sigma^* \to \domain.
\end{align*}
This space has three kinds of structure, all of which will are used in the Fliess Theorem: 
\begin{enumerate}
    \item It is a vector space, since we can take linear combinations of functions.
    \item It has a notion of \kl{left derivatives}, i.e.~for each function $f$ and input string $w \in \Sigma^*$, we can consider the \kl{left derivative} $v \mapsto f(wv)$, which is denoted by $\leftderivative{f}{w}$.
    \item It has a notion of \kl{atom permutations}: for each function $f$ and atom permutation $\pi$, we can consider the function $\pi(f)$, which is the composition $\pi;f$.
\end{enumerate}

\AP
We say that a subset  $U \subseteq \Sigma^* \to \domain$  is
\intro{orbit-finitely spanned} if there is some \kl{orbit-finite} set $Q$, such that
every element of $U$ is a finite linear combination of elements from $Q$. We do
not require the linear combination to be unique, i.e.~we do not require $Q$ to
be a basis. (Choosing a basis can be problematic in the context of orbit-finite
sets, see~\cite[Example 77]{bojanczyk_slightly}.) We are now ready to state the
orbit-finite version of the Fliess Theorem.

\begin{theorem}[Orbit-Finite Fliess Theorem]\label{thm:orbit-finite-fliess}
    The following two conditions are equivalent for every function
    $f : \Sigma^* \to \domain$
    where $\Sigma$ is an orbit-finite alphabet and $\domain$ is a field.
    \begin{enumerate}
      \item \label{it:fliess-weighted} $f$ is computed by a \kl{weighted orbit-finite automaton};
      \item \label{it:fliess-derivatives} $f$ is \kl{equivariant} and its set of
        \kl{left derivatives} is \kl{orbit-finitely spanned}.
        %  there is a finite set $\Gamma$ of derivatives of $f$, such that every derivative of $f$ can be expressed as linear combination
        % \begin{align*}
        % \alpha_1 \pi_1(f_1) + \cdots + \alpha_k \pi_k(f_k),
        % \end{align*}
        % where each $\alpha_i$ is in the field, each $\pi_i$ is an atom permutation, and each $f_i$ is in $\Gamma$.
    \end{enumerate}
\end{theorem}
\begin{proof} Our proof follows the lines of the original theorem, without any significant changes.

    \AP
    We begin with the implication \ref{it:fliess-weighted} $\implies$ \ref{it:fliess-derivatives}. Consider a weighted orbit-finite automaton with state space $Q$. 
    Define the \intro{pre-weight} of a run in the same way as its weight, except that we do not use the final weight. In other words, this is the product of: (1) the initial weight of the first state; and (2) the weights of all transitions. Consider an input string $w$. Define the \intro{configuration} of $w$ to be the linear combination
    \begin{align}
        \label{eq:configuration-wa}
        \sum_\rho \alpha_\rho \cdot q_\rho,
    \end{align}
    where $\rho$ ranges over runs that have input $w$ and non-zero \kl{pre-weight},
    $\alpha_\rho$ is the \kl{pre-weight} of the run $\rho$, and $q_\rho$ is the last state in this run.
    By the assumption that each input string has finitely many runs with non-zero weight, 
    the configuration is a finite sum, i.e.~it belongs to the vector space $\lincomb Q$. 
    The \kl{left derivative} which corresponds to the input string is uniquely determined 
    by this configuration, and the space of \kl{configurations} is \kl{orbit-finitely spanned}. 
    Hence, we get~\ref{it:fliess-derivatives}.

    We now prove the other implication, \ref{it:fliess-derivatives} $\implies$
    \ref{it:fliess-weighted}. Assume~\ref{it:fliess-weighted}, which means that
    there is an \kl{orbit-finite} set $Q \subseteq \Sigma^*$ such that every
    \kl{derivative} of $f$ can be decomposed as a finite linear combination of left
    derivatives
    \begin{align*}
            \sum_i \alpha_i \leftderivative{f}{w_i},
    \end{align*}
    where each string $w_i$ is in $Q$. The following claim shows that strings obtained by taking elements of $Q$ and appending one letter
    (i.e. strings of the form $Q \cdot \Sigma$) can be decomposed in an equivariant way:
    \begin{claim}
      There is an \kl{equivariant function}
        \begin{align*}
       \delta :  Q \times \Sigma \to \lincomb Q
        \end{align*}
        such that the following conditions holds for every $w \in Q$ and $a \in \Sigma$:
        \begin{align*}
        \delta(w,a) = 
        \sum_i \alpha_i w_i 
        \qquad \Rightarrow \qquad 
        \leftderivative f {wa} = \sum_i \alpha_i \leftderivative{f}{w_i}.
        \end{align*}
    \end{claim}
    \begin{proof}
        Observe that Condition~\ref{it:fliess-derivatives} in the theorem's statement already asserts that there exists such a function $\delta$,
        possibly non-equivariant. In this proof we show how to modify it so that it becomes equivariant
        while still satisfying other requirements of the claim. We construct this modified function $\delta'$ as follows:
        for every orbit $Q \times \Sigma$ pick its representative $(w, a)$ and then extend the result to the whole orbit by equivariance:
        \begin{align*}
        \delta'(\pi(w), \pi(a)) := \pi(\delta(w, a))
        \end{align*}
        The new function $\delta'$ is equivariant by construction, and it keeps satisfying other requirements of the claim because
        $f$ is an equivariant function, so its derivatives commute with atom permutation:
        \begin{align*}
        \leftderivative{f}{\pi(w)}= \pi(\leftderivative{f}{w}).
        \end{align*}
        This gives  us that:
        \begin{align*}
                    \leftderivative{f}{wa} = \sum_i \alpha_i \leftderivative{f}{w_i}
                    \quad \iff \quad 
                            \leftderivative{f}{\pi(wa)} = \sum_i \alpha_i \leftderivative{f}{\pi(w_i)},
        \end{align*}
        which completes the proof of the claim.
    \end{proof}        

    Using the function $\delta$ from the above claim, we define a \kl{weighted
    orbit-finite automaton}. The state space is the set $Q$. (We assume without
    loss of generality that $Q$ contains the empty string $\varepsilon$. This
    is not really necessary for the construction, but it makes it more
    intuitive.) The weights are defined as follows: 
    \begin{itemize}
        \item \textbf{Initial weights.} The initial weight of   $\varepsilon$ is $1$. All other states have initial weight $0$.
        \item \textbf{Transition weights.} The weight of a transition 
        \begin{align*}
        w \xrightarrow{a} v
        \end{align*}
    is the coefficient next $v$ in the linear decomposition $\delta(w,a)$.
        \item \textbf{Final weights.} The final weight of a state $w \in Q$ is $f(w)$.
    \end{itemize}
    Remember that \kl{orbit-finite weighted automata} can only admit finitely many runs for each input word.
    Our automaton satisfies this requirement, as all linear combinations returned by $\delta$ are finite. 

    Finally, we justify why this automaton computes the function $f$. A simple inductive proof shows that  
    \begin{align*}
\text{configuration of $w$} = \sum_\rho \alpha_\rho \cdot w_\rho
\quad \implies \quad 
        \leftderivative f w = 
    \sum_\rho \alpha_\rho \cdot \leftderivative f {w_\rho}.
    \end{align*}
    By choice of final weights, the output of the automaton is equal to $f(w)$. 
\end{proof}

We are now ready to complete the proof of \cref{thm:orbit-finite-protocol-to-weighted} by showing the missing link:
\begin{claim}
\label{claim:bilinear-prot-to-of-automaton}
Let $\domain$ be a field.
    If a function $\Sigma^* \to \domain$ is computed by an 
    \kl{orbit-finite bilinear protocol}, it can also be computed by an 
    \kl{orbit-finite weighted automaton}.
\end{claim}
\begin{proof}
    Thanks to the Orbit-Finite Fliess Theorem,
    it is enough to show that if a function is computed by an \kl{orbit-finite bilinear 
    protocol}, then it has an \kl{orbit-finitely spanned vector space} of \kl{left derivatives}. This follows from the same argument as in Section~\ref{sec:field-domain}:
    the vector produced by Alice in a \kl[of bilinear protocol]{bilinear protocol}
    uniquely determines the \kl{left derivative} of her part of the input. 
\end{proof}

\section{Conclusions}
\label{sec:conclusions}
One could possibly consider other inputs, such as trees. It seems that at least some of our results could generalise to such inputs, as long as there would be a suitable theory of automata for the inputs, with theorems in the style of Myhill-Nerode (which is the case for trees). Nevertheless, we leave the the exploration of non-string inputs to future work.



\bibliographystyle{alpha}
\bibliography{bib}



\end{document}
