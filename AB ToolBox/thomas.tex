We introduce a notion of distance over $ \Sigma^* $ defined as follows.

\begin{definition}[Distance]
  Let $u_1, u_2 \in \Sigma^*$ be two strings. The distance between $u_1$ and $u_2$ is denoted by $\norm{u_1, u_2}$ and defined as follows:
  \[\norm{u_1, u_2} = |u_1| + |u_2| - 2 \cdot |u_1 \wedge u_2| \]
  where $u_1 \wedge u_2$ is the longest common prefix of $u_1$ and $u_2$.
\end{definition}
\medskip

Based on this notion of distance, Reutenauer and Schutzenberger define an equivalence relation, $ \RS $ by $ \forall u,v \in \Sigma^*, u \RS v $ if and only if $ \exists c(u,v) \in \mathbb N, \forall w \in \Sigma^*, \norm{f(wu),f(wv)} \le c $. This relation is the central part of their proof that functions that have the Hankel property are rational, as we will discuss later on.\cite{reutenauer1991minimization}

For now, we present an interpretation of this relation in the setting of monotone protocols. For any two signals of Bob, $ q,q' \in Q_b $, we let $ q \Simil q' $ if and only if $ \exists c(q,q'), \norm{f_A[q],f_A[q']} \le c(q,q') $.

\begin{lemma}
    Let $ u,v \in Sigma^* $, $ u \RS v $ if and only if $ q_B(u) \Simil q_B(v) $.
\end{lemma}

\begin{proof}
    We start by proving the left to right implication. Let $ u,v \in \Sigma^* $ such that $ q_B(u) \Simil q_B(v) $ and denote $ q_B(u) $ and $ q_B(v) $ by $ q $ and $ q' $ respectively. Then, there exists $ c(q,q') \in \mathbb N $ such that for any $ w \in \Sigma *, \norm{f_A[q](w),f_A[q'](w)} \le c(q,q') $. Therefore, $$ \norm{f(wu),f(wv)} \le c(q,q') + |f_B[w](u)| + |f_B[w](v)|\le c(q,q') + sup_{w \in \Sigma} |f_B[w](u)| + |f_B[w](v)|. $$

    Let $ c'(u,v) = c(q_B(u),q_B(v)) + sup_{w \in \Sigma} |f_B[w](u)| + |f_B[w](v)| $ for any $ u,v \in \Sigma^* $, for any $ w \in \Sigma^*, \norm{f(wu),f(wv)} \le c'(u,v) $ and since there are finitely many functions of the form $ f_B[w] $, $ c'(u,v) $ is finite. This concludes the proof of the first implication.
    \medskip
    
    Now, we prove the right to left implication. Let $ q,q' \in Q_B $ be two signals equivalent for $ \Simil $ and let $ P(q,q') = \lbrace (u,v) \in \Sigma^* \mid q_B(u) = q, q_B(v) = q', u \RS v \rbrace $. Now, for any $ (u,v) \in P(q,q') $, there exists $ c(u,v) \in \mathbb N $ such that $ \forall w \in \Sigma^*, \norm{f(wu),f(wv)} \le c(u,v) $. Let $ c'(q,q') = inf_{(u,v) \in P(q,q')} c(u,v) + sup_{w \in \Sigma^*} |f_B[w](u)|+|f_B[w](v)| $. Once again, there are only finitely many distinct functions $ f[w] $ for $ w \in \Sigma^* $ so the infimum is taken over natural numbers and is therefore itself a natural number. We claim that for any $ u,v \in \Sigma^* $ such that $ u \Simil v $, and for any $ w \in \Sigma^*, \norm{f_A[q_B(u)](w),f_A[q_B(v)](w)} \le c'(q_B(u),q_B(v)) $, which would conclude the proof.
   \medskip
   
   Let $ u,v \in \Sigma^* $ such that $ u \Simil v $ and let $ q = q_B(u) $ and $ q' = q_B(v) $. Assume that $ \norm{f_A[q](w),f_A[q'](w)} > c(u,v) $ then one is the prefix of the other one. Say $ f_A[q](w) $ is a prefix of $ f_A[q'](w) $. Now, $ f_A[q](w) \cdot f_B[w](u) $ and $ f_A[q'](w) $ coincide at least until the $ c(u,v) $ last letters of $ f_A[q'](w) $. As a consequence, we deduce that $ \norm{f_A[q](w),f_A[q'](w)} \le c(u,v) + |f_B[u](w)| \le c'(q,q') $. Symmetrically, if $ f_A[q'](w) $ is a prefix of $ f_A[q](w) $ then $ \norm{f_A[q](w),f_A[q'](w)} \le c(u,v) + |f_B[v](w)| \le c'(q,q') $.
\end{proof}