\chapter{Aggregator Protocols}

\label{chap:agg-protocols}

In this chapter, we introduce a new class of protocols called \emph{aggregator-protocols}. This is a new way of looking at Alice-Bob protocols. Here we want to look deeply into the way that Alice and Bob "combine" their messages to produce the final output (by message I mean the element of the output domain that each party sends to the other party at each round of communication). The combining operation here again depends on the output domain of the protocol, but in contrast to normal approaches, here we want to focus more on this combining operation and distinguish different types of protocols based on this operation. Let first explain what we mean by aggregator-protocols.

These protocols are similar to normal Alice-Bob protocols, with one key difference: instead of generating partial outputs using messages that contain an element of the output domain, each party has its own local partial output until the end of the communication, and at the end the final output is computed by aggregating these local outputs using a predefined operation.

\begin{myexample}
Consider the following protocol between Alice and Bob. An evil adversary split the input $w\in \Sigma^*$ into two parts $x,y\in \Sigma^*$ such that $w=xy$. Alice receives $x$ and Bob receives $y$. The goal of the protocol is to compute the length of the input string $w$, i.e., $|w|$. The protocol works as follows:
\begin{itemize}
    \item Alice computes the length of her input $|x|$ and keeps it as her local output.
    \item Bob computes the length of his input $|y|$ and keeps it as his local output.
    \item At the end of the communication, both parties send their local outputs to a ``addition-aggregator''. The aggregator computes the final output as $|x| + |y|$.
\end{itemize}
\end{myexample}

We can define different types of aggregator-protocols based on the aggregation operation used to compute the final output from the local outputs of the parties. We have already seen that protocols on the field output domain are equivalent to aggregator-protocols with the scalar product as the aggregation operation. We call these scalar-product protocols. We also mentioned that this is true for every commutative semiring output domain. Also , It is not hard to see that in case of String-output protocols, the aggregation operation is sequence of concatenation. In that case, alice and bob communicating with each other but keep the local output for themselves until the end of the communication, and at the end each of them send $k$ strings to the aggregator, and the aggregator concatenate all the strings in a zig-zag manner to produce the final output string. In the next section, we will define commutative protocols over a monoid domain as another example of aggregator-protocols.

\begin{definition}
    \label{simple-aggregation-definition}
A simple-aggregation protocol over a monoid is given by the following ingredients:
\begin{itemize}
    \item a monoid $(M, \ast, e)$ as the output domain
    \item a number $k \in \mathbb{N}$ of rounds of communication
    \item signal spaces for Alice and Bob, which are the sets $Q_A$ and $Q_B$, respectively
    \item for each round $1 \leq i \leq k$, two strategies:
    \begin{itemize}
        \item Alice's strategy: $\delta_A^i: \Sigma^* \times Q_B^{i-1} \to Q_A$
        \item Bob's strategy: $\delta_B^i: \Sigma^* \times Q_A^{i-1} \to Q_B$
    \end{itemize}
    \item an output function for each party:
    \begin{itemize}
        \item Alice's output function: $f_A: \Sigma^* \times Q_B^k \to M$
        \item Bob's output function: $f_B: \Sigma^* \times Q_A^k \to M$
    \end{itemize}
\end{itemize}
The output of the protocol on inputs $x,y \in \Sigma^*$ is the aggregation of the local outputs of Alice and Bob:
\[
\text{Output}(x,y) = f_A(x, q_B^1, \ldots, q_B^k) \ast f_B(y, q_A^1, \ldots, q_A^k),
\]
where $q_A^i$ and $q_B^i$ are the signals sent by Alice and Bob in round $i$, respectively.
\end{definition}

Here we say that a protocol computes a function $f: \Sigma^* \to M$ if, for every input split $w=xy$, the output of the protocol on inputs $x,y$ equals $f(w)$.


Observe that in simple-aggregation protocols, the parties can simply send each other the sequence of functions $(f_i)_{i=1,2,\dots,k}$ , such that each function $f_i$ is either from $Q_B^{i-1}$ to $Q_A$ or from $Q_A^{i-1}$ to $Q_B$. This sequence of functions can fully describe the behavior of both parties, or in other words, can generate the communication history (which is a sequence of signals).

Hence we can simplify the definition of simple-aggregation protocols in the following way: Alice send one signal to Bob, let call it $q_A(x)$, which is a function from $\Sigma^*$ to $Q_A$ and $x$ is her local input. Similarly, Bob send one signal to Alice, let call it $q_B(y)$, which is a function from $\Sigma^*$ to $Q_B$ and $y$ is his local input. Then both parties compute their local outputs using these functions as follows:
\[\text{Alice's local output} = f_A(x, q_B(y)) \quad , \quad \text{Bob's local output} = f_B(y, q_A(x))\] 
Finally, the aggregator compute the final output as:
\[\text{Output}(x,y) = f_A(x, q_B(y)) \ast f_B(y, q_A(x))\] \footnote{This reminds us of the \emph{Yin-Yang}, so we would like to call this the Yin-Yang protocols. This is because if you consider the local output of Alice as the Yang, then inside it there is the signal from Bob, which is the Yin. Similarly, the local output of Bob is the Yin, which contains the signal from Alice, which is the Yang. Together they form a complete whole. This is just a poetic analogy, but it will be more clear when we look at specific examples of these protocols later in this chapter, more specifically when we make a strong connection between these protocols and rational functions in Section~\ref{sec:monotone-protocols}, As it is known, rational functions can be represented using bimachines, and bimachines itself can be seen as a Yin-Yang computional model.}.

Without loss of generality, we will always assume that all of the elements of the signal spaces $Q_A$ and $Q_B$ are reachable by the respective signal functions $q_A$ and $q_B$. In other words, for every $q_A \in Q_A$, there exists an input string $x \in \Sigma^*$ such that $q_A(x) = q_A$. Similarly, for every $q_B \in Q_B$, there exists an input string $y \in \Sigma^*$ such that $q_B(y) = q_B$.

\begin{definition}
    \label {configuration-definition}
    A configuration of a simple-aggregation protocol over the input space $\Sigma^*$ for each party and input string can be defined as follows: First fixed an enumeration of elements of signal spaces of both parties, i.e., $Q_A = \{q_A^1, q_A^2, \ldots, q_A^{|Q_A|}\}$ and $Q_B = \{q_B^1, q_B^2, \ldots, q_B^{|Q_B|}\}$. Then the configuration of Alice on input $x \in \Sigma^*$ is a vector of size $|Q_B|$ defined as follows:
    \begin{align}
    C_A(x) &= \begin{bmatrix}
           m_{1} \\
           m_{2} \\
           \vdots \\
           m_{|Q_B|}
    \end{bmatrix} \quad \text{where } m_{i} = f_A(x, q_B^i) \quad \text{for } 1 \leq i \leq |Q_B|
    \end{align}
    Similarly, the configuration of Bob on input $y \in \Sigma^*$ can be defined.
\end{definition}

\begin{theorem}
    \label{thm:simple-aggregation-to-normal}
    Suppose that a function $f: \Sigma^* \to M$ is expressible by a normal Alice-Bob protocol over a commutative monoid $(M, \ast, e)$. Then there exists a simple-aggregation protocol that expresses the same function.
\end{theorem}

\begin{proof} 
    We are going to simply simulate the original protocol with a simple-aggregation protocol. In the simulation we replace the operation of combining the messages at each round with signals. Observe that in this setting, there are two kind of messages: the messages that are combinations of previous messages (which we call them \emph{complex terms}), and the messages that are not combinations of previous messages (which we call them \emph{primitive terms}). In the simulation, each signal corresponds to a primitive term is like a pointer to the actual message which the owner at the end of the communication can reconstruct it locally. Now if there exists one signal for each primitive term, which is possible since the number of rounds is finite, then we can have signals for each complex term as well, for the same reason that number of rounds is finite. Now at each round, instead of sending the actual message, each party sends the signal that corresponds to that message. Since the monoid is commutative, the order of combining the messages does not matter, hence alice and bob can simply aggregate their own messages that define the final output locally and send them to the aggregator at the end of the communication.
\end{proof}


\section{Signal Free Protocols}

\begin{definition}
    \label{signal-free-simple-aggregation-definition}
A signal-free simple-aggregation protocol over a monoid is a restriction of simple-aggregation protocols in which the signal spaces $Q_A$ and $Q_B$ are singleton sets. In other words, there is no communication between the parties, and each party computes its local output based only on its input. Such a protocol is therefore specified by:
\begin{itemize}
    \item a monoid $(M, \ast, e)$ as the output domain
    \item two functions $f_A: \Sigma^* \to M$ and $f_B: \Sigma^* \to M$ that are computed by Alice and Bob respectively on their local inputs
\end{itemize}
The final output on inputs $x,y$ is $f_A(x)\ast f_B(y)$.
\end{definition}

\begin{theorem}
\label{thm:signal-free-homomorphism}
Let $(M, \ast, e)$ be a monoid, and let $f_A, f_B: \Sigma^* \to M$ be two functions that define a signal-free aggregation protocol over $M$. Then there exists two element $s_A, s_B \in M$ such that the function $f: \Sigma^* \to M$ computed by the protocol can be expressed as:
\[f(w) = s_A \ast h(w) \ast s_B \quad \text{for every } w \in \Sigma^*\]
where $h: \Sigma^* \to M$ is a homomorphism from the free monoid $\Sigma^*$ to the monoid $M$. (Note that $M$ is not necessarily commutative here.)
\end{theorem}
\begin{proof}
    Let $s_A = f_A(\epsilon)$ and $s_B = f_B(\epsilon)$, where $\epsilon$ is the empty string. To prove that $h$ is a homomorphism, first we show that $h(\epsilon) = e$. This is because, assume by contradiction, that $h(\epsilon) \neq e$. Then we have:
    \[f(\epsilon) = f_A(\epsilon) \ast f_B(\epsilon) = s_A \ast s_B \neq s_A \ast e \ast s_B = f(\epsilon)\]
    which is a contradiction. Therefore, $h(\epsilon) = e$. Then we show that for every $u,v \in \Sigma^*$, we have:
    \[h(uv) = h(u) \ast h(v)\]
    This is because observe that $f_A(x)= s_A \ast h(x)$ and $f_B(y)= h(y) \ast s_B$ for every $x,y \in \Sigma^*$. This is because:
    \[f(x) = f_A(x) \ast f_B(\epsilon) = f_A(x) \ast s_B = s_A \ast h(x) \ast s_B,\]
    which implies that $f_A(x) = s_A \ast h(x)$. Symmetrically, we have:
    \[f(y) = f_A(\epsilon) \ast f_B(y) = s_A \ast f_B(y) = s_A \ast h(y) \ast s_B,\]
    which implies that $f_B(y) = h(y) \ast s_B$.
    Now consider two words $u,v \in \Sigma^*$, we give $u$ to Alice and $v$ to Bob as their inputs. Then we have:
    \begin{align*}
    f(uv) & = f_A(u) \ast f_B(v) \\
    & = (s_A \ast h(u)) \ast (h(v) \ast s_B) \\
    & = s_A \ast h(u) \ast h(v) \ast s_B \\
    & = f_A(uv) \ast f_B(\epsilon) \\
    & = (s_A \ast h(uv)) \ast s_B \\
    \end{align*}
    By cancellation of $s_A$ from the left and $s_B$ from the right, we get:
    \[h(uv) = h(u) \ast h(v)\]
    This completes the proof that $h$ is a homomorphism.
\end{proof}

\subsection{Commutative Signal Free Protocols}
\label{sec:commutative-signal-free}

We say that a function $f:\Sigma^* \to {\mathbb D}$ (where ${\mathbb D}$ is any output domain) is commutative if, for every word $u\in\Sigma^*$ and every permutation $\pi$ of the positions of $u$, we have $f(u)=f(\pi(u))$.

In Theorem~\ref{thm:commutative-signal-free}, we show the following: If, in a signal-free aggregation protocol over a commutative monoid, the neutral element $e$ of the monoid is in the image of both $f_A$ and $f_B$, then the protocol computes a commutative function $f:\Sigma^* \to M$.

\begin{theorem}
\label{thm:commutative-signal-free}
Let $(M, \ast, e)$ be a commutative monoid, and let $f_A, f_B: \Sigma^* \to M$ be two functions that define a signal-free aggregation protocol over $M$. If there exist words $w_A^{e}, w_B^{e} \in \Sigma^*$ such that $f_A(w_A^{e}) = e$ and $f_B(w_B^{e}) = e$, then the function $f: \Sigma^* \to M$ computed by the protocol is commutative. That is, for every word $u \in \Sigma^*$ and every permutation $\pi$ of the positions of $u$, we have $f(u) = f(\pi(u))$.
\end{theorem}

\begin{proof}
    First note that with the given conditions, $f_A(\epsilon)$ and $f_B(\epsilon)$ must be invertible elements in the monoid $M$. This is because first we know that there exist $w_A^{e}, w_B^{e} \in \Sigma^*$ such that $f_A(w_A^{e}) = e$ and $f_B(w_B^{e}) = e$. Now consider the words $w_A^{e}$ and $w_B^{e}$. We have:
    \[f(w_A^{e} w_B^{e}) = f_A(w_A^{e}) \ast f_B(w_B^{e}) = e \ast e = e.\]
    On the other hand, by split-invariance of protocols, we also have:
    \[f(w_A^{e} w_B^{e}) = f_A(\epsilon) \ast f_B(w_A^{e} w_B^{e}) = f_A(\epsilon) \ast f_B(w_A^{e} w_B^{e}).\]
    Combining these two equations, we get:
    \[e = f_A(\epsilon) \ast f_B(w_A^{e} w_B^{e}).\]
    This implies that $f_B(w_A^{e} w_B^{e})$ is the inverse of $f_A(\epsilon)$ in the monoid $M$. A similar argument shows that $f_A(w_A^{e} w_B^{e})$ is the inverse of $f_B(\epsilon)$.


    \begin{claim}
    For every word $u \in \Sigma^*$, we have:
    \[f_A(uv)= f_A(vu) \quad \text{and} \quad f_B(uv)= f_B(vu).\]
    \end{claim}  

    \begin{proof}[Proof of Claim]
    We proof the first equality; the second one follows by a symmetric argument. Let $u,v \in \Sigma^*$. First note that the following holds:
    \begin{equation}
        f(w) = f_A(w) \ast f_B(\epsilon) = f_A(\epsilon) \ast f_B(w) \label{eq:1}
    \end{equation}
    multiplying both sides by $f_B(\epsilon)^{-1}$ from the right and considering commutativity, we get:
    \begin{equation}
        f(w) f_B(\epsilon)^{-1} = f_A(w) = f_B(w) \ast f_A(\epsilon) \ast f_B(\epsilon)^{-1} \label{eq:2}
    \end{equation}
    and symmetrically we have:
    \begin{equation}
        f(w) f_A(\epsilon)^{-1} = f_B(w) = f_A(w) \ast f_B(\epsilon) \ast f_A(\epsilon)^{-1} \label{eq:3}
    \end{equation}

    \noindent 
    From Equation~\eqref{eq:2}, we have:

    \begin{equation}
    f_A(uv) = \underbrace{f(uv)}_{f_A(u) \ast f_B(v)} \ast f_B(\epsilon)^{-1}  = f_A(u) \ast f_B(v) \ast f_B(\epsilon)^{-1} \label{eq:4}
    \end{equation}
    \noindent
    Therefore the sequence of equalities below holds:
    \begin{align*}
    f_A(uv) & = f_A(u) \ast f_B(v) \ast f_B(\epsilon)^{-1} & \text{(from equation~\eqref{eq:4})} \\
    & = f_A(u) \ast f_A(v) \ast f_B(\epsilon) \ast f_A(\epsilon)^{-1} \ast f_B(\epsilon)^{-1} & \text{(from equation~\eqref{eq:3})} \\
    & = f_A(u) \ast f_A(v) \ast f_A(\epsilon)^{-1} & \text{(by cancellation)} \\
    & = f_B(u) \ast f_A(\epsilon) \ast f_B(\epsilon)^{-1} \ast f_A(v) \ast f_A(\epsilon)^{-1} & \text{(from equation~\eqref{eq:2})} \\
    & = f_B(u)  \ast f_B(\epsilon)^{-1} \ast f_A(v) & \text{(by cancellation)} \\
    & = f_A(v) \ast f_B(u)  \ast f_B(\epsilon)^{-1} & \text{(by commutativity)} \\
    & = f_A(vu) & \text{(from equation~\eqref{eq:4})}
    \end{align*}
    This completes the proof of the claim.
    \end{proof}

    \noindent
    For last intermediate step, we use the claim to show the following equalities:
    \begin{equation}
        f(uw) = f(wu) \quad \text{for every } u,w \in \Sigma^* \label{eq:cyclic_commutative}
    \end{equation}
    This is because:
    \begin{align*}
    f(uw) & = f_A(uw) \ast f_B(\epsilon) & \\
    & = f_A(wu) \ast f_B(\epsilon) & \text{(by the claim)} \\
    & = f(wu) &
    \end{align*}
    Also, for every $a,b \in \Sigma$ and every word $u \in \Sigma^*$, we have:
    \begin{equation}
        f(abu) = f(bau) \label{eq:adjacent_commutative}
    \end{equation}
    This is because:
    \begin{align*}
    f(abu) & = f_A(ab) \ast f_B(u) & \\
    & = f_A(ba) \ast f_B(u) & \text{(by the claim)} \\
    & = f(bau) &
    \end{align*}
    And symmetrically, for every $a,b \in \Sigma$ and every word $u \in \Sigma^*$, we have:
    \begin{equation}
        f(uab)= f(uba) \label{eq:adjacent_commutative_2}
    \end{equation}
    with similar reasoning.
    Now it is not had to see that using Equations~\eqref{eq:cyclic_commutative}, \eqref{eq:adjacent_commutative}, and \eqref{eq:adjacent_commutative_2}, we can transform any word $u$ into any permutation $\pi(u)$ of $u$ by a sequence of such transformations, proving that $f(u) = f(\pi(u))$. This completes the proof of the theorem.
\end{proof}

\begin{corollary}
\label{cor:commutative-signal-free-homomorphism}
Let $(M, \ast, e)$ be a commutative monoid, and let $f_A, f_B: \Sigma^* \to M$ be two functions that define a signal-free aggregation protocol over $M$. Then the function $f: \Sigma^* \to M$ computed by the protocol is commutative. That is, for every word $u \in \Sigma^*$ and every permutation $\pi$ of the positions of $u$, we have $f(u) = f(\pi(u))$.
\end{corollary}
\begin{proof}
    This is a direct consequence of Theorems~\ref{thm:signal-free-homomorphism} and~\ref{thm:commutative-signal-free}. It is enough to observe that $f$ can be expressed as:
    \[f(w) = s_A \ast h(w) \ast s_B \quad \text{for every } w \in \Sigma^*\]
    where $h: \Sigma^* \to M$ is a homomorphism from the free monoid $\Sigma^*$ to the monoid $M$, and $s_A, s_B \in M$. Now observe that the inner function $h$ can be expressed with signal-free protocols that satisfy the conditions of Theorem~\ref{thm:commutative-signal-free}. Therefore, $h$ is commutative hence $f$ is also commutative.
\end{proof}
%
\subsection{Addition Protocols}
\label{sec:addition-protocols}


\begin{definition}
\end{definition}
\section{Sequential Protocols}

\begin{definition}
    \label{Sequential-Protocols}
A sequential protocol over a monoid is a simple-aggregation protocol (as defined in Definition~\ref{simple-aggregation-definition}) such that the signal space of Bob, $Q_B$, is a singleton set. In other words, there is no communication from Bob to Alice, and only Alice can send signals to Bob. Such a protocol is therefore specified by: $f_A: \Sigma^* \to M$ and $q_A: \Sigma^* \to Q_A$ for Alice, and $f_B: \Sigma^* \times Q_A \to M$ for Bob. The final output on inputs $x,y$ is $f_A(x)\ast f_B(y, q_A(x))$.
\end{definition}

\begin{myexample}
    \label{example-sequential-protocol}
Let $TM_1,TM_2, \ldots$ be an enumeration of all Turing transducers. Consider the sequential protocol over the monoid $(\mathbb{N}, +, 0)$ where Alice's functions are defined as $f_A(x) = |x|$ and $q_A(x) = 1$ if $TM_{|x|}$ halts on input $  x $ and $f_A(x) =|x| + 1$ and $q_A(x) = 2$ otherwise. Bob's function is defined as $f_B(y, q_A(x)) = |y|+1$ if $q_A(x) = 1$ and $f_B(y, q_A(x)) = |y|$ if $q_A(x) = 2$.

Observe that the output of this protocol is always $|x| + |y| + 1 = |xy|+1$, regardless of whether $TM_{|x|}$ halts on input $ x $ or not. Therefore, the split-invariance condition is satisfied. 

\end{myexample}

\begin{corollary}
    \label{corollary-sequential-protocols}
    The above example shows that signal functions in sequential protocols and more generally in simple-aggregation protocols can be uncomputable functions. 
\end{corollary}

\begin{lemma}
    \label{prefix-lemma-sequential-protocols}
Let $(f_A,f_B,Q_A,Q_B,q_A)$ be ingridients of a sequential protocol over the monoid $(\Gamma^*, \cdot, \varepsilon)$. Now for any two string $w_1, w_2 \in \Sigma^*$, if $q_A(w_1) = q_A(w_1w_2)$ then $f_A(w_1)$ is a prefix of $f_A(w_1w_2)$. More percisely, there exists a string $u \in \Gamma^*$ (output domain) such that $f_A(w_1w_2) = f_A(w_1) \cdot u$. In another words, Alice's local output grows monotonically with respect to the prefix order on strings, whenever the signal sent to Bob does not change.
\end{lemma}

\begin{proof}
    
Assume towards a contradiction that $q_A(w_1) = q_A(w_1 w_2)$ but $f_A(w_1)$ is \emph{not} a prefix of $f_A(w_1 w_2)$. Let $q = q_A(w_1) = q_A(w_1 w_2)$. Then:
\begin{equation}
    f(w_1 w_2)
    = f_A(w_1)\, f_B(w_2,q)
    = f_A(w_1 w_2)\, f_B(\varepsilon,q)
    \label{eq:seq1}
\end{equation}
Since $f_A(w_1)$ is not prefix of $f_A(w_1w_2)$ by assumption, Equation~\eqref{eq:seq1} implies that $f_A(w_1w_2)$ is a strict prefix of $f_A(w_1)$. Write
\[
f_A(w_1) = f_A(w_1 w_2)\, v
\]
for some nonempty $v \in \Gamma^*$. Let
\[
u = f_A(w_1 w_2), \qquad z_1 = f_B(w_2,q)
\]
Then \eqref{eq:seq1} yields
\[
f_B(\varepsilon,q) = v z_1
\]
Now consider $w_1 w_2^2$. Using split-invariance,
\begin{equation}
    f(w_1 w_2^2)
    = f_A(w_1)\, f_B(w_2^2,q)
    = f_A(w_1 w_2)\, f_B(w_2,q)
    \label{eq:seq2}
\end{equation}
Let $z_2 = f_B(w_2^2,q)$. Substituting $f_A(w_1) = u v$ and $f_A(w_1 w_2) = u$ into \eqref{eq:seq2} gives
\[
u v z_2 = u z_1 \quad \Rightarrow \quad v\,z_2 = z_1
\]
Repeating this argument inductively, define
\[
z_k = f_B(w_2^k,q).
\]
The same cancellation shows
\[
z_k = v\,z_{k+1} \qquad \text{for all } k \ge 0
\]
Now since $v \neq \varepsilon$, the length of $z_k$ strictly decreases with $k$. This yields a contradiction.
\end{proof}

%====
\begin{corollary}
If for $w_1,w_2\in\Sigma^*$ we have $q_A(w_1)=q_A(w_1w_2)=:q$, then
\[
   f(w_1)\cdot \bigl(f_B(\varepsilon,q)\bigr)^{-1}
   \text{ is a prefix of }
   f(w_1w_2)\cdot \bigl(f_B(\varepsilon,q)\bigr)^{-1}
\]
Equivalently, there exists a string $u\in\Gamma^*$ such that
\[
   f(w_1w_2)\cdot \bigl(f_B(\varepsilon,q)\bigr)^{-1}
   \;=\;
   f(w_1)\cdot \bigl(f_B(\varepsilon,q)\bigr)^{-1}\cdot u
\]
\end{corollary}

\begin{proof}
For $q = q_A(w_1)=q_A(w_1w_2)$ we have, by definition of the global output of a sequential protocol:
\[
   f(w_1)=f_A(w_1)\,f_B(\varepsilon,q), \qquad
   f(w_1w_2)=f_A(w_1w_2)\,f_B(\varepsilon,q)
\]
By Lemma~\ref{prefix-lemma-sequential-protocols}, the assumption
$q_A(w_1)=q_A(w_1w_2)$ implies that $f_A(w_1)$ is a prefix of $f_A(w_1w_2)$.
Hence there exists $u\in\Gamma^*$ such that
\[
   f_A(w_1w_2)=f_A(w_1)\,u
\]
Substituting this expression into the global output gives
\[
   f(w_1w_2)=f_A(w_1)\,u\,f_B(\varepsilon,q)
            = \bigl(f_A(w_1)\,f_B(\varepsilon,q)\bigr)\,u
            = f(w_1)\,u
\]
Right-cancelling the common suffix $f_B(\varepsilon,q)$ yields
\[
   f(w_1w_2)\cdot \bigl(f_B(\varepsilon,q)\bigr)^{-1}
   \;=\;
   f(w_1)\cdot \bigl(f_B(\varepsilon,q)\bigr)^{-1}\cdot u
\]
which proves that
\[
   f(w_1)\cdot \bigl(f_B(\varepsilon,q)\bigr)^{-1}
   \preceq
   f(w_1w_2)\cdot \bigl(f_B(\varepsilon,q)\bigr)^{-1}
\]
\end{proof}

The above lemma and corollary was not used in the rest of the paper, but is included here as it may be of independent interest.
%====

Before stating the main theorem of this section, to avoid any ambiguity, we will use the definition of subsequential transducers as given in \cite{reutenauer1990subsequential}.

\begin{theorem}
    \label{theorem-main-sequential-protocol}
    For every sequential protocol over the monoid $(\Gamma^*, \cdot, \varepsilon)$, there exists a subsequential transducer that computes the same function.
\end{theorem}

\begin{proof}

        For every function $f: \Sigma^* \to \Gamma^*$ we will define $\widehat{f}: \Sigma^* \to \Gamma^*$ as follows:
        \[
        \widehat{f}(w) = lcp\{ f(wv) : v \in \Sigma^* \}
        \]
        where $lcp$ stands for longest common prefix.
        \begin{definition}
            \label{definition-finitely-many-derivatives}

            Suppose that a function $f: \Sigma^* \to \Gamma^*$ satisfies the property that for every $w_1, w_2 \in \Sigma^*$, $\widehat{f}(w_1)$ is a prefix of $f(w_1 w_2)$. Then we can define the \emph{derivative} of $f$ at $w \in \Sigma^*$ as the function $f_w: \Sigma^* \to \Gamma^*$ defined by:
            \[f_w(v) = \widehat{f}(w)^{-1} f(wv)\]
            for every $v \in \Sigma^*$. And we say that $f$ has \emph{finitely many derivatives} if the set $\{ f_w : w \in \Sigma^* \}$ is finite.

        \end{definition}
        

        Let $(f_A,f_B,Q_A,q_A)$ be the ingredients of a sequential protocol over the monoid $(\Gamma^*, \cdot, \varepsilon)$. Consider the function $f: \Sigma^* \to \Gamma^*$ defined by the sequential protocol, i.e. $f(w) = f_A(w) f_B(\varepsilon, q_A(w))$. First we will show that the function $f$ satisfies the property which is required in Definition~\ref{definition-finitely-many-derivatives}. Then we will show that $f$ has finitely many derivatives, or more precisely, we will show that for every $w_1, w_2 \in \Sigma^*$ such that $q_A(w_1) = q_A(w_2)$, it holds that $f_{w_1} = f_{w_2}$. This will imply that the number of distinct derivatives of $f$ is at most $|Q_A|$, which is finite.

        \noindent
        First define the function $\widehat{u}: Q_A \to \Gamma^*$ as follows:
        \[\widehat{u}(q) = lcp\{ u \in \Sigma^* : u \text{ is a prefix of } f_B(v,q) \text{ for all } v \in \Sigma^* \}\]
        that is, $\widehat{u}(q)$ is the longest common prefix of the strings $f_B(v,q)$ for all $v \in \Sigma^*$. Now we will show that for every $w \in \Sigma^*$, $\widehat{f}(w) = f_A(w) \cdot \widehat{u}(q_A(w))$. Observe that $f_A(w)$ is always a prefix of $f(w \cdot v)$ for any $v \in \Sigma^*$, since:
        \[f(w \cdot v) = f_A(w) \cdot f_B(v,q_A(w))\]
        Thus, there exists a string $u \in \Sigma^*$ such that $\widehat{f}(w) = f_A(w) \cdot u$. By construction, $ \widehat{u}(q_A(w)) $ is a prefix of $ f_B(v,q_A(w)) $ for all $ v \in \Sigma^* $. Therefore, $ f_A(w) \cdot \widehat{u}(q_A(w)) $ is a prefix of $\widehat{f}(w)$. Now it is enough to show that $\widehat{f}(w)$ can not be strictly longer than $ f_A(w) \cdot \widehat{u}(q_A(w)) $. Assume by contradiction that $\widehat{f}(w)$ is strictly longer than $ f_A(w) \cdot \widehat{u}(q_A(w)) $.  Then, there exists a string $u' \in \Sigma^*$ such that $\widehat{f}(w) = f_A(w) \cdot \widehat{u}(q_A(w)) \cdot u'$, with $u' \neq \varepsilon$. By definition of $\widehat{f}$, for any $ v \in \Sigma^* $, $ f_A(w) \cdot \widehat{u}(q_A(w)) \cdot u' $ is a prefix of $ f(w \cdot v) = f_A(w) \cdot f_B(v,q_A(w)) $. Therefore, $\widehat{u}(q_A(w)) \cdot u' $ is a prefix of $ f_B(v,q_A(w)) $ for all $ v \in \Sigma^* $. This contradicts the definition of $ \widehat{u}(q_A(w)) $ as the longest common prefix of the strings $ f_B(v,q_A(w)) $ for all $ v \in \Sigma^* $. Hence, $\widehat{f}(w) = f_A(w) \cdot \widehat{u}(q_A(w))$.


        Now consider two function $f_{w_1}$ and $f_{w_2}$ for some $w_1, w_2 \in \Sigma^*$ such that $q_A(w_1) = q_A(w_2)$. We will show that $f_{w_1} = f_{w_2}$.
        To be more precise, we will show that for every string $w \in \Sigma^*$, $f_w$ is the function that maps any $v \in \Sigma^*$ to $\widehat{u}(q_A(w))^{-1} f_B(v, q_A(w))$. This is because for any $ v \in \Sigma^* $ we have:
        \begin{align*}
        f_w(v) &= \widehat{f}(w)^{-1} \cdot f(wv)\\
        &= (f_A(w) \cdot \widehat{u}(q_A(w)))^{-1} \cdot (f_A(w) \cdot f_B(v,q_A(w)))\\
        &= \widehat{u}(q_A(w))^{-1} \cdot f_B(v,q_A(w))
        \end{align*}
        Therefore, for any $w_1, w_2 \in \Sigma^*$ such that $q_A(w_1) = q_A(w_2)$, it holds that $f_{w_1} = f_{w_2}$. Observe that the number of distinct functions $f_w$ is at most $|Q_A|$, but can be strictly less. Therefore it is possible that for some $q_1, q_2 \in Q_A$ with $q_1 \neq q_2$, it holds that $f_{w_1} = f_{w_2}$. But in any case, the number of distinct derivatives of $f$ is at most $|Q_A|$, which is finite. Hence, $f$ has finitely many derivatives. \footnote{This theorem was proved in collaboration with Thomas Filasto.}
    
    \begin{theorem}
        \label{theorem-subsequential-transducer}
         A function $f: \Sigma^* \to \Gamma^*$ can be computed by a subsequential transducer if and only if it has finitely many derivatives (as defined in Definition~\ref{definition-finitely-many-derivatives}). 
    \end{theorem}
    \noindent
    Also the proof can be found in \cite{reutenauer1990subsequential} (see Theorem 1), I include it here because it is essential for the next steps.
    \begin{proof}
    Let $ u,v \in \Sigma^* $, we start by showing the following equality:
    \begin{equation}
        \label{congruence-equation-derivatives}
        \widehat{f}(uv) = \widehat{f}(u) \cdot \widehat{f_u}(v)
    \end{equation}
    
    \begin{align*}
    \widehat{f}(uv) &= lcp \lbrace f(uvw) \mid w \in \Sigma^* \rbrace\\
    &= lcp \lbrace \widehat{f}(u) \cdot f_u(vw) \mid w \in \Sigma^* \rbrace &(f_u(vw) = \widehat{f}(u)^{-1} \cdot f(uvw))\\
    &= \widehat{f}(u) \cdot lcp \lbrace f_u(vw) \mid w \in \Sigma^* \rbrace\\
    &= \widehat{f}(u) \cdot \widehat{f_u}(v)
    \end{align*}
    Now, we will show that $ f_{u v} = (f_u)_v $. Let $ w \in \Sigma^* $, we have:
    \begin{align*}
    f_{u v}(w) &= \widehat{f}(uv)^{-1} \cdot f(uvw)\\
    &=(\widehat{f}(u) \cdot \widehat{f_u}(v))^{-1} \cdot f(uvw) &\text{(by Equation \ref{congruence-equation-derivatives})}\\ 
    &=(\widehat{f}(u) \cdot \widehat{f_u}(v))^{-1} \cdot (\widehat{f}(u) \cdot f_u(vw)) &(f_u(vw) = \widehat{f}(u)^{-1} \cdot f(uvw))\\
    &=(\widehat{f}(u) \cdot \widehat{f_u}(v))^{-1} \cdot (\widehat{f}(u) \cdot (\widehat{f}_u(v) \cdot (f_u)_v(w))) &((f_u)_v(w) = \widehat{f}_u(v)^{-1} \cdot f_u(vw))\\
    &= (f_u)_v(w)
    \end{align*}
    Now Let $ w,w' \in \Sigma^* $ such that $ f_w = f_{w'} $, then for any $ v \in \Sigma^* $, the following equality holds:
    \begin{equation}
    \label{equation-derivative-equivalence}
    f_w = f_{w'} \implies f_{wv} = f_{w'v}
    \end{equation}
    This is because:
    \begin{align*}
    f_{wv} &= (f_w)_v &(f_{uv} = (f_u)_v)\\
    &= (f_{w'})_v &(f_w = f_{w'})\\
    f_{wv} &= f_{w'v} &(f_{uv} = (f_u)_v)
    \end{align*}
    
    From the above formulae, one can construct a subsequential transducer that computes the same function as $ f $. The states of the transducer are the distinct derivatives of $ f $, i.e. $ Q = \lbrace f_w \mid w \in \Sigma^* \rbrace $. The initial state is $ f_{\varepsilon} $. The state-transition function $ \delta: Q \times \Sigma \to Q $ is defined as follows: for any $ f_w \in Q $ and $ a \in \Sigma $, $ \delta(f_w,a) = f_{wa} $. The output function $ \lambda: Q \times \Sigma \to \Gamma^* $ is defined as follows: for any $ f_w \in Q $ and $ a \in \Sigma $, $ \lambda(f_w,a) = \widehat{f_w}(a) $. Finally, the terminal output function $ t: Q \to \Gamma^* $ is defined as follows: for any $ f_w \in Q $, $ t(f_w) = \widehat{f}(w)^{-1} f(w) $. Having the above formulae in mind, one can verify that the constructed subsequential transducer computes the same function as $ f $.
    \end{proof}
    \noindent
    This completes the proof of the main theorem of this section.
\end{proof}

%==== Normalized Sequential Protocols 
To have a clean characterization of class of protocols, we like to have some kind of normalization for each one of them to avoid unwanted behaviors such as the one shown in Example~\ref{example-sequential-protocol}. We think that if the signal functions of a protocol are computable by some simple computational model, then the protocol is more well-behaved. If we want to unwrap this idea further, let first call this simple computational model $\mathcal{M}$. Then, we can roughly say that if the signal functions of a protocol are computable by $\mathcal{M}$, then we expect that Alice output functions will be computable by some relevent computational model (by relevant we mean a model that is equivalent to the class of protocols we are looking at).

Therefore, we will define normalization for each class of protocols based on the computability of their signal functions. To be more specific, we want to have a signal function that is computable by a Deterministic Finite Semi-Automaton (DFSA). Of course, by theorem~\ref{theorem-subsequential-transducer}, having a regular signal function is immidiate. It is because for any sequential protocol one can construct a subsequential transducer that computes the same function, and the state-transition function of a subsequential transducer is computable by a Deterministic Finite Semi-Automaton. But I think looking into normalization process independently is interesting and may help us in defining normalization for more expressive classes of protocols in the future.
\begin{definition}
    \label{definition-normalized-sequential-protocol}
    A sequential protocol $(f_A,f_B,Q_A,q_A)$ over the monoid $(\Gamma^*, \cdot, \varepsilon)$ is called \emph{normalized} if there exists a Deterministic Finite Semi-Automaton $\mathcal{A} = (Q_A, \Sigma, \delta, q_{A_0})$ such that for every $ w \in \Sigma^* $, $ q_A(w) = \delta^*(q_{A_0}, w) $, where $\delta^*$ is the extended transition function of $\mathcal{A}$.
\end{definition}
\begin{theorem}
    \label{theorem-normalized-sequential-protocol}
    For every sequential protocol over the monoid $(\Gamma^*, \cdot, \varepsilon)$, there exists a normalized sequential protocol that computes the same function.
\end{theorem}
\begin{proof}
    Let $(f_A,f_B,Q_A,q_A)$ be the ingredients of a sequential protocol over the monoid $(\Gamma^*, \cdot, \varepsilon)$. We will first construct another protocol $(f_A',f_B',Q_A,q_A)$ that computes the same function as $(f_A,f_B,Q_A,q_A)$ such that the alice output function $f_A'$ is exactly $\widehat{f}$ as defined in the proof of theorem~\ref{theorem-subsequential-transducer}. Then we will show that $(f_A',f_B',Q_A,q_A)$ can be normalized.

    Observe that because alice can compute $q_A$, she can also compute $\widehat{u}$ as defined in the proof of theorem~\ref{theorem-subsequential-transducer}. Therefore, she can compute $\widehat{f}$ as well. Now define the functions $f_A', f_B'$ as follows:
    \[f_A'(w) = \widehat{f}(w)\]
    \[f_B'(w,q) = \widehat{u}(q)^{-1} f_B(w,q)\]
    for every $ w \in \Sigma^* $ and $ q \in Q_A $. It is easy to see that the sequential protocol $(f_A',f_B',Q_A,q_A)$ computes the same function as $(f_A,f_B,Q_A,q_A)$. Now we will define an equivalence relation on Alice's signal space $Q_A$ as follows: First because we know that every signal $ q \in Q_A $ is reachable by some input string we can choose one string $ w_q \in \Sigma^* $ such that $ q_A(w_q) = q $ for every $ q \in Q_A $. Now we define the equivalence relation $\equiv$ on $Q_A$ as follows:
    \[q_1 \equiv q_2 \iff f_{w_{q_1}} = f_{w_{q_2}}\]
    for every $q_1, q_2 \in Q_A $. It is easy to see that $\equiv$ is indeed an equivalence relation. Now we will show that signals that are equivalent under $\equiv$ behave the same under any input string from bob's side. Let $w$ be an arbitrary string in $\Sigma^*$ and let $q_1, q_2 \in Q_A$ such that $q_1 \equiv q_2$. We want to show that $f_B'(w,q_1) = f_B'(w,q_2)$. Choose two arbitary strings $w_1, w_2 \in \Sigma^*$ such that $q_A(w_1) = q_1$ and $q_A(w_2) = q_2$. Now if we show that $f_{w_1} = f_{w_2}$ then we are done. This is because $f(w_1 w) = \widehat{f}(w_1) f_B'(w,q_1)$ and $f(w_2 w) = \widehat{f}(w_2) f_B'(w,q_2)$. Therefore $f_B'(w,q_1) = \widehat{f}(w_1)^{-1} f(w_1 w)=f_{w_1}(w)$ and $f_B'(w,q_2) = \widehat{f}(w_2)^{-1} f(w_2 w)=f_{w_2}(w)$. We have shown in \ref{theorem-main-sequential-protocol} that for any $w_1, w_2 \in \Sigma^*$ such that $q_A(w_1) = q_A(w_2)$, it holds that $f_{w_1} = f_{w_2}$. Therefore, $f_{w_1} = f_{w_{q_1}} = f_{w_{q_2}} = f_{w_2}$. Hence, $f_B'(w,q_1) = f_B'(w,q_2)$.

    Now we can define the normalized sequential protocol. Let $Q_A' = Q_A /_{\equiv}$ be the quotient set of $Q_A$ under the equivalence relation $\equiv$, and $q_A': \Sigma^* \to Q_A'$ be defined as follows:
    \[q_A'(w) = [q_A(w)]_{\equiv}\]
    for every $ w \in \Sigma^* $. Then $(f_A',f_B',Q_A',q_A')$ defines a normalized sequential protocol that computes the same function as $(f_A,f_B,Q_A,q_A)$. To show that it is indeed normalized, we need to show that $q_A'$ is indeed regular. This follows from the formulae we have shown in the proof of theorem~\ref{theorem-main-sequential-protocol}, by Myhill-Nerode theorem. More precisely, for any $ w_1, w_2 \in \Sigma^* $, if $ q_A'(w_1) = q_A'(w_2) $, then for every string $ v \in \Sigma^* $, $q_A'(w_1 v) = q_A'(w_2 v)$. This is because $ q_A'(w_1) = q_A'(w_2) $ implies that $ q_A(w_1) \equiv q_A(w_2) $, which in turn implies that $ f_{w_1} = f_{w_2} $. Therefore, by equation~\eqref{equation-derivative-equivalence}, for every string $ v \in \Sigma^* $, $ f_{w_1 v} = f_{w_2 v} $, which implies that $ q_A(w_1 v) \equiv q_A(w_2 v) $, and hence $ q_A'(w_1 v) = q_A'(w_2 v) $. Thus, by Myhill-Nerode theorem, there exists a Deterministic Finite Semi-Automaton that computes $q_A'$. This completes the proof.
\end{proof}

If we look at the process of normalization, we can observe that we simply lift the $\widehat{u}$ function into alice's output function, because it was only dependent on the signal sent to bob and not on the local input of bob. Therefore, if we want to define normalization for more expressive classes of protocols, we can follow the same idea. That is, we can lift the part of the $f_B$ function that is only dependent on the signal sent to bob into alice's output function. This way, we expect that to obtain a normalized protocol more easily. But of course, more investigation is needed to see if this idea works in more expressive classes of protocols. Also it is worth mentioning that in this normalization process, we could define the equivalence relation $\equiv$ differently. In fact, we could define it based on the behavior of bob's output function, or more precisely, we could define it as follows:
\[q_1 \equiv q_2 \iff \forall w \in \Sigma^*, f_B(w,q_1) = f_B(w,q_2)\]
for every $q_1, q_2 \in Q_A $. This way, we can avoid the use of derivatives in the normalization process. This alternative definition of $\equiv$ may be more suitable for more expressive classes of protocols, because in those classes, the concept of derivatives may not be well-defined. In fact bob's output function is exactly $f_w$ when alice has the local input $w$. This is because Alice's output function is $\widehat{f}$, and therefore we have:
\begin{align*}
f(wv) &= f_A'(w) f_B'(v, q_A(w))\\
&= \widehat{f}(w) f_B'(v, q_A(w))
\end{align*}
Thus, $f_B'(v, q_A(w)) = \widehat{f}(w)^{-1} f(wv) = f_w(v)$. Hence, defining $\equiv$ based on the behavior of bob's output function is equivalent to defining it based on the behavior of derivatives in this case. We will explore this alternative definition of normalization in next sections when we look at more expressive classes of protocols.

% LTeX: lang=en-US

\subsection{Monotone Protcols}

\begin{definition}
  \label{monotone_protocol_definition}
  A monotone protocol is a simple-aggregation protocol (as defined in Definition~\ref{simple-aggregation-definition}) over the monoid
  $(\Gamma^*, \cdot, \varepsilon)$
  where $\Gamma^*$ is the set of all strings over the alphabet $\Gamma$,
  $\cdot$ is the concatenation operation,
  and $\varepsilon$ is the empty string.
\end{definition}

We can have similar lemma as Lemma~\ref{prefix-lemma-sequential-protocols} for monotone protocols, stating that configurations of Alice grow monotonically with respect to the prefix order on strings, whenever the signal sent by either Alice does not change.

\begin{lemma}
    \label{prefix-lemma-monotone-protocols}
Let $(f_A,f_B,Q_A,Q_B,q_A,q_B)$ be ingridients of a monotone protocol over the monoid $(\Gamma^*, \cdot, \varepsilon)$. Now for any two string $w_1, w_2 \in \Sigma^*$, if $q_A(w_1) = q_A(w_1w_2)$ then for every $1 \leq i \leq |Q_B|$ the $i$-th component of configuration $C_A(w_1)$ is prefix of the $i$-th component of configuration $C_A(w_1w_2)$. More percisely, if we denote the $i$-th component of configuration $C_A(w)$ as $C_A(w)[i]$, then we have:
\[C_A(w_1w_2)[i] = C_A(w_1)[i] \cdot u \] for some $u \in \Gamma^*$ (output domain).
In other words, configurations of Alice grow monotonically with respect to the prefix order on strings, whenever the signal sent to Bob does not change.
\end{lemma}

\begin{lemma}
  \label{lem:concat-ab-rational}
  For every rational function $f \colon \Sigma^* \to \Gamma^*$
  (computable by a bimachine), there exists a monotone prtocol expressing $f$.
\end{lemma}
\begin{proof}
  Let us consider a bimachine computing $f$,
  with a left-to-right automaton
  $\mathcal{A}_L$ and a right-to-left automaton $\mathcal{A}_R$.
  Let us define $Q$ the union of the states of $\mathcal{A}_L$ and $\mathcal{A}_R$.
  Let us define $q_A(u)$ as the state in which $\mathcal{A}_L$ ends
  after reading $u$, and $q_B(v)$ as the state in which $\mathcal{A}_R$ ends
  after reading $v$ (from right to left).

  It we can now define $f_A(u,s)$ to be the output of the bimachine on the word
  $u$ if the left-to-right automaton starts in its initial state, the
  right-to-left automaton starts in state $s \in Q$, and similarly, $f_B(v,s)$
  to be the output of the bimachine on the word $v$ if the left-to-right
  automaton starts in state $s \in Q$, and the right-to-left automaton starts
  in its initial state.
\end{proof}



\begin{definition}
  A simple streaming string transducer (SSST) is 
  given by a finite set $Q$ of states,
  a transition function $\delta \colon Q \times \Sigma \to Q$,
  a finite number $r$ of registers,
  an update function $\rho \colon Q \times \Sigma \times \{ 1, ..., r\} \to 
  \{1, ..., r\} \times \Gamma^*$,
  an initial state $q_0 \in Q$,
  an inital register assignment $I \colon \{1, ..., r\} \to \Gamma^*$,
  and a final output function $F \colon Q \to \{1, ..., r\} \times \Gamma^*$.


  The semantics of a SSST is as follows: when reading a word, the state 
  is initially $q_0$ and the registers are initialized according to $I$.
  When reading a letter $a$ in state $q$, the SSST moves to state
  $\delta(q,a)$ and for each register $i$, it updates register $i$
  by appending the string given by the second component of 
  $\rho(q,a,i)$ to the content of
  register $j$ where $j$ is the first component of $\rho(q,a,i)$.
  After reading the whole input word, if the SSST is in state $q$,
  then the output is obtained by concatenating the string given by
  $F(q)$ to the content of the register indicated by the first component
  of $F(q)$.
\end{definition}

\begin{lemma}
  A function $f \colon \Sigma^* \to \Gamma^*$
  is rational (computable by a bimachine)
  if and only if 
  it is computable by an SSST.
\end{lemma}
\begin{proof}

Given a bimachine computing $f$, one can construnct an SSST with states that are
all the possible states of the left-to-right automaton, and one register per
state of the right-to-left automaton. 
When reading a letter $a$ in state $q$, the SSST moves to state
$\delta(q,a)$. 
Given a state $p$ of the right-to-left automaton, the register 
corresponding to $p$ is updated by 
appending to the register $\delta(p,a)$ the output produced by 
the bimachine when the left-to-right automaton is in state $q$,
the right-to-left automaton is in state $p$, and the letter read is $a$.
Finally, the output function $F$ is defined so that the output is produced
by the register corresponding to the initial state of the right-to-left
automaton.

Conversely, consider an SSST computing $f$, one can 
construct a bimachine that guesses which register will produce the final output,
and only care about updating this register throughout the computation. Since 
there is only one register now, and we only append to it, this means that 
we actually can directly produce the output on the fly as an actual bimachine.
\end{proof}


Let us now consider a function $f$ expressible by a monotone protocol, so that $f(uv) = f_A(u, q_B(v)) \; f_B(v, q_A(u))$ for
suitable functions $f_A, f_B, q_A, q_B$ and finite set $Q$ of signals. Recall the definition of configuration of a simple-aggregation protocol from Definition~\ref{configuration-definition}. Let us order configurations as follows: $C(u) \leq C(v)$ if and only if $q_A(u)
= q_A(v)$ and for all $s \in Q$, there exists $r \in Q$ such that $f_A(u, s)$
is a prefix of $f_A(v, r)$.

We claim that configurations are somehow close to being endowed with a
well-quasi-ordering. Before doing that, let us first normalize the notion of
configuration to contain only signals in $Q$ that can actually be sent by Bob.
That is, for every $s \in Q$ such that there is no $v \in \Sigma^*$ with
$q_B(v) = s$, we remove $s$ from $Q$ and we remove the corresponding word
$f_A(u, s)$ from the configuration $C(u)$.

\begin{lemma}
  \label{lem:pref-wqo}
  For every infinite sequence $(w_n)_{n \in \mathbb{N}}$ of words in $\Gamma^*$
  that is increasing for the prefix order,
  there exist $i < j$ such that $C(w_i) \leq C(w_j)$.
\end{lemma}
\begin{proof}
  Let us consider such an infinite sequence $(w_n)_{n \in \mathbb{N}}$.
  Since $Q$ is finite, there exists an infinite subsequence 
  that has a same state $q \in Q$.

  Let us notice that the sequence of lenghts of the words 
  in $(f_A(w_n, s))_{s \in Q}$ is a sequence of $|Q|$-tuples of natural numbers,
  hence by Dickson's lemma, there exists an infinite subsequence
  where these lenghts are non-decreasing.

  Let us now fix a $q \in Q$ and look at 
  $|f_A(w_n, q)|$ for this $q$ when $n$ ranges over the indices of the
  subsequence defined above. Either $|f_A(w_n, q)|$ is bounded, in which case
  we can extract further a subsequence where $f_A(w_n, q)$ is constant, or
  $|f_A(w_n, q)|$ is unbounded, in which case we can extract an infinite subsequence
  where $|f_A(w_n, q)|$ is strictly increasing. 

  By repeating this process for all the finitely many $q \in Q$, we can extract
  a subsequence $(w_{n_k})_{k \in \mathbb{N}}$ such that for all $q \in Q$,
  either $f_A(w_{n_k}, q)$ is constant, or the lenghts $|f_A(w_{n_k}, q)|$ are
  strictly increasing.

  Let us now take $i = n_0$, and $j$ be large enough so that for every $q \in
  Q$ where $|f_A(w_{n_k}, q)|$ is strictly increasing, for every $s \in Q$, we
  have $|f_A(w_j, q)| > |f_A(w_i, s)|$.

  Let us prove that $C(w_i) \leq C(w_j)$. By construction, we have
  $q_A(w_i) = q_A(w_j)$. Now, let us consider any $s \in Q$. If
  $f_A(w_{n_k}, s)$ is constant, then in particular
  $f_A(w_i, s) = f_A(w_j, s)$, so $f_A(w_i, s)$ is a prefix of
  $f_A(w_j, s)$. 
  Otherwise, $|f_A(w_j, s)| > |f_A(w_i, r)|$
  for all $r \in Q$. Let us now consider a word $v$ such that 
  $q_B(v) = s$ and let us write $w_j = w_i v'$.

  \begin{equation*}
    f_A(w_j, s) \; f_B(v, q_A(w_j))
    f(w_j v) = f(w_i v' v)
    = f_A(w_i, q_B(v' v)) \; f_B(v' v, q_A(w_i))
  \end{equation*}

  Since $|f_A(w_j, s)| > |f_A(w_i, r)|$ for all $r \in Q$,
  this means that $f_A(w_i, q_B(v' v))$ is a prefix of $f_A(w_j, s)$,
  as desired.

  We have proven that $C(w_i) \leq C(w_j)$.
\end{proof}


Now is the time to prove the main result of this section by constructing an
SSST computing $f$ from this well-quasi-ordering on configurations.

\begin{theorem}

  A function $f \colon \Sigma^* \to \Gamma^*$
  is computable by a monotone protocol
  if and only if it is computable by an SSST.

\end{theorem}
\begin{proof}
  The direction from SSST to monotone protocol is straightforward, since SSTs correspond to rational functions 
  and we have already seen that rational functions can be expressed by a monotone protocol in Lemma~\ref{lem:concat-ab-rational}.


  Conversely, let us consider a function $f$ expressible by a monotone protocol. We are going to construct an SSST computing
  $f$. Let us consider as states of the SSST words $u\in \Sigma^*$ that are a
  finite set of representatives of minimal elements of configurations for the
  order $\leq$ defined above. By Lemma~\ref{lem:pref-wqo}, the tree of all
  words ordered by prefixes can be trimmed using $\leq$ so that every branch is
  finite, hence the tree itself is finite due to K\"onig's lemma, proving the
  existence of such a finite set of representatives. To this finite list of
  representatives, we add the empty word $\varepsilon$ if it is not already
  present.

  The initial state of the SSST is $\varepsilon$, and the content of the
  registers is initialized so that for every $s \in Q$, the register
  corresponding to $s$ contains $f_A(\varepsilon, s)$.

  The transition function is defined as follows: $\delta(u, a) = v$, where $v$
  is such that $C(v) \leq C(u a)$ (which exists by definition of the
  representatives). 

  The update function is defined as follows: when reading a letter $a$ in
  state $u$, for each $s \in Q$, let us consider the word
  $f_A(u a, s)$. By definition of the order $\leq$, there exists $r \in Q$ such that
  $f_A(u a, s)$ contains $f_A(\delta(u,a), r)$ as a prefix, hence
  we can write the following update for the register corresponding to $r$:
  \begin{equation*}
    R_s \leftarrow R_r \; w_{u,a,s,v,r}
  \end{equation*}
  where $w_{u,a,s,v,r}$ is the suffix such that
  $f_A(ua, s) = f_A(v, r) \; w_{u,a,s,v,r}$.

  Finally, the output function is defined as follows: when the SSST is in state
  $u$, we output the register $q_B(\varepsilon)$ concatenated with
  $f_B(\varepsilon, q_A(u))$.

  \medskip

  \medskip

  Let us prove by induction on the lenght of $w$ that the SSST computes $f(w)$
  correctly. \textbf{Todo: non-trivial.} 

  \fbox{ 
  to be completed 
  }
  

\end{proof}


As a corollary, we obtain that functions expressible by monotone protocols are exactly the rational functions.

\begin{corollary}

  A function $f \colon \Sigma^* \to \Gamma^*$
  is expressible by a monotone protocol
  if and only if it is rational (computable by a bimachine).
\end{corollary}

