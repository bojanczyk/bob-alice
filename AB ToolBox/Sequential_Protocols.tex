\subsection{Sequential Protocols}

\begin{definition}
    \label{Sequential-Protocols}
A sequential protocol over a monoid is a simple-aggregation protocol (as defined in Definition~\ref{simple-aggregation-definition}) such that the signal space of Bob, $Q_B$, is a singleton set. In other words, there is no communication from Bob to Alice, and only Alice can send signals to Bob. Such a protocol is therefore specified by: $f_A: \Sigma^* \to M$ and $q_A: \Sigma^* \to Q_A$ for Alice, and $f_B: \Sigma^* \times Q_A \to M$ for Bob. The final output on inputs $x,y$ is $f_A(x)\ast f_B(y, q_A(x))$.
\end{definition}

\begin{myexample}
    \label{example-sequential-protocol}
Let $TM_1,TM_2, \ldots$ be an enumeration of all Turing machines. Consider the sequential protocol over the monoid $(\mathbb{N}, +, 0)$ where Alice's functions are defined as $f_A(x) = |x|$ and $q_A(x) = 1$ if $TM_{|x|}$ halts on input $  x $ and $f_A(x) =|x| + 1$ and $q_A(x) = 2$ otherwise. Bob's function is defined as $f_B(y, q_A(x)) = |y|+1$ if $q_A(x) = 1$ and $f_B(y, q_A(x)) = |y|$ if $q_A(x) = 2$.

Observe that the output of this protocol is always $|x| + |y| + 1 = |xy|+1$, regardless of whether $TM_{|x|}$ halts on input $ x $ or not. Therefore, the split-invariance condition is satisfied. 

\end{myexample}

\begin{corollary}
    \label{corollary-sequential-protocols}
    The above example shows that signal functions in sequential protocols and more generally in simple-aggregation protocols can be uncomputable functions. 
\end{corollary}

\begin{lemma}
    \label{prefix-lemma-sequential-protocols}
Let $(f_A,f_B,Q_A,Q_B,q_A)$ be ingridients of a sequential protocol over the monoid $(\Gamma^*, \cdot, \varepsilon)$. Now for any two string $w_1, w_2 \in \Sigma^*$, if $q_A(w_1) = q_A(w_1w_2)$ then $f_A(w_1)$ is a prefix of $f_A(w_1w_2)$. More percisely, there exists a string $u \in \Gamma^*$ (output domain) such that $f_A(w_1w_2) = f_A(w_1) \cdot u$. In another words, Alice's local output grows monotonically with respect to the prefix order on strings, whenever the signal sent to Bob does not change.
\end{lemma}

\begin{proof}
    
Assume towards a contradiction that $q_A(w_1) = q_A(w_1 w_2)$ but $f_A(w_1)$ is \emph{not} a prefix of $f_A(w_1 w_2)$. Let $q = q_A(w_1) = q_A(w_1 w_2)$. Then:
\begin{equation}
    f(w_1 w_2)
    = f_A(w_1)\, f_B(w_2,q)
    = f_A(w_1 w_2)\, f_B(\varepsilon,q)
    \label{eq:seq1}
\end{equation}
Since $f_A(w_1)$ is not prefix of $f_A(w_1w_2)$ by assumption, Equation~\eqref{eq:seq1} implies that $f_A(w_1w_2)$ is a strict prefix of $f_A(w_1)$. Write
\[
f_A(w_1) = f_A(w_1 w_2)\, v
\]
for some nonempty $v \in \Gamma^*$. Let
\[
u = f_A(w_1 w_2), \qquad z_1 = f_B(w_2,q)
\]
Then \eqref{eq:seq1} yields
\[
f_B(\varepsilon,q) = v z_1
\]
Now consider $w_1 w_2^2$. Using split-invariance,
\begin{equation}
    f(w_1 w_2^2)
    = f_A(w_1)\, f_B(w_2^2,q)
    = f_A(w_1 w_2)\, f_B(w_2,q)
    \label{eq:seq2}
\end{equation}
Let $z_2 = f_B(w_2^2,q)$. Substituting $f_A(w_1) = u v$ and $f_A(w_1 w_2) = u$ into \eqref{eq:seq2} gives
\[
u v z_2 = u z_1 \quad \Rightarrow \quad v\,z_2 = z_1
\]
Repeating this argument inductively, define
\[
z_k = f_B(w_2^k,q).
\]
The same cancellation shows
\[
z_k = v\,z_{k+1} \qquad \text{for all } k \ge 0
\]
Now since $v \neq \varepsilon$, the length of $z_k$ strictly decreases with $k$. This yields a contradiction.
\end{proof}

%====
\begin{corollary}
If for $w_1,w_2\in\Sigma^*$ we have $q_A(w_1)=q_A(w_1w_2)=:q$, then
\[
   f(w_1)\cdot \bigl(f_B(\varepsilon,q)\bigr)^{-1}
   \text{ is a prefix of }
   f(w_1w_2)\cdot \bigl(f_B(\varepsilon,q)\bigr)^{-1}
\]
Equivalently, there exists a string $u\in\Gamma^*$ such that
\[
   f(w_1w_2)\cdot \bigl(f_B(\varepsilon,q)\bigr)^{-1}
   \;=\;
   f(w_1)\cdot \bigl(f_B(\varepsilon,q)\bigr)^{-1}\cdot u
\]
\end{corollary}

\begin{proof}
For $q = q_A(w_1)=q_A(w_1w_2)$ we have, by definition of the global output of a sequential protocol:
\[
   f(w_1)=f_A(w_1)\,f_B(\varepsilon,q), \qquad
   f(w_1w_2)=f_A(w_1w_2)\,f_B(\varepsilon,q)
\]
By Lemma~\ref{prefix-lemma-sequential-protocols}, the assumption
$q_A(w_1)=q_A(w_1w_2)$ implies that $f_A(w_1)$ is a prefix of $f_A(w_1w_2)$.
Hence there exists $u\in\Gamma^*$ such that
\[
   f_A(w_1w_2)=f_A(w_1)\,u
\]
Substituting this expression into the global output gives
\[
   f(w_1w_2)=f_A(w_1)\,u\,f_B(\varepsilon,q)
            = \bigl(f_A(w_1)\,f_B(\varepsilon,q)\bigr)\,u
            = f(w_1)\,u
\]
Right-cancelling the common suffix $f_B(\varepsilon,q)$ yields
\[
   f(w_1w_2)\cdot \bigl(f_B(\varepsilon,q)\bigr)^{-1}
   \;=\;
   f(w_1)\cdot \bigl(f_B(\varepsilon,q)\bigr)^{-1}\cdot u
\]
which proves that
\[
   f(w_1)\cdot \bigl(f_B(\varepsilon,q)\bigr)^{-1}
   \preceq
   f(w_1w_2)\cdot \bigl(f_B(\varepsilon,q)\bigr)^{-1}
\]
\end{proof}

The above lemma and corollary was not used in the rest of the paper, but is included here as it may be of independent interest.
%====

Before stating the main theorem of this section, to avoid any ambiguity, we will use the definition of subsequential transducers as given in \cite{reutenauer1990subsequential}.

\begin{theorem}
    For every sequential protocol over the monoid $(\Gamma^*, \cdot, \varepsilon)$, there exists a subsequential machine that computes the same function.
\end{theorem}

\begin{proof}

        for every function $f: \Sigma^* \to \Gamma^*$ we will define $\widehat{f}: \Sigma^* \to \Gamma^*$ as follows:
        \[
        \widehat{f}(w) = lcp\{ f(wv) : v \in \Sigma^* \}
        \]
        where $lcp$ stands for longest common prefix.
        \begin{definition}
            \label{definition-finitely-many-derivatives}
            We will say a function $f: \Sigma^* \to \Gamma^*$ has finitely many derivatives if the set $\{ f_w\coloneqq \widehat{f}(w)^{-1} f(wv) : v \in \Sigma^* \mid w \in \Sigma^* \}$ is finite.
        \end{definition}
        

        First we will show that every function expressed by a sequential protocol has finitely many derivatives. Let $(f_A,f_B,Q_A,q_A)$ be the ingredients of a sequential protocol over the monoid $(\Gamma^*, \cdot, \varepsilon)$. Consider the function $f: \Sigma^* \to \Gamma^*$ defined by the sequential protocol, i.e. $f(w) = f_A(w) f_B(\varepsilon, q_A(w))$. We will show that $f$ has finitely many derivatives. To be more precise, we will show that for every $w_1, w_2 \in \Sigma^*$ such that $q_A(w_1) = q_A(w_2)$, it holds that $f_{w_1} = f_{w_2}$. This will imply that the number of distinct derivatives of $f$ is at most $|Q_A|$, which is finite. First partition $\Sigma^*$ into the sets $S_q = \{ w \in \Sigma^* : q_A(w) = q \}$ for each $q \in Q_A$. Now consider two function $f_{w_1}$ and $f_{w_2}$ where $w_1, w_2 \in S_q$ for some $q \in Q_A$. Now we want to show that $f_{w_1} = f_{w_2}$ First define the function $\widehat{u}: Q_A \to \Gamma^*$ as follows:
        \[\widehat{u}(q) = lcp\{ u \in \Sigma^* : u \text{ is a prefix of } f_B(v,q) \text{ for all } v \in \Sigma^* \}\]
        that is, $\widehat{u}(q)$ is the longest common prefix of the strings $f_B(v,q)$ for all $v \in \Sigma^*$.


        We claim that for every string $w \in \Sigma^*$, $f_w$ is the function that maps any $v \in \Sigma^*$ to $\widehat{u}(q_A(w))^{-1} f_B(v, q_A(w))$. In other words, $f_w$ is entirely determined by the signal $q_A(w)$. 


        Consider $w,v \in \Sigma^*$. Since $f$ is computed by a sequential protocol, by splitting the string $ w \cdot v $ into two strings $ w $ and $ v $ it holds that $ f(w \cdot v) = f_A(w) \cdot f_B(v,q_A(w)) $. By construction, $ \widehat{u}(q_A(w)) $ is a prefix of $ f_B(v,q_A(w)) $ thus, $ f_A(w) \cdot \widehat{u}(q_A(w)) $ is a prefix of $ f(w \cdot v) $, and that for any $ v \in \Sigma^* $.

        Moreover, for any $ v \in \Sigma^* $, $ f_A(w) $ is a prefix of $ f(w \cdot v) $. Hence, it is also a prefix of $ \widehat{f}(w) $, that is $ \widehat{f}(w) = f_A(w) \cdot u $ for some $ u \in \Sigma^* $. But now, for any $ v \in \Sigma^* $, $ f_A(w) \cdot u $ is a prefix of $ f_A(w) \cdot f_B(v,q_A(w)) $ according to definition of $ \widehat{f} $. So in particular, for all $ v \in \Sigma^* $, $ u $ is a prefix of $ f_B(v,q_A(w)) $. Therefore, $ u $ is a prefix of $ \widehat{u}(q_A(w)) $. But since $ f_A(w) \cdot \widehat{u}(q_A(w)) $ is a prefix of $ f(w \cdot v) $ for any $ v \in \Sigma^* $, it is also a prefix of $ \widehat{f}(w) = f_A(w) \cdot u $ and thus $ \widehat{u}(q_A(w)) $ is a prefix of $ u $. Hence, $ u = \widehat{u}(q_A(w)) $ and $ \widehat{f(w)} = f_A(w) \cdot \widehat{u}(q_A(w)) $.
        \medskip

        \noindent
        Now, for any $ v \in \Sigma^* $:
        \begin{align*}
        f_w(v) &= \widehat{f}(w)^{-1} \cdot f(wv)\\
        &= (f_A(w) \cdot \widehat{u}(q_A(w)))^{-1} \cdot (f_A(w) \cdot f_B(v,q_A(w)))\\
        &= \widehat{u}(q_A(w))^{-1} \cdot f_B(v,q_A(w))
        \end{align*}
        which proves the claim. 

        Observe that the number of distinct functions $f_w$ is at most $|Q_A|$, but can be strictly less. Therefore it is possible that for some $q_1, q_2 \in Q_A$ with $q_1 \neq q_2$, it holds that $f_{w_1} = f_{w_2}$. But in any case, the number of distinct derivatives of $f$ is at most $|Q_A|$, which is finite. Hence, $f$ has finitely many derivatives.
    
    \begin{theorem}
        \label{theorem-subsequential-machines}
         A function $f: \Sigma^* \to \Gamma^*$ can be computed by a subsequential machine if and only if it has finitely many derivatives (as defined in Definition~\ref{definition-finitely-many-derivatives}). 
    \end{theorem}
    \noindent
    The proof can be found in \cite{reutenauer1990subsequential} (see Theorem 1).
    This concludes the proof. \footnote{This theorem was proved in collaboration with Thomas Filasto.}
        
\end{proof}