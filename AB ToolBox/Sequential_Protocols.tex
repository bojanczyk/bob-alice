\subsection{Sequential Protocols}

\begin{definition}
    \label{Sequential-Protocols}
A sequential protocol over a monoid is a simple-aggregation protocol (as defined in Definition~\ref{simple-aggregation-definition}) such that the signal space of Bob, $Q_B$, is a singleton set. In other words, there is no communication from Bob to Alice, and only Alice can send signals to Bob. Such a protocol is therefore specified by: $f_A: \Sigma^* \to M$ and $q_A: \Sigma^* \to Q_A$ for Alice, and $f_B: \Sigma^* \times Q_A \to M$ for Bob. The final output on inputs $x,y$ is $f_A(x)\ast f_B(y, q_A(x))$.
\end{definition}

\begin{myexample}
    \label{example-sequential-protocol}
Let $TM_1,TM_2, \ldots$ be an enumeration of all Turing machines. Also for each natural number $n$, let $\langle n \rangle$ be a standard encoding of $n$ as a binary string. Consider the sequential protocol over the monoid $(\mathbb{N}, +, 0)$ where Alice's functions are defined as $f_A(x) = |x|$ and $q_A(x) = 1$ if $TM_{|x|}$ halts on input $ \langle x \rangle$ and $f_A(x) =|x| + 1$ and $q_A(x) = 2$ otherwise. Bob's function is defined as $f_B(y, q_A(x)) = |y|+1$ if $q_A(x) = 1$ and $f_B(y, q_A(x)) = |y|$ if $q_A(x) = 2$.

Observe that the output of this protocol is always $|x| + |y| + 1 = |xy|+1$, regardless of whether $TM_{|x|}$ halts on input $\langle x \rangle$ or not. Therefore, the split-invariance condition is satisfied. 

\end{myexample}

\begin{corollary}
    \label{corollary-sequential-protocols}
    The above example shows that signal functions in sequential protocols and more generally in simple-aggregation protocols can be uncomputable functions. 
\end{corollary}

\begin{lemma}
    \label{prefix-lemma-sequential-protocols}
Let $(f_A,f_B,Q_A,Q_B,q_A)$ be ingridients of a sequential protocol over the monoid $(\Gamma^*, \cdot, \varepsilon)$. Now for any two string $w_1, w_2 \in \Sigma^*$, if $q_A(w_1) = q_A(w_1w_2)$ then $f_A(w_1)$ is a prefix of $f_A(w_1w_2)$. More percisely, there exists a string $u \in \Gamma^*$ (output domain) such that $f_A(w_1w_2) = f_A(w_1) \cdot u$. In another words, Alice's local output grows monotonically with respect to the prefix order on strings, whenever the signal sent to Bob does not change.
\end{lemma}

\begin{proof}
    
    Assume towards a contradiction that $q_A(w_1) = q_A(w_1 w_2)$ but $f_A(w_1)$ is \emph{not} a prefix of $f_A(w_1 w_2)$. Let $q = q_A(w_1) = q_A(w_1 w_2)$. Then:
\begin{equation}
    f(w_1 w_2)
    = f_A(w_1)\, f_B(w_2,q)
    = f_A(w_1 w_2)\, f_B(\varepsilon,q)
    \label{eq:seq1}
\end{equation}

Since $f_A(w_1)$ and $f_A(w_1w_2)$ are incomparable under the prefix relation by assumption, Equation~\eqref{eq:seq1} implies that $f_A(w_1w_2)$ is a strict prefix of $f_A(w_1)$. Write
\[
f_A(w_1) = f_A(w_1 w_2)\, v
\]
for some nonempty $v \in \Gamma^*$. Let
\[
u = f_A(w_1 w_2), \qquad z_1 = f_B(w_2,q).
\]
Then \eqref{eq:seq1} yields
\[
f_B(\varepsilon,q) = v z_1.
\]

Now consider $w_1 w_2^2$. Using split-invariance,
\begin{equation}
    f(w_1 w_2^2)
    = f_A(w_1)\, f_B(w_2^2,q)
    = f_A(w_1 w_2)\, f_B(w_2,q)
    \label{eq:seq2}
\end{equation}
Let $z_2 = f_B(w_2^2,q)$. Substituting $f_A(w_1) = u v$ and $f_A(w_1 w_2) = u$ into \eqref{eq:seq2} gives
\[
u v z_2 = u z_1 \quad \Rightarrow \quad z_2 = v z_1
\]

Repeating this argument inductively, define
\[
z_k = f_B(w_2^k,q).
\]
The same cancellation shows
\[
z_{k+1} = v z_k \qquad \text{for all } k \ge 0
\]
Thus
\[
z_k = v^k z_0,
\quad \text{where } z_0 = f_B(\varepsilon,q)
\]

Since $v \neq \varepsilon$, the length of $z_k$ strictly decreases with $k$. In particular, for sufficiently large $k$, we eventually obtain $z_k = \varepsilon$ (because $f_B$ outputs finite strings and $v$ is repeatedly removed from the left). But then
\[
f(w_1 w_2^{k+1})
= f_A(w_1 w_2)\, z_k
= f_A(w_1 w_2)\, \varepsilon
= f_A(w_1 w_2)
= u,
\]
while we have already shown that $u v$ is always a prefix of $f(w_1 x)$ for all $x$. This yields a contradiction.

    



\end{proof}
