\section{Sequential Protocols}

\begin{definition}
    \label{Sequential-Protocols}
A sequential protocol over a monoid is a simple-aggregation protocol (as defined in Definition~\ref{simple-aggregation-definition}) such that the signal space of Bob, $Q_B$, is a singleton set. In other words, there is no communication from Bob to Alice, and only Alice can send signals to Bob. Such a protocol is therefore specified by: $f_A: \Sigma^* \to M$ and $q_A: \Sigma^* \to Q_A$ for Alice, and $f_B: \Sigma^* \times Q_A \to M$ for Bob. The final output on inputs $x,y$ is $f_A(x)\ast f_B(y, q_A(x))$.
\end{definition}

\begin{myexample}
    \label{example-sequential-protocol}
Let $TM_1,TM_2, \ldots$ be an enumeration of all Turing transducers. Consider the sequential protocol over the monoid $(\mathbb{N}, +, 0)$ where Alice's functions are defined as $f_A(x) = |x|$ and $q_A(x) = 1$ if $TM_{|x|}$ halts on input $  x $ and $f_A(x) =|x| + 1$ and $q_A(x) = 2$ otherwise. Bob's function is defined as $f_B(y, q_A(x)) = |y|+1$ if $q_A(x) = 1$ and $f_B(y, q_A(x)) = |y|$ if $q_A(x) = 2$.

Observe that the output of this protocol is always $|x| + |y| + 1 = |xy|+1$, regardless of whether $TM_{|x|}$ halts on input $ x $ or not. Therefore, the split-invariance condition is satisfied. 

\end{myexample}

\begin{corollary}
    \label{corollary-sequential-protocols}
    The above example shows that signal functions in sequential protocols and more generally in simple-aggregation protocols can be uncomputable functions. 
\end{corollary}

\begin{lemma}
    \label{prefix-lemma-sequential-protocols}
Let $(f_A,f_B,Q_A,Q_B,q_A)$ be ingridients of a sequential protocol over the monoid $(\Gamma^*, \cdot, \varepsilon)$. Now for any two string $w_1, w_2 \in \Sigma^*$, if $q_A(w_1) = q_A(w_1w_2)$ then $f_A(w_1)$ is a prefix of $f_A(w_1w_2)$. More percisely, there exists a string $u \in \Gamma^*$ (output domain) such that $f_A(w_1w_2) = f_A(w_1) \cdot u$. In another words, Alice's local output grows monotonically with respect to the prefix order on strings, whenever the signal sent to Bob does not change.
\end{lemma}

\begin{proof}
    
Assume towards a contradiction that $q_A(w_1) = q_A(w_1 w_2)$ but $f_A(w_1)$ is \emph{not} a prefix of $f_A(w_1 w_2)$. Let $q = q_A(w_1) = q_A(w_1 w_2)$. Then:
\begin{equation}
    f(w_1 w_2)
    = f_A(w_1)\, f_B(w_2,q)
    = f_A(w_1 w_2)\, f_B(\varepsilon,q)
    \label{eq:seq1}
\end{equation}
Since $f_A(w_1)$ is not prefix of $f_A(w_1w_2)$ by assumption, Equation~\eqref{eq:seq1} implies that $f_A(w_1w_2)$ is a strict prefix of $f_A(w_1)$. Write
\[
f_A(w_1) = f_A(w_1 w_2)\, v
\]
for some nonempty $v \in \Gamma^*$. Let
\[
u = f_A(w_1 w_2), \qquad z_1 = f_B(w_2,q)
\]
Then \eqref{eq:seq1} yields
\[
f_B(\varepsilon,q) = v z_1
\]
Now consider $w_1 w_2^2$. Using split-invariance,
\begin{equation}
    f(w_1 w_2^2)
    = f_A(w_1)\, f_B(w_2^2,q)
    = f_A(w_1 w_2)\, f_B(w_2,q)
    \label{eq:seq2}
\end{equation}
Let $z_2 = f_B(w_2^2,q)$. Substituting $f_A(w_1) = u v$ and $f_A(w_1 w_2) = u$ into \eqref{eq:seq2} gives
\[
u v z_2 = u z_1 \quad \Rightarrow \quad v\,z_2 = z_1
\]
Repeating this argument inductively, define
\[
z_k = f_B(w_2^k,q).
\]
The same cancellation shows
\[
z_k = v\,z_{k+1} \qquad \text{for all } k \ge 0
\]
Now since $v \neq \varepsilon$, the length of $z_k$ strictly decreases with $k$. This yields a contradiction.
\end{proof}

%====
\begin{corollary}
If for $w_1,w_2\in\Sigma^*$ we have $q_A(w_1)=q_A(w_1w_2)=:q$, then
\[
   f(w_1)\cdot \bigl(f_B(\varepsilon,q)\bigr)^{-1}
   \text{ is a prefix of }
   f(w_1w_2)\cdot \bigl(f_B(\varepsilon,q)\bigr)^{-1}
\]
Equivalently, there exists a string $u\in\Gamma^*$ such that
\[
   f(w_1w_2)\cdot \bigl(f_B(\varepsilon,q)\bigr)^{-1}
   \;=\;
   f(w_1)\cdot \bigl(f_B(\varepsilon,q)\bigr)^{-1}\cdot u
\]
\end{corollary}

\begin{proof}
For $q = q_A(w_1)=q_A(w_1w_2)$ we have, by definition of the global output of a sequential protocol:
\[
   f(w_1)=f_A(w_1)\,f_B(\varepsilon,q), \qquad
   f(w_1w_2)=f_A(w_1w_2)\,f_B(\varepsilon,q)
\]
By Lemma~\ref{prefix-lemma-sequential-protocols}, the assumption
$q_A(w_1)=q_A(w_1w_2)$ implies that $f_A(w_1)$ is a prefix of $f_A(w_1w_2)$.
Hence there exists $u\in\Gamma^*$ such that
\[
   f_A(w_1w_2)=f_A(w_1)\,u
\]
Substituting this expression into the global output gives
\[
   f(w_1w_2)=f_A(w_1)\,u\,f_B(\varepsilon,q)
            = \bigl(f_A(w_1)\,f_B(\varepsilon,q)\bigr)\,u
            = f(w_1)\,u
\]
Right-cancelling the common suffix $f_B(\varepsilon,q)$ yields
\[
   f(w_1w_2)\cdot \bigl(f_B(\varepsilon,q)\bigr)^{-1}
   \;=\;
   f(w_1)\cdot \bigl(f_B(\varepsilon,q)\bigr)^{-1}\cdot u
\]
which proves that
\[
   f(w_1)\cdot \bigl(f_B(\varepsilon,q)\bigr)^{-1}
   \preceq
   f(w_1w_2)\cdot \bigl(f_B(\varepsilon,q)\bigr)^{-1}
\]
\end{proof}

The above lemma and corollary was not used in the rest of the paper, but is included here as it may be of independent interest.
%====

Before stating the main theorem of this section, to avoid any ambiguity, we will use the definition of subsequential transducers as given in \cite{reutenauer1990subsequential}.

\begin{theorem}
    For every sequential protocol over the monoid $(\Gamma^*, \cdot, \varepsilon)$, there exists a subsequential transducer that computes the same function.
\end{theorem}

\begin{proof}

        For every function $f: \Sigma^* \to \Gamma^*$ we will define $\widehat{f}: \Sigma^* \to \Gamma^*$ as follows:
        \[
        \widehat{f}(w) = lcp\{ f(wv) : v \in \Sigma^* \}
        \]
        where $lcp$ stands for longest common prefix.
        \begin{definition}
            \label{definition-finitely-many-derivatives}

            Suppose that a function $f: \Sigma^* \to \Gamma^*$ satisfies the property that for every $w_1, w_2 \in \Sigma^*$, $\widehat{f}(w_1)$ is a prefix of $f(w_1 w_2)$. Then we can define the \emph{derivative} of $f$ at $w \in \Sigma^*$ as the function $f_w: \Sigma^* \to \Gamma^*$ defined by:
            \[f_w(v) = \widehat{f}(w)^{-1} f(wv)\]
            for every $v \in \Sigma^*$. And we say that $f$ has \emph{finitely many derivatives} if the set $\{ f_w : w \in \Sigma^* \}$ is finite.

        \end{definition}
        

        Let $(f_A,f_B,Q_A,q_A)$ be the ingredients of a sequential protocol over the monoid $(\Gamma^*, \cdot, \varepsilon)$. Consider the function $f: \Sigma^* \to \Gamma^*$ defined by the sequential protocol, i.e. $f(w) = f_A(w) f_B(\varepsilon, q_A(w))$. First we will show that the function $f$ satisfies the property which is required in Definition~\ref{definition-finitely-many-derivatives}. Then we will show that $f$ has finitely many derivatives, or more precisely, we will show that for every $w_1, w_2 \in \Sigma^*$ such that $q_A(w_1) = q_A(w_2)$, it holds that $f_{w_1} = f_{w_2}$. This will imply that the number of distinct derivatives of $f$ is at most $|Q_A|$, which is finite.

        \noindent
        First define the function $\widehat{u}: Q_A \to \Gamma^*$ as follows:
        \[\widehat{u}(q) = lcp\{ u \in \Sigma^* : u \text{ is a prefix of } f_B(v,q) \text{ for all } v \in \Sigma^* \}\]
        that is, $\widehat{u}(q)$ is the longest common prefix of the strings $f_B(v,q)$ for all $v \in \Sigma^*$. Now we will show that for every $w \in \Sigma^*$, $\widehat{f}(w) = f_A(w) \cdot \widehat{u}(q_A(w))$. Observe that $f_A(w)$ is always a prefix of $f(w \cdot v)$ for any $v \in \Sigma^*$, since:
        \[f(w \cdot v) = f_A(w) \cdot f_B(v,q_A(w))\]
        Thus, there exists a string $u \in \Sigma^*$ such that $\widehat{f}(w) = f_A(w) \cdot u$. By construction, $ \widehat{u}(q_A(w)) $ is a prefix of $ f_B(v,q_A(w)) $ for all $ v \in \Sigma^* $. Therefore, $ f_A(w) \cdot \widehat{u}(q_A(w)) $ is a prefix of $\widehat{f}(w)$. Now it is enough to show that $\widehat{f}(w)$ can not be strictly longer than $ f_A(w) \cdot \widehat{u}(q_A(w)) $. Assume by contradiction that $\widehat{f}(w)$ is strictly longer than $ f_A(w) \cdot \widehat{u}(q_A(w)) $.  Then, there exists a string $u' \in \Sigma^*$ such that $\widehat{f}(w) = f_A(w) \cdot \widehat{u}(q_A(w)) \cdot u'$, with $u' \neq \varepsilon$. By definition of $\widehat{f}$, for any $ v \in \Sigma^* $, $ f_A(w) \cdot \widehat{u}(q_A(w)) \cdot u' $ is a prefix of $ f(w \cdot v) = f_A(w) \cdot f_B(v,q_A(w)) $. Therefore, $\widehat{u}(q_A(w)) \cdot u' $ is a prefix of $ f_B(v,q_A(w)) $ for all $ v \in \Sigma^* $. This contradicts the definition of $ \widehat{u}(q_A(w)) $ as the longest common prefix of the strings $ f_B(v,q_A(w)) $ for all $ v \in \Sigma^* $. Hence, $\widehat{f}(w) = f_A(w) \cdot \widehat{u}(q_A(w))$.


        Now consider two function $f_{w_1}$ and $f_{w_2}$ for some $w_1, w_2 \in \Sigma^*$ such that $q_A(w_1) = q_A(w_2)$. We will show that $f_{w_1} = f_{w_2}$.
        To be more precise, we will show that for every string $w \in \Sigma^*$, $f_w$ is the function that maps any $v \in \Sigma^*$ to $\widehat{u}(q_A(w))^{-1} f_B(v, q_A(w))$. This is because for any $ v \in \Sigma^* $ we have:
        \begin{align*}
        f_w(v) &= \widehat{f}(w)^{-1} \cdot f(wv)\\
        &= (f_A(w) \cdot \widehat{u}(q_A(w)))^{-1} \cdot (f_A(w) \cdot f_B(v,q_A(w)))\\
        &= \widehat{u}(q_A(w))^{-1} \cdot f_B(v,q_A(w))
        \end{align*}
        Therefore, for any $w_1, w_2 \in \Sigma^*$ such that $q_A(w_1) = q_A(w_2)$, it holds that $f_{w_1} = f_{w_2}$. Observe that the number of distinct functions $f_w$ is at most $|Q_A|$, but can be strictly less. Therefore it is possible that for some $q_1, q_2 \in Q_A$ with $q_1 \neq q_2$, it holds that $f_{w_1} = f_{w_2}$. But in any case, the number of distinct derivatives of $f$ is at most $|Q_A|$, which is finite. Hence, $f$ has finitely many derivatives. \footnote{This theorem was proved in collaboration with Thomas Filasto.}
    
    \begin{theorem}
        \label{theorem-subsequential-transducer}
         A function $f: \Sigma^* \to \Gamma^*$ can be computed by a subsequential transducer if and only if it has finitely many derivatives (as defined in Definition~\ref{definition-finitely-many-derivatives}). 
    \end{theorem}
    \noindent
    The proof can be found in \cite{reutenauer1990subsequential} (see Theorem 1).
        
\end{proof}

%==== Normalized Sequential Protocols 
To have a clean characterization of class of protocols, we like to have some kind of normalization for each one of them to avoid unwanted behaviors such as the one shown in Example~\ref{example-sequential-protocol}. We think the the roots of such unwanted behaviors are the uncomputable signal functions. Therefore, we will define normalization for each class of protocols based on the computability of their signal functions. To be more specific, we want to have a signal function that is computable by a Deterministic Finite Automaton (DFA).

