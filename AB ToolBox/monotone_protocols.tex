% LTeX: lang=en-US

\section{Monotone Protcols}

\begin{definition}
  \label{monotone_protocol_definition}
  A monotone protocol is a simple-aggregation protocol (as defined in Definition~\ref{simple-aggregation-definition}) over the monoid
  $(\Gamma^*, \cdot, \varepsilon)$
  where $\Gamma^*$ is the set of all strings over the alphabet $\Gamma$,
  $\cdot$ is the concatenation operation,
  and $\varepsilon$ is the empty string.
\end{definition}

We can have similar lemma as Lemma~\ref{prefix-lemma-sequential-protocols} for monotone protocols, stating that configurations of Alice grow monotonically with respect to the prefix order on strings, whenever the signal sent by Alice does not change.

\begin{lemma}
    \label{prefix-lemma-monotone-protocols}
Let $(f_A,f_B,Q_A,Q_B,q_A,q_B)$ be ingredients of a monotone protocol over the monoid $(\Gamma^*, \cdot, \varepsilon)$. Now for any two string $w_1, w_2 \in \Sigma^*$, if $q_A(w_1) = q_A(w_1w_2)$ then for every $1 \leq i \leq |Q_B|$ the $i$-th component of configuration $C_A(w_1)$ is prefix of the $i$-th component of configuration $C_A(w_1w_2)$. More precisely, if we denote the $i$-th component of configuration $C_A(w)$ as $C_A(w)[i]$, then we have:
\[C_A(w_1w_2)[i] = C_A(w_1)[i] \cdot u \] for some $u \in \Gamma^*$ (output domain).
In other words, configurations of Alice grow monotonically with respect to the prefix order on strings, whenever the signal sent to Bob does not change.
\end{lemma}
\begin{proof}
I don't know the proof yet.
\end{proof}

\begin{theorem}[Normalization of Monotone Protocols]
  \label{thm:monotone-protocols-rational}
  For every monotone protocol over the monoid $(\Gamma^*, \cdot, \varepsilon)$,
  there exists a normalized monotone protocol that computes the same function. (A normalized monotone protocol is defined similarly as Definition~\ref{definition-normalized-sequential-protocol}, but for monotone protocols.)
\end{theorem}
\begin{proof}
  (Not finished yet)
  Similar to the proof of Theorem~\ref{theorem-normalized-sequential-protocol}, first we define the function $\delta \colon Q_A \times Q_B \to \Gamma^*$ as follows:
  \[\delta(q_A, q_B) = lcp\{ f_B(v, q_A) \mid v \in \Sigma^* \text{ and } q_B(v) = q_B \}\]
  Then, we define equivalent protocol functions as follows:
  \begin{itemize}
      \item $f_A'(u, q_B) = f_A(u, q_B) \delta(q_A(u), q_B)$
      \item $f_B'(v, q_A) = \delta(q_A, q_B(v))^{-1} f_B(v, q_A)$
  \end{itemize}
  Now consider two following equivalence relations between strings of $\Sigma^*$:
  \[u_1 \equiv_A u_2 \iff \forall v \in \Sigma^*, f_B'(v, q_A(u_1)) = f_B'(v, q_A(u_2)) \]
  \[v_1 \equiv_B v_2 \iff \forall u \in \Sigma^*, f_A'(u, q_B(v_1)) = f_A'(u, q_B(v_2)) \]
  We claim that $\equiv_A$ is a right congruence and $\equiv_B$ is a left congruence. That is, for every $u_1, u_2, a \in \Sigma$, if $u_1 \equiv_A u_2$ then $u_1 a \equiv_A u_2 a$, and for every $v_1, v_2, a \in \Sigma$, if $v_1 \equiv_B v_2$ then $a v_1 \equiv_B a v_2$. 

  %--- I Hope I can complete the proof soon ---
  To be continued...

\end{proof}

%------------------------------------------------------------------------------


\begin{lemma}
  \label{lem:concat-ab-rational}
  For every rational function $f \colon \Sigma^* \to \Gamma^*$
  (computable by a bimachine), there exists a monotone protocol expressing $f$.
\end{lemma}
\begin{proof}
  Let us consider a bimachine computing $f$,
  with a left-to-right automaton
  $\mathcal{A}_L$ and a right-to-left automaton $\mathcal{A}_R$.
  Let us define $Q$ the union of the states of $\mathcal{A}_L$ and $\mathcal{A}_R$.
  Let us define $q_A(u)$ as the state in which $\mathcal{A}_L$ ends
  after reading $u$, and $q_B(v)$ as the state in which $\mathcal{A}_R$ ends
  after reading $v$ (from right to left).

  It we can now define $f_A(u,s)$ to be the output of the bimachine on the word
  $u$ if the left-to-right automaton starts in its initial state, the
  right-to-left automaton starts in state $s \in Q$, and similarly, $f_B(v,s)$
  to be the output of the bimachine on the word $v$ if the left-to-right
  automaton starts in state $s \in Q$, and the right-to-left automaton starts
  in its initial state.
\end{proof}



\begin{definition}
  An \emph{oracle}-sequential transducer (OSeq)
  is a deterministic finite state automaton
  $(Q, \Sigma, \delta, q_0)$
  equipped with output functions $\lambda \colon Q \times \Sigma \to \mathcal{O}$
  where $\mathcal{O}$ is a finite set of oracle functions
  of type $\Sigma^* \to \Gamma^*$,
  and a final output function $F \colon Q \to \Gamma^*$.

  The semantics of an oracle-sequential transducer $\mathcal{A}$
  is defined inductively as follows:
  $\mathcal{A}(q, \varepsilon) = F(q)$, and
  $\mathcal{A}(q, a w) = \lambda(q,a)(w) \cdot \mathcal{A}(\delta(q,a), w)$.
\end{definition}

\begin{lemma}
  A function $f \colon \Sigma^* \to \Gamma^*$
  is rational (computable by a bimachine)
  if and only if 
  it is computable by an oracle-sequential transducer
  where all oracle functions are regular functions of
  bounded size output.
\end{lemma}
\begin{proof}
Given a bimachine computing $f$, one can construct an oracle transducer with
states that are all the possible states of the left-to-right automaton. Then,
$\lambda(q,a)(w)$ is defined as what the bimachine outputs when the
left-to-right automaton is in state $q$, and the right-to-left automaton starts
in its initial state, reads $w$ and then reads $a$, which can be computed by a
regular function of bounded size output by definition.


Conversely, one can compute one big automaton that precomputes all the oracle
functions since their output has bounded size. This is the right-to-left
automaton. The left-to-right automaton is the states of the oracle-sequential
transducer. The output function of the bimachine is
defined as $\lambda(q,a)(w)$ when the left-to-right automaton is in state $q$,
which only depends on the state of the right-to-left and left-to-right automata.
\end{proof}


Let us now consider a function $f$ expressible by a monotone protocol, so that
$f(uv) = f_A(u, q_B(v)) \; f_B(v, q_A(u))$ for suitable functions $f_A, f_B,
q_A, q_B$ and finite set $Q$ of signals. Recall the definition of configuration
of a simple-aggregation protocol from
Definition~\ref{configuration-definition}. Let us order configurations as
follows: $C(u) \leq C(v)$ if and only if $q_A(u) = q_A(v)$ and for all $s \in
Q$, $f_A(u, s)$ is a prefix of $f_A(v, s)$.

We claim that configurations are somehow close to being endowed with a
well-quasi-ordering. Before doing that, let us first normalize the notion of
configuration to contain only signals in $Q$ that can actually be sent by Bob.
That is, for every $s \in Q$ such that there is no $v \in \Sigma^*$ with
$q_B(v) = s$, we remove $s$ from $Q$ and we remove the corresponding word
$f_A(u, s)$ from the configuration $C(u)$.

\begin{lemma}
  \label{lem:pref-wqo}
  For every infinite sequence $(w_n)_{n \in \mathbb{N}}$ of words in $\Gamma^*$
  that is increasing for the prefix order,
  there exist $i < j$ such that $C(w_i) \leq C(w_j)$.
\end{lemma}
\begin{proof}
  Let us consider such an infinite sequence $(w_n)_{n \in \mathbb{N}}$.
  Since $Q$ is finite, there exists an infinite subsequence 
  that has a same state $q \in Q$.

  Let us notice that the sequence of lengths of the words 
  in $(f_A(w_n, s))_{s \in Q}$ is a sequence of $|Q|$-tuples of natural numbers,
  hence by Dickson's lemma, there exists an infinite subsequence
  where these lengths are non-decreasing.

  Let us now fix a $q \in Q$ and look at 
  $|f_A(w_n, q)|$ for this $q$ when $n$ ranges over the indices of the
  subsequence defined above. Either $|f_A(w_n, q)|$ is bounded, in which case
  we can extract further a subsequence where $f_A(w_n, q)$ is constant, or
  $|f_A(w_n, q)|$ is unbounded, in which case we can extract an infinite subsequence
  where $|f_A(w_n, q)|$ is strictly increasing. 

  By repeating this process for all the finitely many $q \in Q$, we can extract
  a subsequence $(w_{n_k})_{k \in \mathbb{N}}$ such that for all $q \in Q$,
  either $f_A(w_{n_k}, q)$ is constant, or the lengths $|f_A(w_{n_k}, q)|$ are
  strictly increasing.

  Let us now take $i = n_0$, and $j$ be large enough so that for every $q \in
  Q$ where $|f_A(w_{n_k}, q)|$ is strictly increasing, for every $s \in Q$, we
  have $|f_A(w_j, q)| > |f_A(w_i, s)|$.

  Let us prove that $C(w_i) \leq C(w_j)$. By construction, we have
  $q_A(w_i) = q_A(w_j)$. Now, let us consider any $s \in Q$. If
  $f_A(w_{n_k}, s)$ is constant, then in particular
  $f_A(w_i, s) = f_A(w_j, s)$, so $f_A(w_i, s)$ is a prefix of
  $f_A(w_j, s)$. 
  Otherwise, $|f_A(w_j, s)| > |f_A(w_i, r)|$
  for all $r \in Q$. Let us now consider a word $v$ such that 
  $q_B(v) = s$ and let us write $w_j = w_i v'$.

  \begin{equation*}
    f_A(w_j, s) \; f_B(v, q_A(w_j))
    f(w_j v) = f(w_i v' v)
    = f_A(w_i, q_B(v' v)) \; f_B(v' v, q_A(w_i))
  \end{equation*}

  Since $|f_A(w_j, s)| > |f_A(w_i, r)|$ for all $r \in Q$, this means that
  $f_A(w_i, q_B(v' v))$ is a prefix of $f_A(w_j, s)$. This defines a map
  $\delta \colon Q \to Q$ such that $f_A(w_i, \delta(r))$ is a prefix of
  $f_A(w_j, r)$ for all $r \in Q$.

  By extracting again the sequence, we can assume that for 
  every $i < j$, such a map $\delta_{i,j} \colon Q \to Q$ exists.
  Using a Ramsey argument, we can extract a subsequence 
  such that this map is the same for all pairs $(i,j)$.

  \textbf{TODO aliaume: there is some work to be done to conclude...} 

  We have proven that $C(w_i) \leq C(w_j)$.
\end{proof}

\newcommand{\lcpbob}[1]{\operatorname{lcp}_{#1}}
\newcommand{\derivbob}[1]{\partial_{#1}}



Now is the time to prove the main result of this section by constructing an
Oracle transducer computing $f$ from this well-quasi-ordering on
configurations. To that end, let us introduce some definitions, leaving $f$ implicit
to unclutter notations. For every $s
\in Q$, let us define the function $\lcpbob{s}(u)$ as the longest common prefix
of all words $f(uw)$ when $w$ ranges over all words such that $q_B(w) = s$.
Then, let us define the function $\derivbob{u}(w)$ as follows: if $q_B(w) = s$,
then $\derivbob{u}(w) = \lcpbob{s}(u)^{-1} f(uw)$.


\begin{theorem}

  A function $f \colon \Sigma^* \to \Gamma^*$
  is computable by a concatenating Alice-Bob communication model
  if and only if it is computable by an Oracle sequential transducer,
  where all oracle functions are regular functions of 
  bounded size output.
\end{theorem}
\begin{proof}
  The direction from oracle transducers to concatenating Alice-Bob communication model is
  straightforward, since oracle transducers correspond to rational functions 
  and we have already seen that rational functions can be computed
  by concatenating Alice-Bob communication models in 
  Lemma~\ref{lem:concat-ab-rational}.


  Conversely, let us consider a function $f$ computable by a concatenating
  Alice-Bob communication model. We are going to construct an oracle transducer
  computing $f$. Let us consider as states of the transducer a set of words
  $u\in \Sigma^*$ that are a finite set of representatives of minimal elements
  of configurations for the order $\leq$ defined above. By
  Lemma~\ref{lem:pref-wqo}, the tree of all words ordered by prefixes can be
  trimmed using $\leq$ so that every branch is finite, hence the tree itself is
  finite due to K\"onig's lemma, proving the existence of such a finite set of
  representatives. To this finite list of representatives, we add the empty
  word $\varepsilon$ if it is not already present. \textbf{TODO aliaume: is the König
  argument correct? unclear.}

  Let us now define the transitions of the automata. Given $u \in \Sigma^*$ a
  representative state, and $a \in \Sigma$ a letter, we know that there exists
  $v \in \Sigma^*$ a representative state such that $C(v) \leq C(u a)$. We pick
  one such $v$ and define a transition from $u$ to $v$ on letter $a$.

  Let us now define the oracle functions. We define $\lambda(u,a)$ to be the
  function that maps a word $w$ to the unique word $x$ such that
  $\derivbob{u}(aw) = x \; \derivbob{v}(w)$ where $v$ is the state reached from
  $u$ on letter $a$.

  \textbf{for omid: it is unclear that 
    the oracle functions even exist!}

\end{proof}
