% LTeX: lang=en-US

\section{Monotone Protcols}

\begin{definition}
  \label{monotone_protocol_definition}
  A monotone protocol is a simple-aggregation protocol (as defined in Definition~\ref{simple-aggregation-definition}) over the monoid
  $(\Gamma^*, \cdot, \varepsilon)$
  where $\Gamma^*$ is the set of all strings over the alphabet $\Gamma$,
  $\cdot$ is the concatenation operation,
  and $\varepsilon$ is the empty string.
\end{definition}

We can have similar lemma as Lemma~\ref{prefix-lemma-sequential-protocols} for monotone protocols, stating that configurations of Alice grow monotonically with respect to the prefix order on strings, whenever the signal sent by Alice does not change.

\begin{lemma}
    \label{prefix-lemma-monotone-protocols}
Let $(f_A,f_B,Q_A,Q_B,q_A,q_B)$ be ingredients of a monotone protocol over the monoid $(\Gamma^*, \cdot, \varepsilon)$. Now for any two string $w_1, w_2 \in \Sigma^*$, if $q_A(w_1) = q_A(w_1w_2)$ then for every $1 \leq i \leq |Q_B|$ the $i$-th component of configuration $C_A(w_1)$ is prefix of the $i$-th component of configuration $C_A(w_1w_2)$. More precisely, if we denote the $i$-th component of configuration $C_A(w)$ as $C_A(w)[i]$, then we have:
\[C_A(w_1w_2)[i] = C_A(w_1)[i] \cdot u \] for some $u \in \Gamma^*$ (output domain).
In other words, configurations of Alice grow monotonically with respect to the prefix order on strings, whenever the signal sent to Bob does not change.
\end{lemma}
\begin{proof}
I don't know the proof yet.
\end{proof}

\begin{theorem}[Normalization of Monotone Protocols]
  \label{thm:monotone-protocols-rational}
  For every monotone protocol over the monoid $(\Gamma^*, \cdot, \varepsilon)$,
  there exists a normalized monotone protocol that computes the same function. (A normalized monotone protocol is defined similarly as Definition~\ref{definition-normalized-sequential-protocol}, but for monotone protocols.)
\end{theorem}
\begin{proof}
  (Not finished yet)
  Similar to the proof of Theorem~\ref{theorem-normalized-sequential-protocol}, first we define the function $\delta \colon Q_A \times Q_B \to \Gamma^*$ as follows:
  \[\delta(q_A, q_B) = lcp\{ f_B(v, q_A) \mid v \in \Sigma^* \text{ and } q_B(v) = q_B \}\]
  Then, we define equivalent protocol functions as follows:
  \begin{itemize}
      \item $f_A'(u, q_B) = f_A(u, q_B) \delta(q_A(u), q_B)$
      \item $f_B'(v, q_A) = \delta(q_A, q_B(v))^{-1} f_B(v, q_A)$
  \end{itemize}
  Now consider two following equivalence relations between strings of $\Sigma^*$:
  \[u_1 \equiv_A u_2 \iff \forall v \in \Sigma^*, f_B'(v, q_A(u_1)) = f_B'(v, q_A(u_2)) \]
  \[v_1 \equiv_B v_2 \iff \forall u \in \Sigma^*, f_A'(u, q_B(v_1)) = f_A'(u, q_B(v_2)) \]
  We claim that $\equiv_A$ is a right congruence and $\equiv_B$ is a left congruence. That is, for every $u_1, u_2, a \in \Sigma$, if $u_1 \equiv_A u_2$ then $u_1 a \equiv_A u_2 a$, and for every $v_1, v_2, a \in \Sigma$, if $v_1 \equiv_B v_2$ then $a v_1 \equiv_B a v_2$. 

  %--- I Hope I can complete the proof soon ---
  To be continued...

\end{proof}

The above theorem was independently interesting because we hope it can be extended to more expressive protocols. So the above theorem is not directly related to the main goal of this section. The main goal of this section is to show that every function computed by a monotone protocol is rational and vice versa which we are going to do it now.

%-----------------------------------------------------------------------------

We want to show that every function computed by a monotone protocol is rational and vice versa. To do so first, we define the notion of \emph{Hankel Property} for every function $f \colon \Sigma^* \to \Gamma^*$. Then we state the Reutenauer-Schützenberger Theorem that every function with Hankel Property is rational and vice versa. Finally, we show that every function computed by a monotone protocol has the Hankel Property.

\begin{definition}[Hankel Property]
  A function $f \colon \Sigma^* \to \Gamma^*$ is said to have the Hankel Property if there exists a constant $K > 0$ and functions $\beta_1, \beta_2, \ldots, \beta_K \colon \Sigma^* \to \Gamma^*$ and $\gamma_1, \gamma_2, \ldots, \gamma_K \colon \Sigma^* \to \Gamma^*$ such that for every strings $u, v \in \Sigma^*$, we have:
  \[ f(uv) = \bigcup_{i=1}^{K} \beta_i(u) \cdot \gamma_i(v) \]
\end{definition}

\begin{theorem}[Reutenauer-Schützenberger Theorem]
  \label{Hankel-property-rational-function}
  Every function $f \colon \Sigma^* \to \Gamma^*$ that has the Hankel Property is rational and vice versa. \cite{reutenauerSchutzenberger1991}
\end{theorem}

So now it is enough to show that every function computed by a monotone protocol has the Hankel Property. We will show in it the next lemma.

\begin{lemma}
  \label{monotone-protocols-hankel-property}
  Every function computed by a monotone protocol has the Hankel Property.
\end{lemma}
\begin{proof}
  Let $(f_A,f_B,Q_A,Q_B,q_A,q_B)$ be ingredients of a monotone protocol over the monoid $(\Gamma^*, \cdot, \varepsilon)$ that computes the function $f \colon \Sigma^* \to \Gamma^*$.
  We define functions $\beta_i \colon \Sigma^* \to \Gamma^*$ and $\gamma_i \colon \Sigma^* \to \Gamma^*$ for every $1 \leq i \leq |Q_B| \times |Q_A|$ as follows:
  \begin{itemize}
      \item $\beta_{q,p}(u) = \emptyset$ if $q_A(u) \neq q$ else $\beta_{q,p}(u) = C_A(u)[p]$
      \item $\gamma_{q,p}(v) = \emptyset$ if $q_B(v) \neq p$ else $\gamma_{q,p}(v) = C_B(v)[q]$
  \end{itemize}
  where $1 \leq q \leq |Q_B|$ and $1 \leq p \leq |Q_A|$.
  Now for every strings $u, v \in \Sigma^*$, we have:
  \[ f(uv) = \bigcup_{(q,p) \in Q_B \times Q_A} \beta_{q,p}(u) \cdot \gamma_{q,p}(v) \]
  Thus, $f$ has the Hankel Property.
\end{proof}