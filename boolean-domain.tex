
\section{Boolean outputs}
\label{sec:boolean-domain}
In this section, we formally describe our model of computation for the simplest output domain, namely the Booleans, and we prove that it defines exactly the regular languages.  The definition that we describe, see Definition~\ref{def:two-party-protocol-boolean} below, has minor differences with respect to the informal description from the introduction. The messages are not necessarily bits, but they belong to some finite set, which is fixed in advance before the input is known. Also,  the two parties send their messages in parallel in each round. 
 These generalisations do not change the expressive power of the protocol (they might influence the number of rounds), but they will be useful in later sections, when we consider restrictions and generalisations. 

\begin{definition}[Boolean protocol]
    \label{def:two-party-protocol-boolean}
  A Boolean protocol 
   is given by the following ingredients: 
  \begin{enumerate}
    \item a finite input alphabet $\Sigma$;
    \item a number of rounds $k \in \set{1,2,\ldots}$;
    \item message spaces for Alice and Bob, which are finite sets $Q_A$ and $Q_B$;
    \item for each round $i \in \set{1,\ldots,k}$, two strategies
    \begin{align*}
    \myoverbrace{\sigma_A^i : \myunderbrace{\Sigma^*}{Alice's \\ local\\ string} \times \myunderbrace{Q_B^{i-1}}{message \\ history}  \to \myunderbrace{Q_A}{new\\ message}}{stategy for Alice in the $i$-th round}
    \qquad \text{and} \qquad 
        \myoverbrace{\sigma_B^i : \myunderbrace{\Sigma^*}{Bob's \\ local\\ string} \times \myunderbrace{Q_A^{i-1}}{message \\ history}  \to \myunderbrace{Q_B}{new\\ message}}{stategy for Bob in the $i$-th round};
    \end{align*}
    \item an output function of type $(Q_A \times Q_B)^{k} \to \set{\text{yes, no}}$.
  \end{enumerate}
\end{definition}

Given a pair of input strings $(w_1,w_2)$, which are called the \emph{local strings} of Alice and Bob, respectively, the  protocol is run as follows. There are $k$ rounds. In each round, both parties send messages, and therefore after $i$ rounds are played, the communication history contains $i$ messages sent by Alice and $i$ messages sent by Bob.  In round  $i \in \set{1,\ldots,k}$,  each of the two parties  looks at their local string and the $i-1$ messages sent by the other party in the previous rounds (only the messages sent by the other party are needed, since the party knows their own messages). Based on this information, each party uses their strategy to produce a new message. At the end of the protocol, the output function is used to determine the value of the function, based on all messages in  the communication history. We say that the protocol computes a language $L \subseteq \Sigma^*$ if for every two strings $w_1,w_2$, the output of the protocol tells us if the concatenation $w_1w_2$ belongs to the language. This corresponds to the split invariance condition that was discussed in the introduction.  Also, the reader will recognise the restriction on the total number of bits (this is bounded by the number of rounds times the logarithm of the size of the message spaces), and the non-uniformity (there is no restriction on the strategies of Alice and Bob). By non-uniformity, the first message sent by Alice could contain an answer to some undecidable problem. However, as we will see, the split invariance restriction will make it impossible to use this information, since the protocol can only compute regular languages, as stated in the \cref{thm:boolean-domain} from the introduction, which we now recall.


\booleandomain*




\begin{proof}
  We begin with the easier right-to-left implication, which says that the protocol can compute every regular language. If the language is  regular, then it  is recognised by a deterministic finite automaton, say  with state space $Q$. To compute the language, we can use a  one-round protocol (as we will see in a moment, this is not a coincidence, since all protocols can be reduced to one round). Alice sends the state in $Q$ of the automaton after reading her local string, and Bob sends the dependency $Q \to \set{\text{yes, no}}$ which says how Alice's state determines acceptance. Once these two pieces of information are known, we can apply the function from Bob's message to the state in Alice's message to determine the output.

  The rest of this proof is devoted to the left-to-right implication, i.e.~to showing that every language computed by the protocol is regular. This is the heart of the proof, and the place where we need to tame the non-uniformity of the protocol. 
  
  Observe that in the right-to-left implication, we have proved that regular languages need only one round. Therefore, if we want to  prove that the protocol can only compute regular languages, then a by-product will be that the protocol collapses to the one-round case.  This is, in fact, the first step of our proof. 
  \begin{lemma}\label{lem:one-round-reduction-boolean}
    For every protocol, there is an equivalent one-round protocol. 
  \end{lemma}
  \begin{proof}
    The general idea is that instead of engaging in interactive communication, each party sends the dependency of their message upon the unknown messages of the other party. Suppose that we are  Alice. The sequence of messages that we send will depend on our local string, and  the messages sent by Bob. Once the local string is fixed, this dependency is captured by a function of type 
    \begin{align*}
    (Q_B)^k \to (Q_A)^k.
    \end{align*}
    Instead of waiting for Bob's messages, Alice sends this function. (This function is not a general function, since it must satisfy the following \emph{causality} constraint: the $i$-th coordinate of the output depends only on the first $i-1$ input coordinates.) At the same time, Bob sends an analogous function of type 
    \begin{align*}
    (Q_A)^k \to (Q_B)^k,
    \end{align*}
    which describes the dependency of his messages. Due to the causality constraints, the two functions combine to create a unique output in $(Q_A \times Q_B)^k$, which can be used to determine the output of the protocol.
  \end{proof}

The proof of the above lemma can incur an exponential cost in the size of the message spaces. This is of little concern to us, since we only care about the protocol having a constant number of rounds.

  To complete the proof of the theorem, we use the  Myhill-Nerode Theorem.  Indeed, consider the strategy of  Alice: 
  \begin{align*}
  \sigma_A : \Sigma^* \to Q_A.
  \end{align*}
  For all we know, this function could be non-computable. However, it classifies all local strings into finitely many categories, with one category for each message in $Q_A$. Furthermore, if two strings $w_1$ and $w'_1$ are in the same category, then they are equivalent in the following sense: 
  \begin{align}\label{eq:myhill-nerode-equivalence}
  w_1 w_2 \in L \Leftrightarrow w'_1 w_2 \in L 
  \qquad \text{for every $w_2 \in \Sigma^*$.}
  \end{align}
  The equivalence described above is the same equivalence as in the Myhill-Nerode Theorem. In particular, equivalence classes of this equivalence are the same as states of the minimal deterministic automaton. Since the number of possible messages in $Q_A$ is finite, it follows that the minimal automaton is finite, and therefore the language is regular.  (Observe that we have shown that the language has at most $Q_A$ equivalence classes of Myhill-Nerode equivalence, which gives an upper bound on the size of the minimal automaton recognizing the language.)
\end{proof}

