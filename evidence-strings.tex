\subsection{Weighted continuous functions}
\label{sec:continuity}
In this section, we give some evidence for Conjecture~\ref{conj:protocol-regular-string-to-string}. The point of departure is the following lemma, which connects weighted automata and string-to-string functions that can be computed in our protocol.

    \begin{lemma}
        \label{lem:postcomposition-weighted-automaton}
        Let $\domain$ be a field, and consider two functions
        \[
        \begin{tikzcd}
        \Sigma^* 
        \ar[r,"f"]
        &
        \Gamma^*
        \ar[r,"g"]
        & 
        \domain,
        \end{tikzcd}
        \]
        such that $f$ is computed by a protocol (with string outputs). If $g$  is computed by a weighted automaton, then the same is true for  $f;g$.
    \end{lemma}
    \begin{proof}
        We will show that the composition $f;g$ is. computed by a protocol (with field outputs). Thanks to \cref{thm:field-domain}, this will imply that $f;g$ is computed by a weighted automaton.

        For each string in $\Gamma^*$, the weighted automaton for $g$ has an associated matrix over the field $\domain$. We modify the protocol for $f$, so that it uses matrices instead of strings. During the execution, instead of sending strings in $\Gamma^*$, the parties send  the corresponding matrices. At the end, instead of concatenating the output strings, the matrices are multiplied, yielding a matrix for the entire output string. Finally, this matrix is applied to the initial state, and then the output function is applied to the resulting vector. All of this is done using addition and multiplication, which is legitimate in a protocol with the output domain $\domain$.
    \end{proof}


The above lemma establishes a property of $f$, namely that weighted automata (over a field) are closed under precomposition with $f$. We think that this is an important property, and therefore we give it a name.


\begin{definition}[Weighted continuity]
    \label{def:weighted-continuity}
    A string-to-string function $f : \Sigma^* \to \Gamma^*$ is called \emph{weighted continuous} if functions computed by weighted automata over a field are closed under precomposition with $f$.
\end{definition}

The name ``continuous'' is inspired by a similar terminology that is used in automata theory for functions that preserve regularity under inverse images, see~\cite[Theorem 4.1]{PinSilva05} or~\cite[Footnote 2]{continuity20}. 
As we have shown in \cref{lem:postcomposition-weighted-automaton}, all string-to-string functions computed by protocols are weighted continuous. In particular, since every regular string-to-string function is computed by a protocol, it follows that every regular string-to-string function is weighted continuous. (To the best of our knowledge, this is a new result.)
We conjecture that the converse is also true.

\begin{conjecture}\label{conj:regular-continuous}
    A string-to-string function is weighted continuous if and only if it is regular.
\end{conjecture}

 One consequence of the above conjecture would be a machine independent characterisation of regular functions. This would be a very valuable contribution. Almost all known characterisations of regular functions have somewhat lengthy definitions, based on specific computational models, and it is something of a  miracle that all of these models are equivalent. A possible exception is the characterisation in~\cite{bojanczykTitoRegular23}; however that characterisation uses the abstract language of category theory, and is less elementary than the one in Conjecture~\ref{conj:regular-continuous}.

As in Conjecture~\ref{conj:protocol-regular-string-to-string}, the content of the conjecture is the left-to-right implication. 
Conjecture~\ref{conj:regular-continuous} is stronger than Conjecture~\ref{conj:protocol-regular-string-to-string}, as explained in the following diagram, which shows the known relations between three kinds of string-to-string functions:
\[
\begin{tikzcd}
\text{regular}
\ar[d,Rightarrow,shift right=2, "\text{\cref{lem:from-regular-to-protocol}}"']
\\
\text{computed by protocols}
\ar[d,Rightarrow, shift right=2, "\text{\cref{lem:postcomposition-weighted-automaton}}"']
\ar[u,Rightarrow, shift right=2,"\text{Conjecture~\ref{conj:protocol-regular-string-to-string}}"']
\\ 
\text{weighted continuous} 
\ar[uu,bend right=89, Rightarrow, shift right=2,"\text{Conjecture~\ref{conj:regular-continuous}}"']
\end{tikzcd}
\]


The proof of \cref{lem:postcomposition-weighted-automaton} could be generalised from fields to commutative semirings. Therefore, we could have an intermediate conjecture, sitting between Conjectures~\ref{conj:protocol-regular-string-to-string} and\ref{conj:regular-continuous}, which uses continuity for commutative semirings. However, it seems likely that continuity for fields is enough, so we stay with fields, and we refer only to fields when talking about weighted coninuity. 
The following theorem gives some evidence for the stronger conjecture, by showing that the weighted coniniuous functions share some well-known properties of the regular string-to-string functions. 

\begin{theorem}\label{thm:evidence-for-the-conjecture}
    If a function $f : \Sigma^* \to \Gamma^*$ is  weighted continuous, then:
    \begin{enumerate}
        \item \label{it:linear-size-outputs} the outputs have at most linear size;
        \item \label{it:linear-time-computable} the outputs can be   computed in linear time;
        \item \label{it:regular-preimages} for every regular language $L \subseteq \Gamma^*$, the preimage $f^{-1}(L)$ is a regular language.
    \end{enumerate}
\end{theorem}

% Before proving the theorem, let us comment on the properties that are listed in it.  By Lemma~\ref{lem:from-regular-to-protocol}, every regular string-to-string  function is computed by a protcol, and therefore every regular string-to-string function has the properties that are listed in the theorem. The fact the regular string-to-string functions have the  first three properties is a folklore result and can be seen directly from the definition of regular string-to-string functions, without protocols. (For the third property, it is useful to know that, as language acceptors, two-way automata recognise exactly the regular languages~\cite[Theorem 2]{shepherdson1959reduction}).
% The fact that regular functions have the  property, about postcomposition with weighted automata over a field, is not known in the literature, to the best of our knowledge. A direct proof is possible, 

\begin{proof}
    For properties~\ref{it:linear-size-outputs} and~\ref{it:linear-time-computable}, we embed strings into numbers. 
    An output string over alphabet $\Gamma$ can be seen as a number in base $|\Gamma|+1$. We do the $+1$ to avoid the digit zero, so that the encoding becomes injective. Let 
    \begin{align*}
    g : \Gamma^* \to \Rat
    \end{align*} 
    be the corresponding encoding. This encoding can be computed by a weighted automaton, see~\cite[Lemma 8.10]{bojanczyk_automata_2025}. By  the assumption on weighted continuity, the composition $f;g$ can be computed by a weighted automaton. This is a weighted automaton that only produces integers on its output, and therefore the same function can also be computed by a weighted automaton over the ring of integers $\Int$, by~\cite[p. 110]{BerstelReutenauer08}. Summing up, we have a weighted automaton over $\Int$ that outputs the representation, in base $|\Gamma|+1$, of  This tells us that 
    \begin{align*}
    g : \Gamma^* \to \Rat(\Gamma)
    \end{align*}
    is easily seen to be 
    This encoding is easily seen to be computable by a weighted automaton over the field $\Rat(\Gamma)$. Therefore, by assumption on weighted continuity, the encodings of outputs of  $f$ can be computed by a weighted automaton. Since a weighted automaton can be evaluated in linear time, it follows that the encodings of outputs of $f$ can be computed in linear time. A string is easily recovered from its encoding, thus proving property~\ref{it:linear-time-computable}. 
    
    We are left with property~\ref{it:regular-preimages}, about regularity of preimages. As mentioned before, this property is sometimes called continuity in the context of automata theory. As we will see, continuity with respect to regular languages, as in property~\ref{it:regular-preimages}, is the special case of  continuity with respect to weighted automata. More precisely, this is the special case that corresponds to weighted automata over the two-element field.  This is because of  the following  folklore correspondence between regular languages and weighted automata over the two-element field. 
        
        \begin{claim}\label{claim:regular-weighted-automata}
            A language $L \subseteq \Gamma^*$ is regular iff its characteristic function $\Gamma^* \to \set{0,1}$ is computed by a weighted automaton over the two-element field 
        \end{claim}
        \begin{proof}
            For the lef-to-right implication, we observe that a weighted automaton over a finite field can be simulated by a deterministic finite automaton. For the other direction, we observe that a weighted automaton can count the parity of  the number of runs in a finite automaton, and if the automaton is deterministic then the number of runs is either zero or one, and thus the parity gives the right answer.
        \end{proof}

        In terms of the correspondence from the above claim, preimages of regular languages become precompositions of weighted automata over the two-element field. In particular, regularity is preserved. 
\end{proof}

One could think that already the two properties in the above theorem are not only necessary for weighted continuous functions, but also sufficient. This is not the case, as shown by the following example.

