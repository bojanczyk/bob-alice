\section{Beyond Boolean outputs}
\label{sec:beyond-boolean-outputs}

In the previous section, we considered Boolean outputs. In this section, we generalise protocols to account for an arbitrary output domain. The inputs remain unchanged -- they will always be strings in this paper.  For the output domain, we use a very general notion, namely a set with some operations. 
\begin{definition}[Output domain]
    An \emph{output domain} consists of: 
    \begin{enumerate}
        \item an underlying set $\domain$; together with
        \item a family of \emph{operations}, each one having  type $\domain^n \to \domain$ for some $n \in \set{0,1,\ldots}$.
    \end{enumerate}
\end{definition}
An output domain is the same thing as a (non-indexed) algebra, in the sense of universal algebra~\cite[p.5]{hobby1988structure}. 
The output domain will typically  be infinite. In principle, the family of operations might be infinite as well \omc{This is another subjective view: the possibility of having infinitely many operations for the algebra does not seem to be useful anywhere in the paper and makes things more complicated than necessary.}, although any protocol will only use finitely many operations.
By abuse of notation, we use the same symbol $\domain$ to denote the output domain and its underlying set, with the operations being implicit. Here are the output domains that will be studied in this paper: 
\begin{itemize}
    \item \intro{Boolean domain} The set is $\set{\text{true, false}}$, and there are no basic operations. (We could add basic operations, such as the Boolean operations $\lor,\land$ and $\neg$, but this will not affect the expressive power of our protocol, so we choose to have no operations.)
    \item \intro{Field domain} The set is a field, such as the rationals or reals, and there are two operations for addition and multiplication. This is not one domain, but a family of domains, with one for each field.
    \item \intro{Monoid domain} The set is a commutative monoid $(M,+,0)$, and there is one basic operation, which is the binary operation $+$. Again, this is not one domain, but a family of domains, with one for each commutative monoid.
    \omc{The text states, “Here are the output domains that will be studied in this paper:” followed by “commutative monoid.” I think it would be better to explain the relationship between this and the unary output alphabet; otherwise, it is unclear how the commutative monoid part is connected to the rest of the paper.}
    \item \intro{String domain} This is a special case of the monoid domain, where the set is $\Gamma^*$ for some finite alphabet $\Gamma$, and the operation is string concatenation.
    \omc{You wrote "commutative monoid" above, while the title is just Monoid. Then here you mention that this is a special case of a monoid. That’s fine, but shouldn’t we remove the word commutative from the previous part?}
\end{itemize}

In the protocol, the output value will be constructed in a constant number of
steps, by using operations from the output domain. We will not distinguish
between the operations that are in the output domain, and other operations
that are derived by composing them, as described in the following definition.

\begin{definition}[Term operation]\label{def:term-operations}
    Consider an output domain $\domain$. An operation 
    \begin{align*}
    t : \domain^n \to \domain
    \end{align*}
    is called a \emph{term operation} if it can be obtained by applying the operations in the domain to variables $x_1,\ldots,x_n$. Each variable can be used multiple times, or not at all.  We write 
    \begin{align*}
    \domain^n \termop \domain
    \end{align*}
    for the set of term operations with $n$ arguments.
\end{definition}

\begin{myexample}
    If the operations are $+$ and $\times$, then after applying the usual distributivity laws,   term operations are the same as polynomials with natural coefficients, such as 
\begin{align*}
x_1x_2^2 + x_1x_2x_3^3  + \myunderbrace{2x_1}{same as $x_1 + x_1$}.
\end{align*}
If the output domain is the Boolean domain, which has no operations, then the only kind of term operation is a single variable, e.g.~$x_2$.  In the case of a string domain, a term operation is some concatenation of the variables, such as $x_1 x_3 x_1 x_2$.
\end{myexample}



We now generalise the protocol to cover functions 
\begin{align*}
f : \Sigma^* \to \domain,
\end{align*}
where $\domain$ is some possibly infinite output domain. As in the Boolean
version of the protocol, the output value is constructed by Alice and Bob, as a
result of an exchange of a constant number of messages. Each message has two
parts, which are called the \emph{signal part} and the \emph{output part}. The
signal part consists of a finite amount of information, and is used to exchange
information between the two parties as in the Boolean protocol. The output part
is a list of elements from the  output domain, and  is meant to be part of the
output value. As in the Boolean case, the above definition slightly deviates
from the informal description in the introduction. In particular, a message can
contain $d$ elements of the output domain. This does not affect the expressive
power of the protocol, but it might affect the number of rounds, and the above
definition will be more convenient later one, where we consider one-round
protocols. Another important difference is that the operations from the output
domain are only applied once at the end of the protocol. This, again does
affect the expressive power, since the two parties can wait with their
operations until all signals have been exchanged.  

\begin{definition}\label{def:two-party-protocol-general} A \intro{two-party protocol}
   is given by the following ingredients: 
  \begin{enumerate}
    \item an \kl{output domain} $\domain$;
    \item a finite input alphabet $\Sigma$;
    \item a \intro{number of rounds} $k \in \set{1,2,\ldots}$;
    \item \intro{signal spaces} for Alice and Bob, which are finite sets $Q_A$ and $Q_B$;
    \item a \intro(protocol){dimension} $d \in \set{0,1,\ldots}$;
    \item for each round $i \in \set{1,\ldots,k}$, a \intro{strategy}
    \begin{align*}
    \myoverbrace{\sigma_A^i : \myunderbrace{\Sigma^*}{Alice's \\ local\\ string} \times \myunderbrace{Q_B^{i-1}}{history of \\ signals \\ from Bob}  \to \myunderbrace{Q_A \times \domain^d}{new\\ message}}{stategy for Alice in the $i$-th round}
    \qquad 
        \myoverbrace{\sigma_B^i : \myunderbrace{\Sigma^*}{Bob's \\ local\\ string} \times \myunderbrace{Q_A^{i-1}}{history of \\ signals \\ from Alice}  \to \myunderbrace{Q_B \times \domain^d}{new\\ message}}{stategy for Bob in the $i$-th round}
    \end{align*}
    \item an \intro[output function@protocol]{output function} of type \begin{align*}
    (Q_A \times Q_B)^{k} \to (\domain^{2dk} \termop \domain).
    \end{align*}
  \end{enumerate}
\end{definition}

\alc{Maybe we should use uniform strategies that take the history $Q_B^{*}$ and
$Q_A^{*}$ instead of fixed number of rounds? This is just to simplify
notations.}



The protocol is executed on a pair of strings $(w_1,w_2)$. In each round, each of the two parties  sends a message (which consists of a signal and some elements of the output domain) that is based on their local string and the history of signals coming from the other party. After all $k$ rounds have been executed, the joint signal history of both players is used, by the output function, to determine a term operation. This operation is then applied to the output history, yielding the final result of the protocol. As in the Boolean case, we are interested in protocols that are split invariant, which means that for every string $w \in \Sigma^*$, the same output is produced for every possible decomposition $w = w_1 w_2$. Such protocols compute a function of type $\Sigma^* \to \domain$. 

\begin{myexample}[Boolean domain]
    Consider the Boolean output domain. In this case, the distinction between output values and signals is irrelevant, since the output values can be sent as signals. Therefore, the general protocol coincides with the Boolean protocol from the previous section, and  can only compute regular languages.  The same remarks apply for a general but finite output domain -- a function $f : \Sigma^* \to \domain$ can be computed by a protocol if and only if for every $d \in \domain$, the inverse image $f^{-1}(d)$ is a regular language.
\end{myexample}

Other examples of output domains are fields and strings. These were discussed  briefly in the introduction in Sections~\ref{sec:intro-field} and~\ref{sec:intro-strings},  and will be discussed at length in Sections~\ref{sec:field-domain} and~\ref{sec:string-outputs}.  

% Nevertheless, we give here one example which concerns strings, and uses both signals and output values in the messages.

% Here are two brief examples.
% \begin{myexample}[Field domain] Assume that the output domain is a field, say the field of rationals, with the operations being addition and multiplication. 
%     Consider the function 
%     \begin{align*}
%     f : \set{0,1}^* \to \mathbb Q,
%     \end{align*}
%     which maps a binary string to the natural number that it represents. For example, the input $01000$ will be mapped to $8$. Here is a protocol. Based on his input string $w_2$, Bob computes two numbers, namely: 
%     \begin{align*}
%     \myunderbrace{y_1 = f(w_2)}{number represented\\  by $w_2$} 
%     \qquad \text{and} \qquad 
%     \myunderbrace{y_2 = 2^{|w_2|}}{the power of two \\ corresponding to the\\ length of $w_2$}.
%     \end{align*}
%     Alice does the same, yielding two numbers $x_1$ and $x_2$, of which only $x_1$ will be used. The final output is then 
%     \begin{align*}
%     x_1\cdot y_2 + x_2.
%     \end{align*}
%     In this protocol, there are no signals. As we will see in the next section, when the output domain is a field, then signals are not needed (unlike the case of strings or Booleans, where signals are needed).
% \end{myexample}


% \begin{myexample}[Conditional reverse]
%     In Example~\ref{ex:reverse-duplicate}, we explained a protocol for reversing the input string. Here is a variant of this protocol, which uses signals. Consider the 
%     \begin{align*}
%     f : \Sigma^* \to \Sigma^*
%     \end{align*}
%     which reverses the string if the first letter is $a$, and otherwise it leaves the string unchanged. To compute this function, Alice needs to send to Bob -- apart from her part of the output string -- a signal to Bob which tells him if the first letter is $a$.
% \end{myexample}



\subsection{One-round protocols}

We have little say to say about  protocols in the case of a general output
domain. The  only result that we have at this level of generality is a
reduction to one-round protocols, similarly to the Boolean case.

\begin{lemma}
\label{lemma:one-round-reduction-general}
  Every protocol is equivalent to a one-round protocol.
\end{lemma}
\begin{proof}
  Same proof as in \cref{lem:one-round-reduction-boolean}. 
  \omc {The idea of the proof is the same but some details are different because of the output part of the messages. should I rewrite the proof to reflect these differences? or just mention that the proof is similar with minor changes?}
\end{proof}

 Later in this paper, we will study protocols for infinite input alphabets. Such protocols will not be a special case of the protocols developped so far, since the signal spaces will be infinite, albeit in a limited way (called orbit-finite).  As we will see, for such signal spaces the reduction to one round will no longer be valid, and the interactive nature of the protocols will be essential.
